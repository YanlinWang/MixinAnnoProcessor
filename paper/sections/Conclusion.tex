\section{Conclusion}\label{sec:conclusion}

Before Java 8, concrete methods and static methods were not allowed
to appear in interfaces.  Java 8 allows static interface methods and
introduces \emph{default methods}, which enables implementations
inside interfaces. This had an important positive consequence that
was probably overlooked: the concept of class
(in Java) is now (almost) redundant and unneeded.  We define a subset of Java,
called Classless Java, where programs and (reusable) libraries can be
easily defined and used.  To avoid for some syntactic boilerplate
caused by Java not being originally designed to work without classes,
we introduce a new annotation, \mixin, which provides default implementations
for various methods (e.g. getters, setters, with-methods) and a
mechanism to instantiate objects. The \mixin annotation helps programmers
to write less cumbersome code while coding in Classless Java. Indeed
we think the obtained gain is so high that Classless Java with the \mixin
annotation can be less cumbersome than full Java.
Interestingly, without classes there is also no subclassing. This scratches an old
  itching point in the long struggle of subtyping versus subclassing:
  According to some authors, from a software engineering perspective,
  interfaces are just a kind of classes. Others consider more
  opportune to consider interfaces are pure types. We do not know how to conciliate
  those two viewpoints and ClassLess Java design.
  We do not have classes purely in the Java sense.
  
 Our language restriction encourage to code in a more flexible way by forcing to either keep a higher
 abstraction level (interfaces are a more abstract concept than classes), or to rely on
 concrete object initialization (the \Q@new I(){...}@ construct).
 
 More importantly, we identify the concept of \emph{language tuning}.
 We identify libraries that are already performing language tuning (Lombock and Cofoja), and 
 we forecast many different kinds of language tuning will emerge on suitable platforms like Java or the C\# CLR.
 We identify varius kinds of safety guarancees that can be offered by language tuning, but the door is open for more flavours of safety guarantees to emerge.