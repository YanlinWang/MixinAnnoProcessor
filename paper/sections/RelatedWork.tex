\section{Related Work}\label{sec:related}
In this section we discuss related work and comparison to Classless Java.

%\yanlin{need to discuss: whether the comments for FTJ is fair.}

%No, was not fair, I know that work, it does modular type checking, and 
%more work after that they extend with more safe/modular typing things.
%also, Java8 IS a language extension on Java
%However, language extensions (including FTJ)
%have a natural drawback: the programmer has to learn new syntax. In contrast,
%our approach is completely compatible with the current Java language, so that
%programmers don't need to pay any learning cost to adapt to this new classless
%programming pattern. Another drawback which is particular for FTJ is that FTJ
%doesn't have type for traits, hence the correctness check of trait is done when
%type-checking classes. This choice makes the design of FTJ simpler but lost
%typechecking efficiency (one trait will be potentially checked multiple times if
%it is used in multiple classes).

\subsection{Multiple Inheritance in Object Oriented Languages}
Many authors have argued in favour or against multiple inheritance.  It provides
expressive power, but it is difficult to model and implement, and can create
programs that are hard to reason about. These difficulties include the famous
diamond (fork-join) problem~\cite{bracha90mixin,Sak89dis}, conflicting methods, etc.
%and the yo-yo problem~\cite{taenzer1989problems}.  
To conciliate the need for expressive power and the need for simplicity, lots of
models have been proposed in the past few years, including C++ virtual
inheritance, mixins~\cite{bracha90mixin} and traits~\cite{scharli03traits}.
They provide novel programming architecture models in the OO paradigm. In terms
of restrictions set on these models, C++ virtual inheritance aims at a more
general model, mixins added some restrictions on the model, and trait model is
the most restricted one (excluding states, instantiation, etc).

\paragraph{C++ Approach.}
C++ tries to provide a general solution to multiple inheritance. Virtual
inheritance in C++ provides another solution to multiple inheritance (especially
the diamond problem by keeping only one copy of the base
class)~\cite{ellis1990annotated}, however suffers from object initialization
problem as pointed out by Malayeri et al.~\cite{malayeri2009cz}. It bypasses all
constructor calls to virtual superclasses, which would potentially cause serious
semantic error. 

\paragraph{Mixins.}
Mixins are a more restricted model than C++ approach. It allows to name
components that can be applied to various classes as reusable functionality
units. However, they suffer from linearisation: the order of mixin application
is relevant in often subtitle and undesired ways. This hinders their usability
and the ability of resolving conflicts: the linearisation (total ordering) of
mixin inheritance cannot provide a satisfactory resolution in all cases and
restricts the flexibility of mixin composition. To fight those limitations, an
algebra of mixin operators is introduced~\cite{ancona2002calculus}, but this
raised the complexity of the approach, especially when constructors and fields
are considered~\cite{marco09FJigsaw}. Scala solution adopts linearised mixins,
(but misleadingly called traits in the language) avoids object initialization
problem by disallowing constructor parameters, causing no ambiguity in cases
such as diamond problem. However this approach has limited expressiveness, and
suffers from all the problems of linearised mixin composition.
Python also offers multiple inheritance via linearised mixins. Indeed in python any class is implicitly a mixin, and mixin composition informally expressed as\\*
\Q@ class A use B,C {...new methods...}@\\*
can be expressed in python as \\*
\Q@ class Aux: ...new methods...@\\*
\Q@ class A(B,C,Aux): pass@ 

\noindent Our approach does not involve linearised mixin problem, in that the semantics of
Java \textbf{extends} clause is unordered and symmetric. Also,our approach
supports proper constructor mechanism.

\paragraph{Traits.}
Simplifying the mixins approach, traits~\cite{scharli03traits} draw a strong
line between units of reuse (traits) and object factories (classes). In this
model, traits as units of reusable code, contain only methods as reusable
functionalities. Thus, no state/state initialization is considered. Classes act
as object factories, requiring functionalities from multiple traits.  Traits
offer a trait algebra with operations like sum, alias and exclusion, provided
for explicit conflict resolution.  Concluding, (pure) traits do not allow state
and they do not offers any reuse instrument to ensure that state is coherently
initialized when finally defined in classes.  Traits can not be
instantiated. This model requires two concepts (traits and classes) to coexist
and cooperate.  Some authors see this as good language design fostering good
software development by helping programmers to think about the structure of
their programs.  However, other authors see the need of two concepts and the
absence of state as drawbacks of this model. Our approach takes interfaces as
units of reuse, and meanwhile generates factory method for instantiation.

% \begin{enumerate}
% \item Mixins allows to name components that can be applied to various classes as
%   reusable functionality units. However, they suffer from linearisation: the
%   order of mixin application is relevant in often subtitle and undesired
%   ways. constraints. This hinders their usability and their ability of resolving
%   conflicts: the linearisation (total ordering) of mixin inheritance cannot
%   provide a satisfactory resolution in all cases and restricts the flexibility
%   of mixin composition.  To fight those limitations, an algebra of mixin
%   operators is introduced~\cite{ancona2002calculus}, but this raised the
%   complexity of the approach, especially when constructors and fields are
%   considered~\cite{marco09FJigsaw}.  Our approach does not have the
%   linearisation problem, in that the semantics of Java \textbf{extends} clause
%   is unordered and symmetric.
% \item Simplifying the mixin algebras approach, traits draw a strong line between
%   units of reuse (traits) and object factories (classes) In this model,
%   traits~\cite{scharli03traits} units of reusable code, containing only methods
%   as reusable functionalities. Thus, no state/state initialization is
%   considered.

%   Classes act as object factories, requiring functionalities from multiple
%   traits.  Traits offers a trait algebra with operations like sum, alias and
%   exclusion, provided for explicit conflict resolution.

%   Concluding, (pure) traits do not allow state and they do not offers any reuse
%   instrument to ensure that state is coherently initialized when finally defined
%   in classes.  Traits can't be instantiated. This model requires two concepts
%   (traits and classes) to coexist and cooperate.

%   Some authors see this as good language design fostering good software
%   development by helping programmers to think about the structure of their
%   programs.  However, other authors see the need of two concepts and the absence
%   of state as drawbacks of this model. Our approach takes interfaces as units of
%   reuse, and meanwhile generates factory method for instantiation.

% \item C++ and Scala also try to provide solutions to multiple inheritance, but
%   both suffer from object initialization problems. Virtual inheritance in C++
%   provides another solution to multiple inheritance (especially the diamond
%   problem by keeping only one copy of the base class)~\cite{ellis1990annotated},
%   however suffers from object initialization problem as pointed out by Malayeri
%   et al.~\cite{malayeri2009cz}. It bypasses all constructor calls to virtual
%   superclasses, which would potentially cause serious semantic error. Scala
%   solution (very similar to linearised mixins, but misleadingly called traits in
%   the language) avoids this problem by disallowing constructor parameters,
%   causing no ambiguity in cases such as diamond problem.  This approach has
%   limited expressiveness, and suffers from all the problems of linearised mixin
%   composition.
%   Python also offers multiple inheritance via linearised mixins. Indeed in python any class is implicitly a mixin, and mixin composition informally expressed as\\*
%   \Q@ class A use B,C {...new methods...}@\\*
%   can be expressed in python as \\*
%   \Q@ class Aux: ...new methods...@\\*
%   \Q@ class A(B,C,Aux): pass@
%   Our approach not only does not involve linearised mixin problem, but also
%   support proper constructor mechanism.
% \end{enumerate}

\subsection{Multiple Inheritance in Java}
Since Java 8 default methods are introduced, concrete method implementation are
allowed to be defined (via the \textbf{default} keyword) inside
interfaces. 
% Since Java supports implementation of multiple interfaces (instead
% of extension of a single class), 
The introduction of default methods opens the
gate for various flavours of multiple inheritance in Java, using interfaces.
Former work by Bono.et. al.~\cite{bono14}. provides details on mimicking traits
through interfaces.

There are proposals for extending Java (before Java8) with traits. For example,
FeatherTrait Java (FTJ)~\cite{Liquori08ftj} by Liquori et al. extends the
calculus of Featherweight Java (FJ)~\cite{Igarashi01FJ} with statically-typed
traits, adding trait-based inheritance in Java.  Except for few, mostly
syntactic details, their work can be emulated/rephrased with Java8 interface.


%\section{Comparing to traits and mixins}\label{sec:comparison}
\subsection{Multiple Inheritance in ClassLess Java}
\begin{comment}
Haoyuan

   - vs both: we do automatic return type refinement, which has useful applications
   (example: Expression Problem)

   - vs traits: we support of methods to create new objects (a replacement to constructors);
   Moreover we have the with and clone methods (we miss more applications for those). Show
   how to model the operations on traits; discuss operations that we cannot model
   (example: renaming).

   - vs mixins: we use the trait model of explicitly resolving conflicts. This is arguably
   better for reasoning.
\end{comment}

Our approach is based on forbidding the use of certain language constructs (classes), in order to rely on the modular composition offered by the rest of the language (interfaces).
Since Java was designed for classes, a direct class-less programming style is verbose and feel unnatural. However, annotation driven code generation is enough to overcome this difficulty, and
the resulting programming style encourages modularity, composability and reusability, by keeping a strong object oriented feel.
We consider our approach as an alternative to traits or mixins.

A comparison on how to to emulate the original operations on traits using Java8 can be find in\cite{bono14}. We briefly recall the main points of their encoding; however we propose a different representation of
 \textbf{exclusion}.
The author of ~\cite{bono14} agree that our revised version is cleaner, typesafe and more direct.\bruno{is this something to 
say in the paper?}

%%\newcommand\shortItem{\vspace{-1ex}}
\begin{itemize}
\begin{comment}%I shoreten up a lot after. I think this version is better but longer, and this content is just repeating bono at all.
\item \textbf{Symmetric sum}: the symmetric composition of two disjoint traits is achieved by simply implementing two interfaces in Java correspondingly, without overriding any method. The composition relies on multiple inheritance on interfaces, which is supported by Java. Below is a simple example.
    \begin{lstlisting}
    interface A { int x(); }
    interface B { int y(); }
    interface C extends A, B {}
    \end{lstlisting}
\item \textbf{Override}: the overriding operation (also known as asymmetric/preferential sum) is modelled by implementing many interfaces, while overriding some methods inside. The code below gives an example of explicitly specifying which super interface take precedence, regarding conflict on a specific method.
    \begin{lstlisting}
    interface A { default int m() {return 1;} }
    interface B { default int m() {return 2;} }
    interface C extends A, B { default int m() {return B.super.m();} }
    \end{lstlisting}
    Here the method \texttt{m()} in interface \texttt{C} simply inherits from \texttt{B.m()}.
\item \textbf{Alias}: an alias operation adds a new name to an old method when creating the new trait. In Java, we just create a new method with reference to the existing method in its super interface. See the example below, where the new method \texttt{k()} is an alias of the existing method \texttt{m()}.
    \begin{lstlisting}
    interface A { default int m() {return 1;} }
    interface B extends A { default int k() {return A.super.m();} }
    \end{lstlisting}
\end{comment}

\item \textbf{Symmetric sum} can be obtained by simple multiple inheritance between interfaces.
\shortItem
    \begin{lstlisting}
    interface A { int x(); }    interface B { int y(); }    interface C extends A, B {}
    \end{lstlisting}
\shortItem
\item \textbf{Overriding} a conflict is obtained by specifying which super interface take precedence.
\shortItem
    \begin{lstlisting}
    interface A { default int m() {return 1;} } 
    interface B { default int m() {return 2;} }
    interface C extends A, B { default int m() {return B.super.m();} }
    \end{lstlisting}
\shortItem
\item \textbf{Alias} is creating  a new method delegating to the existing super interface.
\shortItem
    \begin{lstlisting}
    interface A { default int m() {return 1;} }
    interface B extends A { default int k() {return A.super.m();} }
    \end{lstlisting}
\shortItem

\item \textbf{Exclusion}: exclusion is also supported in Java, where method declarations can hide the default methods correspondingly in the super interfaces.
\shortItem
    \begin{lstlisting}
    interface A { default int m() {return 1;} }
    interface B extends A { int m(); }
    \end{lstlisting}
\shortItem
\end{itemize}

Besides, we support more features than the original trait model:
\begin{itemize}
\item We provide \texttt{of} methods for the annotated interfaces. During annotation processing time, the ``fields'' inside an interface are observed and a static method \texttt{of} is automatically injected to the interface with its arguments correspondingly. Such a method is a replacement to the constructors in original traits, making instantiation more convenient to use.
That is, in our approach there are only interfaces, our model requires a single concept, while the trait model requires traits \emph{and} classes.

\item We provide \texttt{with-} methods as auxiliary constructors. A \texttt{with-} method is generated for each field, just like record update, returning the new object with that field updated
%. A \texttt{clone} method is generated for the interface, returning a copy of the current object.
Furthermore, we do automatic return type refinement for these kind of methods. This feature is comparatively useful in big examples, making operations and behaviours more flexible.%, which we will demonstrate later.
\item We provide two options for generating setters. There are two kind of setters which are commonly used, namely \textit{void setters} and \textit{fluent setters}. The only difference is that a fluent setter returns the object itself after setting, thus supporting a pipeline of such operations. The generation depends on which type of setter is declared in the interface by users.
\end{itemize}

These are the additional features supported by our model, conversely, there are certain operations we cannot model, such as method renaming (as in [Reppy2006]), which breaks structural subtyping.

There are other limitations of our current approach, but they may be addressed
in future work (see Section~\ref{sec:futurework}).



\subsection{ThisType/MyType/Extensibility}
%\marco{ this section is unfair. There are unsolved problems with ThisType in negative positions, namelly it breaks subtyping. The most common solution is to allow calling some methods only when the exact type is known. This demolish most advantages of interfaces.}

In certain situations, our approach allows automatic refinement for return
types. This is part of a bigger topic in class based languages: expressing and
preserving type recursion and (nominal/structural) subtyping at the same time.

One famous attempt in this direction is provided by
\emph{MyType}~\cite{bruce1994paradigmatic}, representing the type of
\textbf{this}, changing its meaning along with inheritance. However when
invoking a method with MyType in a parameter position, the exact type of the
receiver must be known.  This is huge limitation in class based object oriented
programming, and is exasperated by interface-only programming as we propose: no
type is ever exact since classes are not part of the language. A recent
article~\cite{Saito2013933} lights up this topic, since they propose two
features: exact statements \yanlin{do we support exact statements?}\marco{no we
  do not. When is that it may seams we do?} and non-inheritable methods; both
related to our work: any method generated inside of the \Q@of@ method is indeed
non-inheritable, since there is no class name to extends from, and exact
statements (a form of wild-card capture on the exact run-time type) could
capture the ``exact type'' of an object even in a class-less environment.



% The addition of MyType to a language will allow easy definition of
%binary methods, methods with recursive types (i.e., the same type of the
%receiver appears in the argument or return positions of methods), etc. MyType
%greatly enhances the expressiveness and extensibility of object-oriented
%programming languages. 

%In our approach, we are using covariant-return types to simulate some uses of
%MyType. But our approach only works on method positive positions, whereas MyType
%is more general, as it works on any positions. Nevertheless our approach is
%still useful for modeling fluent interfaces and solving expression
%problems,etc. 

\begin{comment}
\subsection{Type-Directed Translations/Syntactic Sugar}
\marco{I'm tring to merge this and the next one}
Language extensions are often implemented as syntactic sugar of the base
language. For example, Scala compiler supports XML syntax in normal Scala code
directly (after Scala ?, users need to import \texttt{scala-xml} library
manually). However, this approach is hard in terms of implementation, because it
requires extending the compiler. Also, this approach does not support combining
multiple extensions into one.

SugarJ~\cite{erdweg11sugarj} is a Java-based extensible programming language
that allows programmers to extend the base language with custom language
features by definitions in meta-DSLs (SDF, Stratego, etc). 
\yanlin{Is new syntax really a ``drawback''? I think for some system, like
  SugarJ, one of purpose IS to introduce these new syntax.}  Drawbacks: new
syntax. To create the extension, programmers have to work with multiple
languages (SDF, stratego, etc) while our approach works totally in Java
environment.

We can model certain types of language extensions with annotations 
only, but those extensions do not introduce new syntax: they 
merely do automatic code generation. 
\end{comment}
\subsection{Meta-programming Competes with Language extensions}
The most obvious solution to features addition to a language is language
extension.  It is often implemented as syntactic extensions that can be
desugared to the base language. For example, the Scala compiler was extended to
directly supports XML syntax. However, this approach does not support combining
multiple extensions into one. We are de facto creating a fork in the language,
and rarely the new fork gain enough traction to become the main language
release.  On this topic we mention SugarJ~\cite{erdweg11sugarj} - a Java-based
extensible language allowing programmers to extend it with custom features by
definitions in meta-DSLs (SDF, Stratego, etc).

On the other side, when the starting language have a flexible enough syntax and
a fast and powerful enough reflection mechanism, we may just need to play with
operator overloading and other language tricks to discover that the language
feature we need can be expressed as a simple library in our language. For
example, consider SQLAlchemy in python.

Java-like languages tend to sit in the middle of two extremes: libraries
can not influence the type system, so many solutions valid in python or other
languages could not be applicable, or may be applicable at the cost of loosing
safety.

Here (compile/load time) code generation comes at the rescue: if for a certain
feature (\mixin in our case) it is possible to use the original language syntax
to \emph{express-describe} any specific instantiation of such feature
(annotating a class and provide getters), then we can insert in the compilation
process a tool that exam and enrich the code before compilation. No need to
modify the original source; for example we can work on temporary files
\yanlin{Here what does temporary files mean? could explain more?}. Java is
a particular good candidate for this kind of manipulation since it already
provide ways to define and integrate such tools in its own compilation process:
in this way there is no need of temporary files, and there is a well defined way
of putting multiple extensions together.

Other languages offer even stronger support to safe code manipulation:
Template Haskell~\cite{sheard2002template}, F\# (type providers)~\footnote{http://research.microsoft.com/fsharp/} and MetaFjig (Active Libraries)~\cite{servetto2010metafjig}
all allow to execute code at compile time and to generate code/classes that are transparently integrated in the program that is being generated/processed/compiled.
In particular, MetaFjig offers a property called \emph{meta-level-soundness}. In short this property ensures by construction that library code (even if wrong or non nonsensical) would never generate ill-typed code. This is roughly equivalent to what we state and manually proof in Lemma 2 for our particular transformation.
Since MetaFjig is not working on annotated classes, there is no direct equivalence on the overall theorem of safety we shown.

\subsection{Formalization of Java8}
We provide a simple and well designed formalization for a subset of Java
including default/static interface methods and object initialization literals
(often called anonymous local inner classes).  A similar formalization was
drafted by Goetz and Field~\cite{goetz12fdefenders} to formalize defender
(default) methods in Java. However this formalization is limited to model
exactly one method inside classes/interfaces.

% As an evidence of the attention and care present in our formalization work,
% while double-checking the behaviour of Java in side cases we have discovered a
% likely bug in the current \texttt{javac}.\marco{refer to before when we explain
%   the issue.}
\yanlin{bug claim removed.}

\section{Future work}\label{sec:futurework}

\subsection{Private state}
The biggest limitation of our approach is the absence in Java8 of support for private/protected methods in interfaces.That is, in Java8 all members of interfaces are public, including static methods.
Since we use abstract methods to encode the state, our state is always all public; however is impossible for the user to know if a certain method maps directly to a field or if it have a default implementation.
If the use wants a constructor that does not directly maps to the fields, (as for secondary constructors in Scala) he can simply define its own \Q@of@ method and delegate on the generated one, as in
\begin{lstlisting}
@Mixin interface Point{
  int x(); int y();
  static Point of(int val){return Point.of(val,val);}  
  }
\end{lstlisting}
However, the generated \Q@of@ method would also be present and public.
If a future version of Java was to support \emph{static private methods in interfaces} we could extend our code generation to handle also encapsulation.
Currently, is possible to use a public nested class with private static methods inside, but this is ugly and cumbersome. We are considering if our annotation processor can take code with \Q!@Private! annotation and turn it into static private methods of a nested class. In this extension,  also the of method could be made private following the same pattern. 

\subsection{State initialization}

As discuss before, the user can trivially define its own \Q@of@ method, and initialize a portion of the state with default values.
However, the initialization code would not be reused/reusable, and subinterfaces would have to repeat such initialization code.
If a field has no setters, a simple alternative is to just define the ``field'' as a default method as in 
\begin{lstlisting}
@Mixin interface Box{ default int val(){return 0;} }
\end{lstlisting}
if setters are required, a possible extension of our code expansion could recognize a field if the getter is provided and the setter is required, and could generate the following code:
\begin{lstlisting}
interface Box{ 
  default int val(){return 0;} //provided
  void val(int _val);//provided
  static Box of(){return new Box(){//generated
    int val=Box.super.val();
    int val(){return val;}
    void val(int _val){val=_val;}
    };}}
\end{lstlisting}
We are unsure of the value of this solution: is very tricky, the user define a method that (contrary to our usual expectation) is actually overridden in a way that the behaviour change, but change only after the first setter is called, plus this code would cache the result instead of re-computing it every time. This can be very relevant and tricky in a non functional setting.

\subsection{Class invariants in ClassLess Java}
Since the objects are created by automatically generated methods,
another limitation of our current approach is that there is no place where the user can dynamically check for class invariants.
In Java often we see code like
\begin{lstlisting}
class Point{ int x; int y;
  Point(int x; int y){this.x=x;this.y=y; assert this.checkInvariant();}
  private boolean checkInvariant(){... x>0,y>0...}
\end{lstlisting} 

We are considering an extension of our annotation where 
default methods with the special name \Q@checkInvariant()@ will be called inside of the \Q@of@ methods.
if multiple interfaces are implemented, and more then one offers
\Q@checkInvariant()@,  a composed implementation could be automatically generated, composing by \Q@&&@ the various competing implementations.

\subsection{Clone, toString, equals and hashCode}
Methods originally defined in \Q@Object@, as \Q@clone@ and \Q@toString@ can be supported by our approach, but they need special care. If an interface annotated with \mixin ask an implementation for \Q@clone@, \Q@toString@, \Q@equals@ or \Q@hashCode@ we can easily generate one from the fields.\footnote{In particular, for clone we can do automatic return type refinement as we do for \Q@with-@ and fluent setters. Note how this would solve most of the Java ugliness related to \Q@clone@ methods.}

However, if the user wish to provide its own implementation, since the method is also implemented in \Q@Object@ we would have a conflict, that we have to explicitly resolve inside of \Q@of@, by implementing the method and delegating it to the user implementation, thus

\begin{lstlisting}
@Mixin interface Point{ int x(); int y();
  default Point clone(){ return Point.of(0,0);}//user defined clone
  }
\end{lstlisting} 
Would expand into 

\begin{lstlisting}
interface Point{ int x(); int y();
  default Point clone(){ return Point.of(0,0);}//user defined clone
  public static Point of(int _x,int_y){
    return new Point(){...
      public Point clone(){ return Point.super.clone();}
      };  }  }
\end{lstlisting} 
