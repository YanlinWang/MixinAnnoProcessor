\section{Related Work}\label{sec:related}
In this section we discuss related work and comparison to Classless Java.

%\yanlin{need to discuss: whether the comments for FTJ is fair.}

%No, was not fair, I know that work, it does modular type checking, and 
%more work after that they extend with more safe/modular typing things.
%also, Java8 IS a language extension on Java
%However, language extensions (including FTJ)
%have a natural drawback: the programmer has to learn new syntax. In contrast,
%our approach is completely compatible with the current Java language, so that
%programmers don't need to pay any learning cost to adapt to this new classless
%programming pattern. Another drawback which is particular for FTJ is that FTJ
%doesn't have type for traits, hence the correctness check of trait is done when
%type-checking classes. This choice makes the design of FTJ simpler but lost
%typechecking efficiency (one trait will be potentially checked multiple times if
%it is used in multiple classes).

\subsection{Multiple Inheritance in Object Oriented Languages}
Many authors have argued in favour or against multiple inheritance.  Multiple
inheritance is very expressive, but difficult to model and implement, and can
cause difficulty (including the famous diamond (fork-join)
problem~\cite{bracha90mixin,Sak89dis}, conflicting methods, etc.) in reasoning about 
programs. To conciliate the need for expressive power and simplicity, many
models have been proposed, including C++ virtual inheritance,
mixins~\cite{bracha90mixin}, traits~\cite{scharli03traits}, and hybrid models
such as CZ~\cite{malayeri2009cz}.  They provide novel programming architecture
models in the OO paradigm. In terms of restrictions set on these models, C++
virtual inheritance aims at a relative general model; the mixin model adds some
restrictions; and the trait model is the most restricted one (excluding
states, instantiation, etc).

\paragraph{C++ Approach.}
C++ tries to provide a general solution to multiple inheritance by
virtual inheritance, dealing with the diamond problem by keeping only
one copy of the base class~\cite{ellis1990annotated}. However C++
suffers from the object initialization problem~\cite{malayeri2009cz}.
It bypasses all constructor calls to virtual superclasses, which can
potentially cause serious semantic errors. In our approach, the \mixin
annotation has full control over object initialization,
and the mechanism is transparent to users. Moreover, customized factory
methods are also allowed: if users are not satisfied with the default
generated \texttt{of} method, they can implement their own.

\paragraph{Mixins.}
Mixins are a more restricted model than the C++ approach. Mixins allow to name
components that can be applied to various classes as reusable functionality
units. However, % they suffer from linearisation: the order of mixin application
% is relevant in often subtle and undesired ways. This hinders their usability
% and the ability of resolving conflicts:
the linearisation (total ordering) of
mixin inheritance cannot provide a satisfactory resolution in some cases and
restricts the flexibility of mixin composition. To fight this limitation, an
algebra of mixin operators is introduced~\cite{ancona2002calculus}, but this
raises the complexity, especially when constructors and fields
are considered~\cite{marco09FJigsaw}. Scala traits are in fact more like linearised mixins.
Scala avoids the object initialization
problem by disallowing constructor parameters, causing no ambiguity in cases
such as diamond problem. However this approach has limited expressiveness, and
suffers from all the problems of linearised mixin composition. Other languages, such as 
Python, also use linearized mixins.
\begin{comment}
Python also offers multiple inheritance via linearised mixins. Indeed in python any class is implicitly a mixin, and mixin composition informally expressed as\\*
\Q@ class A use B,C {...new methods...}@\\*
can be expressed in python as \\*
\Q@ class Aux: ...new methods...@\\*
\Q@ class A(B,C,Aux): pass@ 
\end{comment}
\noindent Java interfaces and default methods do not use
linearisation: the semantics of Java \textbf{extends} clause in
interfaces is unordered and symmetric.

% However, in pure Java, there is no mechanism for creating objects in
% interfaces. Also, our approach supports proper constructor mechanism.

\paragraph{CZ.}
Malayeri and Aldrich proposed a model CZ~\cite{malayeri2009cz} which
aims to do multiple inheritance without the diamond problem. They
divide inheritance into two separate concepts: inheritance dependency
and implementation inheritance. Using a combination of
\texttt{requires} and \texttt{extends}, a program with diamond
inheritance can be transformed to one without diamonds. Moreover,
fields and multiple inheritance can coexist. However untangling
inheritance also untangles the class structure. In CZ, not only the
number of classes, but also the class hierarchy complexity
increases. \mixin does not complicate the class structure, and state
can also coexist with multiple inheritance.

\paragraph{Traits and Java's default methods}
Simplifying the mixins approach, traits~\cite{scharli03traits} draw a strong
line between units of reuse and object factories. Traits, as units of reusable
code, contain only methods as reusable functionality, ignoring states and state
initialization. Classes, as object factories, require functionalities from
(multiple) traits. Java 8 interfaces are closely related to
traits: concrete method implementations are allowed (via the \textbf{default}
keyword) inside interfaces. The introduction of default methods opens the gate
for various flavours of multiple inheritance in Java. Traits offer an algebra
of composition operations like sum, alias and exclusion, providing explicit conflict
resolution. Former work~\cite{bono14} provides details on mimicking the trait
algebra through Java 8 interfaces.  We briefly recall the main points of their
encoding; however we propose a different representation of \textbf{exclusion}.
The first author of~\cite{bono14} agreed (via personal communication) that 
our revised version for exclusion is cleaner, typesafe and more direct.

\newcommand\shortItem{\vspace{-1ex}}
\begin{itemize}
\item \textbf{Symmetric sum} can be obtained by simple multiple inheritance between interfaces.
    \begin{lstlisting}
    interface A { int x(); }    interface B { int y(); }    interface C extends A, B {}
    \end{lstlisting}
\shortItem
\item \textbf{Overriding} a conflict is obtained by specifying which super interface take precedence.
    \begin{lstlisting}
    interface A { default int m() {return 1;} } 
    interface B { default int m() {return 2;} }
    interface C extends A, B { default int m() {return B.super.m();} }
    \end{lstlisting}
\shortItem
\item \textbf{Alias} is creating  a new method delegating to the existing super interface.
    \begin{lstlisting}
    interface A { default int m() {return 1;} }
    interface B extends A { default int k() {return A.super.m();} }
    \end{lstlisting}
\shortItem

\item \textbf{Exclusion}: exclusion is also supported in Java, where method declarations can hide the default methods correspondingly in the super interfaces.
    \begin{lstlisting}
    interface A { default int m() {return 1;} }
    interface B extends A { int m(); }
    \end{lstlisting}
\shortItem
\end{itemize}

There are also proposals for extending Java with traits. For
example, FeatherTrait Java (FTJ)~\cite{Liquori08ftj} extends
FJ~\cite{Igarashi01FJ} with statically-typed traits, adding trait-based
inheritance in Java.  Except for few, mostly syntactic details, their work can
be emulated with Java 8 interfaces. There are also extensions to the original
trait model, which have operations that default methods and interfaces cannot
model, such as method renaming~\cite{reppy2006foundation}, which breaks
structural subtyping.

\paragraph{Traits vs Object Interfaces}
We consider object interfaces to be an alternative to traits or mixins.  In the
trait model two concepts (traits and classes) coexist and cooperate. Some
authors~\cite{BettiniDSS13} see this as good language design fostering good
software development by helping programmers to think about the structure of
their programs.  However, other authors see the need of two concepts and the
absence of state as drawbacks of this model~\cite{malayeri2009cz}.  Object
interfaces are units of reuse, and at the same time they provide factory methods
for instantiation and support state.  Our approach promotes the use of
interfaces instead of classes, in order to rely on the modular composition
offered by interfaces. Since Java was designed for classes, a direct classless
programming style is verbose and unnatural. However, annotation-driven code
generation is enough to overcome this difficulty, and the resulting programming
style encourages modularity, composability and reusability, by keeping a strong
OO feel. In that sense, we promote object interfaces as being both units of
reusable code and object factories. Our practical experience is that, in Java,
separating the two notions leads to a lot of boilerplate code, and is quite
limiting when multiple inheritance with state is required.  Abstract state
operations avoid the key difficulties associated with multiple inheritance and
state, while still being quite expressive.  Moreover the ability to support
constructors adds expressivity, which is not available in approaches
such as Scala's traits/mixins.

%Nevertheless, we
%believe that in other languages the separation can be more practical
%than in Java, and we do not necessarally advocate that merging the two
%notions is a better approach in general.

%There are other limitations of our current approach, but they may be addressed
%in future work (see Section~\ref{sec:futurework}).


\begin{comment}
Besides, we support more features than the original trait model:
\begin{itemize}
\item We provide \texttt{of} methods for the annotated interfaces. During annotation processing time, the ``fields'' inside an interface are observed and a static method \texttt{of} is automatically injected to the interface with its arguments correspondingly. Such a method is a replacement to the constructors in original traits, making instantiation more convenient to use.
That is, in our approach there are only interfaces, our model requires a single concept, while the trait model requires traits \emph{and} classes.

\item We provide \texttt{with-} methods as auxiliary constructors. A \texttt{with-} method is generated for each field, just like record update, returning the new object with that field updated
%. A \texttt{clone} method is generated for the interface, returning a copy of the current object.
Furthermore, we do automatic return type refinement for this kind of methods. This feature is comparatively useful in big examples, making operations and behaviours more flexible.%, which we will demonstrate later.
\item We provide two options for generating setters. There are two kinds of setters which are commonly used, namely \textit{void setters} and \textit{fluent setters}. The only difference is that a fluent setter returns the object itself after setting, thus supporting a pipeline of such operations. The generation depends on which type of setter is declared in the interface by users.
\end{itemize}
\end{comment}



% \begin{enumerate}
% \item Mixins allows to name components that can be applied to various classes as
%   reusable functionality units. However, they suffer from linearisation: the
%   order of mixin application is relevant in often subtitle and undesired
%   ways. constraints. This hinders their usability and their ability of resolving
%   conflicts: the linearisation (total ordering) of mixin inheritance cannot
%   provide a satisfactory resolution in all cases and restricts the flexibility
%   of mixin composition.  To fight those limitations, an algebra of mixin
%   operators is introduced~\cite{ancona2002calculus}, but this raised the
%   complexity of the approach, especially when constructors and fields are
%   considered~\cite{marco09FJigsaw}.  Our approach does not have the
%   linearisation problem, in that the semantics of Java \textbf{extends} clause
%   is unordered and symmetric.
% \item Simplifying the mixin algebras approach, traits draw a strong line between
%   units of reuse (traits) and object factories (classes) In this model,
%   traits~\cite{scharli03traits} units of reusable code, containing only methods
%   as reusable functionalities. Thus, no state/state initialization is
%   considered.

%   Classes act as object factories, requiring functionalities from multiple
%   traits.  Traits offers a trait algebra with operations like sum, alias and
%   exclusion, provided for explicit conflict resolution.

%   Concluding, (pure) traits do not allow state and they do not offers any reuse
%   instrument to ensure that state is coherently initialized when finally defined
%   in classes.  Traits can't be instantiated. This model requires two concepts
%   (traits and classes) to coexist and cooperate.

%   Some authors see this as good language design fostering good software
%   development by helping programmers to think about the structure of their
%   programs.  However, other authors see the need of two concepts and the absence
%   of state as drawbacks of this model. Our approach takes interfaces as units of
%   reuse, and meanwhile generates factory method for instantiation.

% \item C++ and Scala also try to provide solutions to multiple inheritance, but
%   both suffer from object initialization problems. Virtual inheritance in C++
%   provides another solution to multiple inheritance (especially the diamond
%   problem by keeping only one copy of the base class)~\cite{ellis1990annotated},
%   however suffers from object initialization problem as pointed out by Malayeri
%   et al.~\cite{malayeri2009cz}. It bypasses all constructor calls to virtual
%   superclasses, which would potentially cause serious semantic error. Scala
%   solution (very similar to linearised mixins, but misleadingly called traits in
%   the language) avoids this problem by disallowing constructor parameters,
%   causing no ambiguity in cases such as diamond problem.  This approach has
%   limited expressiveness, and suffers from all the problems of linearised mixin
%   composition.
%   Python also offers multiple inheritance via linearised mixins. Indeed in python any class is implicitly a mixin, and mixin composition informally expressed as\\*
%   \Q@ class A use B,C {...new methods...}@\\*
%   can be expressed in python as \\*
%   \Q@ class Aux: ...new methods...@\\*
%   \Q@ class A(B,C,Aux): pass@
%   Our approach not only does not involve linearised mixin problem, but also
%   support proper constructor mechanism.
% \end{enumerate}

% %\section{Comparing to traits and mixins}\label{sec:comparison}
\subsection{Multiple Inheritance in ClassLess Java}
\begin{comment}
Haoyuan

   - vs both: we do automatic return type refinement, which has useful applications
   (example: Expression Problem)

   - vs traits: we support of methods to create new objects (a replacement to constructors);
   Moreover we have the with and clone methods (we miss more applications for those). Show
   how to model the operations on traits; discuss operations that we cannot model
   (example: renaming).

   - vs mixins: we use the trait model of explicitly resolving conflicts. This is arguably
   better for reasoning.
\end{comment}

Our approach is based on forbidding the use of certain language constructs (classes), in order to rely on the modular composition offered by the rest of the language (interfaces).
Since Java was designed for classes, a direct class-less programming style is verbose and feel unnatural. However, annotation driven code generation is enough to overcome this difficulty, and
the resulting programming style encourages modularity, composability and reusability, by keeping a strong object oriented feel.
We consider our approach as an alternative to traits or mixins.

A comparison on how to to emulate the original operations on traits using Java8 can be find in\cite{bono14}. We briefly recall the main points of their encoding; however we propose a different representation of
 \textbf{exclusion}.
The author of ~\cite{bono14} agree that our revised version is cleaner, typesafe and more direct.\bruno{is this something to 
say in the paper?}

%%\newcommand\shortItem{\vspace{-1ex}}
\begin{itemize}
\begin{comment}%I shoreten up a lot after. I think this version is better but longer, and this content is just repeating bono at all.
\item \textbf{Symmetric sum}: the symmetric composition of two disjoint traits is achieved by simply implementing two interfaces in Java correspondingly, without overriding any method. The composition relies on multiple inheritance on interfaces, which is supported by Java. Below is a simple example.
    \begin{lstlisting}
    interface A { int x(); }
    interface B { int y(); }
    interface C extends A, B {}
    \end{lstlisting}
\item \textbf{Override}: the overriding operation (also known as asymmetric/preferential sum) is modelled by implementing many interfaces, while overriding some methods inside. The code below gives an example of explicitly specifying which super interface take precedence, regarding conflict on a specific method.
    \begin{lstlisting}
    interface A { default int m() {return 1;} }
    interface B { default int m() {return 2;} }
    interface C extends A, B { default int m() {return B.super.m();} }
    \end{lstlisting}
    Here the method \texttt{m()} in interface \texttt{C} simply inherits from \texttt{B.m()}.
\item \textbf{Alias}: an alias operation adds a new name to an old method when creating the new trait. In Java, we just create a new method with reference to the existing method in its super interface. See the example below, where the new method \texttt{k()} is an alias of the existing method \texttt{m()}.
    \begin{lstlisting}
    interface A { default int m() {return 1;} }
    interface B extends A { default int k() {return A.super.m();} }
    \end{lstlisting}
\end{comment}

\item \textbf{Symmetric sum} can be obtained by simple multiple inheritance between interfaces.
\shortItem
    \begin{lstlisting}
    interface A { int x(); }    interface B { int y(); }    interface C extends A, B {}
    \end{lstlisting}
\shortItem
\item \textbf{Overriding} a conflict is obtained by specifying which super interface take precedence.
\shortItem
    \begin{lstlisting}
    interface A { default int m() {return 1;} } 
    interface B { default int m() {return 2;} }
    interface C extends A, B { default int m() {return B.super.m();} }
    \end{lstlisting}
\shortItem
\item \textbf{Alias} is creating  a new method delegating to the existing super interface.
\shortItem
    \begin{lstlisting}
    interface A { default int m() {return 1;} }
    interface B extends A { default int k() {return A.super.m();} }
    \end{lstlisting}
\shortItem

\item \textbf{Exclusion}: exclusion is also supported in Java, where method declarations can hide the default methods correspondingly in the super interfaces.
\shortItem
    \begin{lstlisting}
    interface A { default int m() {return 1;} }
    interface B extends A { int m(); }
    \end{lstlisting}
\shortItem
\end{itemize}

Besides, we support more features than the original trait model:
\begin{itemize}
\item We provide \texttt{of} methods for the annotated interfaces. During annotation processing time, the ``fields'' inside an interface are observed and a static method \texttt{of} is automatically injected to the interface with its arguments correspondingly. Such a method is a replacement to the constructors in original traits, making instantiation more convenient to use.
That is, in our approach there are only interfaces, our model requires a single concept, while the trait model requires traits \emph{and} classes.

\item We provide \texttt{with-} methods as auxiliary constructors. A \texttt{with-} method is generated for each field, just like record update, returning the new object with that field updated
%. A \texttt{clone} method is generated for the interface, returning a copy of the current object.
Furthermore, we do automatic return type refinement for these kind of methods. This feature is comparatively useful in big examples, making operations and behaviours more flexible.%, which we will demonstrate later.
\item We provide two options for generating setters. There are two kind of setters which are commonly used, namely \textit{void setters} and \textit{fluent setters}. The only difference is that a fluent setter returns the object itself after setting, thus supporting a pipeline of such operations. The generation depends on which type of setter is declared in the interface by users.
\end{itemize}

These are the additional features supported by our model, conversely, there are certain operations we cannot model, such as method renaming (as in [Reppy2006]), which breaks structural subtyping.

There are other limitations of our current approach, but they may be addressed
in future work (see Section~\ref{sec:futurework}).



\subsection{ThisType and MyType}
%\marco{ this section is unfair. There are unsolved problems with ThisType in negative positions, namelly it breaks subtyping. The most common solution is to allow calling some methods only when the exact type is known. This demolish most advantages of interfaces.}

In certain situations, object interfaces allow automatic refinement for \emph{return
types}. This is part of a bigger topic in class-based languages: expressing and
preserving type recursion and (nominal/structural) subtyping at the same time.

One famous attempt in this direction is provided by
\emph{MyType}~\cite{bruce1994paradigmatic}, representing the type of
\textbf{this}, changing its meaning along with inheritance. However when
invoking a method with MyType in a parameter position, the exact type of the
receiver must be known.  This is a big limitation in class based object oriented
programming, and is exasperated by the interface-based programming we propose: no
type is ever going to be exact since classes are not explicitly used. A recent
article~\cite{Saito2013933} lights up this topic, proposing two
new features: exact statements and nonheritable methods. Both are
related to our work: 1) any method generated inside of the \Q@of@ method is indeed
non-inheritable since there is no class name to extend from; 2) exact
statements (a form of wild-card capture on the exact run-time type) could
capture the ``exact type'' of an object even in a class-less
environment. 
Admittedly, MyType greatly enhances the expressiveness and extensibility of
object-oriented programming languages. Object interfaces use covariant-return types to
simulate some uses of MyType. But our approach only works for refining
return types, whereas MyType is more general, as it also works for
parameter types. Nevertheless, as illustrated with our examples and 
case studies, object interfaces are still very useful in many 
practical applications.
% The addition of MyType to a language will allow easy definition of
%binary methods, methods with recursive types (i.e., the same type of the
%receiver appears in the argument or return positions of methods), etc. 


\begin{comment}
\subsection{Type-Directed Translations/Syntactic Sugar}
\marco{I'm tring to merge this and the next one}
Language extensions are often implemented as syntactic sugar of the base
language. For example, Scala compiler supports XML syntax in normal Scala code
directly (after Scala ?, users need to import \texttt{scala-xml} library
manually). However, this approach is hard in terms of implementation, because it
requires extending the compiler. Also, this approach does not support combining
multiple extensions into one.

SugarJ~\cite{erdweg11sugarj} is a Java-based extensible programming language
that allows programmers to extend the base language with custom language
features by definitions in meta-DSLs (SDF, Stratego, etc). 
\yanlin{Is new syntax really a ``drawback''? I think for some system, like
  SugarJ, one of purpose IS to introduce these new syntax.}  Drawbacks: new
syntax. To create the extension, programmers have to work with multiple
languages (SDF, stratego, etc) while our approach works totally in Java
environment.

We can model certain types of language extensions with annotations 
only, but those extensions do not introduce new syntax: they 
merely do automatic code generation. 
\end{comment}

\subsection{Meta-Programming Competes with Language Extensions}
The most obvious solution to adding features to a language is via syntactic
extensions. Syntactic extensions are often implemented as 
desugarings to the base language. For example,
the Scala compiler was extended to directly support XML syntax.
However, when syntactic extensions are independently created it is hard
to combine multiple extensions into one.
SugarJ~\cite{erdweg11sugarj} is a Java-based extensible language that
aims at making the creation and composition of syntactic sugar
extensions easy, by allowing programmers to extend Java with custom
features (typically for DSLs). However SugarJ's goals are different 
from language tuning: SugarJ aims at creating and composing new syntax; 
whereas language tuning merely reinterprets existing syntax. It is 
clear that reinterpreting existing syntax only can be limiting 
for some applications. However, when language tuning is possible 
it has the advantage that existing tools for the language work 
out-of-the-box (since the syntax is still the same); and composition
of independently developed tunings is straightforward.

Scala-Virtualized~\cite{}\bruno{reference!} is an extension to Scala, 
which allows blending shallow and deep embedding of DSLs. It redefines 
some of Scala's language constructs to method calls, 
which can be overridden by DSL implementors. Thus Scala Virtualized  
can also reinterpret syntax, and be seen as a form of language tuning. 
However, although many Scala's language constructs are supported, 
not all language constructs can be virtualized.

%These extensions are admittedly useful, however they are essentially 
%creating a fork of the language, and rarely the new fork gains enough 
%traction to become the main language release. % On this
% topic we mention


When the base language has a flexible enough syntax and a fast and
powerful enough reflection mechanism, we may just need to play with
operator overloading and other language tricks to discover that the
language feature we need can be expressed as a simple library in our
language. An example of this is SQLAlchemy\cite{}\bruno{reference} in
Python, which uses operator overloading to dynamically turn normal python expressions into database queries
without requiring any syntactic extensions to Python.
Java-like languages tend to sit in the middle of two extremes:
libraries can not influence the type system, so many solutions valid
in Python or other dynamic languages are not applicable, or have the
cost of losing type-safety.

In Java-like languages compile time code generation comes at the rescue: if, for a certain feature
(\mixin in our case), it is possible to use the original language syntax to
\emph{express/describe} any specific instantiation of such feature (annotating a
class and providing getters), then we can insert in the compilation process a tool
that examines and enriches the code before compilation. No need to modify the
original source (for example we can work on temporary files). 
Java is a particular
good candidate for this kind of manipulation since it already provide ways to
define and integrate such tools in its own compilation process via annotation
processing. In this way
there is no need of temporary files, and there is a well-defined way of putting
multiple extensions together.

Other languages offer even stronger support for safe code manipulation:
Template Haskell~\cite{sheard2002template}, F\# (type providers)~\footnote{http://research.microsoft.com/fsharp/} and MetaFjig (Active Libraries)~\cite{servetto2010metafjig}
all allow to execute code at compile time. They generate code/classes that is transparently integrated in the program that is being generated/processed/compiled. In particular, MetaFjig offers a property called \emph{meta-level-soundness}. In short this property ensures by construction that library code (even if wrong or unreasonable) never generates ill-typed code. This is roughly equivalent to 
\textit{Self coherence}, that we have to manually prove.
Since MetaFjig is not working on annotated classes, there is no direct
equivalence on the overall theorem of the shown safety.\bruno{Don't understand 
the end of the sentence!}

\subsection{Formalization of Java8}
We provide a simple formalization for a subset of Java including default/static
interface methods and object initialization literals (often called anonymous
local inner classes).  A similar formalization was drafted by Goetz and
Field~\cite{goetz12fdefenders} to formalize defender (default) methods in
Java. However this formalization is limited to model exactly one method inside
classes/interfaces.\bruno{explain why having multiple methods is significantly
  more challenging.} Our code generation generates several methods that may
interacts with each other and methods defined by user. Thus implementing and
formalizing multiple methods inside interfaces adds complexity greatly.
\marco{I do not see this as a good explanation, I think we may just remove the `one method limitation discussion'. Is their work modelling interactions between abstract and default methods? like making a default method abstract again?}
% As an evidence of the attention and care present in our formalization work,
% while double-checking the behaviour of Java in side cases we have discovered a
% likely bug in the current \texttt{javac}.\marco{refer to before when we explain
%   the issue.}
