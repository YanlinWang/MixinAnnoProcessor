\section{Introduction}\label{sec:intro}

Java 8 introduced \emph{default methods}, allowing interfaces to have
method implementations. The main motivation behind the introduction of
default methods in Java 8 is \emph{interface evolution}. That is, to
allow interfaces to be extended over time, while preserving backwards
compatibility. It soon became clear that default methods could also be
used to emulate something similar to \emph{traits}~\cite{scharli03traits}. The
original notion of traits by Scharli et al. prescribes, among other
things, that: 1) a trait provides a set of methods that implement
behaviour; and 2) a trait does not specify any state variables, so the
methods provided by traits do not access state variables
directly. Java 8 interfaces follow similar principles too. Indeed, a
detailed description of how to emulate trait-oriented programming in
Java 8 can be found in the work by Bono et al.~\cite{bono14}. The Java 8
team designing default methods, was also fully aware of that secondary
use of interfaces, but it was not their objective to model traits:
``The key goal of adding default methods to Java was "interface
evolution", not "poor man's traits"''~\cite{goetz13default}. As a result
they were happy to support the secondary use of interfaces with
default methods as long as it did not make the implementation and
language more complex.
 
Still, the design is quite conservative and appears to be quite limited
 in its current form to model advanced forms of multiple inheritance.
Indeed, our own personal experience of combining default methods 
and multiple interface inheritance in Java to achieve multiple implementation 
inheritance is that many workarounds and boilerplate code are needed. 
In particular, we encountered difficulties because:

\begin{itemize}

\item {\em Interfaces have no constructors.} As a result classes are 
still required to create objects, leading to substantial boilerplate 
code for initialization.

\item {\em Interfaces do not have state.} This creates a tension between 
 using multiple inheritance and having state. Using setter and
  getter methods is a way out of this tension, but this workaround
  requires tedious boilerplate classes that latter implement those
  methods.

\item {\em Useful, general purpose methods, require special care in
  the presence of subtyping.} Methods such as
  \emph{fluent} setters~\cite{fowler2005fluentinterface}, not only require access to the
  internal state of an object, but they also require their return types to be
  refined in subtypes.

\end{itemize}

\noindent Clearly, a way around those difficulties would be to
change Java and just remove these limitations. Scala's own
notion of traits, for example, allows state in traits. Of course adding
state (and other features) to interfaces would complicate the language
and require changes in the compiler, and this would go beyond the
goals of Java 8 development team.

This paper takes a different approach. Rather than trying to get
around the difficulties by changing the language in fundamental ways,
we show that, with a simple language feature, default methods and
interface inheritance are in fact very expressive. Our proposed
language feature enables powerful object-oriented idioms, using
multiple inheritance. We call the language feature \emph{object
  interfaces}, because such interfaces can be instantiated directly,
without the need for an explicit class definition. Moreover, object
interfaces support \emph{abstract state operations}, providing a way
to use multiple inheritance with state in Java. The abstract state
operations include various common utility methods (such as getters and
setters, or clone-like methods). In the presence of subtyping, such
operations often require special care, as their types need to be
refined. Object interfaces provide support for type-refinement and can
automatically produce code that deals with type-refinement
adequately. %With object interfaces, many Java programs can be built
%without using a single class!

Object interfaces do not require changes to the Java runtime or compiler, 
and they also do not introduce any new syntax (apart from a
small annotation). All three features of object interfaces are
achieved by reinterpreting existing Java syntax, and are translated
into regular Java code without loss of type-safety. Since no new
syntax is introduced, it would be incorrect to call object interfaces
a language extension or syntactic sugar. So we use the term
\emph{language tuning} instead. Language tuning sits in between a
lightweight language extension and a glorified library. Language
tuning can offer many features usually implemented by a real language
extension, but because it does not modify the language syntax
pre-existing tools can work transparently on the tuned language.  To
exploit the full-benefits of language tuning, our prototype
implementation of object interfaces uses Java annotations and AST
rewriting, allowing existing Java tools (such as IDE's) to work
out-of-the-box with our implementation. As a result we could
experiment object interfaces with several interesting Java programs,
and conduct various case studies.

To formalize object interfaces, we propose Classless Java (CJ): a
FeatherweightJava-style~\cite{Igarashi01FJ} calculus, which captures the essence
of interfaces with default methods. The semantics of object interfaces
is given as a syntax-directed translation from CJ to itself. In the
resulting CJ code all object interfaces are translated into regular CJ
(and Java) interfaces with default methods. The translation is proved
to be type-safe, ensuring that the translation does not
introduce type-errors in client code. 
%CJ's usefulness goes beyond serving as a
%calculus to formalize object interfaces. During the development
%process of CJ, we encountered a bug in the implementation of default
%methods for the Eclipse Compiler for Java (ECJ). For the program revealing the 
%bug, ECJ behaves differently from both our formalization and Oracle's 
%Java compiler.

To evaluate the usefulness of object interfaces, we illustrate 3
applications. The first application is a simple 
solution to the Expression Problem~\cite{wadler98expression}, supporting independent 
extensibility~\cite{zenger05independentlyextensible}, and without boilerplate code. The second
application shows how embedded DSLs using fluent interfaces~\cite{fowler2005fluentinterface} 
can be easily defined using object interfaces. The last
application is a larger case study for a simple Maze game implemented with 
multiple inheritance. For the last application we show that there is a
significant reduction in the numbers of lines of code when compared 
to an existing implementation~\cite{bono14} using plain Java 8. 
Noteworthy, all applications are implemented 
without defining a single class!

In summary, the contributions of this paper are:
\begin{itemize}

\item {\bf Object Interfaces:} A simple feature that allows various
  powerful multiple-inheritance programming idioms to be expressed
  conveniently in Java.

\item {\bf ClassLess Java (CJ):} A simple formal calculus that models 
the essential features of Java 8 interfaces with default methods, and 
can be used to formally define the translation of object interfaces. 
We prove several properties of the translation\footnote{Proofs and prototype implementation are available in
  the supplementary materials.}.

\item {\bf Implementation and Case Studies:} We have a prototype
  implementation of object interfaces, using Java
  annotations and AST rewriting. Moreover the usefulness of object interfaces is
  illustrated through various examples and case studies.

\item{\bf Language Tuning:} We identify the concept of language tuning 
and we describe object interfaces as an example. We also discuss 
how other existing approaches, such as the annotations in project 
Lombok~\cite{lombok},  can be viewed as language tuning.

\end{itemize}
