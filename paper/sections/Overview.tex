\section{Overview}\label{sec:ep}
Yanlin

* Explain what the Mixin annotations do using examples.

* Motivate the use of multiple inheritance in Java.

* Maybe use Marco's example (Point example).

We provide a Java annotation \mixin to provide default implementations for
various methods and a mechanism to instantiate objects. \mixin annotation helps
programmers to write less cumbersome code and instantiate interfaces in Java.

For example, interface \texttt{Point} annotated with \mixin:

\lstinputlisting[linerange=6-10]{../UseMixinLombok/src/overview/TestPoint.java} % APPLY:linerange=POINT

\texttt{Point} has two (conceptually) member fields \texttt{X} and
\texttt{Y}. Methods \texttt{int X()} and \texttt{int Y()} serve as \emph{getter}
methods. Methods \texttt{void X(int X)} and \texttt{void Y(int Y)} serve as
\emph{setter} methods. Method \texttt{Point withX(int X)} updates field
\texttt{X} and returns \textbf{this}.

A typical and trivial implementation that programmers usually do is:

\lstinputlisting[linerange=82-112]{../UseMixinLombok/src/test/TestLombok.java} % APPLY:linerange=POINTIMPL

\texttt{PointImpl} implements \texttt{Point} and provides a constructor with
quite mechanical code. What's worse, the implementation in \texttt{PointImpl}
may not be reused in a single inheritance language. However, with our approach,
the \mixin annotation will generate a static method \texttt{of} inside
\texttt{Point}. \texttt{of} makes use of Java anonymous classes and achieves the
same implementation as \texttt{PointImpl}.

\lstinputlisting[linerange=15-27]{../UseMixinLombok/src/overview/TestPoint.java} % APPLY:linerange=POINT_OF

Inside the anonymous class in the annotated interface, the following code are
generated:
\begin{itemize}
\item For methods inside the interface with the form \texttt{Tx x()}:
  \begin{itemize}
   \item \texttt{x} is the getter method, with return type
     \texttt{Tx}. Conceptually, it is a member field with name \texttt{x} and
     type \texttt{Tx}.
   \item generate member field \texttt{\_x} of type \texttt{Tx}, initialized
     with \texttt{x}.
   \item generate implemented getter method:
       \begin{lstlisting}
           public Tx x() { return _x; }
       \end{lstlisting}
   \end{itemize}

\item For methods inside the interface with the form \texttt{void x(Tx x)}:
  \begin{itemize}
    \item check if exist method \texttt{Tx x()}. If not, generate error.
    \item generate implemented setter method:
       \begin{lstlisting}
           pubic void x(Tx x) { this.x = x; }
       \end{lstlisting}
    \end{itemize}

\item For methods with the form \texttt{T withX(Tx \_)}:
  \begin{itemize}
  \item if there is no \texttt{x} field, or type \texttt{Tx} does not match,
    then generate error.
  \item implement `withX` using the `of` method.
  \end{itemize}

\item For methods with the form of \texttt{T clone()}: Use \texttt{of} method as
  the constructor, to create a new object with the same field values as the
  current one.

\item For methods with the form of \texttt{T x(Tx \_)}:
  \begin{itemize}
    \item check if exist method \texttt{T x()}, if not, generate error.
    \item inside the inner class, generate
      \begin{lstlisting}
          pubic T x(Tx x) { this.x = x; return this;}
      \end{lstlisting}
  \end{itemize}
\end{itemize}