\begin{abstract}
  Java 8 introduced \emph{default methods}, allowing interfaces to have method
  implementations. When combined with (multiple) interface inheritance, default
  methods provide a basic form of multiple inheritance. However, using this
  combination to simulate multiple inheritance quickly becomes cumbersome, and
  appears to be quite restricted.

  This paper shows that, with a simple language feature, default methods and
  interface inheritance are in fact very expressive. Our proposed language
  feature, called \emph{object interfaces}, enables powerful object-oriented
  idioms, using multiple inheritance, to be expressed conveniently in
  Java. Object interfaces refine conventional Java interfaces in three different
  ways. Firstly, object interfaces have their own object instantiation
  mechanism, providing an alternative to class constructors. Secondly, object
  interfaces support \emph{abstract state operations}, providing a way to use
  multiple inheritance with state in Java. Finally, object interfaces allow type
  refinements that are often tricky to model in conventional class-based
  approaches. Interestingly, object interfaces do not require changes to the
  runtime, and they also do not introduce any new syntax: all three features are
  achieved by reinterpreting existing Java syntax, and are translated into
  regular Java code without loss of type-safety. Since no new syntax is
  introduced, it would be incorrect to call object interfaces a language
  extension or syntactic sugar. So we use the term \emph{language tuning} to
  characterize this kind of language feature. An implementation of object
  interfaces using Java annotations and a formalization of the static and
  dynamic semantics are presented. Moreover, the usefulness of object interfaces
  is illustrated through various examples.
\end{abstract}
