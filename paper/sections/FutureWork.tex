\section{Future work}\label{sec:futurework}

\subsection{Qualifiers in Methods} %{Private state
The biggest limitation of our approach is the inability to model qualifiers
for class methods (private, protected, synchronized, etc.). For example, the absence
of the support for private/protected methods in Java 8 interfaces forces all
members of interfaces to be public, including static methods. Since we use
abstract methods to encode state, our state is always all public. Still, because 
the state can only be accessed by methods, it 
is impossible for the user to know if a certain method maps directly to a field
or if it has a default implementation.  If the user wants a constructor that
does not directly maps to the fields, (as for secondary constructors in Scala)
he can simply define its own \Q@of@ method and delegate on the generated one:
\begin{lstlisting}
@Obj interface Point{  int x(); int y();
  static Point of(int val){return Point.of(val,val);}  }
\end{lstlisting}
However, the generated \Q@of@ method would also be present and public.  If a
future version of Java was to support \emph{static private methods in
  interfaces} we could extend our code generation to handle also encapsulation.
Currently, it is possible to use a public nested class with private static
methods inside, but this is ugly and cumbersome. One possibility is that the
annotation processor takes methods with a \Q!@Private! annotation, and turns it into
static private methods of a nested class. In this extension, also the \Q@of@
method could be made private following the same pattern.

\subsection{State initialization}
As discussed before, the user can trivially define its own \Q@of@ method, and
initialize a portion of the state with default values.  However, the
initialization code would not be reusable, and subinterfaces would have
to repeat such initialization code.  If a field has no setters, a simple
alternative is to just define the ``field'' as a default method as in
\begin{lstlisting}
@Obj interface Box{ default int val(){return 0;} }
\end{lstlisting}
If setters are required, a possible extension of our code expansion could
recognize a field if the getter is provided and the setter is required, and
could generate the following code:
\lstinputlisting[linerange=4-11]{../UseMixinLombok/src/test/TestStateInitialization.java}% APPLY:linerange=STATE_INIT

We are unsure about the value of this workaround because of its trickiness. In
order to define a state with initialization, users have to defining a method trusting that it will be overridden later with a behavioural change, but
such change is visible only after the first time the setter is called. Moreover, this code would cache
the result instead of re-computing it every time. This can be very tricky in a
non functional setting.
%\bruno{Yanlin please polish text/break long sentences.}

\begin{comment}
\subsection{Class Invariants in ClassLess Java}
Since objects are created by automatically generated methods, another limitation
of our current approach is that there is no place where the user can dynamically
check for class invariants. In Java often we see code like
\begin{lstlisting}
class Point{ int x; int y;
  Point(int x; int y){this.x=x;this.y=y; assert this.checkInvariant();}
  private boolean checkInvariant(){... x>0,y>0...}
}
\end{lstlisting} 

We are considering an extension of our annotation where 
default methods with the special name \Q@checkInvariant()@ will be called inside the \Q@of@ methods.
If multiple interfaces are implemented, and more then one offers
\Q@checkInvariant()@,  a composed implementation could be automatically generated, composing by \Q@&&@ the various competing implementations.
\end{comment}
%\bruno{removed the invariants stuff; we need space, I think.}

\subsection{Clone, toString, equals and hashCode}

Methods originally defined in Java class \Q@Object@, as \Q@clone@ and
\Q@toString@, can be supported by our approach with special care. If an
interface annotated with \mixin asks an implementation for \Q@clone@,
\Q@toString@, \Q@equals@ or \Q@hashCode@ we can easily generate one from the
fields.\footnote{In particular, for clone we can do automatic return type
  refinement as we do for \Q@with-@ and fluent setters. Note how this would
  solve most of the Java ugliness related to \Q@clone@ methods.}  However, if
the user wishes to provide his own implementation, since the method is also
implemented in \Q@Object@, a conflict arises. The generated code can resolve the
conflict inside \Q@of@, by implementing the method and delegating it to the user
implementation, thus

\begin{lstlisting}
@Obj interface Point{ int x(); int y();
  default Point clone(){ return Point.of(0,0);}}//user defined clone
\end{lstlisting} 
would expand into 

\begin{lstlisting}
interface Point{ ...
  public static Point of(int _x, int _y){
    return new Point(){...
      public Point clone(){ return Point.super.clone();}};}}
\end{lstlisting} 
