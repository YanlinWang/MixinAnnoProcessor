\section{Case studies}
Haoyuan and Yanlin

\subsection{A Trivial Solution to the Expression Problem}

The \emph{Expression Problem} (EP)~\cite{wadler98expression-problem} is coined
by Wadler about modular extensibility issues in software evolution and has been
a hot topic in programming languages since. Today we know of various solutions
to the EP that either rely on \emph{new programming language
  features}~\cite{chambers95multimethods,clifton00multiJava,madsen89virtual,nystrom06j,bruce98astatically,McDirmid01jiazzi,garrigue98polymorphic,zenger01extensible,loh06open,wehr11javaGI},
or can be used as \emph{design patterns}~\cite{gof} in existing
languages~\cite{torgersen04theexpression,oliveira06extensible,wouter08datatypes,oliveira09modular,oliveira12extensibility}. In
this paper we show that the EP can be solved with our \mixin annotation in a
trivial way. We will use a canonical example of implementing arithmetic
expressions gradually to illustrate the usage. 

\paragraph{\textbf{Initial System}}
The initial system expresses arithmetic expressions with only literals and
addition structure:

\lstinputlisting[linerange=25-35]{../UseMixinLombok/src/test/TestExpression.java}% APPLY:linerange=EXPRESSION_INIT

\texttt{Exp} is the common super-interface with an evaluation operation
\texttt{eval()} inside. Sub-interfaces \texttt{Lit} and \texttt{Add} extend
interface \texttt{Exp} with default implementations for \texttt{eval}
operation. The number field (\texttt{x}) of a literal is represented as a getter
method \texttt{x()} and expression fields (\texttt{e1,e2}) of an addition as
getter methods \texttt{e1()} and \texttt{e2()}.

\paragraph{\textbf{Adding a New Variant}} 
It is easy to add new data variants to the code in the initial system in a
modular way. For example, the following code how to add the subtraction variant.

\lstinputlisting[linerange=39-44]{../UseMixinLombok/src/test/TestExpression.java}% APPLY:linerange=EXPRESSION_SUB


\paragraph{\textbf{Adding a New Operation}}
Adding new operations to the previous system is still straightforward (although
requires longer code). The following code shows an example of adding a new
operation \texttt{print}. 

\lstinputlisting[linerange=48-57]{../UseMixinLombok/src/test/TestExpression.java}% APPLY:linerange=EXPRESSION_PRINT

The basic idea is to define interfaces for extending old interfaces.  The
interface \texttt{ExpP} extending interface \texttt{Exp} is defined with the
extra function \texttt{print()}. Interfaces \texttt{LitP} and \texttt{AddP} are
defined with default implementations of operation \texttt{print()}, extending
base interfaces \texttt{Lit} and \texttt{Add}, respectively. Importantly, note
that in \texttt{AddP}, the types of ``\emph{fields}'' \texttt{e1,e2} are refined via
the \emph{return types refinement} of getter methods \texttt{e1()} and
\texttt{e2()}. 

\subsection{Other case studies}
\marco{should we put here the DB with fluent setters?}
