\section{Interaction of interface methods with interface composition}
From Java8 interfaces can have three type of methods:
abstract methods, default methods, static methods.
\begin{itemize}
\item Static methods are handled in a very clean way: they are visible only on the interface they are explicitly declared. This means that 
the following code is ill-typed.
\begin{lstlisting}
interface A0 { static int m(){return 1;} }
interface B0 extends A0 {}
...
B0.m()//ill typed
\end{lstlisting}
Note how this is different w.r.t. the way static methods are handled in classes.
In this way static methods have simply no interaction with interface composition (\Q@extends@ or \Q@implements@).

\item Abstract methods composition is accepted when there is one that is the most specific.  For example in 
method \texttt{Integer m()} is visible in \Q@C1@.

\begin{lstlisting}
interface A1 { Object m(); }
interface B1 { Integer m();}
interface C1 extends A1,B1 {} //accepted
\end{lstlisting}

\item Default methods conflict with any other default or abstract method. For
example the following code is rejected due to conflicting methods.
\begin{lstlisting}
interface A2 { default int m() {return 1;}}
interface B2 { int m(); }
interface C2 { default int m() {return 2;}}
interface D2 extends A2,B2 {} //rejected due to conflicting methods
interface E2 extends A2,C2 {} //rejected due to conflicting methods
\end{lstlisting}
Note how this is in contrast with what happens in most trait models, where \Q@D2@ would be accepted, and the implementation in \Q@A2@ would be part of the behaviour of \Q@D2@.

\item The method in the current interface wins over any methods defined in its
super-interfaces, provided that the method conform to the subtype of all methods
in its super-interfaces, i.e., the method is the most specific one.
This also override rejection due to conflicting methods. For example, the following code is accepted, but would be rejected (see before) if the method \Q@m@ was not redefined.
\begin{lstlisting}
interface D3 extends A2,B2 { int m(); } //accepted
interface E3 extends A2,C2 { default int m(){return 42;} } //accepted
\end{lstlisting}
\end{itemize}

While trying to formally encode the Java specification we have done some tests to clarify corner case behaviour.
Consider the following correct declarations:
\begin{lstlisting}
interface A1{T m(); }
interface A2 extends A1{default T m(){ ... } }
interface A3 extends A2{T m(); } 

interface B1{default T m(){ ... } }
interface B2 extends B1{T m(); }
interface B3 extends B2{default T m(){ ... } }
\end{lstlisting}

What happens if we define a new interface \Q@M@ extending one \Q@A@${}_i$ and one \Q@B@${}_i$? we have 9 cases, that can fit nicely a table:\\*
\noindent
\begin{tabular}{|l|l|l|l|}
\hline
\textbf{M extends} & \textbf{A1}                  & \textbf{A2} & \textbf{A3} \\ \hline
\textbf{B1}        & conservative error                        & conflict error      & conservative error       \\ \hline
\textbf{B2}        & both abstract, accepted                        & conservative error       & both abstract, accepted       \\ \hline
\textbf{B3}        & \textbf{conservative error} &conflict  error       & conservative error      \\ \hline
\end{tabular}
\\*
We try to classify the results out of the table:
\begin{itemize}
\item \textbf{conflict error} happens when the method from \Q@A@${}_i$ and one \Q@B@${}_i$ are both implemented. This is also considered an error in most trait models.
\item \textbf{both abstract, accepted} happens when the method from \Q@A@${}_i$ and one \Q@B@${}_i$ are both abstract. This is also considered correct in all trait models.
\item \textbf{conservative error}
 happens when the method from \Q@A@${}_i$ and one \Q@B@${}_i$ is implemented in only one side.
This is different from what we would expect in a trait model, but is coherent with the conservative idea that a method defined in an interface
should not silently satisfy a method in another one.
\end{itemize}

During our experimentation, we found a bug in ECJ (eclipse version of javac): The case \textbf{B3},\textbf{A1}
is accepted by ECJ4.5.1 and rejected by javac.
By email communication with Brian Goetz (lead Java8 designer) we have confirmation that the expected behaviour is rejection, and that this is a bug in ECJ.
%Table~\ref{table:javabug} shows the method overriding bug in Java. In the
%example, there are 6 interfaces \texttt{A1,A2,A3,B1,B2,B3} with methods all
%named as \texttt{m} which are either abstract or default methods. If we define a
%new interface \texttt{M} that extends two of these interfaces, then the method
%overriding result is shown in the table. For example, row2 col2 means \texttt{M
%  extends A1,B1}. The result is \texttt{ERROR} because the abstract method
%\texttt{T m();} in \texttt{A1} conflicts with the method \texttt{default T m()
%  {...}} in \texttt{B1}. Readers may also figure out other extending cases by
%following this way of interpretation.

% The interesting case in row4 col2 where
%Indeed, to be coherent with the idea of \textbf{conservative error}, the case should not be accepted.
%We do not see how this should behave differently from \textbf{B1},\textbf{A1}, and \textbf{B3},\textbf{A3}.
%We fear that the only retro-compatible fix for this strange behaviour is to accept all the cases of \textbf{conservative error} in a future version of Java.
%the experimental result is \texttt{B3.m}. Because if we think consistently, this

%\lstinputlisting[linerange=4-9]{../UseMixinLombok/src/methodshadowing/test5/Test5.java} % APPLY:linerange=JAVABUG

%In our approach, we choose to not model this strange behaviour (a bug?).
%Our auxiliary function $\mBody(\m,\C)$ enforce the \textbf{conservative error} strategy.
%The rest of our formalization is parametric with the definition of $\mBody(\m,\C)$, thus if Java change its resolution strategy to a more permissive one, only minor adaptations in $\mBody(\m,\C)$ would be needed.

\section{Formal Semantics}\label{sec:formal}

This section presents a formalization of ClassLess Java: a minimal
FeatherweightJava-like calculus which models the essence of Java 
interfaces with default methods. 

\begin{figure}[t]
\begin{grammar}
\production{
\e
}{
  \x\mid\MCall\e\m\es\mid\MCall{\C}\m\es\mid\MCall{\C\QM{.super}}\m\es\mid\x\QM=\e\QM;\e'\mid\obj
  }{expressions}\\
\production{
\obj
}{
\QM{new}\ \C\oR\cR\oC\fields\
\mh_1\oC\QM{return}\ \e_1\QM{;}\!\cC
\ldots
\mh_n\oC\QM{return}\ \e_n\QM{;}\!\cC
\cC
  }{object creation}\\
\production{\field}{\T\ \f \QM= \x\QM;}{field declaration}\\
\production{
\II
}{
 \ann\ \QM{interface}\ \C\ \QM{extends}\ \Cs\ \oC \methods\ \cC
  }{interface declaration}\\
\production{
\method
}{
 \QM{static}\ \mh\ \oC\QM{return}\ \e\QM{;}\!\cC
\mid
\QM{default}\ \mh\ \oC\QM{return}\ \e\QM{;}\cC
\mid
\mh\QM{;}
  }{method declaration}\\
\production{
\mh
}{
 \T_0\ \m\ \oR\T_1\ \x_1\ldots\T_n\ \x_n\cR
  }{method header}\\
\production{
\ann
}{
  \mixinAnn|\emptyset
  }{annotations}\\
\production{\Gamma}{
\x_1{:}\C_1\ldots\x_n{:}\C_n
}{environment}
\end{grammar}
\caption{Grammar of ClassLess Java}
\label{Grammar}
\end{figure}

\subsection{Syntax}

Figure~\ref{Grammar} shows the syntax of ClassLess Java.%
  The syntax formalizes a minimal
version of Java 8, focusing on interfaces, default methods and object
creation literals.  There is no syntax for classes.
To help readability we use many metavariables to represent identifiers: $C,x,f$ and $m$; however they all maps to a single set of identifiers as in Java.  Expressions
consist of conventional constructs such as variables ($\x$), method
calls ($\e\QM.\m\QM(\es)$) and static method calls
($\C\QM.\m\QM(\es)$). For simplicity the degenerate case of calling a
static method over the $\this$ receiver is not considered.  A more
interesting type of expressions are super calls
($\C\QM{.super.}\m\QM(\es)$), whose semantic is to call the (non
static) method $\m$ over the $\this$ receiver, but statically
dispatching to the version of the method as visible in the interface
$\C$. A simple form of field updates ($\x\QM=\e\QM;\e'$) is also
modelled. In the syntax of field updates $\x$ is expected to be a
field name. After updating the field $\x$ using the value of $\e$, the
expression $\e'$ is executed. To blend the statement based nature of
Java and the expression based nature of our language, we consider a
method body of the form \Q@return@ $\x\QM=\e\QM;\e'$ to represent
$\x\QM=\e\QM;\QM{return}\ \e'$ in Java.  Finally, there is an object
initialization expression from an interface $\C$, where (for
simplicity) all the fields are initialized with a variable present in
scope. Note how our language is a subset of Java 8.
To  be compatible with java, the concrete syntax for an interface
  declaration with empty supertype list  would also
  omit the \Q@extends@ keyword.
% The single
%non-Java 8 piece of syntax is the \mixin annotation, which is the only
%one interesting piece of syntax in this article.
%??? how is @Mixin not java8? in the sense that is java5?
  Following standard
practise, we consider a global Interface Table (\metaVar{IT}) mapping
from interface names $\C$ to interface declarations $\II$.

The environment $\Gamma$ is a mapping from variables to types.  As
usual, we allow a functional notation for $\Gamma$ to do variable
lookup.  Moreover, to help us defining auxiliary functions, a functional
notation is also allowed for a set of methods $\methods$, using the
method name $\m$ as a key.  That is, we define $\methods(\m)=\method$
iff there is a unique $\method\in\methods$ whose name is $\m$.  For
convenience, we define $\methods(\m)=\none$ otherwise; moreover
$\m\in\dom(\methods)\ \miff\ \methods(\m)=\method$.

\subsection{Typing}

Typing statement $\Gamma \vdash \e\in\C$ reads ``in the environment
$\Gamma$, expression $\e$ has type $\C$.''. 
Before discussing the typing rules we discuss some of the used notation.
As a shortcut, we write
$\Gamma \vdash \e\in\C<:\C'$ instead of $\Gamma \vdash \e\in\C$ and
$\C<:\C'$.  

We omit the definition of
the usual traditional subtyping relation between interfaces, that is the transitive and reflexive closure of the declared \Q@extends@ relation, that can be read off the interface table and given by the
\textbf{extends} clauses in the interfaces. %
\footnote{\marco{find a better place for} Notice how there are no classes, thus there is no
  subclassing.  We believe that this approach may scratch an old
  itching point in the long struggle of subtyping versus subclassing:
  According to some authors, from a software engineering perspective,
  interfaces are just a kind of classes. Others consider more
  opportune to consider interfaces are pure types. In this vision our
  language would have no subclassing. We do not know how to conciliate
  those two viewpoints and ClassLess Java design. We do not have
  Classes purely in the Java sense.}
The auxiliary notation $\Gamma^\mh$ trivially
extracts the environment from a method header, by collecting the all types
and names of the method parameters.  The
notation $\m^\mh$ and $\C^\mh$ denotes, respectivelly, extracting the
method name and the return type from a method header. $\mBody(\m,\C)$,
defined in Section~\ref{},
returns the full method declaration as seen by $\C$, that is the
method $\m$ can be declared in $\C$ or inherited from another
interface. 
$\textsf{mtype}(\m,\C)$ and $\textsf{mtypeS}(\m,\C)$ return the type
signature from a method (using $\mBody(\m,\C)$ internally).
$\textsf{mtype}(\m,\C)$ is defined only for non static methods, while
$\textsf{mtypeS}(\m,\C)$ only on static ones. We use $\dom(\C)$ to
denote the set of methods that are defined for type $\C$, that is:
$\m\in\dom(\C)\ \miff \ \mBody(\m,\C)=\method$.

\begin{figure}[t]
$
\begin{array}{l}

%% T-Invk
\inferrule[(T-Invk)]{
 \Gamma \vdash \e \in \C_0 \\\\
\forall i\in 1..n\ \ \Gamma \vdash \e_i \in \_<:\C_i \\\\
  \textsf{mtype}(\m,\C_0) \!=\! \C_1\ldots\C_n \!\!\to\! \C
%\textsf{mmodifier}(\m,\C) \neq \textbf{static}
 }{
 \Gamma \vdash \e\QM.\m\QM(\e_1\ldots\e_n\QM) \in \C }
\quad\quad

%%T-StaticInvk
\inferrule[(T-StaticInvk)]{
\forall i\in 1..n\  \ \Gamma \vdash \e_i\in \_<:\C_i \\\\
\textsf{mtypeS}(\m,\C_0) \!=\! \C_1\ldots\C_n \!\to\! \C
%\textsf{mmodifier}(\m,\C) = \textbf{static}
}{
\Gamma \vdash \C_0\QM.\m\QM(\e_1\ldots\e_n\QM) \in \C}
\quad\quad

%%T-SuperInvk
\inferrule[(T-SuperInvk)]{
\Gamma(\this) <: \C_0 \\\\
\forall i\in 1..n\ \ \Gamma \vdash \e_i\in \_<:\C_i \\\\
  \textsf{mtype}(\m,\C_0) \!=\! \C_1\ldots\C_n \!\!\to\! \C
%\textsf{mmodifier}(\m,\C_0) \neq \textbf{static} \\\\
}{\Gamma \vdash \C_0\QM.\QM{super}\QM.\m\QM(\e_1\ldots\e_n\QM) \in \C}


%%T-Var
\\[5ex]
\inferrule[(T-Var)]{
\Gamma(\x)=\C
}{
\Gamma \vdash \x \in\C}
\quad\quad

%%T-Obj
\inferrule[(T-Obj)]{
\forall i\in 1..k\ \ \Gamma(\x_i)\subtype\T_i\\\\
\forall i\in 1..n\ \ 
%\Gamma_i
\Gamma,\f_1{:}\T_1,\ldots,\f_k{:}\T_k,\,\QM{this}{:}\C,\Gamma^{\mh_i}
\vdash\e_i\in \_\subtype\C^{\mh_i}\\\\
\sigvalid(\mh_1\ldots\mh_n,I)\quad\quad\quad\quad
%\forall i\in 1..n\ \mh_i\subtype\mBody(\m^{\mh_i},\C)\\\\
\alldefined(\mh_1\ldots\mh_n,I)
%\forall\m\mbox{ such that }
%\mBody(\m,\C)=\mh\QM; \exists i\in 1..n\ \m^{\mh_i}=\m
%\forall i\in 1\ldots n\ \Gamma_i=\Gamma,\f_1{:}\T_1,\ldots,\f_k{:}\T_k,\,\QM{this}{:}\C,\Gamma^{\mh_i}
}{
\Gamma \vdash\QM{new}\ \C\oR\cR\oC\T_1\ \f_1\QM=\x_1\QM;\ldots\T_k\ \f_k\QM=\x_k\QM;\
\mh_1\oC\QM{return}\ \e_1\QM{;}\!\cC
\ldots
\mh_n\oC\QM{return}\ \e_n\QM{;}\!\cC
\cC
\in\C
}
\\[5ex]



%%T-update
\quad
\inferrule[(T-update)]{
\Gamma \vdash \e\in\_<:\Gamma(\x)\\\\
\Gamma \vdash \e'\in\C
}{
\Gamma \vdash \x\QM=\e\QM;\e'\in\C }
\quad\quad\quad

%%T-Intf
 \inferrule[(T-Intf)]{
IT(\C) = \ann\ \QM{interface}\ \C\ \QM{extends}\ \C_1\ldots\C_n \
\oC\methods\ \cC\\\\
 \forall \QM{default}\ \mh\ \oC\QM{return}\ \e\QM;\cC \in \methods,
\ \ \Gamma^{\mh},\,\QM{this}{:}\C\vdash\e\in \_\subtype\C^{\mh} \\\\
 \forall \QM{static}\ \mh\ \oC\QM{return}\ \e\QM;\cC \in \methods,
\ \ \Gamma^{\mh}\vdash\e\in \_\subtype\C^{\mh} \\\\
\dom(\C)=\dom(\C_1)\cup\ldots\cup\dom(\C_n)\cup\dom(\methods)
 }{
\C \text{ OK}
}
\end{array}$
\caption{CJ Typing}
\label{ET}
\end{figure}

In Figure~\ref{ET} we show the typing rules.  We discuss first the
most interesting rules, that is \rn{t-Obj} and \rn{t-Intf}. Rule
\rn{t-Obj} is the most complex typing rule. Firstly, we need to
ensure that all field initializations are type correct, by looking up the type of
each variable assigned to a field in the typing environment and verifying that such type is a
subtype of the field type. Secondly, we check that all method bodies are
well-typed. To do this the enviroment used to check the method body
needs to be extended appropriately: we add all fields and their types;
add $\this:I$; and add the arguments (and types) of the respective
method.
Now we need to check if the object is a valid extension for that specific interface.
This can be logically diveded in two steps:
First  we check that all method headers are valid
with respect to the corresponding method already present in $\C$;

\noindent$\begin{array}{l}
\InTextDef{10em}{\sigvalid(\mh_1\ldots\mh_n,I)
}{
\forall i\in 1..n\ \ \mh_i\QM;\subtype\mBody(\m^{\mh_i},\C)
}
\end{array}$

Here we require that for all newly declared methods, there is a
method with the same name defined in the interface $\C$, and that such method is a subtype of the newly introduced one. We define subtyping between methods in a general form that will be useful also later.

\noindent$\begin{array}{l}
\InTextDef{18em}{
\T\ \m\oR\T_1\x_1\ldots \T_n\x_n\cR \subtype \T' \m\oR\T_1\x_1'\ldots\T_n\x_n'\cR
}{
\T\subtype \T'
}\\
\InTextDef{18em}{\method \subtype
\QM{default}\ \mh\,\mbox{\Q@\{return \_;\}@}
}{
\method\subtype\mh\QM;
}\\
\InTextDef{18em}{\QM{default}\ \mh\,\mbox{\Q@\{return \_;\}@}\subtype\method
}{
\mh\QM;\subtype\method
}\\
\end{array}$

Two method headers are subtypes if all the parameter types are the same and the return types are subtypes.
That is, we allows return type specialization as introduced in Java5.
Default methods are subtypes if their method headers are subtypes.


Finally, all abstract methods in the interface (that is methods that need
to be explicitly overridden) have been
implemented. That is, we define a method with the same name.

\noindent$\begin{array}{l}
\InTextDef{11em}{\alldefined(\mh_1\ldots\mh_n,I)}{
\forall\m\mbox{ such that } \mBody(\m,\C)=\mh\QM; \exists i\in 1..n\ \m^{\mh_i}=\m}
\end{array}$



The rule \rn{(t-intf)} checks that an interface $\C$ is correctly
typed.  First we check that the body of all the default and static
methods are well typed.  Then we check that $\dom(\C)$ is the same of
$\dom(\C_1)\cup\ldots\cup\dom(\C_n)\cup\dom(\methods)$.  This is not a
trivial check, since $\dom(\C)$ is defined using $\mBody$, that is
undefined in many cases: notably if a method $\method\in\methods$ is
not compatible with some method in $\dom(\C_1)\ldots\dom(\C_n)$ or if
any method in both $\dom(\C_i)$ and $\dom(\C_j)$ ($i,j\in 1..n$) is
conflicting.

\subsection{Auxiliary Definitions}

\begin{comment}
\subsubsection{Auxiliary function: \textsf{mtype}}
- \textsf{mtype(m, C)} : the signature of method m in C.

\[ \inferrule{
  IT(T) = \text{\emph{ann} interface } C \{ \overline{M} \} \\
  E \spc m(\overline{D} \spc \overline{x}) \{ \text{return } e; \} \in M}
{ \textsf{mtype(m,T)} = \overline{D} \to E } \]

\[ \inferrule{
  IT(T) = \text{\emph{ann} interface } C \{ \overline{M} \} \\
  m \notin M}
{ \textsf{mtype(m,T)} = \emptyset } \]

\[ \inferrule{
  IT(T) = \text{\emph{ann} interface } C \text{ extends } C_1,...,C_k \{ \overline{M} \} \\
  E \spc m(\overline{D} \spc \overline{x}) \{ \text{return } e; \} \in M}
{ \textsf{mtype(m,T)} = \overline{D} \to E } \]

\[ \inferrule{
  IT(T) = \text{\emph{ann} interface } C_0 \text{ extends } \overline{C} \{
  \overline{M} \} \\
  m \notin M}
{ \textsf{mtype(m,T)} = \bigcup \textsf{mtype}(m,\overline{D}) } \]
\end{comment}


Defining \mBody{} is not trivial, and requires quite a lot of attention to the specific model of Java Interfaces, and on how it differs w.r.t. Java Class model.
$\mBody(\m,\C)$ denotes the actual method $\m$ (body included) that interface $\C$ owns. It can either be defined originally in $\C$ or in its supertypes, and then passed to $\C$ via inheritance.

We use internally a special modifier $\conflicted$ to denote the case of two methods with conflicting implementation.\\*
\noindent$\begin{array}{l}
\InTextDef{15ex}{
\mBody(\m,\C_0)
}{
\override(\methods(\m),
\shadow(\m,\tops(\m,\Cs)))
}\\
\InTextWith{\metaVar{IT}(\C_0) =
\ \ann\ \QM{interface}\ \C_0\ \QM{extends}\ \Cs\ \oC\methods\cC}
\end{array}$

As you can see, we are delegating the work to three others auxiliary functions:\\*
 $\tops(\Cs), \shadow(\m,\Cs)$ and $\override(\method,\method')$
${}_{}$\\*

\tops{} recovers from the interface table only the ``needed'' methods, that is,
the non static ones that are not transitively reachable by following another, less specific, superinterface chain. Formally:\\*

\noindent$\begin{array}{l}
\InTextAssert{35ex}{
\method\in\tops(\m,\Cs)}{
}\\
\InTextWith{
  \mBody(\m,\C)=\method,\ \method\mbox{ not a static method, and }}\\ 
\tab\tab  \C\in\Cs\ \mbox{ such that }
\forall \C'\in\Cs\setminus\C \mbox{ not } \C' \subtype \C

\end{array}$

\shadow{} choose the most specific version of a method, that is the unique version available, or a conflicted version from a set of possibilities.
We do not model overloading, so it is an error if multiple versions are available with different parameter types. Formally:\\*
\noindent
$\begin{array}{l}
\InTextDef{15ex}{
\shadow()}{\none}\\
\InTextDef{15ex}{\shadow(\method)}{\method}\\
\InTextDef{15ex}{\shadow(\overline{\mh\QM;})}{\Aux{mostSpecific}(\overline{\mh\QM;})}\\
\InTextDef{15ex}{\shadow(\methods)}{\conflicted\mh\QM;}\\
\InTextWith{\methods\mbox{ not of the form }\overline{\mh\QM;}\mbox{ and }
\Aux{mostSpecific}(\methods)\in\{\mh\QM;,\QM{default}\ \mh\mbox{\Q@\{return \_;\}@}\}}\\
\end{array}$

Where $\Aux{mostSpecific}$ return the most specific method , that is a method whose type is the subtype of all the others.
Since method subtyping is a partial ordering, this may be not defined, this in turn makes \shadow{}, and the whole \mBody{} not defined for that specific $\m$. Rule  \rn{t-intf}  relies on this behaviour.

\noindent$\begin{array}{l}
\InTextDef{18ex}{\Aux{mostSpecific}(\methods)}{\method}\\
\InTextWith{\method \in \methods\ \mand\ \forall \method' \in \methods :  \method \subtype \method'}

\end{array}$

${}_{}$\\*
The override function models how the implementation in an interface can override implementation in the superinterface; even in case of a conflict.
Note how we use the special value $\none$, and how (forth case) overriding can solve a conflict.

\noindent$\begin{array}{l}
\InTextDef{20ex}{\override(\none,\none)}{\none}\\
\InTextDef{20ex}{\override(\method,\none)}{\method}\\

\InTextDef{20ex}{\override(\none,\method)}{\method}\\
\InTextWith{\method\mbox{ not of the form } \conflicted\ \mh\QM;}\\
\InTextDef{20ex}{\override(\method,\method')}{\method}\\
\InTextWith{\method'\in\{\mh\QM;,\QM{default}\ \mh\,\mbox{\Q@\{return \_;\}@}, \conflicted\ \mh\QM; \},
\method\subtype\method'}
\end{array}$


%The \textsf{shadow} function takes two same methods (with the same name and types of arguments), and return the method which shadows the other during inheritance.
%exmples to motivate our design
%interface A{static String m(){return "A";}}
%interface C extends A{
%	default String dm(){
%		  this.m();//wrong in java
%		  A.m();
%		  C.m();//wrong in java
%		}
%}
%
%
%(1) Static methods are not inherited. Also, if one of $\{body_1,body_2\}$ is null, \textsf{shadow} simply returns the other one. Hence

\begin{comment}
%Now it is correct, but may be we do not need it?
We abbreviate typing statements on
sequences in a simple way, writing $\Gamma \vdash \overline{t}:\overline{C}$ as
shorthand for $\Gamma \vdash t_1:C_1,..., \Gamma \vdash t_n:C_n$.
\end{comment}



\begin{comment}
\subsubsection{Method Typing}


\[
\inferrule
{ }
{T_0 \spc m(\overline{T} \spc \overline{x}); \text{ OK IN I} }
\quad \textsc{(T-Meth)}
\]

\[
\inferrule
{\overline{x}:\overline{T} \vdash e:S \\ S <: T_0}
{T_0 \spc m(\overline{T} \spc \overline{x}) \text{ \{ return } e;\} \text{ OK IN
    I} \\\\ \Gamma \vdash \textbf{this}:I }
\quad \textsc{(T-MethBody)}
\]

\[
\inferrule
{IT(I)=\text{interface } I \text{ extends } \overline{J} \text{\{...\}} \\
\forall i,\text{if \textsf{mtype}}(m,J_i) = \overline(T) \to U_0, \text{then }
T_0 <: U_0 }
{T_0 \spc m(\overline{T} \spc \overline{x}); \text{ OK IN I} }
\quad \textsc{(T-MethExt)}
\]

\[
\inferrule
{\textbf{this}:I, \overline{x}:\overline{T} \vdash e:S \\ S <: T_0 \\\\
IT(I)=\text{interface } I \text{ extends } \overline{J} \text{\{...\}} \\\\
\forall i,\text{if \textsf{mtype}}(m,J_i) = \overline(T) \to U_0, \text{then }
T_0 <: U_0 }
{T_0 \spc m(\overline{T} \spc \overline{x}) \text{ \{ return } e;\} \text{ OK IN
    I} }
\quad \textsc{(T-MethBodyExt)}
\]

\[ \inferrule
{\textbf{interface } I \textbf{ extends } \overline{J} \{ \overline{M} \} \\\\
 \forall J_i \in \overline{J}, J_i \text{ OK} \\\\
 \forall m \in \overline{M}, \textsf{mbody}(m,I) \neq \error \\\\
 \forall J_i \in \overline{J}, \forall m \text{ inside } J_i,
 \textsf{mbody}(m,I) \neq \error }
{I \text{ OK}}
\quad \textsc{(T-Intf)}
 \]

Interface $I$ type checks well, if:
\begin{itemize}
\item All its super-interfaces $\overline{J}$ OK.
\item All methods inside interface $I$ are OK.
\item All methods that $I$ is inheriting from super-interfaces are OK.
\end{itemize}
\subsubsection{Subtyping}

\[ \inferrule{}{T <: T} \]

\[ \inferrule{S <: T \\ T <: U}{S <: U}\]

\[ \inferrule{\emph{ann} \spc \textbf{interface} \spc C_0 \spc \textbf{extends} \spc C_1,...,C_k \{...\}}
{C_0 <: C_1 \\ ... \\ C_0 <: C_k} \]
\end{comment}
%\subsubsection{Interface Table}

