%\documentclass[preprint]{sigplanconf}

\documentclass[preprint]{llncs}


\usepackage{color}

\usepackage{amsmath}
\usepackage{stmaryrd}
\usepackage{graphicx}
\usepackage{amssymb}
\usepackage{fancyvrb}
\usepackage{url}
\usepackage{pstricks,pst-node,pst-tree}
\usepackage{theorem}
%% \usepackage{mathpartir}
\usepackage{bbm}
\usepackage{pgf}
\usepackage{multirow}

\usepackage{listings}
\usepackage{verbatim}
\usepackage{graphicx}
\usepackage{wrapfig}

\usepackage[T1]{fontenc}
\usepackage[scaled=0.85]{beramono}

% "define" code highlights for Java and Scala
\lstdefinelanguage{JavaScala}{
  morekeywords={public,int,interface,implements,default,
    abstract,case,catch,class,def,%
    do,else,extends,false,final,finally,%
    for,if,implicit,import,match,mixin,%
    new,null,object,override,package,%
    private,protected,requires,return,sealed,%
    super,this,throw,trait,true,try,%
    type,val,var,while,with,yield},
  otherkeywords={=>,<-,<\%,<:,>:,\#,@},
  sensitive=true,
  morecomment=[l]{//},
  morecomment=[n]{/*}{*/},
  morestring=[b]",
  morestring=[b]',
  morestring=[b]"""
}

\lstset{ %
language=JavaScala,                % choose the language of the code
columns=flexible,
lineskip=-1pt,
basicstyle=\ttfamily\small,       % the size of the fonts that are used for the code
numbers=none,                   % where to put the line-numbers
numberstyle=\ttfamily\tiny,      % the size of the fonts that are used for the line-numbers
stepnumber=1,                   % the step between two line-numbers. If it's 1 each line will be numbered
numbersep=5pt,                  % how far the line-numbers are from the code
backgroundcolor=\color{white},  % choose the background color. You must add \usepackage{color}
showspaces=false,               % show spaces adding particular underscores
showstringspaces=false,         % underline spaces within strings
showtabs=false,                 % show tabs within strings adding particular underscores
morekeywords={var},
%  frame=single,                   % adds a frame around the code
tabsize=2,                  % sets default tabsize to 2 spaces
captionpos=none,                   % sets the caption-position to bottom
breaklines=true,                % sets automatic line breaking
breakatwhitespace=false,        % sets if automatic breaks should only happen at whitespace
title=\lstname,                 % show the filename of files included with \lstinputlisting; also try caption instead of title
escapeinside={(*}{*)},          % if you want to add a comment within your code
keywordstyle=\ttfamily\bfseries,
% commentstyle=\color{Gray},
% stringstyle=\color{Green}
}


%%\usepackage{natbib}
%%\bibpunct();A{},
%%\let\cite=\citep

%include lhs2TeX.fmt
%include lhs2TeX.sty
%include forall.fmt

\pagestyle{plain}

%{\theorembodyfont{\sffamily} \newtheorem{theorem}{Theorem}}
%{\theorembodyfont{\sffamily} \newtheorem{lemma}{Lemma}}
%\newtheorem{theorem}{Theorem}
%\newtheorem{lemma}{Lemma}
%\newenvironment{proof}{\textbf{Proof:\hspace{4mm}}}{$\Box$}
\newcommand{\authornote}[3]{{\color{#2} {\sc #1}: #3}}
\newcommand\william[1]{\authornote{william}{green}{#1}}
%%\newcommand\tom[1]{\authornote{tom}{red}{#1}}
\newcommand\bruno[1]{\authornote{bruno}{red}{#1}}
\newcommand\yanlin[1]{\authornote{yanlin}{green}{#1}}
%%\newcommand\bruno[1]{}

\newcommand{\hl}[1]{\textcolor{red}{#1}}

\newcommand\sem[1]{\llbracket #1 \rrbracket_r}
\newcommand\sems[1]{\llbracket #1 \rrbracket_s}
\newcommand\tsem[1]{\llbracket #1 \rrbracket}
\newcommand{\rbm}[1]{\raisebox{-2.0ex}[0.5ex]{#1}}
\newcommand\nat[0]{\mathbb{N}}
\newcommand\unit[0]{\mathbbm{1}}

%\renewcommand{\mathindent}{0pt}


\begin{comment}
\author{Bruno C. d. S. Oliveira and Yanlin Wang}
\institute{The University of Hong Kong\\
\email{\{bruno,ylwang\}@cs.hku.hk}\\
\authorrunning{Bruno Oliveira and Yanlin Wang}} % abbreviated author list (for running head)
\end{comment}

%%%%%%%%%%%%%%%%%%%%%%%%%%%%%%%%%%%%%%%%%%%%%%%%%%%%%%%%%%%%%%%%%%%%%%%%%%%%%%%%
\begin{document}

\title{Classless Java}
%%\subtitle{Solving The Expression Problem with Covariant Return Types}
%% Poor Man's Family Polymorphism

\maketitle

\begin{abstract}

\end{abstract}

%if False

%\category{D.3.2}{Programming Languages}
%                {Language Classifications}
%                [Functional Languages]
%\category{F.3.3}{Logics and Meanings of Programs}
%                {Studies of Program Constructs}
%                []

%\terms
%Languages
%
%\keywords
%Mixins, explicit effects, monads, aspect-oriented programming, parametricity,
%interference

%endif

%===============================================================================

\section{Introduction}\label{sec:intro}

\begin{itemize}

\item Using annotattions to implement a rich notion of traits with a 
mechanism to instantiate objects (the of method). Goal 1: is to reduce 
the amount of code that is required to program with interfaces and default 
methods. Goal 2: To provide a convenient means to do multiple inheritance 
in Java.

\item Specify the system more formally.

\item Show that we can model all trait operations

\item Implementation using Lombok

\item Case studies: The expression problem, Trivially
Case Studies from traits paper. 

\end{itemize}

\section{Overview}\label{sec:ep}
Yanlin

* Explain what the Mixin annotations do using examples. 

* Motivate the use of multiple inheritance in Java.

* Maybe use Marco's example.

\section{Comparing to traits and mixins}
Haoyuan

   - vs both: we do automatic return type refinement, which has useful applications 
   (example: Expression Problem)

   - vs traits: we support of methods to create new objects (a replacement to constructors);
   Moreover we have the with and clone methods (we miss more applications for those). Show 
   how to model the operations on traits; discuss operations that we cannot model 
   (example: renaming).

   - vs mixins: we use the trait model of explicitly resolving conflicts. This is arguably 
   better for reasoning. 

\section{Formal Semantics}\label{sec:typesafety}
Yanlin and Haoyuan

We need to show 2 things:

1) The dynamic semantics: what's the code that gets generated by a mixin annotation;

2) The type system: what programs to reject; properties: generation of type-safe/checkable code.

\bruno{The implementation is still missing the type system (rejecting some programs)!}

\section{Implementation}
Haoyuan

discuss implementation in lombok; and limitations.

\bruno{The implementation does not support separate compilation yet. Can we fix this?}

\section{Case studies}
Haoyuan and Yanlin

\subsection{A Trivial Solution to the Expression Problem}

\subsection{Other case studies} 
\bruno{The case studies still need to be implemented!}

Collections example from traits paper? 

Other case studies using multiple inheritance?

\section{Related Work}\label{sec:related}

- traits (original, variations, scala)
- mixins (original, scala)
- multiple inheritance 
- expression problem
- ...

%===============================================================================
\section{Conclusion}\label{sec:conclusion}

%===============================================================================

\bibliographystyle{splncs}
\bibliography{paper}

\appendix

\end{document}

%%% Local Variables:
%%% mode: latex
%%% TeX-master: "."
%%% End:
