%\documentclass[preprint]{sigplanconf}
%\documentclass[preprint]{llncs}
\documentclass[a4paper,UKenglish]{lipics}

\newenvironment{grammar}{$\begin{array}[t]{lcll}}{\end{array}$}
\newcommand{\production}[3]{#1&{:}{:}=&#2 & \mbox{{\small{#3}}}}
\newcommand{\productionMore}[2]{&{}&#1 & \mbox{{\small{#2}}}}
%\newcommand{\terminale}[1]{{\text{\tt #1}}}
\newcommand{\nonTerminal}[1]{\mathit{#1}\xspace}
\newcommand\Q\lstinline
\newcommand{\metaVar}[1]{\textit{#1}\xspace}
\newcommand{\ann}{\metaVar{ann}}
\newcommand{\x}{\metaVar{x}}
\newcommand{\e}{\metaVar{e}}
\newcommand{\T}{\metaVar{T}}
\newcommand{\xs}{\metaVar{xs}}
\newcommand{\m}{\metaVar{m}}
\newcommand{\es}{\metaVar{es}}
\newcommand{\C}{\metaVar{C}}
\newcommand{\f}{\metaVar{f}}
\newcommand{\MCall}[3]{#1\mbox{\Q@.@}#2\oR#3\cR}
\newcommand{\ctx}{{\cal{E}}}
\newcommand{\emptyctx}{[\ ]}
\newcommand{\val}{\metaVar{v}}
\newcommand{\vals}{\metaVar{vs}}
\newcommand{\Aux}[1]{\textsf{#1}}
\newcommand\QM[1]{\mbox{\Q@#1@}}
\newcommand\oC{\mbox{\Q@\{@}}
\newcommand\cC{\mbox{\Q@\}@}}
\newcommand\oR{\mbox{\Q@(@}}
\newcommand\cR{\mbox{\Q@)@}}
\newcommand{\this}{\mbox{\Q@this@\xspace}}
\newcommand{\mixinAnn}{\mbox{\Q$@Mixin$\xspace}}
\newcommand{\method}{\metaVar{meth}}
\newcommand{\mh}{\metaVar{mh}}
\newcommand{\obj}{\metaVar{obj}}

\newcommand{\spc}{\ }


\usepackage{listings}
\usepackage{multicol}
\usepackage{microtype}%if unwanted, comment out or use option "draft"
\usepackage[table,xcdraw]{xcolor}
\usepackage{color}
\usepackage{amsmath}
\usepackage{stmaryrd}
\usepackage{graphicx}
\usepackage{amssymb}
\usepackage{fancyvrb}
\usepackage{url}
\usepackage{pstricks,pst-node,pst-tree}
\usepackage{theorem}
%% \usepackage{mathpartir}
\usepackage{bbm}
\usepackage{pgf}
\usepackage{multirow}

\usepackage{listings}
\usepackage{verbatim}
\usepackage{graphicx}
\usepackage{wrapfig}

\usepackage[T1]{fontenc}
\usepackage[scaled=0.85]{beramono}
\usepackage{mathpartir}

% "define" code highlights for Java and Scala
\lstdefinelanguage{JavaScala}{
  morekeywords={public,int,interface,implements,default,
    abstract,case,void,catch,class,def,static,%
    do,else,extends,false,final,finally,%
    for,if,implicit,import,match,mixin,%
    new,null,object,override,package,%
    private,protected,requires,return,sealed,%
    super,this,throw,trait,true,try,%
    type,var,while,yield},
  otherkeywords={=>,<-,<\%,<:,>:,\#,@},
  sensitive=true,
  morecomment=[l]{//},
  morecomment=[n]{/*}{*/},
  morestring=[b]",
  morestring=[b]',
  morestring=[b]"""
}

\lstset{ %
language=JavaScala,                % choose the language of the code
columns=flexible,
lineskip=-1pt,
basicstyle=\ttfamily\small,       % the size of the fonts that are used for the code
numbers=none,                   % where to put the line-numbers
numberstyle=\ttfamily\tiny,      % the size of the fonts that are used for the line-numbers
stepnumber=1,                   % the step between two line-numbers. If it's 1 each line will be numbered
numbersep=5pt,                  % how far the line-numbers are from the code
backgroundcolor=\color{white},  % choose the background color. You must add \usepackage{color}
showspaces=false,               % show spaces adding particular underscores
showstringspaces=false,         % underline spaces within strings
showtabs=false,                 % show tabs within strings adding particular underscores
morekeywords={var},
%  frame=single,                   % adds a frame around the code
tabsize=2,                  % sets default tabsize to 2 spaces
captionpos=none,                   % sets the caption-position to bottom
breaklines=true,                % sets automatic line breaking
breakatwhitespace=false,        % sets if automatic breaks should only happen at whitespace
title=\lstname,                 % show the filename of files included with \lstinputlisting; also try caption instead of title
escapeinside={(*}{*)},          % if you want to add a comment within your code
keywordstyle=\ttfamily\bfseries,
aboveskip=3pt,
belowskip=-1pt
% commentstyle=\color{Gray},
% stringstyle=\color{Green}
}


%%\usepackage{natbib}
%%\bibpunct();A{},
%%\let\cite=\citep

%include lhs2TeX.fmt
%include lhs2TeX.sty
%include forall.fmt

%\pagestyle{plain}

%{\theorembodyfont{\sffamily} \newtheorem{theorem}{Theorem}}
%{\theorembodyfont{\sffamily} \newtheorem{lemma}{Lemma}}
%\newtheorem{theorem}{Theorem}
%\newtheorem{lemma}{Lemma}
%\newenvironment{proof}{\textbf{Proof:\hspace{4mm}}}{$\Box$}
\newcommand{\authornote}[3]{{\color{#2} {\sc #1}: #3}}
\newcommand\bruno[1]{\authornote{bruno}{red}{#1}}
\newcommand\yanlin[1]{\authornote{yanlin}{purple}{#1}}
%\newcommand{\hl}[1]{\textcolor{red}{#1}}
\newcommand\marco[1]{\authornote{marco}{blue}{#1}}
\newcommand\haoyuan[1]{\authornote{haoyuan}{cyan}{#1}}

\newcommand\sem[1]{\llbracket #1 \rrbracket_r}
\newcommand\sems[1]{\llbracket #1 \rrbracket_s}
\newcommand\tsem[1]{\llbracket #1 \rrbracket}
\newcommand{\rbm}[1]{\raisebox{-2.0ex}[0.5ex]{#1}}
\newcommand\nat[0]{\mathbb{N}}
\newcommand\unit[0]{\mathbbm{1}}

\newcommand\mixin{\mixinAnn\xspace}
\renewcommand{\paragraph}[1]{\vspace{5pt}\noindent{\bf #1}}
\newenvironment{listing}{\vspace{-3pt}\begin{lstlisting}}{\end{lstlisting}\vspace{-3pt}}
\usepackage{xspace}

% Author macros::begin %%%%%%%%%%%%%%%%%%%%%%%%%%%%%%%%%%%%%%%%%%%%%%%%
\title{Classless Java: Tuning Java Interfaces}
\titlerunning{Classless Java}

\author[1]{John Q. Open}
\author[2]{Joan R. Access}
\affil[1]{Dummy University Computing Laboratory\\
  Address, Country\\
  \texttt{open@dummyuni.org}}
\affil[2]{Department of Informatics, Dummy College\\
  Address, Country\\
  \texttt{access@dummycollege.org}}
\authorrunning{J.\,Q. Open and J.\,R. Access}
% mandatory. First: Use abbreviated first/middle names. Second (only in severe
% cases): Use first author plus 'et. al.'

\begin{comment}
\Copyright{John Q. Open and Joan R. Access}
% mandatory, please use full first names. LIPIcs license is "CC-BY";
% http://creativecommons.org/licenses/by/3.0/

\subjclass{Dummy classification -- please refer to
  \url{http://www.acm.org/about/class/ccs98-html}}
% mandatory: Please choose ACM 1998 classifications from
% http://www.acm.org/about/class/ccs98-html . E.g., cite as "F.1.1 Models of
% Computation".
\keywords{Dummy keyword -- please provide 1--5 keywords}% mandatory: Please provide 1-5 keywords
% Author macros::end %%%%%%%%%%%%%%%%%%%%%%%%%%%%%%%%%%%%%%%%%%%%%%%%%

%Editor-only macros:: begin (do not touch as author)%%%%%%%%%%%%%%%%%%%%%%%%%%%%%%%%%%
\serieslogo{}%please provide filename (without suffix)
\volumeinfo%(easychair interface)
  {Billy Editor and Bill Editors}% editors
  {2}% number of editors: 1, 2, ....
  {Conference title on which this volume is based on}% event
  {1}% volume
  {1}% issue
  {1}% starting page number
\EventShortName{}
\DOI{10.4230/LIPIcs.xxx.yyy.p}% to be completed by the volume editor
% Editor-only macros::end %%%%%%%%%%%%%%%%%%%%%%%%%%%%%%%%%%%%%%%%%%%%%%%
\end{comment}

%%%%%%%%%%%%%%%%%%%%%%%%%%%%%%%%%%%%%%%%%%%%%%%%%%%%%%%%%%%%%%%%%%%%%%%%%%%%%%%%
\begin{document}
\maketitle

\begin{abstract}

Java 8 introduced \emph{default methods}, allowing interfaces to
have method implementations. When combined with (multiple) interface
inheritance, default methods provide a basic form of multiple
inheritance. However, using this combination to simulate multiple
inheritance quickly becomes cumbersome, and appears to be quite
restricted.

This paper shows that, with a simple language feature, default methods
and interface inheritance are in fact very expressive. Our proposed
language feature, called \emph{object interfaces}, enables powerful
object-oriented idioms, using multiple inheritance, to be expressed
conveniently in Java. Object interfaces refine conventional Java
interfaces in three different ways. Firstly, object interfaces have
their own object instantiation mechanism, providing an alternative to
class constructors. Secondly, object interfaces support \emph{abstract
  state operations}, providing a way to use multiple inheritance with
state in Java. Finally, object interfaces allow type refinements that
are often tricky to model in conventional class-based
approaches. Interestingly, object interfaces do not require changes to
the runtime, and they also do not introduce any new syntax: all three features are achieved by reinterpreting
existing Java syntax, and are translated into regular Java code
without loss of type-safety. Since no new syntax is introduced, it
would be incorrect to call object interfaces a language extension or
syntactic sugar. So we use the term \emph{language tuning} to
characterize this kind of language feature. An implementation of
object interfaces using Java annotations and a formalization of the
static and dynamic semantics are presented. Moreover the usefulness of
object interfaces is illustrated through various examples.

\end{abstract}

%if False

%\category{D.3.2}{Programming Languages}
%                {Language Classifications}
%                [Functional Languages]
%\category{F.3.3}{Logics and Meanings of Programs}
%                {Studies of Program Constructs}
%                []

%\terms
%Languages
%
%\keywords
%Mixins, explicit effects, monads, aspect-oriented programming, parametricity,
%interference

%endif

%===============================================================================

\section{Introduction}\label{sec:intro}

\Objectoriented languages strive to offer great code reuse.
They couple flexibility and rigour, expressive power and
modular reasoning.  Two main OO models emerged to this end:
\prototypebased (PB)~\cite{Ungar87self} and \classbased languages such as
Java, C\# or Scala.  In \prototypebased
languages objects inherit from other objects. Thus objects own
both behaviour and state (and objects are all you have).
In \classbased languages an object is an instance of a specific class,
and classes inherit from other classes.  Objects own state,
while classes contain behaviour and the structure of the state.

Regardless of the OO model, inheritance is a key mechanism is OO 
languages. Inheritance provides a modularization mechanism, which 
is used to reuse implementations from inherited classes/objects. 
Unfortunately, as widely acknowledged in the literature~\cite{scharli03traits,Sak89dis,bracha90mixin,malayeri2009cz}
multiple inheritance, especially when combined with state, has several
tricky issues, including several variants of the famous 
\emph{diamond problem}~\cite{bracha90mixin,Sak89dis}. Many of the problems related to 
inheritance arise from the direct use of \emph{fields} to model state. 
Inheriting two fields with the same name raises 
the question of whether the two fields should be kept, or only one 
field should exist. Initialization 
of the fields is also problematic, since initialization code may be inherited from 
multiple parents. Finally, an additional problem with mutable fields is that their 
type cannot be type-refined in extensions, which can cause modularity problems
~\cite{wadler98expression,eptrivially}.

To address those limitations, this paper presents a third alternative
OO model called \emph{\interfacebased} \objectoriented programming
languages (IB), where objects implement interfaces directly and fields
are not directly supported. In IB interfaces own the implementation for the
behaviour, which is structurally defined in their
interface. Programmers do not define objects directly but delegate the
task to \emph{object interfaces}, whose role is similar to
non-abstract classes in \classbased \objectoriented programming
languages (CB). Objects are instantiated by static factory methods in
object interfaces.

Due to the absence of fields, a key challenge in IB lies in how to model state, which is
fundamental to have stateful objects. All abstract operations in an
object interface are interpreted as \emph{abstract state
  operations}. The abstract state operations include various common
utility methods (such as getters and setters, or clone-like
methods). Objects are only responsible to define the ultimate
behaviour of a method. % and, for example, if such a method is just a setter.
Anything related to state is completely contained in the
instances and does not leak into the inheritance logic.  In CB, the structure of the state is fixed and can only grow
by inheritance.  In contrast, in IB the state is never
fixed, and methods such as abstract setters and getters
can always receive an explicit implementation down in the inheritance
chain, improving \textbf{modularity and flexibility}.  That is, the
concept of abstract state is more fluid.

Object interfaces provide support for automatic type-refinement.  In
contrast, in CB special care and verbose explicit type-refinement are
required to produce code that deals with subtyping adequately. We
believe that such verbosity hindered and slowed down the discovery of
useful programming patterns involving type-refinement. Our previous work~\cite{eptrivially} 
on the Expression Problem~\cite{wadler98expression} in Java-like languages shows how
easy it is to solve the problem using only type-refinement. However it
took nearly 20 years since the formulation of the problem for that
solution to be presented in the literature. In IB, due to its emphasis
on type-refinement, that solution should have been more obvious.

One advantage of abstract state operations and
type-refinement is that it allows a new approach to
\emph{type-safe covariant mutable state}. That is, in IB,
it is possible to type-refine \emph{mutable} state in subtypes. This is
typically forbidden in CB: it is widely known that \emph{naive} type-refinement of
mutable fields is not type-safe. Although covariant refinement of mutable
fields is supported by some type systems~\cite{bruce98astatically,bruce1994paradigmatic,ernst06virtual,Saito2013933}, this requires
significant complexity and restrictions to ensure that all uses of
covariant state are indeed type-safe.

\begin{comment}
\marcoT{%

In this paper we show how to support type-safe
\textbf{field removal},
\textbf{field type refinement} allowing a kind of covariant setters refinement,
and \textbf{multiple inheritance}.}
\end{comment}

\begin{comment}
Another advantage of IB is the support for
multiple inheritance. The literature shows how
easy it is to combine multiple sources of pure behaviour
using traits~\cite{scharli03traits}. In Java
multiple \emph{interface} inheritance has been supported since
inception, and Java 8 default methods~\cite{goetz12fdefenders} bring some of the
advantages of traits to Java. The literature~\cite{Sak89dis,bracha90mixin,malayeri2009cz}
is also rich on how hard it is to combine multiple sources
of behaviour \textbf{and} state with multiple \emph{implementation}
inheritance of classes. In IB there is only multiple
interface inheritance, yet programmers can still use state via
abstract state operations. IB enables powerful
idioms using multiple inheritance and state.
\end{comment}

%Traits or Java8 interfaces still assume a CB model:
%But traits still require classes, which
%are responsible for object construction and adding state.

%To retain the easiness and modularity of combining multiple sources of
%pure behaviour, in IB state is just a special kind of behaviour
%implementation.
%\marco{NO NO!
%the advantage is not just multiple composition
%(by the way, that is the term you are searching for, not multiple inheritance)
%By not having state reified in the ideal/platonic level, it is not fixed
%what is state and what is not, thus you can (model an encoding of) REMOVE fields and you can refine
%fields.}
% Objects are the only responsibles to define the
%ultimate behaviour of a method and, for example, if such method is
%just a setter. Anything related to state is completely contained in
%the instances and do not leak in the interfaces.

IB could be explained by defining a novel language, with new syntax
and semantics. However, this would have a steep learning curve.  We
take a different approach instead. For the sake of providing a more
accessible explanation, we will embed our ideas directly into Java.
Our IB embedding relies on the
new features of Java 8: interface \emph{static methods} and
\emph{default methods}, which allow interfaces to have method
implementations. In the context of Java, what we propose is a programming
style, where we never use classes (more precisely, we never
  use the \Q@class@ keyword).  We call this restricted version of
Java \emph{Classless Java}. 
\begin{comment}
\textcolor{red}{Classless Java provides us
with a way to address the famous diamond problem~\cite{bracha90mixin,Sak89dis}
based on Java 8's interface model: interfaces can offer different implementations
using the default methods, whereas conflicts are supposed to be explicitly resolved,
otherwise the program will be rejected by the compiler. An interface that inherits
a few default implementations of a method from its supertypes can define a new implemention
to hide the conflicts among them, or use the \Q@super@ keyword to specify the one it wants. Examples
and more discussion will be presented afterwards.}
\end{comment}

%Still, the design is quite conservative and appears to be quite limited
% in its current form to model advanced forms of multiple inheritance.
%Indeed, our own personal experience of combining default methods
%and multiple interface inheritance in Java to achieve multiple implementation
%inheritance is that many workarounds and boilerplate code are needed.
%In particular, we encountered difficulties because:

%\begin{itemize}
%
%\item {\em Interfaces have no constructors.} As a result, classes are
%still required to create objects, leading to substantial boilerplate
%code for initialization.
%
%\item {\em Interfaces do not have state.} This creates a tension between
% using multiple inheritance and having state. Using setter and
%  getter methods is a way out of this tension, but this workaround
%  requires tedious boilerplate classes that later implement those
%  methods.
%
%\item {\em Useful, general purpose methods require special care in
%  the presence of subtyping.} Methods such as
%  \emph{fluent} setters~\cite{fowler2005fluentinterface}, not only require access to the
%  internal state of an object, but they also require their return types to be
%  refined in subtypes.
%
%\end{itemize}

%\noindent Clearly, a way around those difficulties would be to change
%Java and just remove these limitations. Scala's own notion of
%traits~\cite{scala-overview}, for example, allows state in traits. Of
%course adding state (and other features) to interfaces would
%complicate the language and require changes to the compiler, and this
%would go beyond the goals of Java 8 development team.

%This paper takes a different approach. Rather than trying to get
%around the difficulties by changing the language in fundamental ways,
%we show that, with a simple language feature, default methods and
%interface inheritance are in fact very expressive.





%With object interfaces, many Java programs can be built
%without using a single class!

Using Java annotation processors, we produce an implementation of
Classless Java, which allows us to stick to pure Java 8. 
%By annotating
% the interfaces that represent object interfaces with \mixin, code for interface instantiation and
%type refinement can be automatically generated. Such code should not be needed in the first place in
%a real IB language, and the annotation processor allows us to
%transparently hide it from Java programmers. 
The implementation works
by performing AST rewriting, allowing most existing Java tools (such as
IDEs) to work out-of-the-box with our implementation. Moreover, the
implementation blends Java's conventional CB style and IB smoothly.
We apply object interfaces to several interesting
Java programs and conduct various case studies. Finally, we also
discuss the behaviour of Classless Java and
its properties.

%Since Java was not designed to be used in this way, our style can be verbose, especially about object instantiation.


%
%Object interfaces do not require changes to the Java runtime or compiler,
%and they also do not introduce any new syntax. All three features of object interfaces are
%achieved by reinterpreting existing Java syntax, and are translated
%into regular Java code without loss of type-safety. Since no new
%syntax is introduced, it would be incorrect to call object interfaces
%a language extension or syntactic sugar. So we use the term
%\emph{language tuning} instead. Language tuning sits in between a
%lightweight language extension and a glorified library. Language
%tuning can offer many features usually implemented by a real language
%extension, but because it does not modify the language syntax
%pre-existing tools can work transparently on the tuned language.  To
%exploit the full benefits of language tuning,



%To formalize object interfaces, we propose Classless Java (CJ): a
%FeatherweightJava-style~\cite{Igarashi01FJ} calculus, which captures the essence
%of interfaces with default methods. The semantics of object interfaces
%is given as a syntax-directed translation from CJ to itself. In the
%resulting CJ code, all object interfaces are translated into regular CJ
%(and Java) interfaces with default methods. The translation is proved
%to be type-safe, ensuring that the translation does not
%introduce type-errors in client code.
%CJ's usefulness goes beyond serving as a
%calculus to formalize object interfaces. During the development
%process of CJ, we encountered a bug in the implementation of default
%methods for the Eclipse Compiler for Java (ECJ). For the program revealing the
%bug, ECJ behaves differently from both our formalization and Oracle's
%Java compiler.

\begin{comment}
To evaluate the usefulness of object interfaces, we illustrate
\numOfCaseStudies \bruno{needs updates}\yanlin{updated}
applications. The first application is a simple
solution to the Expression Problem~\cite{wadler98expression}, supporting independent
extensibility~\cite{zenger05independentlyextensible}, and without boilerplate code. The second
application shows how embedded DSLs using fluent interfaces~\cite{fowler2005fluentinterface}
can be easily defined using object interfaces. The third
application is a case study for a simple Maze game implemented with
multiple inheritance. For this application we show that there is a
significant reduction in the number of lines of code when compared
to an existing implementation~\cite{bono14} using plain Java 8. The last
application is a relatively larger case study of refactoring of a simple interpreter, showing that our
approach can benefit modularity and scale to real code base. Noteworthy, the first three applications are implemented
without defining a single class! The last one contains several classes that cannot be
converted to interfaces due to the limitation of our implementation of \mixin.
\end{comment}

\begin{comment}
While the Java embedding has obvious advantages from the practical
point-of-view, it also imposes some limitations that a new IB language
would not have. Supporting proper encapsulation is
difficult in Java due to limitations of Java interfaces. In particular
in Java interfaces the visibility of all methods is public. Thus
modelling private state is difficult in current Java 8. However,
 using existing design patterns~\cite{BettiniDSS13} we can emulate hiding
methods from interfaces. Furthermore Java 9 will allow private methods
in interfaces~\cite{srikanth16private}.
\end{comment}

%We discuss
%these limitations, possible workarounds, and native language support in Section~\ref{sec:futurework}.

In summary, the contributions of this paper are:
\begin{itemize}

\item {\bf IB and Object Interfaces:} which enable
  powerful programming idioms using multiple-inheritance,
  type-refinement and abstract state operations.

\item {\bf Classless Java:} a practical realization of IB in
  Java. Classless Java is implemented using annotation processing,
  allowing most tools to work transparently with our approach.
  Existing Java projects can use our approach and still be
  backward compatible with their clients, in a way that is specified
  by our safety properties.

\item {\bf Type-safe covariant mutable state:} we show how the
 combination of abstract state operations and type-refinement enables
 a form of mutable state that can be covariantly refined in a type-safe way.

%\item{\bf Type preservation guarantees:}\bruno{needs rephrasing}
%We discuss our formalization of a subset of Java8 type system and how we use this
%to characterize safety properties about our annotations.
%\item {\bf Classless Java (CJ):} A simple formal calculus that models
%the essential features of Java 8 interfaces with default methods, and
%can be used to formally define the translation of object interfaces.
%We prove several properties of the translation\footnote{Proofs and prototype implementation are available in
%  the supplementary materials.}.

\item {\bf Applications and case studies:} we illustrate the usefulness of IB
  through various examples and case studies\footnote{\url{https://github.com/YanlinWang/classless-java}}. 
  An extended version with a formal translation to Java can be found in the 
  companion technical report~\cite{classless}.

%\item{\bf Language Tuning:} We identify the concept of language tuning
%and describe object interfaces as an example. We also discuss
%how other existing approaches, such as the annotations in project
%Lombok~\cite{lombok},  can be viewed as language tuning.

\end{itemize}


\section{A Running Example: Animals}\label{sec:ep}
\begin{figure*}
\centering
\saveSpaceFig
\begin{tabular}{c|c}
\includegraphics[height=3.3cm]{pdfs/PegasusDetail.pdf}\hspace{0pt} &
\begin{minipage}{7cm}
\vspace{-95pt}
\lstinputlisting[linerange=6-13]{../UseMixinLombok/src/pegasus/simple/java8/Main.java}% APPLY:linerange=PEGASUS_JAVA
%basicstyle=\ttfamily\scriptsize
\end{minipage}
\end{tabular}
\caption{The animal system (left: complete structure, right: code for simplified animal system).}\label{fig:pegasus}
\saveSpaceFig
\end{figure*}

This section illustrates how our programming style, supported by
\mixinAnn{}, enables powerful programming idioms based on multiple
inheritance and type refinements.  We propose a standard example:
\Q@Animal@s with a 2-dimensional \Q@Point2D@ representing their
\Q@location@, subtypes \Q@Horses@, \Q@Bird@s, and \Q@Pegasus@.
Birds can \Q@fly@, thus their locations need to be 3-dimensional
\Q@Point3D@s (field type refinement). We model \Q@Pegasus@
(a well-known creature in Greek mythology) as a kind of
\Q@Animal@ with the skills of both \Q@Horse@s and \Q@Bird@s (multiple
inheritance). A simple class diagram illustrating the basic system is
given on the left side of Figure~\ref{fig:pegasus}.
%\footnote{Some
%  research argues in favor of using subtyping for modeling taxonomies,
%  other research argues against this practice, we do not wish to take
%  sides in this argument, but to provide an engaging example.}

%\bruno{Should we provide a one sentence summary in the abstract of how much code
%is needed in Java (without Obj) vs the approach with CJ, for the Pegasus example?}
%\marco{Not in the abstract. But I think is good to have that number and to put in in the example section.
%I will set up the sentence and someone can compute the number.}

\subsection{Simple Multiple Inheritance with Default
  Methods}\label{sec:simple}

Before modelling the complete animal system, we  start with a
simple version without locations. This version serves the purpose of illustrating how
Java 8 default methods can already model simple forms of multiple inheritance.
\texttt{Horse} and \texttt{Bird} are subtypes
of \texttt{Animal}, with methods \texttt{run()} and \texttt{fly()},
respectively. Pegasus can not only \emph{run} but also \emph{fly}! This is the
place where \emph{``multiple inheritance''} is needed, because
\texttt{Pegasus} needs to obtain \texttt{fly} and \texttt{run}
functionality from both \texttt{Horse} and \texttt{Bird}.
A first attempt to model the animal system is given on the right side
of Figure~\ref{fig:pegasus}.
Note that the implementations of the methods \texttt{run}
and \texttt{fly} are defined inside interfaces, using default
methods. Moreover, because interfaces support multiple interface
inheritance, the interface for \texttt{Pegasus} can inherit behaviour
from both \texttt{Horse} and \texttt{Bird}. Although Java interfaces
do not allow instance fields, no form of state is needed so far to
model the animal system.

\para{Instantiation}
To use \texttt{Horse}, \texttt{Bird} and \texttt{Pegasus}, some
objects must be created first. A first problem with using
interfaces to model the animal system is simply that interfaces
cannot be directly instantiated. Classes, such as:

\lstinputlisting[linerange=17-19]{../UseMixinLombok/src/pegasus/simple/java8/Main.java}% APPLY:linerange=PEGASUS_INST

\noindent are needed for instantiation. Now a \texttt{Pegasus} animal can be created
using the class constructor:

\begin{lstlisting}
Pegasus p = new PegasusImpl();
\end{lstlisting}

\noindent There are some annoyances here. Firstly, the sole
purpose of the classes is to provide a way to instantiate
objects. Although (in this case) it takes only one line of code to
provide each of those classes, this code is essentially boilerplate
code, which does not add behavior to the system. Secondly,
the namespace gets filled with three additional types. For example,
both \texttt{Horse} and \texttt{HorseImpl} are needed: \texttt{Horse}
is needed because it needs to be an interface so that \texttt{Pegasus}
can use multiple inheritance; and \texttt{HorseImpl} is needed to
provide object instantiation.
Note that, for this very simple animal system, plain Java 8 anonymous
classes can be used to avoid these problems.  We could have simply
instantiated \texttt{Pegasus} using:

\begin{lstlisting}
Pegasus p = new Pegasus() {}; // anonymous class
\end{lstlisting}

\noindent However, as we shall see, once the system gets a little more
complicated, the code for instantiation quickly becomes more
complex and verbose (even with anonymous classes).

\subsection{Object Interfaces and Instantiation}

To model the animal system with object interfaces all that a user
needs to do is to add an \mixinAnn{} annotation to the \texttt{Horse},
\texttt{Bird}, and \texttt{Pegasus} interfaces:

\lstinputlisting[linerange=8-12]{../UseMixinLombok/src/pegasus/simple/lombok/Main.java}% APPLY:linerange=PEGASUS_LOMBOK
\noindent The effect of the annotations is that a static \emph{factory} method called
\texttt{of} is automatically added to the interfaces. With the
\texttt{of} method a \texttt{Pegasus} object is instantiated as follows:

\begin{lstlisting}
Pegasus p = Pegasus.of();
\end{lstlisting}

\noindent The \texttt{of} method provides an alternative to a
constructor, which is missing from interfaces. The following code
shows the code corresponding to the \texttt{Pegasus} interface
after the \mixinAnn{} annotation is processed:

\begin{lstlisting}
interface Pegasus extends Horse, Bird {
  // generated code not visible to users
  static Pegasus of() { return new Pegasus() {}; }
}
\end{lstlisting}

\noindent Note that the generated code is transparent to users, who
only see the original code with the \mixin annotation. Compared to the pure
Java solution in Section~\ref{sec:simple}, the solution using object interfaces
has the advantage of providing a direct mechanism for object
instantiation, which avoids adding boilerplate classes to the
namespace.

\subsection{Object Interfaces with State}

The animal system modelled so far is a simplified version of the
system presented in the left-side of Figure~\ref{fig:pegasus}.
The example is still not sufficient to appreciate the advantages of IB
programming.
Now we model the complete animal system where an \Q@Animal@ includes a \Q@location@
representing its position in space. We use 2D points to keep track of locations.

\para{\Q@Point2D@: simple immutable data with fields}
Points can be modelled with interfaces.
 In IB
 state is accessed and
manipulated using abstract methods.  The usual approach to model
points in Java is to use a class with fields for the coordinates.
In Classless Java interfaces are used instead:

\begin{lstlisting}
interface Point2D { int x(); int y(); }
\end{lstlisting}

\noindent The encoding over Java is now inconvenient: creating a new point object is cumbersome, even
with anonymous classes:

\begin{lstlisting}
Point2D p = new Point2D() {
  public int x() {return 4;}
  public int y() {return 2;}
}
\end{lstlisting}

\noindent However this cumbersome syntax is not required for every
object allocation. As programmers do, for ease or reuse, the boring
repetitive code can be encapsulated in a method. A generalization of the
\texttt{of} static factory method is appropriate:% in this case:
\begin{lstlisting}
interface Point2D { int x(); int y();
  static Point2D of(int x, int y) {
    return new Point2D() {
      public int x(){return x;}
      public int y(){return y;}
    };  }  }
\end{lstlisting}

\vspace{-5pt}
\para{\Q@Point2D@ with object interfaces}
This obvious ``constructor'' code is generated by the \mixin
annotation.  By annotating the interface \Q@Point2D@, a variation of the shown
static method \texttt{of} will be generated, mimicking the functionality of a
simple-minded constructor. \mixin first looks at the abstract methods and detects
what the fields are, then generates an \Q@of@ method with one parameter for each
of them. We can just write:

\begin{lstlisting}
@Obj interface Point2D { int x(); int y(); }
\end{lstlisting}

\noindent A field or factory parameter is generated for every
abstract method that takes no parameters.
%(except for methods with special
%names).
 An example of using \Q@Point2D@, where we ``clone'' an existing point
 but use
 \Q@42@ as the x-coordinate, is:
\begin{lstlisting}
Point2D p = Point2D.of(42,myPoint.y());
\end{lstlisting}

\para{\texttt{with-} methods in object interfaces}
The pattern of creating a new object by reusing most information from an old
object is very common when programming with immutable
data-structures. As such, it is
supported by \mixin as \Q@with-@ methods:
\begin{lstlisting}
@Obj interface Point2D {
    int x();    int y(); // getters
    // with- methods
    Point2D withX(int val);
    Point2D withY(int val);
}
\end{lstlisting}

\noindent Using \texttt{with-} methods, the point \texttt{p} can also be created
by:

\begin{lstlisting}
Point2D p = myPoint.withX(42);
\end{lstlisting}

\noindent If there is a large number of fields, \texttt{with-} methods
will save programmers from writing large amounts of tedious code that
simply copies field values.
\begin{comment}
is expanded by \mixin into\footnote{
Note how we actually generate a real field \Q@int x=_x;@.
This provides a more uniform translation that can work also for mutable data structures, where setters are required.
}

\begin{lstlisting}
interface Point2D { int x(); int y();
    static Point2D of(int _x, int _y){ return new Point2D(){
        int x=_x; int y=_y;
        public int x(){return x;}   public int y(){ return y; }
        Point2D withX(int val){ return Point2D.of(val,this.y()); }
        Point2D withY(int val){ return Point2D.of(this.x(),val); }
    }; }
  Point2D withX(int val);    Point2D withY(int val); }
\end{lstlisting}
\end{comment}
Moreover, if the programmer wants a different implementation, he may
provide an alternative implementation using \Q@default@ methods. For example:
\begin{lstlisting}
@Obj interface Point2D {
    int x(); int y();
    default Point2D withX(int val){ /*myCode*/ }
    Point2D withY(int val); }
\end{lstlisting}

%\begin{comment}
%\marco{I re-enabled this code, I think is needed for understandability}
\noindent is expanded into
\begin{lstlisting}
interface Point2D {
    int x(); int y();
    default Point2D withX(int val){ /*myCode*/ }
    Point2D withY(int val);
    static Point2D of(int _x, int _y){
      return new Point2D(){
        int x=_x;    int y=_y;
        public int x(){return x;}
        public int y(){return y;}
        public Point2D withY(int val){
          return of(x(),val);}  }; } }
\end{lstlisting}

\noindent Only code for methods needing implementation is generated. Thus,
programmers can easily customize the behaviour for their special needs.
Also, since \mixin interfaces offer the \Q@of@ factory method, only interfaces where all the abstract methods
can be synthesized can be object interfaces. A non-\mixin interface is like an abstract class in Java.
%\end{comment}

%Firstly, to model \texttt{Point2D} that has x-coordinate and y-coordinate by an
%interface, we immediately run into the problem of expressing the fields
%\texttt{x} and \texttt{y}.
%interfaces. Method \texttt{withX, withY} creates a new instance of
%\texttt{Point2D} with updated field \texttt{x,y}, respectively.

% Firstly, to model \texttt{Animal} by an interface, we immediately run into the
% problem of expressing the field \texttt{point}. Since in Java there is no way to
% define member fields inside interfaces, we propose to simulate fields by
% abstract methods inside interfaces:

%\lstinputlisting[linerange=42-48]{../UseMixinLombok/src/pegasus/TestAnimal.java}% APPLY:linerange=POINT2D

%\paragraph{Instantiation}
%In Java, to implement an interface like \texttt{Point2D}, a typical and trivial
%approach that programmers usually do is creating a class extending the interface
%and providing implementation for all methods inside. For example, this is the
%implementation for interface \texttt{Point2D}:

%\begin{lstlisting}
%class Point2DImpl implements Point2D {
%    private int _x;
%    private int _y;
%    public Point2DImpl(int x, int y) {
%        this._x = x;
%        this._y = y;
%    }
%    public int x() {
%        return _x;
%    }
%    public int y() {
%        return _y;
%    }
% %Your implementation of with is wrong
%    public Point2D withX(int x) {
%        x(x);
%        return this;
%    }
%    public void x(int x) {
%        _x = x;
%    }
%    public void y(int y) {
%        _y = y;
%    }
%    public Point2D withY(int y) {
%        y(y);
%        return this;
%    }
%}
%\end{lstlisting}
%
%\texttt{Point2DImpl} implements \texttt{Point2D} and provides a constructor with
%quite mechanical code. What's worse, the implementation in \texttt{Point2DImpl}
%may not be reused in a single inheritance language.

%
%\begin{lstlisting}
%  // inside interface Point2D
%  static Point of(int x, int y) {
%      return new Point() {
%          int _x = x;
%          int _y = y;
%          public int x() {
%            return _x;
%          }
%          public int y() {
%            return _y;
%          }
%          public Point2D withX(int x) {
%            x(x);
%            return this;
%          }
%          public void x(int x) {
%            _x = x;
%          }
%          public void y(int y) {
%            _y = y;
%          }
%          public Point2D withY(int y) {
%            y(y);
%            return this;
%          }
%    }
%  }
%\end{lstlisting}
%
%\lstinputlisting[linerange=-]{} % APPLY:linerange=POINT_OF

\para{\Q@Animal@ and \Q@Horse@: simple mutable data with fields}
2D points are mathematical entities, thus we choose an immutable data structure to
model them. Animals are real world entities, and when an animal moves,
it is the \emph{same} animal with a different location. We model this with
mutable state.

%Now we proceed to define \texttt{Animal} with \texttt{point} ``member
%field''.
%%Not again, before we used only getters!
% Again, we model this member field with getter and setter methods:
\lstinputlisting[linerange=58-60]{../UseMixinLombok/src/pegasus/TestAnimal.java}% APPLY:linerange=ANIMAL

\noindent Here we declare an abstract getter and a setter for the mutable ``field''
\Q@location@.  Without the \mixin annotation, there is no convenient way to
instantiate \texttt{Animal}.  For \texttt{Horse}, the \mixin annotation is used
and an implementation of \texttt{run()} is defined using a \Q@default@
method. The implementation of \texttt{run()} further illustrates the convenience of \texttt{with-} methods:

\lstinputlisting[linerange=64-66]{../UseMixinLombok/src/pegasus/TestAnimal.java}% APPLY:linerange=HORSE

\noindent Creating and using \texttt{Horse} is quite simple:

\lstinputlisting[linerange=10-12]{../UseMixinLombok/src/pegasus/TestAnimal.java}% APPLY:linerange=USINGHORSE

\noindent Note how the \texttt{of}, \texttt{withX} and
\texttt{location} methods (generated automatically) give a
basic interface for dealing with animals.

In summary, state (mutable or not) in object interfaces
relies on a notion of abstract state, and state is not directly
available to programmers. Instead programmers use methods, called
\emph{abstract state operations}, to interact with state.

% by method
%\texttt{run()}: method \texttt{withX} returns a new point object with field
%\texttt{x} updated by the argument to \texttt{withX}. Without these
%\texttt{with} methods, operations like \texttt{run()} would be much harder to define.

\subsection{Object Interfaces and Subtyping}
\Q@Bird@s are \Q@Animal@s, but while \Q@Animal@s only need 2D
locations, \Q@Bird@s need 3D locations. Therefore when the \texttt{Bird}
interface extends the \Q@Animal@ interface, the notion of points needs to
be \emph{refined}. Such kind of refinement is challenging
in typical \classbased approaches. Fortunately, with object interfaces,
we are able to provide a simple and effective solution.

\para{Unsatisfactory \classbased solutions to field type refinement}
In Java if we want to define an animal class with a field we have a set of
unsatisfactory options in front of us:
\begin{itemize}
\item Define a \Q@Point3D@ field in \Q@Animal@: this is bad since all animals
  would require more than needed.
  %Also it requires the programmer to predict the future, or
  Also it requires adapting the old code to accommodate for new evolutions.

\item Define a \Q@Point2D@ field in \Q@Animal@ and define an extra \Q@int z@
  field in \Q@Bird@.  This solution is very ad-hoc, requiring to basically
  duplicate the difference between \Q@Point2D@ and \Q@Point3D@ inside \Q@Bird@.
  %Again, there are many reasons this would be bad,
  The most dramatic criticism is that it would not scale to a scenario when
  \Q@Bird@ and \Q@Point3D@ are from different programmers.

\item Redefine getters and setters in \Q@Bird@, always put \Q@Point3D@ objects
  in the field and cast the value out of the \Q@Point2D@ field to \Q@Point3D@
  when implementing the overridden getter.  This solution scales to the multiple
  programmers approach, but requires ugly casts and can be implemented in a
  wrong way leading to bugs.
\end{itemize}

We may be tempted to assume that a language extension is needed.
%Instead, with object interfaces, another approach is possible
Instead, the \emph{restriction} of (object) interfaces to have no
fields enlightens us that another approach is possible; often in programming languages ``freedom is slavery''.

\para{Field type refinement with object interfaces}
Object interfaces address the challenge of type-refinement as follows:
\begin{itemize}
\item by \emph{covariant method overriding}, the return type of
  \texttt{location()} is refined to \texttt{Point3D};
\item by \emph{overloading}, a new setter for location is defined with a more
  precise type;
\item a \Q@default@ setter implementation with the old signature is provided.
\end{itemize}

\noindent Thus the code for the \Q@Bird@ interface is:

\lstinputlisting[linerange=70-79]{../UseMixinLombok/src/pegasus/TestAnimal.java}% APPLY:linerange=BIRD
%Interface \texttt{Point3D} extends \texttt{Point2D} with a new abstract method
%\texttt{int z()} (treated as a getter for member field \texttt{z}). Note that
%the return type of various methods (e.g. with- methods, getters) get refined
%either by covariant method overriding or automatically by our annotation
%processor. Besides \emph{with}, other methods (including \emph{clone},
%\emph{of}) also do type-refinements automatically.


\noindent From the type perspective, the key is the covariant method
overriding of \texttt{location()}. However, from the semantics
perspective the key is the implementation for the setter with the old
signature (\Q@location(Point2D)@). The key to the setter
implementation is a new type of \Q@with@ method, called
 a (functional) property updater.

\para{\Q@Point3D@ and property updaters}
The \Q@Point3D@ interface is defined as follows:

\lstinputlisting[linerange=52-55]{../UseMixinLombok/src/pegasus/TestAnimal.java}% APPLY:linerange=POINT3D

\noindent \Q@Point3D@ includes a
\Q@with@ method, taking a \Q@Point2D@ as an argument.
Other wither methods (such as \Q@withX@) functionally update a field one at a time.  This can be
inefficient, and sometimes hard to maintain.  Often we want to update multiple
fields simultaneously, for example using another object as source.  Following
this idea, the method \Q@with(Point2D)@ is an example of a (functional)
property updater: it takes a certain type of object and returns a copy of the
current object where all the fields that match fields in the parameter
object are updated to the corresponding value in the parameter. The idea is that
the result should be like \Q@this@, but modified to be as similar as possible to the parameter.

With the new \Q@with@ method we may use the information for
\Q@z@ already stored in the object to forge an appropriate \Q@Point3D@
to store. Note how all the information about what fields sit in
\Q@Point3D@ and \Q@Point2D@ is properly encapsulated in the
\Q@with@ method, and is transparent to the implementer of \Q@Bird@.

Property updaters never break class invariants, since they
internally call operations that were already deemed
safe by the programmer. For example a list object
would not offer a setter for its \texttt{size} field (which should be kept hidden), thus
a property updater would not attempt to set it.

% To implement the old setter in a convenient way, \mixin supports one
% last type of operations: property updater \texttt{with}
% methods. Unlike the \texttt{withX} (where \texttt{X} stands for a
% field name) methods presented so far, property updaters take several
% fields at once, contained in an interface, and copy those fields into
% fields of another interface.

%Symmetrically, we could offer an imperative property updater that
%calls the setters instead of the withers.  \Q@Point3D set(Point2D
%val)@.

\begin{figure}
\saveSpaceFig
\lstinputlisting[linerange=88-114]{../UseMixinLombok/src/pegasus/TestAnimal.java}% APPLY:linerange=GENERATED_POINT3D
\caption{Generated boilerplate code.}
\label{fig:boilerplate}
\saveSpaceFig
\end{figure}

\para{Generated boilerplate}
To give an idea of how much code \mixin is generating, we show the
generated code for \texttt{Point3D} in Figure~\ref{fig:boilerplate}.
%\marcoT{Overall, for the whole
%animals-with-locations, an @@@ lines code example,
%the generate/completed code is composed by
%@@@ lines.}
Writing such code by hand is error-prone. For
example a distracted programmer may swap the arguments of calls to
\Q@Point3D.of@.  Note how \Q@with-@ methods are automatically refined in their
return types, so that code like:

\begin{lstlisting}
Point3D p = Point3D.of(1,2,3); p = p.withX(42);
\end{lstlisting}

\noindent will be accepted. If the programmer wishes to suppress this behavior
and keep the signature as it was, it is sufficient to redefine the \Q@with-@
methods in the new interface repeating the old signature.  Again, the philosophy
is that if the programmer provides something directly, \mixin does not touch it.
The cast in \Q@with(Point2D)@ is trivially safe because of the \Q@instanceof@
test. The idea is that if the parameter is a subtype of the current exact type,
then we can just return the parameter, as something that is just ``more'' than
\Q@this@.

\begin{figure*}
\saveSpaceFig
\centering
\hspace{.01in}{%\fontsize{8}{10}\selectfont
\begin{tabular}{|l|l|l|l|}
\hline
& \textbf{Operation}  & \textbf{Example}                  & \textbf{Description } \\ \hline
\multirow{4}{*}{\parbox{2.3cm}{State operations (for a field \texttt{x})}} & \textbf{``fields''/getters}        &   \Q@int x()@                  & Retrieves value from field \texttt{x}.          \\ \cline{2-4}
& {\bf withers}        &   \Q@Point2D withX(int val)@                & Clones
object; updates field \texttt{x} to \texttt{val}.             \\ \cline{2-4}
& \textbf{setters}        & \Q@void x(int val)@ & Sets the field
\texttt{x} to a  new value \texttt{val}.        \\ \cline{2-4}
& \textbf{fluent setters}        & \Q@Point2D x(int val)@ &Sets the field
\texttt{x} to \texttt{val} and returns \texttt{\this}.  \hspace{-.1in}         \\ \hline
\multirow{3}{*}{Other operations} &
\textbf{factory methods} &
%\begin{tabular}{l}
\Q@static Point2D of(int _x,int _y)@
%\end{tabular}
 & Factory method (generated).        \\
\cline{2-4}
& \textbf{functional updaters}        & \Q@Point3D with(Point2D val)@
& Updates all matching fields in \texttt{val}.        \\ \hline
%\cline{2-4}
%& \textbf{imperative updaters}        & \Q@Point3D set(Point2D val)@ &
%Set all matching fields in \texttt{val}.        \\ \hline
\end{tabular}
}
\caption{Abstract state operations for a field \texttt{x}, together with other operations, supported by the \mixin
  annotation. }

\label{fig:abstractstate}

\end{figure*}

\para{Summary of operations in Classless Java}
%Method bodies can only refer to other methods;
%Objects are (conceptually) closures and the state is
%composed by captured local variables.}
%Methods interacting with state are called \emph{abstract state operations}.
In summary, object interfaces provide support for
different types of abstract state operations: four field-based state
operations; and functional updaters. Object instantiation is directly
supported by \texttt{of} factory methods.
Figure~\ref{fig:abstractstate} summarizes the six operations supported
by \mixin. The field-based abstract state operations are determined by
naming conventions and the types of the methods. Fluent setters are a variant of
conventional setters, and are discussed in more detail in Section~\ref{sec:dsls}.

\begin{comment}
\textcolor{red}{Note that we are breaking the naming conventions for getters and setters in Java;
they have the same names as their fields in our case, which makes code more concise,
but also presents a nice form of fluent setters with method chaining (see Section~\ref{sec:dsls}).
Fluent setters are useful, and the difference from \Q@with-@ methods is that they do not create new objects.}
\end{comment}

\subsection{Advanced Multiple Inheritance}
Finally, defining \texttt{Pegasus} is as simple as we did in the simplified
(and stateless) version on the right of Figure~\ref{fig:pegasus}.
 Note how even the non-trivial pattern for field type refinement is
transparently composed, and \texttt{Pegasus} has a \Q@Point3D@
\Q@location@.%  This works because \Q@Horse@ do not perform
% any field type refinement, otherwise we may have to choose/create a
% common subtype in order for \texttt{Pegasus} to exists.

\lstinputlisting[linerange=83-83]{../UseMixinLombok/src/pegasus/TestAnimal.java}% APPLY:linerange=PEGASUS


\begin{comment}
\subsection{A Running Example: \texttt{Point}}
Suppose we want to create a point component that models the a point in space,
that has x-coordinate and y-coordinate. For example, if we create the
\texttt{Point} interface in Java, it would look like this:

\begin{lstlisting}
interface Point {
    int x();
    int y();
}
\end{lstlisting}

\texttt{Point} has two (conceptually) member fields \texttt{x} and \texttt{y},
representing the two coordinates of a point. Methods \texttt{int x()} and
\texttt{int y()} serve as \emph{getter} methods.
% Methods \texttt{void X(int X)} and \texttt{void Y(int Y)} serve as
% \emph{setter} methods. Method \texttt{Point withX(int X)} updates field
% \texttt{X} and returns \textbf{this}.

\subsection{Naive Implementation}
In Java, to implement an interface like \texttt{Point}, a typical and trivial
approach that programmers usually do is creating a class extending the interface
and providing implementation for all methods inside. For example, this is the
implementation for interface \texttt{Point}:

\lstinputlisting[linerange=-]{} % APPLY:linerange=POINTIMPL

\texttt{PointImpl} implements \texttt{Point} and provides a constructor with
quite mechanical code. What's worse, the implementation in \texttt{PointImpl}
may not be reused in a single inheritance language.

\subsection{The Classless Java Approach}
Instead of writing a whole another class to provide the implementation for
\texttt{Point}, we annotate on interface \texttt{Point} directly with \mixin:

\lstinputlisting[linerange=-]{} % APPLY:linerange=POINT

The \mixin annotation will generate a static method \texttt{of} inside
\texttt{Point}. The method \texttt{of} mimic the functionality of constructors,
it takes arguments same as constructors and return objects similar to
constructors. It makes use of Java anonymous classes and achieves the same
implementation as \texttt{PointImpl}.

With \CJ, we provide a Java annotation \mixin to provide default
implementations for various methods and a mechanism to instantiate
objects. \mixin annotation helps programmers to write less cumbersome code and
instantiate interfaces in Java.

\lstinputlisting[linerange=-]{} % APPLY:linerange=POINT_OF
\end{comment}


\begin{comment}
\subsection{More On \mixin}
Besides the benefit of freeing programmers from writing boilerplate code, our
\mixin annotation can also allow programs to mimic multiple inheritance in a
restricted form easily.

\lstinputlisting[linerange=42-48]{../UseMixinLombok/src/pegasus/TestAnimal.java}% APPLY:linerange=POINT2D

\lstinputlisting[linerange=52-55]{../UseMixinLombok/src/pegasus/TestAnimal.java}% APPLY:linerange=POINT3D

\lstinputlisting[linerange=58-60]{../UseMixinLombok/src/pegasus/TestAnimal.java}% APPLY:linerange=ANIMAL

\lstinputlisting[linerange=64-66]{../UseMixinLombok/src/pegasus/TestAnimal.java}% APPLY:linerange=HORSE

\lstinputlisting[linerange=70-79]{../UseMixinLombok/src/pegasus/TestAnimal.java}% APPLY:linerange=BIRD

\lstinputlisting[linerange=83-83]{../UseMixinLombok/src/pegasus/TestAnimal.java}% APPLY:linerange=PEGASUS

Interface \texttt{Point3D} extends \texttt{Point/Point2D} with a new abstract
method \texttt{int z()} (treated as a getter for member field
\texttt{z}). Interface \texttt{Horse} shows the usage and advantage of
\emph{with} methods by method \texttt{run()}: method \texttt{withX} returns a
new point object with field \texttt{x} updated by the argument to
\texttt{withX}. Without these \texttt{with} methods, operations like
\texttt{run()} would be much harder to define. Note that the return type of
various methods get refined automatically by our annotation processor. Besides
\emph{with}, other methods (including \emph{clone}, \emph{of}) also do
type-refinements automatically.

The \emph{``multiple inheritance''} case appears at interface
\texttt{Pegasus}. Pegasuses can not only \emph{run} but also \emph{fly}!
Interface \texttt{Pegasus} obtains \texttt{fly} and \texttt{run} functionality
through interface \texttt{Horse} and \texttt{Bird}. Using \mixin annotation,
actually there is no code that programmers have to write at all. The idea of using
default methods inside interfaces was proposed in ~\cite{}. It enables us to do
multiple inheritance, which otherwise is hard to do in Java-like languages that
do not support multiple inheritance, easily.
\end{comment}



\section{Formal Semantics}\label{sec:formal}
\subsection{Grammar}

\marco{Tentative Grammar ClassLessJava}

\begin{figure}[h]
\begin{grammar}
\production{
\e
}{
  \x\mid\MCall\e\m\es\mid\MCall{\C}\m\es\mid\MCall{\C\QM{.super}}\m\es\mid\obj
  }{expressions}\\
\production{
\obj
}{
\QM{new}\ \C\oR\cR\oC\T_1\ \f_1\QM;\ldots\T_k\ \f_k\QM;\
\mh_1\oC\QM{return}\ \e_1\QM{;}\!\cC
\ldots
\mh_n\oC\QM{return}\ \e_n\QM{;}\!\cC
\cC
  }{object creation}\\
\production{
\metaVar{I}
}{
 \ann\ \QM{interface}\ \C_0\ \QM{extends}\ \C_1\ldots\C_k \oC\method_1\ldots\method_n\cC
  }{interface declaration}\\
\production{
\method
}{
 \QM{static}\ \mh\ \oC\QM{return}\ \e\QM{;}\!\cC
\mid
\QM{default}\ \mh\oC\QM{return}\ \e\QM{;}\cC
\mid
\mh\QM{;}
  }{method declaration}\\
\production{
\mh
}{
 \T_0\ \m\ \oR\T_1\ \x_1\ldots\T_n\ \x_n\cR
  }{method header}\\
\production{
\ann
}{
  \mixinAnn|\emptyset
  }{annotations}\\
\end{grammar}
\caption{Grammar of ClassLess Java}
\label{Grammar}
\end{figure}

In Figure~\ref{Grammar} we show the syntax of ClassLess Java.

To be compatible with java, the concrete syntax for an interface declaration with empty supertype list $\C_1\ldots\C_k$ would also omit the \Q@extends@ keyword.

Yanlin and Haoyuan

We need to show 2 things:

1) The dynamic semantics: what's the code that gets generated by a mixin annotation;

2) The type system: what programs to reject; properties: generation of type-safe/checkable code.

\bruno{The implementation is still missing the type system (rejecting some
  programs)!}

\footnote{Future work: updating multiple fields in one method call,
  \texttt{with(T v)}}


\subsection{Typing Rules}
\subsubsection{Expression Typing}

\[
\inferrule{}{\Gamma \vdash x \in \Gamma(x)} \quad \textsc{(T-Var)}
\]

\[
\inferrule{\Gamma \vdash e_0 \in C_0 \\
  \textsf{mtype}(m,C_0) = \overline{D} \to E \\
  \Gamma \vdash \overline{e} \in \overline{C} \\
  \overline{C} <: \overline{D} }
{ \Gamma \vdash e_0.m(\overline{e}) \in E }
\quad \textsc{(T-Invk)}
\]

\[
\inferrule{\textsf{mtype}(m,C) = \overline{D} \to E \\
\Gamma \vdash \overline{e} \in \overline{C} \\
\overline{C} <: \overline{D} \\
\textsf{mmodifier}(m,C) = \textbf{static} }
{\Gamma \vdash C.m(\overline{e}) \in E}
\quad \textsc{(T-StaticInvk)}
\]

\[
\inferrule{\textsf{mtype}(m,C) = \overline{D} \to E \\
\Gamma \vdash \overline{e} \in \overline{C} \\
\overline{C} <: \overline{D} \\
\textsf{checkSuper}(C.super.m(\overline{e}), A, C) \\\\
\text{\yanlin{check whether A <: C, where A is the enclosing interface of the
    expression}} }
{\Gamma \vdash C.\textbf{super}.m(\overline{e}) \in E}
\quad \textsc{(T-SuperInvk)}
\]


\subsubsection{Method Typing} 
\[ 
\inferrule 
{ }
{T_0 \spc m(\overline{T} \spc \overline{x}); \text{ OK IN I} }
\quad \textsc{(T-Meth)}
\]

\[ 
\inferrule
{\overline{x}:\overline{T} \vdash e:S \\ S <: T_0}
{T_0 \spc m(\overline{T} \spc \overline{x}) \text{ \{ return } e;\} \text{ OK IN
    I} \\\\ \Gamma \vdash \textbf{this}:I }
\quad \textsc{(T-MethBody)}
\]

\[ 
\inferrule
{IT(I)=\text{interface } I \text{ extends } \overline{J} \text{\{...\}} \\
\forall i,\text{if \textsf{mtype}}(m,J_i) = \overline(T) \to U_0, \text{then }
T_0 <: U_0 }
{T_0 \spc m(\overline{T} \spc \overline{x}); \text{ OK IN I} }
\quad \textsc{(T-MethExt)}
\]

\[ 
\inferrule
{\overline{x}:\overline{T} \vdash e:S \\ S <: T_0 \\ 
IT(I)=\text{interface } I \text{ extends } \overline{J} \text{\{...\}} \\
\forall i,\text{if \textsf{mtype}}(m,J_i) = \overline(T) \to U_0, \text{then }
T_0 <: U_0 }
{T_0 \spc m(\overline{T} \spc \overline{x}) \text{ \{ return } e;\} \text{ OK IN
    I} \\\\ \Gamma \vdash \textbf{this}:I }
\quad \textsc{(T-MethBodyExt)}
\]

\subsubsection{Context}

\[ \Gamma = \epsilon \mid (x, C) : \Gamma  \]

\subsubsection{Subtyping}

\[ \inferrule{}{T <: T} \]

\[ \inferrule{S <: T \\ T <: U}{S <: U}\]

\[ \inferrule{\emph{ann} \spc \textbf{interface} \spc C_0 \spc \textbf{extends} \spc C_1,...,C_k \{...\}}
{C_0 <: C_1 \\ ... \\ C_0 <: C_k} \]

\subsubsection{Interface Table}
The Interface Table (IT) mapping from types \texttt{T} to interface declarations \texttt{L}. From the
interface table, we can read off the subtype relation between interfaces. The
subtype relation is given by the \textbf{extends} clauses in \texttt{IT}.


\subsection{Auxiliary Definitions}
\yanlin{todo: add more auxiliary functions}

\yanlin{todo: add interface context}

\begin{comment}
\subsubsection{Auxiliary function: \textsf{mtype}}
- \textsf{mtype(m, C)} : the signature of method m in C.

\[ \inferrule{
  IT(T) = \text{\emph{ann} interface } C \{ \overline{M} \} \\
  E \spc m(\overline{D} \spc \overline{x}) \{ \text{return } e; \} \in M}
{ \textsf{mtype(m,T)} = \overline{D} \to E } \]

\[ \inferrule{
  IT(T) = \text{\emph{ann} interface } C \{ \overline{M} \} \\
  m \notin M}
{ \textsf{mtype(m,T)} = \emptyset } \]

\[ \inferrule{
  IT(T) = \text{\emph{ann} interface } C \text{ extends } C_1,...,C_k \{ \overline{M} \} \\
  E \spc m(\overline{D} \spc \overline{x}) \{ \text{return } e; \} \in M}
{ \textsf{mtype(m,T)} = \overline{D} \to E } \]

\[ \inferrule{
  IT(T) = \text{\emph{ann} interface } C_0 \text{ extends } \overline{C} \{
  \overline{M} \} \\
  m \notin M}
{ \textsf{mtype(m,T)} = \bigcup \textsf{mtype}(m,\overline{D}) } \]
\end{comment}

\subsubsection{Auxiliary function: \textsf{mbody}}

$\textsf{mbody}(m,C)$ denotes the actual body of method $m$ that interface $C$ owns. It can either be defined originally in $C$ or in its supertypes, and then passed to $C$ via inheritance.

The body of a method $m$ contains the relevant information with respect to that method, like the type of $m$ as well as the modifier. Below shows how the functions $mtype$ and $mmodifier$ are derived from $mbody$.

\[ \textsf{mbody}(m,C) = \textit{modifier } E \spc m(\overline{D} \spc \overline{x}) \{ \text{return } e; \} \] \[ \Rightarrow \textsf{mtype}(m,C) = \overline{D} \to E,\ \textsf{mmodifier}(m,C) = \textit{modifier}\]

\[ \inferrule{
  IT(C) = \text{\emph{ann} interface } C \{ \overline{M} \} \\
  m \notin M}
{ \textsf{mbody}(m,C) = \emptyset } \]

\[ \inferrule{
  IT(C) = \text{\emph{ann} interface } C \{ \overline{M} \} \\
  m \in M}
{ \textsf{mbody}(m,C) = m } \]

\[ \inferrule{
  IT(C_0) = \text{\emph{ann} interface } C_0 \text{ extends } \overline{C} \{
  \overline{M} \} \\
  m \notin M
  }
{ \textsf{mbody}(m,C_0) = \textsf{fold}_\textsf{shadow}\cdot\textsf{map}_{\textsf{mbody,}m}\cdot\textsf{subset}(\overline{C}) } \]

\[ \inferrule[(ReviewedRule)]{
  IT(C_0) = \text{\emph{ann} interface } C_0 \text{ extends } \overline{C} \{
  \methods \} \\
  \methods(\m)=\method
\\
\Aux{tops}(\overline{C})=\C_1\ldots\C_n  
  }
{ \mBody(\m,\C_0) = \textsf{replace}(\method, \shadow(\mBody(\m,C_1)\ldots\mBody(\m,C_n))
 } \]
\subsubsection{Auxiliary function: \Aux{tops}}

\[ \textsf{tops}(\Cs)=\{\ \C\in\Cs\ |\ \nexists \C'\in\Cs\setminus\C,\ \C' \subtype \C\ \} \]

\subsubsection{Auxiliary function: \textsf{shadow}}



The \textsf{shadow} function takes two same methods (with the same name and types of arguments), and return the method which shadows the other during inheritance.
%exmples to motivate our design
%interface A{static String m(){return "A";}}
%interface C extends A{
%	default String dm(){
%		  this.m();//wrong in java
%		  A.m();
%		  C.m();//wrong in java
%		}
%}
%
%
%(1) Static methods are not inherited. Also, if one of $\{body_1,body_2\}$ is null, \textsf{shadow} simply returns the other one. Hence
\begin{equation*}
\begin{array}{ll}
\shadow()=\emptyset\\
\shadow(\methods)=\method &\mif\ \method \in \methods, \forall \method' \in \methods\setminus\method :  \method' \subtype  \method,\\
\mbox{ otherwise}\\
\shadow(\methods)=\conflicted\mh&\mif\ \shadow(\Aux{removeBody}(\methods))=\mh\\
\Aux{removeBody}(\QM{default}\ \mh\mbox{\Q@\{return \_;\}@})
=\mh\QM;\\
\Aux{removeBody}(\mh\QM)
=\mh\QM;
\\
%\conflictError \shadow \_\\
\_\subtype\emptyset\\
%\mh\mbox{\Q@\{return \_;\}@}\shadow \mh & \mif\ \mh\shadow\mh'\\
\_\ \mh\mbox{\Q@\{return \_;\}@}\subtype \mh & \mif\ \mh\subtype\mh'\\

\T \m\oR\T_1\x_1\ldots \T_n\x_n\cR \subtype \T' \m\oR\T_1\x_1'\ldots\T_n\x_n'\cR & \mif\ \T\subtype \T'\\
\Aux{replace}(\method,\emptyset)=
\Aux{replace}(\emptyset,\method)=\method\\
\Aux{replace}(\method,\_\ \mh\mbox{\Q@\{return \_;\}@})
=

\\
%\textsf{shadow}(body_1, body_2)=\emptyset & \textsf{if }body_1.\textsf{modifier}=body_2.\textsf{modifier}=\textbf{static}\\
%\textsf{shadow}(body_1, body_2)=body_1 & \textsf{if }body_2=\emptyset\textsf{ or }body_2.\textsf{modifier}=\textbf{static}\\
%\textsf{shadow}(body_1, body_2)=body_2 \hspace{.1in}  & \textsf{if }body_1=\emptyset\textsf{ or }body_1.\textsf{modifier}=\textbf{static}
\end{array}
\end{equation*}

%(2) If $body_1.\textsf{returnType}=body_2.\textsf{returnType}$, \textsf{shadow} tends to return a default method. If both $body_1$ and $body_2$ are default methods, \textsf{shadow} throws an error.
%\begin{equation*}
%\begin{array}{ll}
%\textsf{shadow}(body_1, body_2)=\textsf{ERROR} & \textsf{if }body_1.\textsf{modifier}=body_2.\textsf{modifier}=\textbf{default}\\
%\textsf{shadow}(body_1, body_2)=body_1 \hspace{.1in} & \textsf{if }body_1.\textsf{modifier}=\textbf{default} \\
%\textsf{shadow}(body_1, body_2)=body_2 \hspace{.1in} & \textsf{if }body_2.\textsf{modifier}=\textbf{default} \\
%\textsf{shadow}(body_1, body_2)=body_1\textsf{ (or }body_2\textsf{)} \hspace{.1in} & \textsf{otherwise}
%\end{array}
%\end{equation*}
%
%(3) If $body_1.\textsf{returnType}<:body_2.\textsf{returnType}$, \textsf{shadow} tends to choose the one with the subtype (namely $body_1$), but only when both methods are abstract, otherwise it gives an error. The other direction $body_2.\textsf{returnType}<:body_1.\textsf{returnType}$ follows the same rule. It also gives an error if there is no subtyping relationship between two return types.
%\begin{equation*}
%\begin{array}{ll}
%\textsf{shadow}(body_1, body_2)=body_1 & \textsf{if }body_1.\textsf{modifier}=body_2.\textsf{modifier}=\emptyset\\
%& \textsf{and }body_1.\textsf{returnType}<:body_2.\textsf{returnType}\\
%\textsf{shadow}(body_1, body_2)=body_2 & \textsf{if }body_1.\textsf{modifier}=body_2.\textsf{modifier}=\emptyset\\
%& \textsf{and }body_2.\textsf{returnType}<:body_1.\textsf{returnType}\\
%\textsf{shadow}(body_1, body_2)=\textsf{ERROR} \hspace{.1in} & \textsf{otherwise}
%\end{array}
%\end{equation*}

\subsubsection{Auxiliary function: \textsf{replace}}

The \textsf{replace} function takes two same methods (with the same name and types of arguments), and gives the result of the first method overriding the second one.

\begin{equation*}
\begin{array}{ll}
\textsf{replace}(body_1, body_2)=body_1 & \textsf{if }body_2=\emptyset\\
\textsf{replace}(body_1, body_2)=body_2 & \textsf{if }body_1=\emptyset\\
\textsf{replace}(body_1, body_2)=body_1 & \textsf{if }body_1.\textsf{returnType}<:body_2.\textsf{returnType}\\
\textsf{replace}(body_1, body_2)=\textsf{ERROR} \hspace{.1in} & \textsf{otherwise}
\end{array}
\end{equation*}




\section{What  \mixin Generates}\label{sec:translation}

We now show what the \mixin annotation generates. We present a formal definition for
most of the generated methods; however in our formalism we do not consider
% setters, so we do not formally define the generation of the setters.
%We do not include
casts or \Q@instanceof@, so we do not include the \Q@with@ method.
For the same reason we do not include \Q@void@ returning setters, since they are just a minor variation over the more interesting fluent setters, and they would require special handling just for the conventional \Q@void@ type.

\subsection{Translation Function}${}_{}$\\*

\noindent$[\![\mixinAnn\ \QM{interface}\ \C_0\ \QM{extends}\ \Cs\ \oC \methods\ \cC ]\!] =
\emptyset\ \QM{interface}\ \C_0\ \QM{extends}\ \Cs\ \oC
\methods\ \methods' \cC
$\\*${}_{}$\tab
where  $\valid(\C_0)$,\Q@of@$\notin\dom(\methods)$ and $\methods'=\ofMethod(\C_0) \ \otherMethod(\C_0,\methods)$.\\*

To translate an annotated interface, we add the \Q@of@ method, and then we add some other methods.
However, first of all we check if the interface is valid for annotation:

\noindent$\valid(\C_0)$  holds if $\forall \m\in\dom(\C_0),$ if $\mh\QM; = \mBody(\m, \C_0),$ one case is satisfied:
$\isField(\method)$,
$\isWith(\method, \C_0)$
or
$\isSetter(\method,\C_0)$
%$\isClone(\method, \C_0)$.

That is, we can categorize all the \emph{not implemented} methods in a pattern that we know how to implement.

%To complete our formal definition we would need to add setters and \Q@with@.
Moreover, we check that the method \Q@of@ is not already defined by the user.
In our simplified formalization we consider this to be just an error.
In our prototype we keep \Q@overloading@ into account, and so we check that an of method with the same signature of the one we would like to generate is not already present.\marco{Do we check it?}


In the following we will write $\QM{with#}\m$ to append $\m$ to \QM{with}, following the camelCase rule, so the first letter of
$\m$ must be lower-case and is turned in upper-case upon merging.
For example \QM{with#foo}=\QM{withFoo}.
Special names $\specialName(\m)$ are  \QM{with} and all the identifiers of form $\QM{with#}\m$.

\subsection{$\ofMethod$}${}_{}$\\*
We now formally define $\ofMethod$, the function that generates the method \QM{of}, that behaves like a factory. To avoid boring digressions about well known ways to find unique names, for the sake of this formalization we assume that no-args methods do not start with underscore, and we prefix method names with underscore to obtain valid  parameter names.\\*
\noindent$\begin{array}{l}
\ofMethod(\C_0) = \
 \QM{static}\ \C_0\ \QM{of} \oR \C_1\ \QM_\m_1\QM,\ldots \C_n\ \QM_\m_n\cR\
\QM{\{}
\QM{return new}\ \C_0 \oR\cR\ \QM{\{} \\
\tab\tab\tab\tab\tab\tab\tab \C_1\ \m_1 = \QM_\m_1\QM;\ldots \C_n\ \m_n = \QM_\m_n\QM; \\
\tab\tab\tab\tab\tab\tab\tab
\C_1\ \m_1\oR\cR\ \QM{\{return }\ \m_1\QM{;\}}\ \ldots
\C_n\ \m_n\oR\cR\ \QM{\{return }\ \m_n\QM{;\}}\\
\tab\tab\tab\tab\tab\tab\tab\withMethod(\C_1,\m_1,\C_0,\es_1)\ldots\withMethod(\C_n,\m_n,\C_0,\es_n)\\
\tab\tab\tab\tab\tab\tab\tab\setterMethod(\C_1,\m_1,\C_0)\ldots\setterMethod(\C_n,\m_n,\C_0)\\
%\tab\tab\tab\tab\tab\tab\tab\cloneMethod(\C_0,\es)\\
\tab\tab\tab\tab\tab\tab\tab\withMethod(\C_0)\\
\tab\tab\tab\tab\tab\tab\QM{\};\}} \\
\end{array}$
\\*
with $\fieldsFunc(\C_0)=\C_1\ \m_1\QM{();},\ldots \C_n\ \m_n\QM{();}$,\\*
and $\es_i=\m_1\QM,\ldots\QM, \m_{i-1}\QM,\QM{_val,}\m_{i+1}\QM,\ldots\QM, \m_n$\\*
%and $\es=\m_1\QM,\ldots\QM, \m_n$\\*
\haoyuan{Should we include the $\withMethod(\C_0)$ in the formalization?}

The function $\fieldsFunc(\C_0)$ (formally defined later) denotes all the fields in the current interface.
For methods inside the interface with the form $\C_i\ \m_i$\QM{();}
  \begin{itemize}
   \item $\m_i$ is the field name, and have type $\C_i$.
   \item $\m_i$\QM{()} is the getter, that just return the current field value.
   \item if a method \Q@with#@$\m_i$ is required, then it is implemented by calling the \Q@of@ method using
    the current value for all the fields except for $\m_i$. Such new value is provided as parameter. This correspond to the expressions $\es_i$.
\item \QM_$\m_i$\QM($\C_i\ $\QM{ _val)} is the setter. In our prototype we use name $\m_i$, here we use the underscore to avoid modelling overloading.
%   \item similarly, for the \Q@clone@ method, \Q@of@ is called using the current value for all the fields.
   %\item To complete our generation, we need to generate setters, fluent setters and the with method.
   %\item \marco{should we just formalize setters?}
   \end{itemize}

\subsection{Other auxiliary functions}${}_{}$\\*


\noindent$\begin{array}{ll}
\withMethod:&\withMethod(\C,\m,\C_0,\es)=
\C_0\ \QM{with#}\m\oR \C\ \QM{_val}\cR\ \QM{\{}
\QM{return} \C_0\QM{.of(}\es\QM{);\}} \\
&\mbox{iff }
\mBody(\QM{with#}\m,\C_0) \mbox{ is of form }\mh\QM;\\
&\withMethod(\C,\m,\C_0,\es)=\emptyset\mbox{ otherwise}\\
\setterMethod:&\setterMethod(\C,\m,\C_0)=
\C_0\ \QM_\m\oR \C\ \QM{_val}\cR\ \QM{\{}
 \m\QM{= _val;return this;\}} \\
&\mbox{iff }
\mBody(\QM_\m,\C_0) \mbox{ is of form }\mh\QM;\\
&\setterMethod(\C,\m,\C_0)=\emptyset\mbox{ otherwise}\\
%\cloneMethod:&\cloneMethod(\C_0,\es)=
%\C_0\ \QM{clone()\{return}\ \C_0\QM{.of(}\es\QM{);\}} \\
%&\mbox{iff }
%\mBody(\QM{clone},\C_0) \mbox{ is of form }\mh\QM;\\
%&\cloneMethod(\C_0,\es)=\emptyset\mbox{ otherwise}\\
\end{array}$

\haoyuan{method name of setter? $m$ or $\_m$?}

As you can see above, \Q@with-@ and setter methods are generated if needed.
We can discover if there is the need of generating such methods by checking if the method is unimplemented in $\C_0$. Note that we do not need to check if its header is a subtype of what we would generate, this is ensured by $\valid(\C_0)$.



\noindent$\begin{array}{ll}
\otherMethod:& \C_0\ \QM{with#}\m\oR \C\ \QM{_val}\cR\QM;\in
\otherMethod(\C_0,\methods)
\\&
 \mbox{iff }
\C\ \m\QM{();}\in \fieldsFunc(\C_0), \isWith(\mBody(\QM{with#}\m, \C_0))
\\&\mbox{ and } \QM{with#}\m\notin\dom(\methods)\\
& \C_0\ \QM_\m\oR \C\ \QM{_val}\cR\QM;\in
\otherMethod(\C_0,\methods)
\\&
 \mbox{iff }
\C\ \m\QM{();}\in \fieldsFunc(\C_0), \isSetter(\mBody(\QM_\m, \C_0))
\\&\mbox{ and } \QM_\m\notin\dom(\methods)\\
%&\C_0\ \QM{clone}\oR\cR\QM;\in
%\otherMethod(\C_0)
%\\&  \mbox{ iff }
%\isClone(\mBody(\QM{clone}, \C_0))
%\mbox{ and } \QM{clone}\notin\dom(\methods)\\
\end{array}$

Other methods that we need to generate in the interface are \Q@with-@ and setters. %\Q@clone@.
%A complete formalization would also generate the \Q@with@.
This is needed only if we need to refine the return type.
To discover if this is the case, we check if such \Q@with-@ or setter %\Q@clone@
 is required by $\C_0$, but is not already present in the methods directly declared in $\C_0$.

\noindent$\begin{array}{ll}
\fieldsFunc:&\method\in\fieldsFunc(\C_0) \mbox{ iff }
\method\in \dom(\C_0)\ \mand\ \isField(\method)
\\
\isField:&\isField(\C\ \m\oR\cR\QM;)\tab \mif\ \mnot\ \specialName(\m)\\
\isWith:&\isWith(\C'\ \QM{with#}\m \oR \C\ \x\cR\QM;, \C_0)
\tab \mif\ \C_0 <: \C', \mBody(\m, \C_0) = \C\ \m\oR\cR\QM;\\
& \mand\ \mnot\ \specialName(\m)\\
\isSetter:&\isSetter(\C'\ \QM_\m \oR \C\ \x\cR\QM;, \C_0)
\tab \mif\ \C_0 <: \C', \mBody(\m, \C_0) = \C\ \m\oR\cR\QM;\\
& \mand\ \mnot\ \specialName(\m)\\

%\isClone:&\isClone(\C\ \QM{clone}\oR\cR\QM;, \C_0)\tab \mif\ \C_0 <: \C \\
%\isImplemented:&\isImplemented(\method) \tab\mbox{iff }\method\mbox{ not of form }\mh\QM;
%\QM{default}\ \mh\mbox{\Q@\{return \_;\}@}) = \QM{true} \\
%&\isImplemented(\QM{static}\ \mh\mbox{\Q@\{return \_;\}@}) = \QM{true} \\
\end{array}$

We have not formally modelled non fluent setters and the \Q@with@ method; informally
\begin{itemize}
\item For methods inside the interface with the form \Q@void @$\m$\QM($\C\ \x$\QM{);}:
  \begin{itemize}
    \item Check if exist method $\C\ \m$\Q@();@. If not, generate error (that is, is not $\valid(\C_0)$).
    \item Generate implemented setter method inside \Q@of@:\\*
           \Q@public void @$\m$\Q@(@$\C$\Q@ _val) { @$\m$\Q@=_val;}@
    Note how there is no need to refine the return type for non fluent setters, thus we do not need to generate the method header in the interface body itself.
    \end{itemize}
\item For methods with the form $\C'\ $\QM{with(}$\C\ \x$\QM{);}:
  \begin{itemize}
    \item As for before, check that $\C'$ is a supertype of the current interface type $\C_0$.
    \item Generate implemented \Q@with@ method inside \Q@of@:\\*
           \Q@public @$\C_0\ $\Q@with(@$\C$\Q@ _val) { @\\*
           \Q@  if(_val instanceof @$\C_0$\Q@){return (@$\C_0$\Q@)_val;}@\\*
${}_{}$\Q@  return @$\C_0$\Q@.of(@$\e_1\ldots\e_n$\Q@);}@\\*
where with $\m_1\ldots\m_n$  fields of $\C_0$,
$\e_i=$\Q@_val.@$\m_i$\Q@()@ if $\C$ has a $\m_i$\Q@()@ method; otherwise
$\e_i=\m_i$.
    \item If needed, as for \Q@with-@ and setters, generate the method header with refined return type in the interface.
 \end{itemize}

%\item For methods with the form $\C'\ \m$\QM($\C\ \x$\QM{);}:
 % \begin{itemize}
  %  \item As for before, check if exist method $\C\ \m$\Q@();@. Also, check that $\C'$ is a supertype of the current interface type $\C_0$.
   % \item Generate implemented setter method inside \Q@of@:\\*
    %       \Q@public @$\C_0\ \m$\Q@(@$\C$\Q@ _val) { @$\m$\Q@=_val; return this;}@
   % \item If needed, as for \Q@with-@ and clone, generate the method header with refined return type in the interface.
 % \end{itemize}
\end{itemize}

\marco{ insert somewhere description of  fluent setter~\cite{the lombock thread, something about fluent stuff}.
This allows for convenient and chains of setters, as we will show later \marco{insert forward reference when available}}.




\subsection{Results}
\textbf{THEOREM. }
For a given $\II_0\ldots\II_n$ interface table such that
$\forall\II\in\II_0\ldots\II_n, \II$ OK, then in the interface table
$[\![\II_0]\!]\II_1\ldots\II_n$
$\forall\II\in[\![\II_0]\!]\II_1\ldots\II_n$ either $\II$ OK or $\II$ is a subtype of $\II_0$.

To understand this theorem statement, we need to understand that there are three kind of guarantees that we can offer
\marco{should we repeat the 3 points in the email?}\\

To prove the theorem we introduce two lemmas below. The complete proof is available in Appendix \ref{subsec:proof1}, \ref{subsec:proof2} and \ref{subsec:proof3}.\\

\noindent\textbf{LEMMA 1. }
If $\II_0$ satisfies $\dom(\C_0)=\dom(\C_1)\cup\ldots\cup\dom(\C_n)\cup\dom(\methods)$, then $[\![\II_0]\!]$ satisfies $\dom'(\C_0)=\dom(\C_1)\cup\ldots\cup\dom(\C_n)\cup\dom(\methods)\cup\dom(\methods')$.\\

\noindent\textbf{LEMMA 2. }
If $\II_0$ OK, then $[\![\II_0]\!]$ OK.





%\subsubsection{Derived notations}
% Below shows how the functions $mtype$ and $mmodifier$ are derived from $mbody$.
%\[ \textsf{mbody}(m,C) = \textit{modifier } E \spc m(\overline{D} \spc \overline{x}) \{ \text{return } e; \} \] \[ \Rightarrow \textsf{mtype}(m,C) = \overline{D} \to E,\ \textsf{mmodifier}(m,C) = \textit{modifier}\]

%\marco{we also need to define a function that gives all the methods of an interface, something line}
%\[
%\Aux{methodsOf}_\C=\{\m|\mBody(\m,\C)=\method\}
%\]





\begin{comment}
\begin{figure}[tbp]
\centering
\includegraphics[width=5in]{screenshot.png}
\caption{Screenshot.}\label{screenshot_png}
\end{figure}

\haoyuan{I tried to understand the current algorithm, and did more experiments in eclipse.
Now I borrow some ideas from the current version, and give a new version of the algorithm in text. See below.

(1) I guess the function \textsf{tops} is not necessary. The first step is still
\[\textsf{mbody}(m,C_i)\in\overline{meth}\textrm{ (excluding \textbf{static} methods)}\]

(2) Assume the context is ``interface $C_0$ extends $\overline{C}$ \{$meth'$;...\}''. First handle
\[\textsf{override}(meth',\overline{meth}) \eqno{(*)}\]

(3) If $meth'\ne\none$, $(*)$ returns $meth'$ if
\[\forall meth\in\overline{meth},meth'\subtype meth\]
even if there are conflicts in $\overline{meth}$.

(4) If $meth'=\none$, we need to figure out
\[\textsf{mostSpecific}(\overline{meth})\]
and it should be the one that ``overrides'' all the others in $\overline{meth}$. It means we should not only deal with the return types of methods, but also look into the subtyping relation of interfaces. But for abstract methods, only return types are taken into consideration.
}
\end{comment}

%\text{\yanlin{shouldn't mostSpecific be: $\forall \method' \in \methods : \method \subtype
%  \method'$ ?}}

%(2) If $body_1.\textsf{returnType}=body_2.\textsf{returnType}$, \textsf{shadow} tends to return a default method. If both $body_1$ and $body_2$ are default methods, \textsf{shadow} throws an error.
%\begin{equation*}
%\begin{array}{ll}
%\textsf{shadow}(body_1, body_2)=\textsf{ERROR} & \textsf{if }body_1.\textsf{modifier}=body_2.\textsf{modifier}=\textbf{default}\\
%\textsf{shadow}(body_1, body_2)=body_1 \hspace{.1in} & \textsf{if }body_1.\textsf{modifier}=\textbf{default} \\
%\textsf{shadow}(body_1, body_2)=body_2 \hspace{.1in} & \textsf{if }body_2.\textsf{modifier}=\textbf{default} \\
%\textsf{shadow}(body_1, body_2)=body_1\textsf{ (or }body_2\textsf{)} \hspace{.1in} & \textsf{otherwise}
%\end{array}
%\end{equation*}
%
%(3) If $body_1.\textsf{returnType}<:body_2.\textsf{returnType}$, \textsf{shadow} tends to choose the one with the subtype (namely $body_1$), but only when both methods are abstract, otherwise it gives an error. The other direction $body_2.\textsf{returnType}<:body_1.\textsf{returnType}$ follows the same rule. It also gives an error if there is no subtyping relationship between two return types.
%\begin{equation*}
%\begin{array}{ll}
%\textsf{shadow}(body_1, body_2)=body_1 & \textsf{if }body_1.\textsf{modifier}=body_2.\textsf{modifier}=\emptyset\\
%& \textsf{and }body_1.\textsf{returnType}<:body_2.\textsf{returnType}\\
%\textsf{shadow}(body_1, body_2)=body_2 & \textsf{if }body_1.\textsf{modifier}=body_2.\textsf{modifier}=\emptyset\\
%& \textsf{and }body_2.\textsf{returnType}<:body_1.\textsf{returnType}\\
%\textsf{shadow}(body_1, body_2)=\textsf{ERROR} \hspace{.1in} & \textsf{otherwise}
%\end{array}
%\end{equation*}

%\subsubsection{Auxiliary function: \textsf{replace}}
%
%The \textsf{replace} function takes two same methods (with the same name and types of arguments), and gives the result of the first method overriding the second one.
%
%\begin{equation*}
%\begin{array}{ll}
%\textsf{replace}(body_1, body_2)=body_1 & \textsf{if }body_2=\emptyset\\
%\textsf{replace}(body_1, body_2)=body_2 & \textsf{if }body_1=\emptyset\\
%\textsf{replace}(body_1, body_2)=body_1 & \textsf{if }body_1.\textsf{returnType}<:body_2.\textsf{returnType}\\
%\textsf{replace}(body_1, body_2)=\textsf{ERROR} \hspace{.1in} & \textsf{otherwise}
%\end{array}
%\end{equation*}


\section{Bridging between IB and CB in Java}\label{sec:imp}


%\footnote{ Rewriting libraries and
%  application in this language/model would result in more reusable
%  libraries, but it would be a taunting task.}
Creating a new language/extension would be an
elegant way to illustrate the point of IB. However,
significant amounts of engineering would be needed to build a practical
language and achieve a similar level of integration and tool support
as Java. To be practical, %To have an approach that can both illustrate IB programming and be practical, 
we have instead implemented
\mixin as an annotation in Java 8, and a \emph{compilation agent}.
That is, the Classless Java style of programming
is supported by library.

Disciplined use of Classless Java (avoiding class
declarations as done in Section~\ref{sec:ep}) illustrates what \emph{pure} IB is like.
However, using \mixin, CB and IB programming can be mixed together,
harvesting the practical convenience of using existing Java libraries, the full
Java language and IDE support.
The key to our implementation is compilation agents, which
 allow us to rewrite the Java AST just
before compilation. We discuss the advantages and limitations of our approach.

\subsection{Compilation Agents}
Java supports compilation agents, where Java libraries can interact with the Java compilation process,
acting as a man in the middle between the
generation of AST and bytecode.

This process is facilitated by frameworks like Lombok~\cite{lombok}:
a Java library that aims at reducing Java boilerplate code via 
annotations. \mixin was created using Lombok.
Figure~\ref{fig:lombok}~\cite{neildo2011blog} illustrates the flow of
the \mixin annotation.
First Java source code is parsed into an abstract syntax tree (AST).
The AST is then captured by Lombok:
each annotated node is passed to
the corresponding (Eclipse or Javac) handler. The handler is
free to modify the information of the annotated node, or even inject new nodes (like methods, inner classes,
etc). Finally, the Java compiler works on the modified AST to generate bytecode.
%Note how during the compilation,
%no source code is changed, and no new source/temporary files are created.


%There are a number of annotations provided by the
%original Lombok, including \Q:@Getter:, \Q:@Setter:,
%\Q:@ToString: for generating getters, setters and \QM{toString}
%methods, respectively.  Furthermore, Lombok provides a number of
%interfaces for users to create custom transformations, as extensions
%to the original framework.
%A transformation is based on a handler, which acts on the AST for the
%annotated node and returns a modified AST for analysis and
%generation afterwards. Such a handler can either be a Javac handler or
%an Eclipse handler.

\begin{figure}[t]
\saveSpaceFig
\centering
\includegraphics[width=3in]{pdfs/lombok3.png}
\caption{The flow chart of \mixin annotation processing.
}
\label{fig:lombok}
\saveSpaceFig
\end{figure}

\paragraph{Advantages of Lombok}
The Lombok compilation agent has advantages with respect to alternatives like
pre-processors, or other Java annotation processors.
Lombok offers in Java an expressive power similar to that of Scala/Lisp macros;
except, for the syntactic convenience of quote/unquote templating.

\paragraph{Direct modification of the AST}
Lombok alters the generation process of the class files,
by directly modifying the AST. Neither the source code is modified nor
new Java files are generated. Moreover, and probably more importantly,
Lombok supports generation of code \emph{inside} a class/interface,
which conventional Java annotation processors, such as \texttt{javax.annotation}, do not support.

\paragraph{Modularity}
While general preprocessing acts across module boundaries, compilation
agents act modularly on each class/compilation unit. It makes sense to
apply the transformations to one class/interface at a time, and only to
annotated classes/interfaces. This allows library code to be reused
without being reprocessed or recompiled, making our
approach 100\% compatible with existing Java libraries, which can be
used and extended normally.

\paragraph{Tool support}
Features written in Lombok integrate and are supported directly in the
language and are also supported by most tools.  
In Figure~\ref{fig:screenshot},
\mixin generates an \Q@of@ method in \Q@Point2D@, and \Q@of@, \Q@withX@, \Q@withY@ methods in \Q@Point3D@.
In Eclipse, the processing is
performed transparently and the information of the interface from
compilation is captured in the ``Outline'' window.
This includes all
the methods inside the interface, including the generated ones.
Moreover, as a useful IDE feature, the auto-completion also works for these newly generated methods.

\begin{comment}
\paragraph{Clarity against obfuscation}
Preprocessors bring great power, which can easily be misused producing
code hard to understand. Thus code quality and maintainability are reduced.
Compilation agents start from Java syntax, but they can reinterpret it.
Preserving the syntax avoids syntactic conflicts, allowing many
tools to work transparently.
\end{comment}

\begin{figure*}[t]
\saveSpaceFig
\saveSpaceFig
\centering
\includegraphics[width=5.2in]{pdfs/screenshot4.png}
\caption{Generated methods shown in the Outline window of Eclipse and auto-completion.}
\label{fig:screenshot}
\saveSpaceFig
\end{figure*}

\begin{comment}
\paragraph{Direct modification of the AST}
Lombok alters the generation process of the class files,
by directly modifying the AST. Neither the source code is modified nor
new Java files are generated. Moreover, and probably more importantly,
Lombok supports generation of code \emph{inside} a class/interface,
which conventional Java annotation processors do not support. For
example, the standard \texttt{javax.annotation} processor, which is part of the
Java platform, only allows generation of \emph{new code}, and the
new code has to be written in \emph{new files}. Modification and/or
reinterpretation of existing code are not supported. 

\paragraph{Modularity}
While general preprocessing acts across module boundaries, compilation
agents act modularly on each class/compilation unit. It makes sense to
apply the transformations to one class/interface at a time, and only to
annotated classes/interfaces. This allows library code to be reused
without the need of being reprocessed and recompiled, making our
approach 100\% compatible with existing Java libraries, which can be
used and extended normally. Of course, Java libraries can also receive
and use instances of object interfaces as normal objects.

\paragraph{Tool support}
Features written in Lombok integrate and are supported directly in the
language, and are often also supported by (most) tools.  For example in Eclipse, the processing is
performed transparently and the information of the interface from
compilation is captured in the ``Outline'' window.
This includes all
the methods inside the interface as well as the generated ones.
In Figure~\ref{fig:screenshot},
\mixin generates an \Q@of@ method in \Q@Point2D@, and \Q@of@, \Q@withX@, \Q@withY@ methods in \Q@Point3D@.
These methods are also visible to users, showing their types and modifiers in Outline window.
Moreover, as a useful IDE feature, the auto-completion also works for these newly generated methods.

\paragraph{Clarity against obfuscation}
Preprocessors bring great power, which can easily be misused producing
code particularly hard to understand. Thus code quality and maintainability are reduced.
Compilation agents start from Java syntax, but they can reinterpret it.
Preserving the syntax avoids syntactic conflicts, and allows many
tools to work transparently.
\end{comment}

\subsection{\mixin AST Reinterpretation}

Of course, careless reinterpretation of the AST could still be
surprising for badly designed rewritings.  \mixin reinterprets
the syntax with the sole goal of \emph{enhancing and completing code}:
we satisfy the behaviour of abstract methods; add method
implementations; and refine return types.  We consider this to be
quite easy to follow and reason about, since it is similar to what
happens in normal inheritance.  Refactoring operations like renaming
and moving should work transparently in conjunction with our
annotation, since they rely on the overall type structure of the
class, which we do not arbitrarily modify but just complete.

Thus, in addition to the advantages of Lombok, Classless Java offers
 more advantages with respect to arbitrary (compilation agent driven) AST rewriting.

%\marco{The section No reuse of the type system
%is controversial.
%We do need to repeat the type checking, plus we aim to make untypable stuff well typed
%(for example anyone using the of method would not be well typed before).}
%\item \textbf{No reuse of the type system.}
%As we mentioned above, badly designed rewritings can arise from the great power of Lombok. A simple piece of source code
%\begin{lstlisting}
%interface M { int m(); }
%\end{lstlisting}
%can be reinterpreted as
%\begin{lstlisting}
%interface M { void m(String s); }
%\end{lstlisting}
%in which case the type of method \Q@m@ is changed. Our \mixin annotation does not introduce this kind of rewritings,
%and hence the type system is reused. Moreover, Lombok can also modify unbounded types, which is easy to understand,
%for instance, the following code
%\begin{lstlisting}
%interface M { T m(); } // T is unbounded
%\end{lstlisting}
%is transformed into
%\begin{lstlisting}
%interface M { int m(); } // No error message
%\end{lstlisting}
%in which case the user will see the unbounded type in source code, but without error message from the compilation, since
%Lombok has modified the return type of \Q@m@. However, our \mixin annotation can still keep such errors and warnings.


%\item \textbf{Lack of reuse.}  %not sure here... I think most preprocessors support decent reuse, even the C one
%Reusability is yet another concern in using preprocessors.
% In Lombok, implementations of features are
%encapsulated in various annotation handlers,
% in which case some behaviours are allowed to reuse the code by invoking methods
%in other handlers, where tedious replicated code is avoided.



\paragraph{Syntax and type errors}
Some preprocessors (like the C one) can produce syntactically invalid code.
Lombok ensures only syntactically valid code is produced. %; however, type errors can appear.
Classless Java additionally guarantees that no type errors are introduced
in generated code and client code. We discuss these two guarantees in
more detail next:

\begin{itemize}

\item{\bf Self coherence}: the generated code itself is well-typed. 
In our case, it means that either \mixin{} produces (in a controlled way) an
understandable error or the interface can be successfully annotated and the generated code
 (e.g. the \texttt{of} methods in Figure~\ref{fig:screenshot}) is well-typed.

\item{\bf Client coherence}: all the client code (for example method calls)
  that is well-typed before code generation is also well-typed after the generation.
The annotation just adds more behaviour without removing any functionality.

\end{itemize}

\paragraph{Heir coherence} Another form of guarantee that could be
useful in AST rewriting is heir coherence. That is, interfaces
(and in general classes) inheriting the instrumented code are
well-typed if they were well-typed without the instrumentation.
In a strict sense, our rewriting \emph{does not} guarantee heir coherence.  The reason
is that this would forbid adding any (default or abstract) method to
the annotated interfaces, or even doing type refinement. Indeed consider
the following:

\begin{lstlisting}
interface A { int x(); A withX(int x); }
@Obj interface B extends A {}
interface C extends B { A withX(int x); }
\end{lstlisting}

\noindent This code is correct before the translation, but \mixin would  generate in \Q@B@  a method ``\Q@B withX(int x);@''.
This would break \Q@C@.
Similarly, an expression of the form ``\Q@new B(){.. A withX(int x){..}}@''
would be correct before translation, but ill-typed after the translation.

Our automatic type refinement is a useful and convenient feature, but
not transparent to the heirs of the annotated interface.  They need to
be aware of the annotation semantics and provide the right type while
refining methods. To support heir coherence, we need
to give up automatic type refinement, which is an essential part of IB programming.
However, Java libraries almost always break heir
coherence during evolution and still claim backward compatibility. In practice, adding any method to any
non-final class of a Java library is enough to break heir
coherence.  We think return type refinement breaks heir coherence ``less" than normal library evolution, and
if no automatic type-refinements are needed, then \mixin can claim a
form of heir coherence.
%Section~\ref{sec:translation} 
Formal definition/proofs for our safety claims are in the
technical report.%~\footnote{http://www.cs.hku.hk/research/techreps/document/TR-2016-02.pdf}.

\subsection{Limitations}
Our prototype implementation has certain limitations:
\begin{itemize}
\item Lombok allows writing handlers for either javac or ejc(Eclipse's own compiler).
Our current implementation only realizes ejc version. The implementation for
  the \texttt{javac} version is still missing.
\item Simple generics is supported:
type parameters can be used, but generic method typing is 
delegated to the Java compiler instead of  %neither formalized in this paper nor 
being explicitly checked by \mixin.
\item 
Due to limited support in Lombok for separate compilation, i.e., 
accessing information of code defined in different files, \mixin 
%In the same spirit, we have a mature \mixin annotation, which
%does not support separate compilation yet: it 
requires that all
  related interfaces have to appear in a single Java file.
Reusing the logic inside the experimental Lombok annotation \lstinline{@Delegate},
we also offer a less polished annotation supporting
separate compilation.
%  has the same behavior as \mixin. A limitation is
%  that users need to put references to the super types
% of the annotated type together with the annotation, for instance,
%  \lstinline{@Delegate(types = Point2D.class)}.
%   Another limitation is the methods generated by \lstinline{@Delegate} will
%  not be visible in the Outline window, but they can still be auto-completed in Eclipse.
\end{itemize}


%\paragraph{Lombok does language tuning}
%We consider Lombok to be the most developed example of language
%tuning.  While the authors of Lombok do not introduce a specific term
%for what they are doing, their slogan \emph{``Spice up your java''}
%seems to be in line with the philosophy of language tuning. Some
%other examples of language tuning in Lombok include the \Q@val@ type,
%similar to \Q@auto@ in C\# or C++04.  Another library doing language
%tuning is CoFoJa~\cite{cofoja}, where annotations are used to insert
%pre-post conditions in generated bytecode.



\section{Case studies}
Haoyuan and Yanlin

\subsection{A Trivial Solution to the Expression Problem}

The \emph{Expression Problem} (EP)~\cite{wadler98expression-problem} is coined
by Wadler about modular extensibility issues in software evolution and has been
a hot topic in programming languages since. Today we know of various solutions
to the EP that either rely on \emph{new programming language
  features}~\cite{chambers95multimethods,clifton00multiJava,madsen89virtual,nystrom06j,bruce98astatically,McDirmid01jiazzi,garrigue98polymorphic,zenger01extensible,loh06open,wehr11javaGI},
or can be used as \emph{design patterns}~\cite{gof} in existing
languages~\cite{torgersen04theexpression,oliveira06extensible,wouter08datatypes,oliveira09modular,oliveira12extensibility}. In
this paper we show that the EP can be solved with our \mixin annotation in a
trivial way. We will use a canonical example of implementing arithmetic
expressions gradually to illustrate the usage. 

\paragraph{\textbf{Initial System}}
The initial system expresses arithmetic expressions with only literals and
addition structure:

\lstinputlisting[linerange=23-40]{../UseMixinLombok/src/test/TestExpression.java}% APPLY:linerange=EXPRESSION_INIT

\texttt{Exp} is the common super-interface with an evaluation operation
\texttt{eval()} inside. Sub-interfaces \texttt{Lit} and \texttt{Add} extend
interface \texttt{Exp} with default implementations for \texttt{eval}
operation. The number field (\texttt{x}) of a literal is represented as a getter
method \texttt{x()} and expression fields (\texttt{e1,e2}) of an addition as
getter methods \texttt{e1()} and \texttt{e2()}.

\paragraph{\textbf{Adding a New Variant}} 
It is easy to add new data variants to the code in the initial system in a
modular way. For example, the following code how to add the subtraction variant.

\lstinputlisting[linerange=44-51]{../UseMixinLombok/src/test/TestExpression.java}% APPLY:linerange=EXPRESSION_SUB


\paragraph{\textbf{Adding a New Operation}}
Adding new operations to the previous system is still straightforward (although
requires longer code). The following code shows an example of adding a new
operation \texttt{print}. 

\lstinputlisting[linerange=55-73]{../UseMixinLombok/src/test/TestExpression.java}% APPLY:linerange=EXPRESSION_PRINT

The basic idea is to define interfaces for extending old interfaces.  The
interface \texttt{ExpP} extending interface \texttt{Exp} is defined with the
extra function \texttt{print()}. Interfaces \texttt{LitP} and \texttt{AddP} are
defined with default implementations of operation \texttt{print()}, extending
base interfaces \texttt{Lit} and \texttt{Add}, respectively. Importantly, note
that in \texttt{AddP}, the types of ``\emph{fields}'' \texttt{e1,e2} are refined via
the \emph{return types refinement} of getter methods \texttt{e1()} and
\texttt{e2()}. 

\subsection{Other case studies}
\marco{should we put here the DB with fluent setters?}


\section{Related Work}\label{sec:related}
In this section we discuss related work and comparison to Classless Java.




%\yanlin{need to discuss: whether the comments for FTJ is fair.}

%No, was not fair, I know that work, it does modular type checking, and 
%more work after that they extend with more safe/modular typing things.
%also, Java8 IS a language extension on Java
%However, language extensions (including FTJ)
%have a natural drawback: the programmer has to learn new syntax. In contrast,
%our approach is completely compatible with the current Java language, so that
%programmers don't need to pay any learning cost to adapt to this new classless
%programming pattern. Another drawback which is particular for FTJ is that FTJ
%doesn't have type for traits, hence the correctness check of trait is done when
%type-checking classes. This choice makes the design of FTJ simpler but lost
%typechecking efficiency (one trait will be potentially checked multiple times if
%it is used in multiple classes).

\subsection{Multiple Inheritance in Object Oriented Languages}
Many authors have argued in favour or against multiple inheritance.  It provides
expressive power, but it is difficult to model and implement, and can create
programs that are hard to reason about.  These difficulties include the famous
diamond (fork-join) problem~\cite{bracha90mixin,Sak89dis}, conflicting methods, etc.
%and the yo-yo problem~\cite{taenzer1989problems}.  
To conciliate the need for
expressive power and the need for simplicity, lots of models have been proposed
in the past few years, including mixins~\cite{bracha90mixin} and
traits~\cite{scharli03traits}.  They provide novel programming architecture
models in the OO paradigm.

\begin{enumerate}
\item Mixins allows to name components that can be applied to various classes as
  reusable functionality units. However, they suffer from linearisation: the
  order of mixin application is relevant in often subtitle and undesired
  ways. constraints. This hinders their usability and their ability of resolving
  conflicts: the linearisation (total ordering) of mixin inheritance cannot
  provide a satisfactory resolution in all cases and restricts the flexibility
  of mixin composition.  To fight those limitations, an algebra of mixin
  operators is introduced~\cite{ancona2002calculus}, but this raised the
  complexity of the approach, especially when constructors and fields are
  considered~\cite{marco09FJigsaw}.  Our approach does not have the
  linearisation problem, in that the semantics of Java \textbf{extends} clause
  is unordered and symmetric.
\item Simplifying the mixin algebras approach, traits draw a strong line between
  units of reuse (traits) and object factories (classes) In this model,
  traits~\cite{scharli03traits} units of reusable code, containing only methods
  as reusable functionalities. Thus, no state/state initialization is
  considered.

  Classes act as object factories, requiring functionalities from multiple
  traits.  Traits offers a trait algebra with operations like sum, alias and
  exclusion, provided for explicit conflict resolution.

  Concluding, (pure) traits do not allow state and they do not offers any reuse
  instrument to ensure that state is coherently initialized when finally defined
  in classes.  Traits can't be instantiated. This model requires two concepts
  (traits and classes) to coexist and cooperate.

  Some authors see this as good language design fostering good software
  development by helping programmers to think about the structure of their
  programs.  However, other authors see the need of two concepts and the absence
  of state as drawbacks of this model. Our approach takes interfaces as units of
  reuse, and meanwhile generates factory method for instantiation.

\item C++ and Scala also try to provide solutions to multiple inheritance, but
  both suffer from object initialization problems. Virtual inheritance in C++
  provides another solution to multiple inheritance (especially the diamond
  problem by keeping only one copy of the base class)~\cite{ellis1990annotated},
  however suffers from object initialization problem as pointed out by Malayeri
  et al.~\cite{malayeri2009cz}. It bypasses all constructor calls to virtual
  superclasses, which would potentially cause serious semantic error. Scala
  solution (very similar to linearised mixins, but misleadingly called traits in
  the language) avoids this problem by disallowing constructor parameters,
  causing no ambiguity in cases such as diamond problem.  This approach has
  limited expressiveness, and suffers from all the problems of linearised mixin
  composition.
  Python also offers multiple inheritance via linearised mixins. Indeed in python any class is implicitly a mixin, and mixin composition informally expressed as\\*
  \Q@ class A use B,C {...new methods...}@\\*
  can be expressed in python as \\*
  \Q@ class Aux: ...new methods...@\\*
  \Q@ class A(B,C,Aux): pass@
  Our approach not only does not involve linearised mixin problem, but also
  support proper constructor mechanism.
\end{enumerate}

\subsection{Multiple Inheritance in Java}
Since Java 8 default methods are introduced, concrete method implementation are
allowed to be defined (via the \textbf{default} keyword) inside
interfaces. 
% Since Java supports implementation of multiple interfaces (instead
% of extension of a single class), 
The introduction of default methods opens the
gate for various flavours of multiple inheritance in Java, using interfaces.
Former work by Bono.et. al.~\cite{bono14}. provides details on mimicking traits
through interfaces.

There are proposals for extending Java (before Java8) with traits. For example,
FeatherTrait Java (FTJ)~\cite{Liquori08ftj} by Liquori et al. extends the
calculus of Featherweight Java (FJ)~\cite{Igarashi01FJ} with statically-typed
traits, adding trait-based inheritance in Java.  Except for few, mostly
syntactic details, their work can be emulated/rephrased with Java8 interface.


\section{Comparing to traits and mixins}

\begin{comment}
Haoyuan

   - vs both: we do automatic return type refinement, which has useful applications
   (example: Expression Problem)

   - vs traits: we support of methods to create new objects (a replacement to constructors);
   Moreover we have the with and clone methods (we miss more applications for those). Show
   how to model the operations on traits; discuss operations that we cannot model
   (example: renaming).

   - vs mixins: we use the trait model of explicitly resolving conflicts. This is arguably
   better for reasoning.
\end{comment}

Our approach is based on code generation by Java annotations. The model we generate encourages composability and reusability in object-oriented programming, and is considered to be an alternative to traits or mixins,  meanwhile achieving better performance in some situations. Hence it is necessary for us to make a comparison between this approach and traits (or mixins) we commonly used before.

Our approach is quite different from mixins, in the sense that we use the trait model of explicitly resolving conflicts. Just as [Scharli2003] demonstrated in the paper, mixin inheritance is a good approach of achieving code reuse, nevertheless, the mixin model is not so expressive to resolve conflicts from many mixins. In the trait model, aliases and exclusions are provided for explicit conflict resolution. Such operations can actually be modelled in the mechanism of our approach.

Here we present how the original operations on traits are supported by our model.
\begin{itemize}
\item \textbf{Symmetric sum}: the symmetric composition of two disjoint traits is achieved by simply implementing two interfaces in Java correspondingly, without overriding any method. The composition relies on multiple inheritance on interfaces, which is supported by Java.
\item \textbf{Override}: the overriding operation (also known as asymmetric sum) is modelled by implementing many interfaces, while overriding some methods inside. The code below gives an example of explicitly specifying which super interface to refer to, regarding two methods with the same name.
    \begin{lstlisting}
    interface A { default int m() {return 1;} }
    interface B { default int m() {return 2;} }
    interface C extends A, B { default int m() {return B.super.m();} } 
    \end{lstlisting}
    Here the method \texttt{m()} in interface \texttt{C} simply inherits from \texttt{B.m()}.
\item \textbf{Alias}: an alias operation adds a new name to an old method when creating the new trait. In Java, we just create a new method with reference to the existing method in its super interface. See the example below, where the new method \texttt{k()} is an alias of the existing method \texttt{m()}.
    \begin{lstlisting}
    interface A { default int m() {return 1;} }
    interface B extends A { default int k() {return A.super.m();} }
    \end{lstlisting}
\item \textbf{Exclusion}: exclusion is also supported in Java, where method declarations can hide the default methods correspondingly in the super interfaces. See the example below.
    \begin{lstlisting}
    interface A { default int m() {return 1;} }
    interface B extends A { int m(); }
    \end{lstlisting}
\end{itemize}

Besides, we support \texttt{of} methods in our model, as a replacement to the constructors in original traits. Furthermore, we also support \texttt{with} and \texttt{clone} methods as auxiliary constructors, making the creation of instances more flexible and convenient. Conversely, there are certain operations we cannot model, such as method renaming (as in [Reppy2006]), which breaks structural subtyping.

A further feature leads to return type refinement in our model. Generally speaking, when we use inheritance to create a new trait, with the return type of an existing method being refined, the new \texttt{of} method keeps this consistency. This feature is very useful in many applications; we will see how it makes a difference in our Expression Problem example, in Section [Case Study]. 

\subsection{ThisType/MyType/Extensibility}
%\marco{ this section is unfair. There are unsolved problems with ThisType in negative positions, namelly it breaks subtyping. The most common solution is to allow calling some methods only when the exact type is known. This demolish most advantages of interfaces.}

In certain situations, our approach allows automatic refinement for return types. This is part of a bigger topic in class based languages: expressing and preserving type recursion and (nominal/structural) subtyping at the same time. 

One famous attempt in this direction is provided by
\emph{MyType}~\cite{bruce1994paradigmatic}, representing the type of
\textbf{this}, changing its meaning along with inheritance.  However when
invoking a method with MyType in a parameter position, the exact type of the
receiver must be known.  This is huge limitation in class based object oriented
programming, and is exasperated by interface-only programming as we propose: no
type is ever exact since classes are not part of the language.  A recent
article~\cite{Saito2013933} lights up this topic, since they propose two
features: exact statements \yanlin{do we support exact statements?}\marco{no we do not. When is that it may seams we do?} and
non-inheritable methods; both related to our work: any method generated inside
of the \Q@of@ method is indeed non-inheritable, since there is no class name to
extends from, and exact statements (a form of wild-card capture on the exact
run-time type) could capture the ``exact type'' of an object even in a
class-less environment.



% The addition of MyType to a language will allow easy definition of
%binary methods, methods with recursive types (i.e., the same type of the
%receiver appears in the argument or return positions of methods), etc. MyType
%greatly enhances the expressiveness and extensibility of object-oriented
%programming languages. 

%In our approach, we are using covariant-return types to simulate some uses of
%MyType. But our approach only works on method positive positions, whereas MyType
%is more general, as it works on any positions. Nevertheless our approach is
%still useful for modeling fluent interfaces and solving expression
%problems,etc. 

\begin{comment}
\subsection{Type-Directed Translations/Syntactic Sugar}
\marco{I'm tring to merge this and the next one}
Language extensions are often implemented as syntactic sugar of the base
language. For example, Scala compiler supports XML syntax in normal Scala code
directly (after Scala ?, users need to import \texttt{scala-xml} library
manually). However, this approach is hard in terms of implementation, because it
requires extending the compiler. Also, this approach does not support combining
multiple extensions into one.

SugarJ~\cite{erdweg11sugarj} is a Java-based extensible programming language
that allows programmers to extend the base language with custom language
features by definitions in meta-DSLs (SDF, Stratego, etc). 
\yanlin{Is new syntax really a ``drawback''? I think for some system, like
  SugarJ, one of purpose IS to introduce these new syntax.}  Drawbacks: new
syntax. To create the extension, programmers have to work with multiple
languages (SDF, stratego, etc) while our approach works totally in Java
environment.

We can model certain types of language extensions with annotations 
only, but those extensions do not introduce new syntax: they 
merely do automatic code generation. 
\end{comment}
\subsection{Meta-programming competes with Language extensions}
The most obvious solution to add features to a language is language extension.
They are often implemented as syntactic extensions that can be desugared to the base
language. For example, the Scala compiler was extended to directly supports XML syntax. However, this approach does not support combining multiple extensions into one. We are de facto creating a fork in the language, and rarely the new fork gain enough traction to become the main language release.
On this topic we mention SugarJ~\cite{erdweg11sugarj}; a Java-based extensible language allowing programmers to extend it with custom features by definitions in meta-DSLs (SDF, Stratego, etc). 

On the other side, when the starting language have a flexible enough syntax and a fast and powerful enough reflection, we may just need to play with operator overloading and other language tricks to discover that the language feature we need can be expressed as a simple library in our language. For example, consider SQL alchemy in python.

Java-like language tends to sits in the middle of this two extremes:
libraries can not influence the type system, so many solutions valid in python or other languages could not be applicable, or may be applicable at the cost of loosing safety.

Here (compile/load time) code generation come at the rescue: 
if for a certain feature (\mixin in our case) it is possible to use the original language syntax to
\emph{express-describe} any specific instantiation of such feature
(annotating a class and provide getters), then we can insert in the compilation process a tool that exam and enrich the code before compilation. No need to modify the original source; for example we can work on temporary files.
Java is a particular good candidate for this kind of manipulation since it already provide ways to define and integrate such tools in its own compilation process: in this way there is no need of temporary files, and there is a well defined way of putting multiple extensions together.

Other languages offers even stronger support to safe code manipulation:
Template Haskell~\cite{}, F\# (type providers)~\cite{} and MetaFjig (Active Libraries)~\cite{}
all allows to execute code at compile time and to generate code/classes that are transparently integrated in the program that is being generated/processed/compiled.
In particular, MetaFjig offers a property called \emph{meta-level-soundness}. In short this property ensures by construction that library code (even if wrong or non nonsensical) would never generate ill-typed code. This is roughly equivalent to what we state and manually proof in Lemma 2 for our particular transformation.
Since MetaFjig is not working on annotated classes, there is no direct equivalence on the overall theorem of safety we shown.

\subsection{Formalization of Java8}
We provide a simple and well designed formalization for a subset of Java including default/static interface methods and object initialization literals (often called anonymous local inner classes).
A similar formalization was drafted by\marco{ I use the term draft because I seams to remember it was just a technical report, I'm right?}
Goetz and Field~\cite{goetz12fdefenders} to formalize defender (default) methods
in Java. However this formalization is limited to model exactly one
method inside classes/interfaces.

As a evidence of the attention and care present in our formalization work, while double-checking the behaviour of Java in side cases we have discovered a likely bug in the current \texttt{javac}.\marco{refer to before when we explain the issue.}

\section{Future work}

\subsection{Private state}
The biggest limitation of our approach is the absence in Java8 of support for private/protected methods in interfaces.That is, in Java8 all members of interfaces are public, including static methods.
Since we use abstract methods to encode the state, our state is always all public; however is impossible for the user to know if a certain method maps directly to a field or if it have a default implementation.
If the use wants a constructor that does not directly maps to the fields, (as for secondary constructors in Scala) he can simply define its own \Q@of@ method and delegate on the generated one, as in
\begin{lstlisting}
@Mixin interface Point{
  int x(); int y();
  static Point of(int val){return Point.of(val,val);}  
  }
\end{lstlisting}
However, the generated \Q@of@ method would also be present and public.
If a future version of Java was to support \emph{static private methods in interfaces} we could extend our code generation to handle also encapsulation.
Currently, is possible to use a public nested class with private static methods inside, but this is ugly and cumbersome. We are considering if our annotation processor can take code with \Q!@Private! annotation and turn it into static private methods of a nested class. In this extension,  also the of method could be made private following the same pattern. 

\subsection{State initialization}

As discuss before, the user can trivially define its own \Q@of@ method, and initialize a portion of the state with default values.
However, the initialization code would not be reused/reusable, and subinterfaces would have to repeat such initialization code.
If a field has no setters, a simple alternative is to just define the ``field'' as a default method as in 
\begin{lstlisting}
@Mixin interface Box{ default int val(){return 0;} }
\end{lstlisting}
if setters are required, a possible extension of our code expansion could recognize a field if the getter is provided and the setter is required, and could generate the following code:
\begin{lstlisting}
interface Box{ 
  default int val(){return 0;} //provided
  void val(int _val);//provided
  static Box of(){return new Box(){//generated
    int val=Box.super.val();
    int val(){return val;}
    void val(int _val){val=_val;}
    };}}
\end{lstlisting}
We are unsure of the value of this solution: is very tricky, the user define a method that (contrary to our usual expectation) is actually overridden in a way that the behaviour change, but change only after the first setter is called, plus this code would cache the result instead of re-computing it every time. This can be very relevant and tricky in a non functional setting.

\subsection{Class invariants in ClassLess Java}
Since the objects are created by automatically generated methods,
another limitation of our current approach is that there is no place where the user can dynamically check for class invariants.
In Java often we see code like
\begin{lstlisting}
class Point{ int x; int y;
  Point(int x; int y){this.x=x;this.y=y; assert this.checkInvariant();}
  private boolean checkInvariant(){... x>0,y>0...}
\end{lstlisting} 

We are considering an extension of our annotation where 
default methods with the special name \Q@checkInvariant()@ will be called inside of the \Q@of@ methods.
if multiple interfaces are implemented, and more then one offers
\Q@checkInvariant()@,  a composed implementation could be automatically generated, composing by \Q@&&@ the various competing implementations.

\subsection{Clone, toString, equals and hashCode}
Methods originally defined in \Q@Object@, as \Q@clone@ and \Q@toString@ can be supported by our approach, but they need special care. If an interface annotated with \mixin ask an implementation for \Q@clone@, \Q@toString@, \Q@equals@ or \Q@hashCode@ we can easily generate one from the fields.\footnote{In particular, for clone we can do automatic return type refinement as we do for \Q@with-@ and fluent setters. Note how this would solve most of the Java ugliness related to \Q@clone@ methods.}

However, if the user wish to provide its own implementation, since the method is also implemented in \Q@Object@ we would have a conflict, that we have to explicitly resolve inside of \Q@of@, by implementing the method and delegating it to the user implementation, thus

\begin{lstlisting}
@Mixin interface Point{ int x(); int y();
  default Point clone(){ return Point.of(0,0);}//user defined clone
  }
\end{lstlisting} 
Would expand into 

\begin{lstlisting}
interface Point{ int x(); int y();
  default Point clone(){ return Point.of(0,0);}//user defined clone
  public static Point of(int _x,int_y){
    return new Point(){...
      public Point clone(){ return Point.super.clone();}
      };  }  }
\end{lstlisting} 


\section{Future work}\label{sec:futurework}
In this section we discuss potential future work (not a full list).
\begin{comment}
\subsection{Encapsulation}
The biggest limitation of our approach is the inability to model visibility restrictions. For example, the absence
of support for private/protected methods in Java 8 interfaces forces all
members of interfaces to be public, including static methods. Since we use
abstract methods to encode state, our state is always all public. Still, because 
the state can only be accessed by methods, it 
is impossible for the user to know if a certain method maps directly to a field
or if it has a default implementation.  If the user wants a constructor that
does not directly maps to the fields, (as for secondary constructors in Scala)
he can simply define its own \Q@of@ method and delegate on the generated one:
\begin{lstlisting}
@Obj interface Point{  int x(); int y();
  static Point of(int val){return Point.of(val,val);}  }
\end{lstlisting}
However, the generated \Q@of@ method would also be present and public.  If a
future version of Java was to support \emph{static private methods in
  interfaces} we could extend our code generation to handle also encapsulation.
  
However, interfaces as a whole can have public or 
package private (java default) visibility.

We can add a second annotation \Q!@Exposed!  that leverages on this edge: An
interface without exposed works as usual, but if any method of a public \mixin
interface is annotated with \Q!@Exposed!, we can apply a translation where a new
(package private) interface type is introduced, and the original interface become
just a facade.  For example:
\begin{lstlisting}
@Obj public interface Person{
  void name(String val);
  @Exposed default void rename(String newName){ if(/*valid name*/){ this.name(val);}}
  @Exposed String name();
  @Exposed static Person from(String val){ if(/*valid name*/){return Person.of(val);}
    throw /*invalid name*/}  }
\end{lstlisting}
becomes
\begin{lstlisting}
public interface Person{
  void rename(String newName)
  String name();
  static Person from(String val){ return Person$.from(val);} }
  
@Obj interface Person$ extends Person{//will be further expanded by @Obj
  void name(String val);
  default void rename(String newName){ if(/*valid name*/){ this.name(val);}}
  String name();
  static Person from(String val){ if(/*valid name*/){return Person$.of(val);}
    throw /*invalid name*/}  }
\end{lstlisting}

This is not a perfect solution, since
\Q@Person$@ can still be seen inside the \Q@Person@ package and heirs of
\Q@Person$@,
however it is surprising we achieve such of a good result without any language
support for privacy in interfaces.
\end{comment}

%\subsection{Qualifiers in Methods} %{Private state
\paragraph{Qualifiers in Methods} %{Private state
The biggest limitation of our approach is the inability to model qualifiers
for class methods (private, protected, synchronized, etc.). For example, the absence
of support for private/protected methods in Java 8 interfaces forces all
members of interfaces to be public, including static methods. Since we use
abstract methods to encode state, our state is always all public. Still, because 
the state can only be accessed by methods, it 
is impossible for the user to know if a certain method maps directly to a field
or if it has a default implementation.  If the user wants a constructor that
does not directly maps to the fields, (as for secondary constructors in Scala)
he can simply define its own \Q@of@ method and delegate on the generated one:
\begin{lstlisting}
@Obj interface Point{  int x(); int y();
  static Point of(int val){return Point.of(val,val);}  }
\end{lstlisting}
However, the generated \Q@of@ method would also be present and public.  If a
future version of Java was to support \emph{static private methods in
  interfaces} we could extend our code generation to handle also encapsulation.
Currently, it is possible to use a public nested class with private static
methods inside, but this is ugly and cumbersome. One possibility is that the
annotation processor takes methods with a \Q!@Private! annotation, and turns it into
static private methods of a nested class. In this extension, also the \Q@of@
method could be made private following the same pattern.

% \yanlin{future work on state initialization removed for now. Because I think the
%   other two future works are fine. But this one is still not clear and
%   controversial. The proposed workaround forces users to make decision: if he
%   wants state with initialization then he also has to define the setter. Or
%   else, it is ambiguous: could be interpreted as a state initializer or just
%   a normal default method.}
\begin{comment}
\subsection{State initialization}
As discussed before, the user can trivially define its own \Q@of@ method, and
initialize a portion of the state with default values.  However, the
initialization code would not be reusable, and subinterfaces would have
to repeat such initialization code.  If a field has no setters, a simple
alternative is to just define the ``field'' as a default method as in
\begin{lstlisting}
@Obj interface Box{ default int val(){return 0;} }
\end{lstlisting}
If setters are required, a possible extension of our code expansion could
recognize a field if the getter is provided and the setter is required, and
could generate the following code:
\lstinputlisting[linerange=4-11]{../UseMixinLombok/src/test/TestStateInitialization.java}% APPLY:linerange=STATE_INIT

We are unsure about the value of this workaround because of its trickiness. In
order to define a state with initialization, users have to define a method trusting that it will be overridden later with a behavioral change, but
such change is visible only after the first time the setter is called. Moreover, this code would cache
the result instead of re-computing it every time. This can be very tricky in a
non-functional setting.
%\bruno{Yanlin please polish text/break long sentences.}
\end{comment}

\begin{comment}
\subsection{Class Invariants in ClassLess Java}
Since objects are created by automatically generated methods, another limitation
of our current approach is that there is no place where the user can dynamically
check for class invariants. In Java often we see code like
\begin{lstlisting}
class Point{ int x; int y;
  Point(int x; int y){this.x=x;this.y=y; assert this.checkInvariant();}
  private boolean checkInvariant(){... x>0,y>0...}
}
\end{lstlisting} 

We are considering an extension of our annotation where 
default methods with the special name \Q@checkInvariant()@ will be called inside the \Q@of@ methods.
If multiple interfaces are implemented, and more then one offers
\Q@checkInvariant()@,  a composed implementation could be automatically generated, composing by \Q@&&@ the various competing implementations.
\end{comment}
%\bruno{removed the invariants stuff; we need space, I think.}

%\subsection{Clone, toString, equals and hashCode}
\paragraph{Clone, toString, equals and hashCode}
Methods originally defined in Java class \Q@Object@, as \Q@clone@ and
\Q@toString@, can be supported by our approach with special care. If an
interface annotated with \mixin asks an implementation for \Q@clone@,
\Q@toString@, \Q@equals@ or \Q@hashCode@ we can easily generate one from the
fields.\footnote{In particular, for clone we can do automatic return type
  refinement as we do for \Q@with-@ and fluent setters. Note how this would
  solve most of the Java ugliness related to \Q@clone@ methods.}  However, if
the user wishes to provide his own implementation, since the method is also
implemented in \Q@Object@, a conflict arises. The generated code can resolve the
conflict inside \Q@of@, by implementing the method and delegating it to the user
implementation, thus

\begin{lstlisting}
@Obj interface Point{ int x(); int y();
  default Point clone(){ return Point.of(0,0);}}//user defined clone
\end{lstlisting} 
would expand into 

\begin{lstlisting}
interface Point{ ...//as before
  public static Point of(int _x, int _y){
    return new Point(){...
      public Point clone(){ return Point.super.clone();}};}}
\end{lstlisting} 


%===============================================================================
\section{Conclusion}\label{sec:conclusion}

Before Java 8, concrete methods and static methods where not allowed
to appear in interfaces.  Java 8 allows static interface methods and
introduces \emph{default methods}, which allow for implementation
insides interfaces. This had an important positive consequence that
was probably overlooked by the Java design team: the concept of class
(in java) is now redundant and unneeded.  We define a subset of Java,
called ClassLess Java, where programs and (reusable) libraries can be
easily defined and used.  To avoid for some syntactic boilerplate
caused by Java not being originally designed to work without classes,
we introduce a new annotation:\mixin provide default implementations
for various methods (e.g. getters, setters, with-methods) and a
mechanism to instantiate objects. \mixin annotation helps programmers
to write less cumbersome code while coding in ClassLess Java; indeed
we think the obtained gain is so high that ClassLess Java with \mixin
annotation can be less cumbersome than full Java.\bruno{May need rewriting}


%===============================================================================

%\bibliographystyle{splncs}
\bibliographystyle{plain}
\bibliography{paper}

\appendix
\newpage

\section{Appendix}\label{sec:appendix}

\subsection{LEMMA 1 and Proof}\label{subsec:lemma1}

\textbf{LEMMA 1. }
For any expression $e$ under an interface table $\II$ IT where $\Gamma\vdash e\in\C^\II$, $\II$ has $\weakAnn$ annotation and $[\![\II]\!] = \II'$, then under the interface table $\II'$ IT, $\Gamma\vdash e\in\C^\II$.
\begin{proof}
By induction on the typing rules: by the grammar shown in Figure~\ref{Grammar}, there are 6 cases for an arbitrary expression $e$:
\begin{itemize}
\item Variables are typed in the same exact way.
\item Field update. The type preservation is ensured by induction.
\item A method call (normal, static or super). The corresponding method declaration won't be ``removed'' by the translation, also the types remain unchanged. The only work $\weakAnn$ does is adding a static method $\QM{of}$ to the interface, however, a pre-condition of the translation is $\QM{of}\notin\dom(\C^\II)$, so adding $\QM{of}$ method has no way to affect any formerly well typed method call.
\item An object creation. Adding the $\QM{of}$ method doesn't introduce unimplemented methods to an interface, moreover, the static method is not inheritable, hence after translation such an object creation still type checks and has the right type by induction.
\end{itemize}
\end{proof}
\textbf{LEMMA 1b. }
For any expression $e$ under an interface table $\II$ IT where there is no heir of $\C^\II$,  $\Gamma\vdash e\in\C^\II$, $\II$ has \mixin annotation and $[\![\II]\!] = \II'$, then under the interface table $\II'$ IT, $\Gamma\vdash e\in\_<:\C^\II$.
\begin{proof}
The proof follows the same scheme of the Lemma1, but for the case of method call the return type may be refined with a subtype. This is still ok since we require $\_<:\C^\II$. On the other side, this weaker result still allows the application on the method call typing rules, since in the premises the types of the actual parameter are required to be a subtype of the formal one.
\end{proof}
\subsection{LEMMA 2 and Proof}\label{subsec:lemma2}

\textbf{LEMMA 2. }
If $\II$ has $\weakAnn$ annotation and $\C^{\II}$ OK in $\II$ IT, then $[\![\II]\!]$ OK in $[\![\II]\!]$ IT.
\begin{proof}

By the rule \rn{t-Intf} in Figure~\ref{ET}, we divide the proof into two parts.

\noindent\textbf{Part I.} For each default or static method in the domain of $[\![\C^\II]\!]$, the type of the return value is compatible with the method's return type.

Since $\II$ OK, and by \textbf{LEMMA 1}, all the existing default and static methods are well typed in $[\![\II]\!]$, except for the new method \QM{of}. It suffices to prove that it still holds for $\ofMethod(\C)$.


By the definition of $\ofMethod(\C)$, the return value is an object $$\QM{return new}\ \C^\II \oR\cR\ \QM{\{}...\ \QM{\}}$$
To prove it is of type $\C^\II$, we use the typing rule \rn{t-Obj}.

\begin{itemize}
\item All field initializations are type correct. By the definition of $\ofMethod(\C^\II)$ in Section~\ref{subsec:ofmethod}, the fields $m_1,\ldots,m_n$ are initialized by $\QM{of}$'s arguments, and types are compatible.
\item All method bodies are well-typed.
    \begin{itemize}
    \item Typing of the $i$-th getter $m_i$. \[\Gamma, m_i:\C_i, \QM{this}:\C^\II \vdash m_i\in \C_i\]
        We know that $\C_i=\C^{\mh_i}$ since the $i$-th getter has its return type the same as the corresponding field $m_i$.
    \item Typing of the \QM{with-} method of an arbitrary field $m_i$. By Section~\ref{subsec:ofmethod}, if the \QM{with-} method of $m_i$ is well-defined, it has the form \[\C^\II\ \QM{with#}\m_i\oR \C_i\ \QM{_val}\cR\ \QM{\{}\QM{return}\ \C^\II\QM{.of(}\es_i\QM{);\}}\]
        $\es_i$ is obtained by replacing $m_i$ with $\QM{_val}$ in the list of fields, and since they have the same type $\C_i$, the arguments $\es_i$ are compatible with $\C^\II.\QM{of}$ method. Hence \[\Gamma, m_1:\C_1\ldots m_n:\C_n, \QM{this}:\C^\II,\ \QM{_val} : \C_i \vdash \C^\II\QM{.of}\oR\es_i\cR\in \C^\II\]
        We know that $\C^\II=\C^{\mh_i}$ by the return type of $\QM{with#}\m_i$ shown as above.
    \item Typing of the $i$-th setter \QM_$m_i$. If the \QM_$m_i$ method is well-defined, it has the form
        \[\C^\II\ \QM_\m_i\oR \C_i\ \QM{_val}\cR\ \QM{\{} \m_i\QM{= _val;return this;\}}\]
        By \rn{t-Update}, the assignment ``$\m_i\QM{= _val;}$'' is correct since $\m_i$ and $\QM{_val}$ have the same type $I_i$, and the return type is decided by \QM{this}. \[\Gamma, \QM{this}:\C^\II,\ \QM{_val} : \C_i \vdash\QM{this}\in \C^\II\]
        We know that $\C^\II=\C^{\mh_i}$ by the return type of $\QM_\m_i$ shown as above.\\
        \haoyuan{Can we write $\Gamma, \QM{this}:\C^\II,\ \QM{_val} : \C_i, \C^{\mh_i} = \C^\II \vdash\QM{this}\in \C^\II = \C^{\mh_i}$?}\\
        \haoyuan{Can we write $\QM{\{} \m_i\QM{= _val;return this;\}} \in \C^\II$?}
      \marco{Not without defining what that would mean. Do we need to? it seams good as it is now...}
    \end{itemize}
\item All method headers are valid with respect to the domain of $\C^\II$. Namely $$\sigvalid(\mh_1\ldots\mh_n,I)$$
    For convenience, we use ``$\method$ in $\ofMethod(\C^\II)$'' to denote that $\method$ is one of the implemented methods in the return expression of $\ofMethod(\C^\II)$, namely \Q@new@ $\C^\II$\Q@(){...}@.
    \begin{itemize}
    \item For the $i$-th getter $m_i$,
        \begin{align*}
        &\C_i\ \m_i\oR\cR\ \QM{\{...\}}\mbox{ in }\ofMethod(\C^\II)\\
        \mimply\hspace{.2in}& \C_i\ \m_i\oR\cR\QM; \in \fieldsFunc(\C^\II)\\
        \mimply\hspace{.2in}& \C_i\ \m_i\oR\cR\QM; = \mBody(m_i,\C^\II)\\
        \mimply\hspace{.2in}& \C_i\ \m_i\oR\cR\QM; <: \mBody(m_i,\C^\II)
        \end{align*}
    \item For the $\QM{with#}\m_i$ method,
        \begin{align*}
        &\C^\II\ \QM{with#}\m_i\oR \C_i\ \QM{_val}\cR\ \QM{\{...\}}\mbox{ in }\ofMethod(\C^\II)\\
        \mimply\hspace{.2in}& \mBody(\QM{with#}\m_i,\C^\II) \mbox{ is of form }\mh\QM;\\
        \mbox{with}\hspace{.2in}& \valid(\C^\II)\\
        \mimply\hspace{.2in}& \isWith(\mBody(\QM{with#}\m_i,\C^\II),\C^\II)\\
        \mimply\hspace{.2in}& \C^\II\ \QM{with#}\m_i\oR \C_i\ \QM{_val}\cR\QM; <: \mBody(\QM{with#}\m_i,\C^\II)
        \end{align*}
    \item For the $i$-th setter $\QM_\m_i$,
        \begin{align*}
        &\C^\II\ \QM_\m_i\oR \C_i\ \QM{_val}\cR\ \QM{\{...\}}\mbox{ in }\ofMethod(\C^\II)\\
        \mimply\hspace{.2in}& \mBody(\QM_\m_i,\C^\II) \mbox{ is of form }\mh\QM;\\
        \mbox{with}\hspace{.2in}& \valid(\C^\II)\\
        \mimply\hspace{.2in}& \isSetter(\mBody(\QM_\m_i,\C^\II),\C^\II)\\
        \mimply\hspace{.2in}& \C^\II\ \QM_\m_i\oR \C_i\ \QM{_val}\cR\QM; <: \mBody(\QM_\m_i,\C^\II)
        \end{align*}
    \end{itemize}
\item All abstract methods in the domain of $\C^\II$ have been implemented. Namely $$\alldefined(\mh_1\ldots\mh_n,I)$$
    Here we simply refer to $\valid(\C^\II)$, since it guarantees each abstract method to satisfy $\isField$, $\isWith$ or $\isSetter$. But that object includes all implementations for those cases. A getter $m_i$ is generated if it satisfies $\isField$; a \Q@with-@ method is generated for the case $\isWith$, by the definition of $\withMethod$; a setter for $\isSetter$, similarly, by the definition of $\setterMethod$. Hence it is of type $\C^\II$ by \rn{t-Obj}.
\end{itemize}

\noindent\textbf{Part II.} Next we check that in $[\![\II]\!]$, $$\dom([\![\II]\!])=\dom(\C_1)\cup\ldots\cup\dom(\C_n)\cup\dom(\methods)\cup\dom(\method')$$

Since $\II$ OK, we have $\dom(\II)=\dom(\C_1)\cup\ldots\cup\dom(\C_n)\cup\dom(\methods)$, and hence it is equivalent to prove $$\dom([\![\II]\!])=\dom(\C^\II)\cup\dom(\method')$$
This is obvious since a pre-condition of the translation is $\QM{of}\notin\dom(\C^\II)$, so $\method'$ doesn't overlap with $\dom(\C^\II)$. The definition of $\dom$ is based on $\mBody$, and here the new domain $\dom([\![\II]\!])$ is only an extension to $\dom(\C)$ with the \QM{of} method, namely $\method'$. Also note that after translation, there are still no methods with conflicted names, since the \QM{of} method was previously not in the domain, hence $[\![\II]\!]$ is well-formed, which finishes our proof.
\end{proof}

\textbf{LEMMA 2b. }
If $\II$ has \mixin annotation $\C^{\II}$ OK in $\II$ IT
and there is no heir of $\C^\II$, then $[\![\II]\!]$ OK in $[\![\II]\!]$ IT.
\begin{proof}
\noindent\textbf{Part I.} Similarly to what already argued for Lemma2,
since $\II$ OK, and by \textbf{LEMMA 1b}, all the existing default and static methods are well typed in $[\![\II]\!]$,IT.
The translation function delegate its work to $\weakAnn$ in such way that we can refer to
Lemma2  to complete this part. Note that all the methods that are added (directly) by \mixin are abstract, and thus there is no body to typecheck.


\noindent\textbf{Part II.}
Similar to what already argued for Lemma2, but we need to notice that the the newly added methods are valid refinements for already present methods in $\dom(\C^\II)$ before the translation.
Thus by last clause of the definition of $\override(\_)$, $\mBody(\_)$ is defined on the same method names.
\end{proof}


\subsection{THEOREM and Proof}\label{subsec:theorem}

\noindent\textbf{THEOREM }$\weakAnn$\\*
If a given interface table $\II$ IT is OK\\*
 where $\II$ has $\weakAnn$,
$\valid(\C^{\II})$  and $\QM{of}\notin\dom(\C^{\II})$,\\*
then the interface table $[\![\II]\!]$ IT is OK.

\begin{proof}
\textbf{LEMMA 2} already proves that $[\![\II]\!]$ is OK. On the other hand, for any $\II'\in $ IT$\setminus\II$, by \textbf{LEMMA 1}, we know that all its methods
are still well-typed, and the generated code in translation of $\weakAnn$ is only a static method \QM{of}, which has no way to affect the domain
of $\II'$, so after translation rule \rn{t-Intf} can still be applied, which finishes our proof.
\end{proof}

\noindent\textbf{THEOREM }\Q!@Obj!\\*
If a given interface table $\II$ IT is OK\\*
 where $\II$ has \Q!@Obj!,
$\valid(\C^{\II})$  and $\QM{of}\notin\dom(\C^{\II})$, and there is no heir of $\C^{\II}$,\\*
then the interface table $[\![\II]\!]$ IT is OK.


\begin{proof}
Similar to what already argued for \textbf{THEOREM }$\weakAnn$
we can apply \textbf{LEMMA 2b} and \textbf{LEMMA 1b}. Then we finish by \textbf{THEOREM }$\weakAnn$.
\end{proof}


\end{document}

%%% Local Variables:
%%% mode: latex
%%% TeX-master: "."
%%% End:
