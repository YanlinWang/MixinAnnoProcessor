\section{Applications and Case Studies}

This section illustrates applications and larger case studies for
Classless Java. The first application shows how a useful pattern,
using multiple inheritance and type-refinement, can be conveniently
encoded in Classless Java. The second application shows how to model
embedded DSLs based on fluent APIs. The two larger case studies, take
existing projects and refactor them into Classless Java. The first
case study shows a significant reduction in code size, while the
second case study maintains the same amount of code, but improves
modularity.

\subsection{The Expression Problem with Object Interfaces}\label{subsec:ep}

As the first application for Classless Java, we illustrate a useful
programming pattern that improves modularity and extensibility of
programs. This useful pattern is based on an existing solution to
the \emph{Expression Problem} (EP)~\cite{wadler98expression}, which is a well-known
problem about modular extensibility issues in software evolution. Recently, a
new solution~\cite{eptrivially} using only covariant type refinement was
proposed. When this solution is modeled with interfaces and default methods, it
can even provide independent extensibility~\cite{zenger05independentlyextensible}: the ability to assemble a system
from multiple, independently developed extensions. Unfortunately, the
required instantiation code makes a plain Java solution verbose and cumbersome
to use. The \mixin annotation is enough to remove the boilerplate code, making
the presented approach very appealing. Our last case study, presented
in Section~\ref{subsec:int}, is essentially a (much larger) application of this
pattern to an existing program. Here we illustrate the pattern in the
much smaller Expression Problem.

\paragraph{Initial System}
In the formulation of the EP, there is an initial system that models
arithmetic expressions with only literals and addition, and an initial
operation \texttt{eval} for expression evaluation.
As shown in Figure~\ref{fig:ep}, \texttt{Exp} is the common
super-interface with operation \texttt{eval()}
inside. Sub-interfaces \texttt{Lit} and \texttt{Add} extend interface
\texttt{Exp} with default implementations for the \texttt{eval} operation. The
number field \texttt{x} of a literal is represented as a getter method
\texttt{x()} and expression fields (\texttt{e1} and \texttt{e2}) of an addition
as getter methods \texttt{e1()} and \texttt{e2()}.

\begin{figure*}
\centering
\saveSpaceFig
\begin{tabular}{l|l}
\lstinputlisting[linerange=33-43]{../UseMixinLombok/src/casestudy/ep/TestExpression.java}% APPLY:linerange=EXPRESSION_INIT
&
\lstinputlisting[linerange=54-64]{../UseMixinLombok/src/casestudy/ep/TestExpression.java}% APPLY:linerange=EXPRESSION_PRINT
\end{tabular}
\caption{The Expression Problem (left: initial system, right: code for adding
  print operation).}\label{fig:ep}
\saveSpaceFig
\end{figure*}% \bruno{Yanlin: code is overflowing the margins. Note that
  % there are 2 spare lines that can be used on the right!}

\paragraph{Adding a New Type of Expressions}
In the OO paradigm, it is easy to add new types of expressions. For example, the
following code shows how to add subtraction.

\lstinputlisting[linerange=47-50]{../UseMixinLombok/src/casestudy/ep/TestExpression.java}% APPLY:linerange=EXPRESSION_SUB

%\marco{text under need to be improved}
\paragraph{Adding a New Operation} The difficulty of the EP in OO
languages arises from adding new operations. For example, adding a pretty printing
operation would typically change all existing code. However, a solution
%to the EP forbids this and it also forbids using casts. In other words, it should be possible
should add operations in a type-safe and modular way. This
turns out to be easily achieved with the assistance of \mixin.  The code in
Figure~\ref{fig:ep} (on the right) shows how to add the new operation \texttt{print}.
Interface \texttt{ExpP} extends \texttt{Exp} with the extra method
\texttt{print()}. Interfaces \texttt{LitP} and \texttt{AddP} are defined with
default implementations of \texttt{print()}, extending base interfaces
\texttt{Lit} and \texttt{Add}, respectively. Importantly, note that in
\texttt{AddP}, the types of ``\emph{fields}'' (i.e. the getter methods)
\texttt{e1} and \texttt{e2} are refined. If the types were not refined then
the \texttt{print()} method in \texttt{AddP} would fail to type-check.

\paragraph{Independent Extensibility}
To show that our approach supports independent extensibility, a
new operation \texttt{collectLit} which collects all
literal components in an expression is defined. For space reasons,
we omit some code:

\begin{lstlisting}[]
interface ExpC extends Exp {
    List<Integer> collectLit(); }
@Obj interface LitC extends Lit, ExpC {...}
@Obj interface AddC extends Add, ExpC {
    ExpC e1(); ExpC e2(); ...}
\end{lstlisting}

\noindent Now we combine the two extensions together:
%(\texttt{print} and \texttt{collectLit}) 

\lstinputlisting[linerange=88-90]{../UseMixinLombok/src/casestudy/ep/TestExpression.java}% APPLY:linerange=INDEPENDENT_EXTENSIBILITY

\noindent \texttt{ExpPC} is the new expression interface supporting
\texttt{print} and \texttt{collectLit} operations; \texttt{LitPC} and
\texttt{AddPC} are the extended variants. Notice that except for the routine of
\textbf{extends} clauses, no glue code is required. Return types of
\texttt{e1,e2} must be refined to \texttt{ExpPC}.
Creating a simple expression of type \texttt{ExpPC} is
as simple as:
\begin{lstlisting}
ExpPC e8 = AddPC.of(LitPC.of(3), LitPC.of(4));
\end{lstlisting}
\noindent Without Classless Java, tedious instantiation code would need
to be defined manually.

\subsection{Embedded DSLs with Fluent Interfaces}\label{sec:dsls}
Since the style of fluent interfaces was invented in Smalltalk as method
cascading, more and more languages (Java, C++, Scala, etc) came to support fluent interfaces. In most languages, to create fluent
interfaces, programmers have to either hand-write everything or create a wrapper
around the original non-fluent interfaces using \textbf{this}. In Java, there
are several libraries (including jOOQ, op4j, fluflu, JaQue, etc) providing useful
fluent APIs. However most of them only provide a fixed set of predefined fluent
interfaces. %Fluflu enables the creation of a fluent API and implements control
% over method chaining by using Java annotations. However methods that returns
% \textbf{this} are still hand-written.

The \mixin annotation can also be used to create fluent interfaces.  When
creating fluent interfaces with \mixin, there are two main advantages:
\begin{enumerate}
\item Instead of forcing programmers to hand-write code using \textbf{return
    this}, our approach with \mixin annotation removes this verbosity and
  automatically generates fluent setters.
\item The approach supports extensibility: the return types of fluent setters are
  automatically refined.
\end{enumerate}

\noindent We use embedded DSLs of two simple SQL query languages to illustrate.
The first query language \texttt{Database}  models
\texttt{select}, \texttt{from} and \texttt{where} clauses:
\lstinputlisting[linerange=6-10]{../UseMixinLombok/src/casestudy/sql/DatabaseTest.java}% APPLY:linerange=FLUENT_DATABASE
% \bruno{Wouldn't it be better to have a static method create() that calls of("","","")) and initializes all the fields? It seems awkward
% to have to use the of explicitly to initialized all fields.}

\noindent The main benefit that fluent methods give
us is the convenience of method chaining:

\lstinputlisting[linerange=33-33]{../UseMixinLombok/src/casestudy/sql/DatabaseTest.java}% APPLY:linerange=FLUENT_QUERY1

\noindent Note how all the logic for the fluent setters is automatically provided by the \mixin annotation.

\paragraph{Extending the Query Language} The previous query language can be extended with a new feature
\texttt{orderBy} which orders the result records by a field that users
specify. With \mixin programmers just need to extend the interface \texttt{Database} with new
features, and the return type of fluent setters in
\texttt{Database} is automatically refined to \texttt{ExtendedDatabase}:
\lstinputlisting[linerange=14-18]{../UseMixinLombok/src/casestudy/sql/DatabaseTest.java}% APPLY:linerange=FLUENT_DATABASE_EXT
In this way, when a query is created using \texttt{ExtendedDatabase},
all the fluent setters return the correct type, and not the old \texttt{Database} type, which would prevent calling
\texttt{orderBy}.

\lstinputlisting[linerange=39-39]{../UseMixinLombok/src/casestudy/sql/DatabaseTest.java}% APPLY:linerange=FLUENT_QUERY2

Languages like Smalltalk and Dart offer method cascading and avoid the need for fluent setters.
This is achieved at the price of introducing additional syntax and 
intrinsically relies on an imperative setting. 
Our approach supports both fluent setters and (functional) 
fluent withers.

\subsection{A Maze Game}
This case study is a simplified variant of a Maze game, which is often
used~\cite{gof,bono14}
to evaluate code reuse ability related to inheritance and design
patterns. In the game, there is a player with the goal of collecting
as many coins as possible. She may enter a room with several doors to
be chosen among. This is a good example because it involves code reuse
(different kinds of doors inherit a common type, with different
features and behavior), multiple inheritance (a special kind of door
may require features from two other door types) and it also shows how
to model operations \texttt{symmetric sum}, \texttt{override} and
\texttt{alias} from trait-oriented programming. The game has been
implemented using plain Java 8 and default methods by Bono
et. al~\cite{bono14}, and the code for that implementation is
available online. We reimplemented the game using \mixin. Due to space
constraints, we omit the code here. The following table summarizes
the number of lines of code and classes/interfaces in each implementation:

\vspace{5pt}
{\fontsize{8}{10}\selectfont
\begin{tabular}{ccc}
\hline \rowcolor[HTML]{C0C0C0}
            & SLOC   & \# of classes/interfaces \\ \hline
Bono et al. & 335    & 14                       \\
Ours        & 199    & 11                       \\
\rowcolor[HTML]{C0C0C0}
Reduced by  & 40.6\% & 21.4\%                   \\ \hline
\end{tabular}
}
\vspace{5pt}


\noindent The \mixin annotation reduced the interfaces/classes used
in Bono et al.'s implementation by 21.4\% (from 14 to 11), due to the replacement of instantiation classes with
generated \texttt{of} methods. The number of source lines of code 
(SLOC)
was reduced by 40\% due to both the removal of instantiation overhead and
generation of getters/setters. %\bruno{explain why? we generate code for getters/setters?}.
%To ensure a fair comparison, we used the same coding style as Bono et al.'s.

\begin{comment}
\subsection{Refactoring a Compiler}
The last case study applies our approach to a larger code base of a simple
one-pass compiler for \lstinline{C0} (a subset of \lstinline{C}). This compiler
is a handwritten, monolithic compiler used for educational purposes at Aarhus
University, Denmark\footnote{Source code available at
  http://cs.au.dk/~mis/dOvs/Czero.java}. \lstinline{C0} is restricted to
integers, several control structures and function declaration/definition and
basic I/O statements. All the code (including parsing, code generation for
various structures, error handling) is entangled in one big file. Following our
approach with \mixin, we refactored the code to several smaller interfaces
(including \lstinline{Constants}, \lstinline{MemberFields}, interfaces for
parsing different structures, etc). After the refactoring, the new code does not
reduce code as previous case studies. SLOC for original code without comments
and blank lines is 828, while ours is 830 (for fair comparison, we put all
interfaces into one file when calculating SLOC). However, the case study shows
two things: Classless Java can be applied to real code base such as compilers;
code becomes more modular without the sacrifice of code amount/simplicity.
\marco{Why we do not cite the result of the former study? the one with object algebraes
where the number of lines grown quite a while?}
\end{comment}

\subsection{Refactoring an Interpreter}\label{subsec:int}
\begin{comment}
The last case study refactors the code from an interpreter for
a Lisp-like language
\lstinline{Mumbler}\footnote{https://github.com/cesquivias/mumbler/tree/master/simple},
which is created as a tutorial for the Truffle
Framework~\cite{wurthinger2013one}.  Keeping a balance between
simplicity and useful features, \lstinline{Mumbler} contains numbers,
booleans, lists (encoding function calls and special forms such as
if-expression, lambdas, etc). In the original code base, which
consists of 626 SLOC of Java, only one operation \texttt{eval} is
supported. Extending Mumbler to support one more operation, such as a
pretty printer (making it 661 SLOC), would normally require changing the existing code base
directly.

Our refactoring applies the pattern presented in
Section~\ref{subsec:ep} to the existing Mumbler code base to improve
its modularity and extensibility. Using the refactored code base (560 SLOC) it  
becomes possible to add new operations modularly, and to support
independent extensibility (677 SLOC). We add one more
operation \texttt{print} to both the original and the refactored code
base. In the original code base the pretty printer is added
non-modularly by modifying the existing code. In the refactored code
base the pretty printer is added modularly. 
Although the code in the refactored version is slightly increased (by 2.4\% SLOC), the
modularity is greatly increased, allowing for improved reusability and maintainability.
\end{comment}

The last case study refactors the code from an interpreter for
a Lisp-like language
\lstinline{Mumbler}\footnote{https://github.com/cesquivias/mumbler/tree/master/simple},
which is created as a tutorial for the Truffle
Framework~\cite{wurthinger2013one}.  Keeping a balance between
simplicity and useful features, \lstinline{Mumbler} contains numbers,
booleans, lists (encoding function calls and special forms such as
if-expression, lambdas, etc). In the original code base, which
consists of 626 SLOC of Java, only one operation \texttt{eval} is
supported. Extending Mumbler to support one more operation, such as a
pretty printer \texttt{print}, would normally require changing the existing code base
directly.

Our refactoring applies the pattern presented in
Section~\ref{subsec:ep} to the existing Mumbler code base to improve
its modularity and extensibility. Using the refactored code base it
becomes possible to add new operations modularly, and to support
independent extensibility. We add one more
operation \texttt{print} to both the original and the refactored code
base. In the original code base the pretty printer is added
non-modularly by modifying the existing code. In the refactored code
base the pretty printer is added modularly. 
Although the code in the refactored version is slightly increased (by 2.4\% SLOC), the
modularity is greatly increased, allowing for improved reusability and maintainability:

\vspace{0.5em}
{\fontsize{8}{10}\selectfont
\begin{tabular}{llll}
\hline
\rowcolor[HTML]{C0C0C0} 
Code               & SLOC & Code               & SLOC \\ \hline
original (\texttt{eval})         & 626  & original (\texttt{eval}+\texttt{print}) & 661  \\ 
refactored (\texttt{eval})         & 560  & refactored (\texttt{eval}+\texttt{print}) & 677  \\ \hline
\end{tabular}
}

