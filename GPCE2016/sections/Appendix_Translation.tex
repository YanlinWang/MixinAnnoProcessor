\section{What  \mixin Generates}\label{sec:appendixtranslation}

This section presents a formal definition for most of the generated methods by $\mixinAnn$. 

\subsection{Translation}\label{subsec:translation}

The translation functions of $\mixinAnn$ and $\weakAnn$
are presented in Figure~\ref{figure:translation}. Note that
it is necessary to explicitly check if the interface is valid
for annotation:

\noindent$\begin{array}{ll}
\InTextDef{7ex}{\valid(\C_0)}{\forall \m\in\dom(\C_0),\mbox{ if }\mh\QM; =
            \mBody(\m, \C_0),}\\
            \hspace{.24in}\mbox{ one of the following cases is satisfied:}\\
\tab\tab
\isField(\method), \isWith(\method, \C_0) \mbox{ or }
\isSetter(\method,\C_0)\\
\InTextDef{24ex}{\isField(\C\ \m\oR\cR\QM;)}{
\mnot\ \specialName(\m)}\\
\InTextDef{24ex}{\isWith(\C'\ \QM{with#}\m \oR \C\ \x\cR\QM;, \C_0)}{}\\
\hspace{.3in}\C_0 <: \C', \mBody(\m, \C_0) = \C\ \m\oR\cR\QM;
\ \mand\ \mnot\ \specialName(\m)\\
\InTextDef{24ex}{\isSetter(\C'\ \QM_\m \oR \C\ \x\cR\QM;, \C_0)}{}\\
\hspace{.3in}\C_0 <: \C', \mBody(\m, \C_0) = \C\ \m\oR\cR\QM;
\ \mand\ \mnot\ \specialName(\m)\\

%\isClone:&\isClone(\C\ \QM{clone}\oR\cR\QM;, \C_0)\tab \mif\ \C_0 <: \C \\
%\isImplemented:&\isImplemented(\method) \tab\mbox{iff }\method\mbox{ not of form }\mh\QM;
%\QM{default}\ \mh\mbox{\Q@\{return \_;\}@}) = \QM{true} \\
%&\isImplemented(\QM{static}\ \mh\mbox{\Q@\{return \_;\}@}) = \QM{true} \\
\end{array}$

\noindent That is, we can categorize all abstract methods in a pattern that we
know how to implement: it is either a field getter, a with method or a setter.

Moreover, we check that the method \Q@of@ is not already defined by the user.
In the formalization an existing definition of the \Q@of@ method is an error. However,
in the prototype (which also needs to account for overloading), the check is more complex
as it just checks that an \Q@of@ method with the same signature of the one being
generated is not already present.

We write $\QM{with#}\m$ to append $\m$ to
\QM{with}, following the camelCase rule. The first letter of $\m$
must be lower-case and is changed to upper-case upon appending. For
example \QM{with#foo}=\QM{withFoo}.  Special names $\specialName(\m)$
are \QM{with} and all identifiers of the form $\QM{with#}\m$.

\paragraph{The $\otherMethod$ function:}
$\otherMethod(\C_0,\methods)$ is defined as follows:

\noindent$\begin{array}{ll}
\InTextDef{16em}{\C_0\ \QM{with#}\m\oR \C\ \QM{_val}\cR\QM;\in
\otherMethod(\C_0,\methods)}{}\\
\hspace{.25in}\isWith(\mBody(\QM{with#}\m, \C_0), \C_0),\ 
\QM{with#}\m\notin\dom(\methods)\\
\InTextDef{16em}{\C_0\ \QM_\m\oR \C\ \QM{_val}\cR\QM;\in
\otherMethod(\C_0,\methods)}{}\\
\hspace{.25in}\isSetter(\mBody(\QM_\m, \C_0), \C_0),\ \QM_\m\notin\dom(\methods)\\
\end{array}$

\noindent The methods generated in the interface are \Q@with-@ and setters. %\Q@clone@.
%A complete formalization would also generate the \Q@with@.
The methods are generated when they are unimplemented in $\C_0$, because
the return types need to be refined.
To determine whether the methods need to be generated,
we check if such \Q@with-@ or setter methods %\Q@clone@
are required by $\C_0$, but not declared directly in $\C_0$.


\paragraph{The $\ofMethod$ function:}\label{subsec:ofmethod}
The function $\ofMethod$ generates the method \QM{of}, as an object factory. To
avoid boring digressions into well-known ways to find unique
names, %for the sake of this formalization
we assume that all methods with no parameters do not start with an underscore,
and we prefix method names with underscores to obtain valid parameter names for
\QM{of}.
\noindent$\begin{array}{l}
\InTextDef{5em}{\ofMethod(\C_0)}{
 \QM{static}\ \C_0\ \QM{of} \oR \C_1\ \QM_\m_1\QM,\ldots \C_n\ \QM_\m_n\cR\
\QM{\{}}\\
\tab\QM{return new}\ \C_0 \oR\cR\ \QM{\{}\\
\tab\tab \C_1\ \m_1 = \QM_\m_1\QM;\ldots \C_n\ \m_n = \QM_\m_n\QM; \\
\tab\tab
\C_1\ \m_1\oR\cR\ \QM{\{return }\ \m_1\QM{;\}}\ \ldots
\C_n\ \m_n\oR\cR\ \QM{\{return }\ \m_n\QM{;\}}\\
\tab\tab\withMethod(\C_1,\m_1,\C_0,\es_1)\ldots\withMethod(\C_n,\m_n,\C_0,\es_n)\\
\tab\tab\setterMethod(\C_1,\m_1,\C_0)\ldots\setterMethod(\C_n,\m_n,\C_0)\\
%\tab\tab\cloneMethod(\C_0,\es)\\
%\tab\tab\withMethod(\C_0)\\
\tab\QM{\};\}} \\
\InTextWith{\C_1\ \m_1\QM{();},\ldots \C_n\ \m_n\QM{();} = \fieldsFunc(\C_0)}\\
\hspace{.5in}\mbox{ and }\es_i=\m_1\QM,\ldots\QM, \m_{i-1}\QM,\QM{_val,}\m_{i+1}\QM,\ldots\QM, \m_n
\end{array}$

\noindent Note that, the function $\fieldsFunc(\C_0)$ denotes all the fields in the current interface:

\noindent$\begin{array}{ll}
\InTextDef{16ex}{\method\in\fieldsFunc(\C_0)}{}\\
\hspace{.3in}\isField(\method)\ \mand\
\method=\mBody(m^\method,\C_0)
\end{array}$

\noindent For methods inside the interface with the form $\C_i\ \m_i$\QM{();}
  \begin{itemize}
   \item $\m_i$ is the field name, and has type $\C_i$.
   \item $\m_i$\QM{()} is the getter and just returns the current field value.
   \item if a method \Q@with#@$\m_i()$ is required, then it is implemented by calling the \Q@of@ method using
    the current value for all the fields except for $\m_i$. Such new value is
    provided as a parameter. This corresponds to the expressions $\es_i$.
\item \QM_$\m_i$\QM($\C_i\ $\QM{ _val)} is the setter. In our prototype we use name $\m_i$, here we use the underscore to avoid modeling overloading.
%   \item similarly, for the \Q@clone@ method, \Q@of@ is called using the current value for all the fields.
   %\item To complete our generation, we need to generate setters, fluent setters and the with method.
   %\item \marco{should we just formalize setters?}
   \end{itemize}

The auxiliary functions are defined below. Note that we do not need to check if some header is a subtype of what we would generate, this is ensured by $\valid(\C_0)$.

\noindent$\begin{array}{l}
\InTextDef{10em}{\withMethod(\C,\m,\C_0,\es)}{}\\
\hspace{.2in}\C_0\ \QM{with#}\m\oR \C\ \QM{_val}\cR\ \QM{\{}
\QM{return}\ \C_0\QM{.of(}\es\QM{);\}} \\
\InTextWith{\mBody(\QM{with#}\m,\C_0) \mbox{ having the form }\mh\QM;}\\
\InTextDef{10em}{\withMethod(\C,\m,\C_0,\es)}{\emptyset\mbox{ otherwise}}\\
\InTextDef{10em}{\setterMethod(\C,\m,\C_0)}{}\\
\hspace{.2in}\C_0\ \QM_\m\oR \C\ \QM{_val}\cR\ \QM{\{}
 \m\QM{= _val;return this;\}} \\
\InTextWith{
\mBody(\QM_\m,\C_0) \mbox{ having the form }\mh\QM;}\\
\InTextDef{10em}{\setterMethod(\C,\m,\C_0)}{\emptyset\mbox{ otherwise}}\\
%\cloneMethod:&\cloneMethod(\C_0,\es)=
%\C_0\ \QM{clone()\{return}\ \C_0\QM{.of(}\es\QM{);\}} \\
%&\mbox{iff }
%\mBody(\QM{clone},\C_0) \mbox{ is of form }\mh\QM;\\
%&\cloneMethod(\C_0,\es)=\emptyset\mbox{ otherwise}\\
\end{array}$

\subsection{Other Features}\label{subsec:otherfeatures}

We do not formally model non-fluent setters and the \Q@with@ method.
An informal explanation of how those methods are generated is given next:
\begin{itemize}
\item For methods inside the interface with the form \Q@void @$\m$\QM($\C\ \x$\QM{);}:
  \begin{itemize}
    \item Check if method $\C\ \m$\Q@();@ exists. If not, generate error (that
      is, $\valid(\C_0)$ is false).
    \item Generate the implemented setter method inside \Q@of@:\\*
      \Q@public void @$\m$\Q@(@$\C$\Q@ _val) { @$\m$\Q@=_val;}@\\*
      There is no need to refine the return type for non-fluent setters, thus we
      do not need to generate the method header in the interface body itself.
    \end{itemize}
\item For methods with the form $\C'\ $\QM{with(}$\C\ \x$\QM{);}:
  \begin{itemize}
   \item $\C$ must be an interface type (no classes or primitive types).
    \item As before, check that $\C'$ is a supertype of the current interface type $\C_0$.
    \item Generate implemented \Q@with@ method inside \Q@of@:\\*
           \Q@public @$\C_0\ $\Q@with(@$\C$\Q@ _val) { @\\*
           \Q@  if(_val instanceof @$\C_0$\Q@){return (@$\C_0$\Q@)_val;}@\\*
${}_{}$\Q@  return @$\C_0$\Q@.of(@$\e_1\ldots\e_n$\Q@);}@\\*
 with $\e_i=$\Q@_val.@$\m_i$\Q@()@ if $\C$ has a $\m_i$\Q@()@ method where
 $\m_1\ldots\m_n$ are fields of $\C_0$; otherwise $\e_i=\m_i$.
    \item If needed, as for \Q@with-@ and setters, generate the method headers with refined return types in the interface.
 \end{itemize}

%\item For methods with the form $\C'\ \m$\QM($\C\ \x$\QM{);}:
 % \begin{itemize}
  %  \item As for before, check if exist method $\C\ \m$\Q@();@. Also, check that $\C'$ is a supertype of the current interface type $\C_0$.
   % \item Generate implemented setter method inside \Q@of@:\\*
    %       \Q@public @$\C_0\ \m$\Q@(@$\C$\Q@ _val) { @$\m$\Q@=_val; return this;}@
   % \item If needed, as for \Q@with-@ and clone, generate the method header with refined return type in the interface.
 % \end{itemize}
\end{itemize}

