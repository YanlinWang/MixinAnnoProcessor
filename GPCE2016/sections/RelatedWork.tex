\section{Related Work}\label{sec:related}
%In this section we discuss related work and how it compares to Classless Java.

%\yanlin{need to discuss: whether the comments for FTJ is fair.}

%No, was not fair, I know that work, it does modular type checking, and
%more work after that they extend with more safe/modular typing things.
%also, Java8 IS a language extension on Java
%However, language extensions (including FTJ)
%have a natural drawback: the programmer has to learn new syntax. In contrast,
%our approach is completely compatible with the current Java language, so that
%programmers don't need to pay any learning cost to adapt to this new classless
%programming pattern. Another drawback which is particular for FTJ is that FTJ
%doesn't have type for traits, hence the correctness check of trait is done when
%type-checking classes. This choice makes the design of FTJ simpler but lost
%typechecking efficiency (one trait will be potentially checked multiple times if
%it is used in multiple classes).

\begin{comment}
\paragraph{Models of \objectoriented programming}
There are traditionally two big styles of OOP: \classbased and
\prototypebased. The \classbased tradition started with Simula~\cite{Dahl66simula},
and is the dominant style of OOP today. Major languages such
as Java, C\#, C++ or Scala favour a \classbased OOP style. Very often
\classbased languages are statically typed and they have interfaces or
some other similar mechanisms such as traits. However, in contrast to
IB, interfaces or traits in CB are not meant to
replace classes, but rather, they complement classes. Normally
interfaces or traits define the signatures of the methods and
\emph{may} have some behaviour (such as in Java 8 interfaces or Scala
traits). However classes are still responsible for object
instantiation and handling state. \Prototypebased languages started
with Self~\cite{Ungar87self} and are still widely used today via languages like
JavaScript. Frequently \prototypebased languages are dynamically
typed. IB is clearly closer to CB. However the
key difference is that classes are not needed in IB. Object interfaces recover the responsibility of object
instantiation, and abstract state operations provide a way to deal
with state. A primary motivation of IB is to improve
modularity and flexibility. Central to these goals are the
more expressive and automated forms of type-refinement, and support
for multiple interface inheritance.
\end{comment}

\para{Multiple inheritance in \objectoriented languages}
Many authors have argued in favor or against multiple inheritance.
Multiple \emph{implementation}
inheritance is very expressive, but difficult to model and implement. It can
cause difficulties (including the famous diamond
problem~\cite{bracha90mixin,Sak89dis}, conflicting methods, etc.) in reasoning about
programs. To allow for expressive power and simplicity, many
models have been proposed, including C++ virtual inheritance,
mixins~\cite{bracha90mixin}, traits~\cite{scharli03traits}, and hybrid models
such as CZ~\cite{malayeri2009cz}.  They provide novel programming architecture
models in the OO paradigm. In terms of restrictions set on these models, C++
virtual inheritance aims at a relative general model; the mixin model adds some
restrictions; and the trait model is the most restricted one (excluding
state, instantiation, etc).

%\paragraph{C++ Approach.}
C++ has a general solution to multiple inheritance by
virtual inheritance, dealing with the diamond problem by keeping only
one copy of the base class~\cite{ellis1990annotated}. However it
suffers from object initialization problem~\cite{malayeri2009cz}.
It bypasses constructor calls to virtual superclasses, which can
cause serious semantic errors. In our approach, the \mixin
annotation has full control over object initialization,
and the mechanism is transparent to users. If users are not satisfied with the default
 \texttt{of} method, customized factory methods can be provided.

%\paragraph{Mixins.}
Mixins are more restricted than the C++ approach. Mixins allow naming
components that can be applied to various classes as reusable functionality
units. However, % they suffer from linearization: the order of mixin application
% is relevant in often subtle and undesired ways. This hinders their usability
% and the ability of resolving conflicts:
the linearization (total ordering) of
mixin inheritance cannot provide a satisfactory resolution in some cases and
restricts the flexibility of mixin composition. To fight against this limitation, an
algebra of mixin operators is introduced~\cite{ancona2002calculus}, but this
raises the complexity, especially when constructors and fields
are considered~\cite{marco09FJigsaw}. Scala traits~\cite{scala-overview} are in fact more like linearized mixins.
Scala avoids the object initialization
problem by disallowing constructor parameters, causing no ambiguity in cases
such as the diamond problem. However this approach has limited expressiveness, and
suffers from all the problems of linearized mixin composition. % Other languages, such as
% Python, also use linearized mixins.
\begin{comment}
Python also offers multiple inheritance via linearized mixins. Indeed, in python any class is implicitly a mixin, and mixin composition informally expressed as\\*
\Q@ class A use B,C {...new methods...}@\\*
can be expressed in python as \\*
\Q@ class Aux: ...new methods...@\\*
\Q@ class A(B,C,Aux): pass@
\end{comment}
\noindent Java interfaces and default methods do not use
linearization: the semantics of Java \textbf{extends} clause in
interfaces is unordered and symmetric.

% However, in pure Java, there is no mechanism for creating objects in
% interfaces. Also, our approach supports proper constructor mechanism.

%\paragraph{CZ.}
Malayeri and Aldrich proposed a model CZ~\cite{malayeri2009cz} which
aims to do multiple inheritance without the diamond problem.
Inheritance is divided into two concepts: inheritance dependency
and implementation inheritance. Using a combination of
\texttt{requires} and \texttt{extends}, a program with diamond
inheritance is transformed to one without diamonds. Moreover,
fields and multiple inheritance can coexist. However untangling
inheritance also untangles the class structure. In CZ, not only the
number of classes, but also the class hierarchy complexity
increases. IB does not complicate the hierarchical structure, and state
also coexists with multiple inheritance.

%\paragraph{Traits and Java's default methods}
Simplifying the mixins approach, traits~\cite{scharli03traits} draw a strong
line between units of reuse and object factories. Traits, as units of reusable
code, contain only methods as reusable functionality, ignoring state and state
initialization. Classes, as object factories, require functionality from
(multiple) traits. Java 8 interfaces are closely related to
traits: concrete method implementations are allowed (via the \textbf{default}
keyword) inside interfaces. The introduction of default methods opens the gate
for various flavors of multiple inheritance in Java. Traits offer an algebra
of composition operations like sum, alias, and exclusion, providing explicit conflict
resolution. Former work~\cite{bono14} provides details on mimicking the trait
algebra through Java 8 interfaces.
\begin{comment}
We briefly recall the main points of their
encoding; however we propose a different representation of \textbf{exclusion}.
The first author of~\cite{bono14} agreed (via personal communication) that
our revised version for exclusion is cleaner, typesafe and more direct.
\newcommand\shortItem{\vspace{-1ex}}
\begin{itemize}
\item \textbf{Symmetric sum} can be obtained by simple multiple inheritance between interfaces.
    \begin{lstlisting}
    interface A { int x(); }
    interface B { int y(); }
    interface C extends A, B {}
    \end{lstlisting}
\shortItem
\item \textbf{Overriding} a conflict is obtained by specifying which super interface take precedence.
    \begin{lstlisting}
    interface A { default int m() {return 1;} }
    interface B { default int m() {return 2;} }
    interface C extends A, B {
        default int m() {return B.super.m();} }
    \end{lstlisting}
\shortItem
\item \textbf{Alias} is creating  a new method delegating to the existing super interface.
    \begin{lstlisting}
    interface A { default int m() {return 1;} }
    interface B extends A {
        default int k() {return A.super.m();} }
    \end{lstlisting}
\shortItem

\item \textbf{Exclusion}: exclusion is also supported in Java, where method declarations can hide the default methods correspondingly in the super interfaces.
    \begin{lstlisting}
    interface A { default int m() {return 1;} }
    interface B extends A { int m(); }
    \end{lstlisting}
\shortItem
\end{itemize}
\end{comment}
There are also proposals for extending Java with traits. For
example, FeatherTrait Java (FTJ)~\cite{Liquori08ftj} extends
FJ~\cite{Igarashi01FJ} with statically-typed traits, adding trait-based
inheritance in Java.  Except for few, mostly syntactic details, their work can
be emulated with Java 8 interfaces. There are also extensions to the original
trait model, with operations (e.g. renaming~\cite{reppy2006foundation}, which breaks
structural subtyping) that default methods and interfaces cannot
model.

\para{Traits vs Object Interfaces.}
We consider object interfaces an alternative to traits or mixins. In
trait model two concepts (traits and classes) coexist and cooperate. Some
authors~\cite{BettiniDSS13} see this as good language design fostering good
software development by helping programmers think about the program structures.
However, others see the need of two concepts and the
absence of state as drawbacks of this model~\cite{malayeri2009cz}.  Object
interfaces are units of reuse, and meanwhile provide factory methods
for instantiation and support state. Our approach promotes the use of
interfaces in order to exploit the modular composition
offered by interfaces. Since Java was designed for classes, a direct classless
programming style is verbose and unnatural. However, annotation-driven code
generation is enough to overcome this difficulty and the resulting programming
style encourages modularity, composability and reusability. %, by keeping a strong OO feel.
In that sense, we promote object interfaces as being both units of
reuse and object factories. Our practical experience shows that
separating the two notions leads to lots of boilerplate code, and is quite
limiting when multiple inheritance with state is required.  Abstract state
operations avoid the key difficulties associated with multiple inheritance and
state, while still being quite expressive. Moreover the ability to support
constructors adds expressivity, which is not available in approaches
such as Scala's traits/mixins.
%\footnote{Scala traits have
%  no constructor, but \Q@var@s and (\Q@lazy@) \Q@val@s can be
%  ``pre-initialized''.}

%Nevertheless, we
%believe that in other languages the separation can be more practical
%than in Java, and we do not necessarily advocate that merging the two
%notions is a better approach in general.

%There are other limitations of our current approach, but they may be addressed
%in future work (see Section~\ref{sec:futurework}).



% \footnote{The original
%  notion of traits by Scharli et al. prescribes, among other things,
%  that: 1) a trait provides a set of methods that implement behavior;
%  and 2) a trait does not specify any state variables, so the methods
%  provided by traits do not access state variables directly. Java 8
%  interfaces follow similar principles too. Indeed, a detailed
%  description of how to emulate trait-oriented programming in Java 8
%  can be found in the work by Bono et al.~\cite{bono14}. The Java 8
%  team designing default methods, was also fully aware of that
%  secondary use of interfaces, but it was not their objective to model
%  traits: ``The key goal of adding default methods to Java was
%  "interface evolution", not "poor man's
%  traits"''~\cite{goetz13default}. As a result, they were happy to
%  support the secondary use of interfaces with default methods as long
%  as it did not make the implementation and language more complex.}

\begin{comment}
\marcoT{We could use Scala Traits to encode IB, but the simpler Java8
  model is a better starting point: scala traits can declare \Q@def, val, var@ and
  \Q@lazy val@ However, while \Q@def@s can be overridden by
  \Q@val@/\Q@var@s, \Q@val@/\Q@var@s can not be overridden.
   A\Q@val@/\Q@var@ member is predestined to be
  instance state, and this is in sharp contrast with our model where
  the state is only a private detail of instances, and do not
  contaminate interface/trait/classes.
  }
\end{comment}




\begin{comment}
Besides, we support more features than the original trait model:
\begin{itemize}
\item We provide \texttt{of} methods for the annotated interfaces. During annotation processing time, the ``fields'' inside an interface are observed and a static method \texttt{of} is automatically injected to the interface with its arguments correspondingly. Such a method is a replacement to the constructors in original traits, making instantiation more convenient to use.
That is, in our approach there are only interfaces, our model requires a single concept, while the trait model requires traits \emph{and} classes.

\item We provide \texttt{with-} methods as auxiliary constructors. A \texttt{with-} method is generated for each field, just like record update, returning the new object with that field updated
%. A \texttt{clone} method is generated for the interface, returning a copy of the current object.
Furthermore, we do automatic return type refinement for this kind of methods. This feature is comparatively useful in big examples, making operations and behaviors more flexible.%, which we will demonstrate later.
\item We provide two options for generating setters. There are two kinds of setters which are commonly used, namely \textit{void setters} and \textit{fluent setters}. The only difference is that a fluent setter returns the object itself after setting, thus supporting a pipeline of such operations. The generation depends on which type of setter is declared in the interface by users.
\end{itemize}
\end{comment}



% \begin{enumerate}
% \item Mixins allows to name components that can be applied to various classes as
%   reusable functionality units. However, they suffer from linearization: the
%   order of mixin application is relevant in often subtitle and undesired
%   ways. constraints. This hinders their usability and their ability of resolving
%   conflicts: the linearization (total ordering) of mixin inheritance cannot
%   provide a satisfactory resolution in all cases and restricts the flexibility
%   of mixin composition.  To fight those limitations, an algebra of mixin
%   operators is introduced~\cite{ancona2002calculus}, but this raised the
%   complexity of the approach, especially when constructors and fields are
%   considered~\cite{marco09FJigsaw}.  Our approach does not have the
%   linearization problem, in that the semantics of Java \textbf{extends} clause
%   is unordered and symmetric.
% \item Simplifying the mixin algebras approach, traits draw a strong line between
%   units of reuse (traits) and object factories (classes) In this model,
%   traits~\cite{scharli03traits} units of reusable code, containing only methods
%   as reusable functionalities. Thus, no state/state initialization is
%   considered.

%   Classes act as object factories, requiring functionalities from multiple
%   traits.  Traits offers a trait algebra with operations like sum, alias and
%   exclusion, provided for explicit conflict resolution.

%   Concluding, (pure) traits do not allow state and they do not offers any reuse
%   instrument to ensure that state is coherently initialized when finally defined
%   in classes.  Traits can't be instantiated. This model requires two concepts
%   (traits and classes) to coexist and cooperate.

%   Some authors see this as good language design fostering good software
%   development by helping programmers to think about the structure of their
%   programs.  However, other authors see the need of two concepts and the absence
%   of state as drawbacks of this model. Our approach takes interfaces as units of
%   reuse, and meanwhile generates factory method for instantiation.

% \item C++ and Scala also try to provide solutions to multiple inheritance, but
%   both suffer from object initialization problems. Virtual inheritance in C++
%   provides another solution to multiple inheritance (especially the diamond
%   problem by keeping only one copy of the base class)~\cite{ellis1990annotated},
%   however suffers from object initialization problem as pointed out by Malayeri
%   et al.~\cite{malayeri2009cz}. It bypasses all constructor calls to virtual
%   superclasses, which would potentially cause serious semantic error. Scala
%   solution (very similar to linearised mixins, but misleadingly called traits in
%   the language) avoids this problem by disallowing constructor parameters,
%   causing no ambiguity in cases such as diamond problem.  This approach has
%   limited expressiveness, and suffers from all the problems of linearised mixin
%   composition.
%   Python also offers multiple inheritance via linearised mixins. Indeed in python any class is implicitly a mixin, and mixin composition informally expressed as\\*
%   \Q@ class A use B,C {...new methods...}@\\*
%   can be expressed in python as \\*
%   \Q@ class Aux: ...new methods...@\\*
%   \Q@ class A(B,C,Aux): pass@
%   Our approach not only does not involve linearised mixin problem, but also
%   support proper constructor mechanism.
% \end{enumerate}

% %\section{Comparing to traits and mixins}\label{sec:comparison}
\subsection{Multiple Inheritance in ClassLess Java}
\begin{comment}
Haoyuan

   - vs both: we do automatic return type refinement, which has useful applications
   (example: Expression Problem)

   - vs traits: we support of methods to create new objects (a replacement to constructors);
   Moreover we have the with and clone methods (we miss more applications for those). Show
   how to model the operations on traits; discuss operations that we cannot model
   (example: renaming).

   - vs mixins: we use the trait model of explicitly resolving conflicts. This is arguably
   better for reasoning.
\end{comment}

Our approach is based on forbidding the use of certain language constructs (classes), in order to rely on the modular composition offered by the rest of the language (interfaces).
Since Java was designed for classes, a direct class-less programming style is verbose and feel unnatural. However, annotation driven code generation is enough to overcome this difficulty, and
the resulting programming style encourages modularity, composability and reusability, by keeping a strong object oriented feel.
We consider our approach as an alternative to traits or mixins.

A comparison on how to to emulate the original operations on traits using Java8 can be find in\cite{bono14}. We briefly recall the main points of their encoding; however we propose a different representation of
 \textbf{exclusion}.
The author of ~\cite{bono14} agree that our revised version is cleaner, typesafe and more direct.\bruno{is this something to 
say in the paper?}

%%\newcommand\shortItem{\vspace{-1ex}}
\begin{itemize}
\begin{comment}%I shoreten up a lot after. I think this version is better but longer, and this content is just repeating bono at all.
\item \textbf{Symmetric sum}: the symmetric composition of two disjoint traits is achieved by simply implementing two interfaces in Java correspondingly, without overriding any method. The composition relies on multiple inheritance on interfaces, which is supported by Java. Below is a simple example.
    \begin{lstlisting}
    interface A { int x(); }
    interface B { int y(); }
    interface C extends A, B {}
    \end{lstlisting}
\item \textbf{Override}: the overriding operation (also known as asymmetric/preferential sum) is modelled by implementing many interfaces, while overriding some methods inside. The code below gives an example of explicitly specifying which super interface take precedence, regarding conflict on a specific method.
    \begin{lstlisting}
    interface A { default int m() {return 1;} }
    interface B { default int m() {return 2;} }
    interface C extends A, B { default int m() {return B.super.m();} }
    \end{lstlisting}
    Here the method \texttt{m()} in interface \texttt{C} simply inherits from \texttt{B.m()}.
\item \textbf{Alias}: an alias operation adds a new name to an old method when creating the new trait. In Java, we just create a new method with reference to the existing method in its super interface. See the example below, where the new method \texttt{k()} is an alias of the existing method \texttt{m()}.
    \begin{lstlisting}
    interface A { default int m() {return 1;} }
    interface B extends A { default int k() {return A.super.m();} }
    \end{lstlisting}
\end{comment}

\item \textbf{Symmetric sum} can be obtained by simple multiple inheritance between interfaces.
\shortItem
    \begin{lstlisting}
    interface A { int x(); }    interface B { int y(); }    interface C extends A, B {}
    \end{lstlisting}
\shortItem
\item \textbf{Overriding} a conflict is obtained by specifying which super interface take precedence.
\shortItem
    \begin{lstlisting}
    interface A { default int m() {return 1;} } 
    interface B { default int m() {return 2;} }
    interface C extends A, B { default int m() {return B.super.m();} }
    \end{lstlisting}
\shortItem
\item \textbf{Alias} is creating  a new method delegating to the existing super interface.
\shortItem
    \begin{lstlisting}
    interface A { default int m() {return 1;} }
    interface B extends A { default int k() {return A.super.m();} }
    \end{lstlisting}
\shortItem

\item \textbf{Exclusion}: exclusion is also supported in Java, where method declarations can hide the default methods correspondingly in the super interfaces.
\shortItem
    \begin{lstlisting}
    interface A { default int m() {return 1;} }
    interface B extends A { int m(); }
    \end{lstlisting}
\shortItem
\end{itemize}

Besides, we support more features than the original trait model:
\begin{itemize}
\item We provide \texttt{of} methods for the annotated interfaces. During annotation processing time, the ``fields'' inside an interface are observed and a static method \texttt{of} is automatically injected to the interface with its arguments correspondingly. Such a method is a replacement to the constructors in original traits, making instantiation more convenient to use.
That is, in our approach there are only interfaces, our model requires a single concept, while the trait model requires traits \emph{and} classes.

\item We provide \texttt{with-} methods as auxiliary constructors. A \texttt{with-} method is generated for each field, just like record update, returning the new object with that field updated
%. A \texttt{clone} method is generated for the interface, returning a copy of the current object.
Furthermore, we do automatic return type refinement for these kind of methods. This feature is comparatively useful in big examples, making operations and behaviours more flexible.%, which we will demonstrate later.
\item We provide two options for generating setters. There are two kind of setters which are commonly used, namely \textit{void setters} and \textit{fluent setters}. The only difference is that a fluent setter returns the object itself after setting, thus supporting a pipeline of such operations. The generation depends on which type of setter is declared in the interface by users.
\end{itemize}

These are the additional features supported by our model, conversely, there are certain operations we cannot model, such as method renaming (as in [Reppy2006]), which breaks structural subtyping.

There are other limitations of our current approach, but they may be addressed
in future work (see Section~\ref{sec:futurework}).




\para{Automatic generation of getters and setters}
% \bruno{Need to discuss automatic generation of setter and
%   getters, and then discuss this text about injections. See reviews
%   and reply.}
is an old idea used in languages such as
Self~\cite{Ungar1987self}, Dart~\cite{Dart} and Newspeak~\cite{Bracha_thenewspeak}.
% By declaring fields in Newspeak and other languages,
The programmers specifies field signatures and (critically) the intention of storing such information,
then the language generates getters and setters.
Once state is abstracted away, it is well known that state
access can be replaced with computation, but the type of the field stays the same.
We do the opposite, the idea of the field is generated starting from signatures of
getters and setters. % Our fluid approach to state is specular:
In our approach, the intention of storing the information is not expressed by the programmer and set in stone
but can vary by inheritance.
In this case, the underling type of the field can be changed by our fluid state,
 and \Q@with@ methods provide the right injection from the old type to the new.

\para{ThisType and MyType}
%\marco{ this section is unfair. There are unsolved problems with ThisType in negative positions, namely it breaks subtyping. The most common solution is to allow calling some methods only when the exact type is known. This demolishes most advantages of interfaces.}
Object interfaces support automatic type-refinement.
Type refinement is part of a bigger topic in \classbased languages: expressing and
preserving type recursion and (nominal/structural) subtyping at the same time.
One famous attempt in this direction is
\emph{MyType}~\cite{bruce1994paradigmatic}, representing the type of
\textbf{this}, changing its meaning along with inheritance. However when
invoking a method with MyType in parameter positions, the exact type of the
receiver must be known. This is a big limitation in \classbased OO
programming and is exasperated by the \interfacebased programming we propose: no
type is ever going to be exact since classes are not explicitly used. A recent
article~\cite{Saito2013933} proposes two
new features: exact statements and nonheritable methods. Both are
related to our work: 1) any method generated inside the \Q@of@ method is indeed
non-inheritable since there is no class name to extend from; 2) exact
statements (a form of wild-card capture on the exact run-time type) could
capture the ``exact type'' of an object even in a class-less
environment.
Admittedly, MyType greatly enhances the expressivity and extensibility of
\objectoriented programming languages. Object interfaces %use covariant-return types to
simulate some uses of MyType. But this approach only works for refining
return types, whereas MyType is more general, as it also works for
parameter types. Our approach to covariantly refine state can recover
some of the additional expressivity of MyType. As illustrated with our examples, object interfaces are still very useful in many
practical applications, yet they do not require additional
complexity from the type system.

% The addition of MyType to a language will allow easy definition of
%binary methods, methods with recursive types (i.e., the same type of the
%receiver appears in the argument or return positions of methods), etc.


\begin{comment}

\paragraph{Comparison with other Lombok annotations}\marco{if we keep it, will go in related work}
The Lombok project provides a set of predefined annotations, including constructor
generators similar as ours (e.g., \Q:@NoArgsConstructor:,
\Q:@RequiredArgsConstructor: and \Q:@AllArgsConstructor:). They
generate various kinds of constructors for \emph{classes}, with or without
constructor arguments. This set of annotations is of great use, especially when
used together with other features provided in Lombok (e.g.,
\Q:@Data:). Moreover, the implementation of these annotations in Lombok
gives us hints on how to implement \mixin. However, none of these annotations
can model what we are doing with \mixin - generating constructor-methods
(\textbf{of}) for \emph{interfaces}. Apart from constructors, \mixin also
provides other convenient features (including generating fluent setters, type
refinement, etc), which the base Lombok project does not provide.
Finally, while \mixin is formalized, none of Lombok's annotations have been
studied in a formal way.



\subsection{Type-Directed Translations/Syntactic Sugar}
\marco{I'm tring to merge this and the next one}
Language extensions are often implemented as syntactic sugar of the base
language. For example, Scala compiler supports XML syntax in normal Scala code
directly (after Scala ?, users need to import \texttt{scala-xml} library
manually). However, this approach is hard in terms of implementation, because it
requires extending the compiler. Also, this approach does not support combining
multiple extensions into one.

SugarJ~\cite{erdweg11sugarj} is a Java-based extensible programming language
that allows programmers to extend the base language with custom language
features by definitions in meta-DSLs (SDF, Stratego, etc).
\yanlin{Is new syntax really a ``drawback''? I think for some system, like
  SugarJ, one of purpose IS to introduce these new syntax.}  Drawbacks: new
syntax. To create the extension, programmers have to work with multiple
languages (SDF, stratego, etc) while our approach works totally in Java
environment.

We can model certain types of language extensions with annotations
only, but those extensions do not introduce new syntax: they
merely do automatic code generation.
\end{comment}
