%Table~\ref{table:javabug} shows the method overriding bug in Java. In the
%example, there are 6 interfaces \texttt{A1,A2,A3,B1,B2,B3} with methods all
%named as \texttt{m} which are either abstract or default methods. If we define a
%new interface \texttt{M} that extends two of these interfaces, then the method
%overriding result is shown in the table. For example, row2 col2 means \texttt{M
%  extends A1,B1}. The result is \texttt{ERROR} because the abstract method
%\texttt{T m();} in \texttt{A1} conflicts with the method \texttt{default T m()
%  {...}} in \texttt{B1}. Readers may also figure out other extending cases by
%following this way of interpretation.

% The interesting case in row4 col2 where
%Indeed, to be coherent with the idea of \textbf{conservative error}, the case should not be accepted.
%We do not see how this should behave differently from \textbf{B1},\textbf{A1}, and \textbf{B3},\textbf{A3}.
%We fear that the only retro-compatible fix for this strange behavior is to accept all the cases of \textbf{conservative error} in a future version of Java.
%the experimental result is \texttt{B3.m}. Because if we think consistently, this

%\lstinputlisting[linerange=4-9]{../UseMixinLombok/src/methodshadowing/test5/Test5.java} % APPLY:linerange=JAVABUG

%In our approach, we choose to not model this strange behavior (a bug?).
%Our auxiliary function $\mBody(\m,\C)$ enforce the \textbf{conservative error} strategy.
%The rest of our formalization is parametric with the definition of $\mBody(\m,\C)$, thus if Java changes its resolution strategy to a more permissive one, only minor adaptations in $\mBody(\m,\C)$ would be needed.

\begin{comment}
\subsubsection{Auxiliary function: \textsf{mtype}}
- \textsf{mtype(m, C)} : the signature of method m in C.

\[ \inferrule{
  IT(T) = \text{\emph{ann} interface } C \{ \overline{M} \} \\
  E \spc m(\overline{D} \spc \overline{x}) \{ \text{return } e; \} \in M}
{ \textsf{mtype(m,T)} = \overline{D} \to E } \]

\[ \inferrule{
  IT(T) = \text{\emph{ann} interface } C \{ \overline{M} \} \\
  m \notin M}
{ \textsf{mtype(m,T)} = \emptyset } \]

\[ \inferrule{
  IT(T) = \text{\emph{ann} interface } C \text{ extends } C_1,...,C_k \{ \overline{M} \} \\
  E \spc m(\overline{D} \spc \overline{x}) \{ \text{return } e; \} \in M}
{ \textsf{mtype(m,T)} = \overline{D} \to E } \]

\[ \inferrule{
  IT(T) = \text{\emph{ann} interface } C_0 \text{ extends } \overline{C} \{
  \overline{M} \} \\
  m \notin M}
{ \textsf{mtype(m,T)} = \bigcup \textsf{mtype}(m,\overline{D}) } \]
\end{comment}

%The \textsf{shadow} function takes two same methods (with the same name and types of arguments), and return the method which shadows the other during inheritance.
%exmples to motivate our design
%interface A{static String m(){return "A";}}
%interface C extends A{
%	default String dm(){
%		  this.m();//wrong in java
%		  A.m();
%		  C.m();//wrong in java
%		}
%}
%
%
%(1) Static methods are not inherited. Also, if one of $\{body_1,body_2\}$ is null, \textsf{shadow} simply returns the other one. Hence

\begin{comment}
%Now it is correct, but may be we do not need it?
We abbreviate typing statements on
sequences in a simple way, writing $\Gamma \vdash \overline{t}:\overline{C}$ as
shorthand for $\Gamma \vdash t_1:C_1,..., \Gamma \vdash t_n:C_n$.
\end{comment}



\begin{comment}
\subsubsection{Method Typing}


\[
\inferrule
{ }
{T_0 \spc m(\overline{T} \spc \overline{x}); \text{ OK IN I} }
\quad \textsc{(T-Meth)}
\]

\[
\inferrule
{\overline{x}:\overline{T} \vdash e:S \\ S <: T_0}
{T_0 \spc m(\overline{T} \spc \overline{x}) \text{ \{ return } e;\} \text{ OK IN
    I} \\\\ \Gamma \vdash \textbf{this}:I }
\quad \textsc{(T-MethBody)}
\]

\[
\inferrule
{IT(I)=\text{interface } I \text{ extends } \overline{J} \text{\{...\}} \\
\forall i,\text{if \textsf{mtype}}(m,J_i) = \overline(T) \to U_0, \text{then }
T_0 <: U_0 }
{T_0 \spc m(\overline{T} \spc \overline{x}); \text{ OK IN I} }
\quad \textsc{(T-MethExt)}
\]

\[
\inferrule
{\textbf{this}:I, \overline{x}:\overline{T} \vdash e:S \\ S <: T_0 \\\\
IT(I)=\text{interface } I \text{ extends } \overline{J} \text{\{...\}} \\\\
\forall i,\text{if \textsf{mtype}}(m,J_i) = \overline(T) \to U_0, \text{then }
T_0 <: U_0 }
{T_0 \spc m(\overline{T} \spc \overline{x}) \text{ \{ return } e;\} \text{ OK IN
    I} }
\quad \textsc{(T-MethBodyExt)}
\]

\[ \inferrule
{\textbf{interface } I \textbf{ extends } \overline{J} \{ \overline{M} \} \\\\
 \forall J_i \in \overline{J}, J_i \text{ OK} \\\\
 \forall m \in \overline{M}, \textsf{mbody}(m,I) \neq \error \\\\
 \forall J_i \in \overline{J}, \forall m \text{ inside } J_i,
 \textsf{mbody}(m,I) \neq \error }
{I \text{ OK}}
\quad \textsc{(T-Intf)}
 \]

Interface $I$ type checks well, if:
\begin{itemize}
\item All its super-interfaces $\overline{J}$ OK.
\item All methods inside interface $I$ are OK.
\item All methods that $I$ is inheriting from super-interfaces are OK.
\end{itemize}
\subsubsection{Subtyping}

\[ \inferrule{}{T <: T} \]

\[ \inferrule{S <: T \\ T <: U}{S <: U}\]

\[ \inferrule{\emph{ann} \spc \textbf{interface} \spc C_0 \spc \textbf{extends} \spc C_1,...,C_k \{...\}}
{C_0 <: C_1 \\ ... \\ C_0 <: C_k} \]
\end{comment}
%\subsubsection{Interface Table}

