\section{What  \mixin Generates}\label{sec:translation}

\begin{figure*}[t]\label{figure:translation}
\centering
$\begin{array}{l}
\InTextDef{16em}{[\![\mixinAnn\ \QM{interface}\ \C_0\ \QM{extends}\ \Cs\ \oC \methods\ \cC ]\!]
}{
[\![\weakAnn\ \QM{interface}\ \C_0\ \QM{extends}\ \Cs\ \oC
\methods\ \methods' \cC
]\!]}\\
\InTextWith{\methods'=\otherMethod(\C_0,\methods)}\\

\InTextDef{16em}{[\![\weakAnn\ \QM{interface}\ \C_0\ \QM{extends}\ \Cs\ \oC \methods\ \cC ]\!]
}{
\QM{interface}\ \C_0\ \QM{extends}\ \Cs\ \oC
\methods\ \ofMethod(\C_0) \cC
}\\
\InTextWith{\valid(\C_0),\QM{of}\notin\dom(\C_0) }
\end{array}$
\caption{The translation functions of $\mixinAnn$ and $\weakAnn$.}
\end{figure*}

\begin{figure*}[t]\label{figure:valid}
\centering
\hspace{-.3in}
\begin{minipage}[b]{0.5\textwidth}
$\begin{array}{l}
\InTextDef{16em}{\C_0\ \QM{with#}\m\oR \C\ \QM{_val}\cR\QM;\in
\otherMethod(\C_0,\methods)}{}\\
\hspace{.25in}\isWith(\mBody(\QM{with#}\m, \C_0), \C_0),\
\QM{with#}\m\notin\dom(\methods)\\
\InTextDef{16em}{\C_0\ \QM_\m\oR \C\ \QM{_val}\cR\QM;\in
\otherMethod(\C_0,\methods)}{}\\
\hspace{.25in}\isSetter(\mBody(\QM_\m, \C_0), \C_0),\ \QM_\m\notin\dom(\methods)\\
\InTextDef{7ex}{\valid(\C_0)}{\forall \m\in\dom(\C_0),\mbox{ if }\mh\QM; =
            \mBody(\m, \C_0),}\\
            \hspace{.24in}\mbox{ one of the following cases is satisfied:}\\
\tab\tab
\isField(\method), \isWith(\method, \C_0) \mbox{ or }
\isSetter(\method,\C_0)\\
\end{array}$
\end{minipage}
\vline
\hspace{.1in}
\begin{minipage}[b]{0.43\textwidth}
$\begin{array}{l}
\InTextDef{24ex}{\isField(\C\ \m\oR\cR\QM;)}{
\mnot\ \specialName(\m)}\\
\InTextDef{24ex}{\isWith(\C'\ \QM{with#}\m \oR \C\ \x\cR\QM;, \C_0)}{}\\
\hspace{.3in}\C_0 <: \C', \mBody(\m, \C_0) = \C\ \m\oR\cR\QM;
\ \mand\ \mnot\ \specialName(\m)\\
\InTextDef{24ex}{\isSetter(\C'\ \QM_\m \oR \C\ \x\cR\QM;, \C_0)}{}\\
\hspace{.3in}\C_0 <: \C', \mBody(\m, \C_0) = \C\ \m\oR\cR\QM;
\ \mand\ \mnot\ \specialName(\m)\\
\end{array}$
\end{minipage}
\caption{The $\otherMethod$ and $\valid$ functions (left) and auxiliary functions (right).}
\end{figure*}

This section gives an overview of what $\mixinAnn$ generates in Classless Java. We provide a
formalization of Classless Java in Appendix~\ref{sec:formal}, which models the essence of
Java interfaces with default methods, including the syntax as well as the typing rules.
Since the formalized part of Classless Java does not consider casts or \Q@instanceof@,
the \Q@with@ method is not included in the
formal translation. For the same reason \Q@void@ returning setters are
not included, since they are just a minor variation over the more
interesting fluent setters, and they would require special handling
just for the conventional \Q@void@ type.

\begin{figure*}[t]\label{figure:ofmethod}
\centering
\hspace{-.2in}
\begin{minipage}[b]{0.53\textwidth}
$\begin{array}{l}
\InTextDef{5em}{\ofMethod(\C_0)}{
 \QM{static}\ \C_0\ \QM{of} \oR \C_1\ \QM_\m_1\QM,\ldots \C_n\ \QM_\m_n\cR\
\QM{\{}}\\
\tab\QM{return new}\ \C_0 \oR\cR\ \QM{\{}\\
\tab\tab \C_1\ \m_1 = \QM_\m_1\QM;\ldots \C_n\ \m_n = \QM_\m_n\QM; \\
\tab\tab
\C_1\ \m_1\oR\cR\ \QM{\{return }\ \m_1\QM{;\}}\ \ldots
\C_n\ \m_n\oR\cR\ \QM{\{return }\ \m_n\QM{;\}}\\
\tab\tab\withMethod(\C_1,\m_1,\C_0,\es_1)\ldots\withMethod(\C_n,\m_n,\C_0,\es_n)\\
\tab\tab\setterMethod(\C_1,\m_1,\C_0)\ldots\setterMethod(\C_n,\m_n,\C_0)\\
%\tab\tab\cloneMethod(\C_0,\es)\\
%\tab\tab\withMethod(\C_0)\\
\tab\QM{\};\}} \\
\InTextWith{\C_1\ \m_1\QM{();},\ldots \C_n\ \m_n\QM{();} = \fieldsFunc(\C_0)}\\
\hspace{.5in}\mbox{ and }\es_i=\m_1\QM,\ldots\QM, \m_{i-1}\QM,\QM{_val,}\m_{i+1}\QM,\ldots\QM, \m_n
\end{array}$
\end{minipage}
\vline
\hspace{.2in}
\begin{minipage}[b]{0.4\textwidth}
$\begin{array}{l}
\InTextDef{16ex}{\method\in\fieldsFunc(\C_0)}{}\\
\hspace{.3in}\isField(\method)\ \mand\
\method=\mBody(m^\method,\C_0)\\
\InTextDef{10em}{\withMethod(\C,\m,\C_0,\es)}{}\\
\hspace{.2in}\C_0\ \QM{with#}\m\oR \C\ \QM{_val}\cR\ \QM{\{}
\QM{return}\ \C_0\QM{.of(}\es\QM{);\}} \\
\InTextWith{\mBody(\QM{with#}\m,\C_0) \mbox{ having the form }\mh\QM;}\\
\InTextDef{10em}{\withMethod(\C,\m,\C_0,\es)}{\emptyset\mbox{ otherwise}}\\
\InTextDef{10em}{\setterMethod(\C,\m,\C_0)}{}\\
\hspace{.2in}\C_0\ \QM_\m\oR \C\ \QM{_val}\cR\ \QM{\{}
 \m\QM{= _val;return this;\}} \\
\InTextWith{
\mBody(\QM_\m,\C_0) \mbox{ having the form }\mh\QM;}\\
\InTextDef{10em}{\setterMethod(\C,\m,\C_0)}{\emptyset\mbox{ otherwise}}\\
%\cloneMethod:&\cloneMethod(\C_0,\es)=
%\C_0\ \QM{clone()\{return}\ \C_0\QM{.of(}\es\QM{);\}} \\
%&\mbox{iff }
%\mBody(\QM{clone},\C_0) \mbox{ is of form }\mh\QM;\\
%&\cloneMethod(\C_0,\es)=\emptyset\mbox{ otherwise}\\
\end{array}$
\end{minipage}
\caption{The generated $\QM{of}$ method (left) and auxiliary functions (right).}
\end{figure*}

\subsection{Translation}

For the purposes of the formalization, the translation is divided into
two parts for more convenient discussion on formal properties later. To this aim we introduce the annotation
$\weakAnn$. Its role is only in the translation process, hence is
not part of the Classless Java language.  $\weakAnn$ generates the
constructor method $\QM{of}$, while \mixin automatically refines the
return types and calls $\weakAnn$.

Figure~\ref{figure:translation} presents the translation functions. In the first function, $\mixinAnn$
injects refined methods to interface $\C_0$, while in the second function, $\weakAnn$ invokes
$\ofMethod(\C_0)$ and generates the $\QM{of}$
method for $\C_0$, if such a method does not exist in its domain, and all the abstract methods are
valid for the annotation. Figure~\ref{figure:valid} and Figure~\ref{figure:ofmethod}\haoyuan{Refs are wrong!?} present more details on the auxiliary
functions. The formal definition is available in Appendix~\ref{subsec:translation}.

The function $\ofMethod$ generates the static method $\QM{of}$, as an object factory. It detects all the
field methods of $\C_0$, take them as arguments, and in the return statement is an anonymous class which generates getters,
fluent setters and with-methods in certain cases. Some other features of $\mixinAnn$, including non-fluent setters and the $\QM{with}$ method, however, are not formalized in Classless Java. Appendix~\ref{subsec:otherfeatures} gives an informal explanation of those methods.

\subsection{Guarantees}

To understand to what extent our approach is correct (and more in general, what it means to say that a language tuning is correct) we identify three types of guarantees:
\begin{itemize}
\item \textit{Self coherence}: the generated code itself is well-typed; that is,
  type errors are not present in code the user has not written.
In our case it means that either \mixin{} produces (in a controlled way) an
understandable error, or the interface can be successfully annotated and the generated code is well-typed.
We guarantee \textit{Self coherence}.

\item \textit{Client coherence}: all the client code (for example method calls)
  that is well-typed before code generation is also well-typed after the generation.
The annotation just adds more behavior without removing any functionality.
We guarantee \textit{Client coherence}.

\item \textit{Heir coherence}: interfaces (and in general classes) inheriting the instrumented code are well-typed if they were well-typed without the instrumentation.
This would forbid adding any (default or abstract) method to the annotated interfaces, including type refinement.
\mixin  does not guarantee \textit{Heir coherence}.
Indeed consider the following:

\begin{lstlisting}
interface A { int x(); A withX(int x); }
@Obj interface B extends A {}
interface C extends B { A withX(int x); }
\end{lstlisting}

\noindent This code is correct before the translation, but \mixin would  generate in \Q@B@  a method ``\Q@B withX(int x);@''.
This would break \Q@C@. \\*
Similarly, an expression of the form ``\Q@new B(){.. A withX(int x){..}}@''
would be correct before translation, but would be ill-typed after the translation.
\end{itemize}

\noindent This means that our automatic type refinement
is a useful and convenient feature, but not transparent to the heirs of the annotated interface. They need to be aware of the annotation semantics and provide the right type while refining methods.

\subsection{Results}

To formally characterize the behavior of our annotation and the two levels of guarantees that we offer, we provide some notations and two theorems:
\begin{itemize}
\item We denote with $\C^{\II}$ the name of an interface.
\item An interface table
IT is OK if under such interface table, all interfaces are OK.
\item Since interface tables are just represented as sequences of interfaces we write IT = $\II$ IT' to select a specific interface in a table.
\item IT contains an heir of $\C$ if there is an interface that extends it, or a \Q@new@ that instantiate it.
\end{itemize}


\begin{thm}[@ObjOf tuning]
If a given interface table $\II$ IT is OK
 where $\II$ has $\weakAnn$,
$\valid(\C^{\II})$  and $\QM{of}\notin\dom(\C^{\II})$,
then the interface table $[\![\II]\!]$ IT is OK.
\end{thm}

\begin{thm}[@Obj tuning]
If a given interface table $\II$ IT is OK
 where $\II$ has $\mixinAnn$,
$\valid(\C^{\II})$  and $\QM{of}\notin\dom(\C^{\II})$, and there is no heir of $\C^{\II}$,
then the interface table $[\![\II]\!]$ IT is OK.
\end{thm}

Informally, the theorems mean that for a client program that
typechecks before the translation is applied, if the annotated type has
no subtypes and no objects of that type are created, then type safety
is guaranteed after the successful translation.

The second
step of \mixin, namely what $\weakAnn$ does in the formalization, is
guaranteed to be type-safe for the three kinds of coherence by the $\weakAnn$  tuning theorem.
\mixin
tuning is more interesting: since \mixin does not guarantee heir
coherence, we explicitly exclude the presence of heirs. In this way
\mixin tuning guarantees only self and client coherence. The formal
theorem proofs are available in the supplementary materials.

%~\ref{subsec:lemma1},~\ref{subsec:lemma2} and~\ref{subsec:theorem}.\\

\begin{comment}
\begin{figure}[tbp]
\centering
\includegraphics[width=5in]{screenshot.png}
\caption{Screenshot.}\label{screenshot_png}
\end{figure}

\haoyuan{I tried to understand the current algorithm, and did more experiments in eclipse.
Now I borrow some ideas from the current version, and give a new version of the algorithm in text. See below.

(1) I guess the function \textsf{tops} is not necessary. The first step is still
\[\textsf{mbody}(m,C_i)\in\overline{meth}\textrm{ (excluding \textbf{static} methods)}\]

(2) Assume the context is ``interface $C_0$ extends $\overline{C}$ \{$meth'$;...\}''. First handle
\[\textsf{override}(meth',\overline{meth}) \eqno{(*)}\]

(3) If $meth'\ne\none$, $(*)$ returns $meth'$ if
\[\forall meth\in\overline{meth},meth'\subtype meth\]
even if there are conflicts in $\overline{meth}$.

(4) If $meth'=\none$, we need to figure out
\[\textsf{mostSpecific}(\overline{meth})\]
and it should be the one that ``overrides'' all the others in $\overline{meth}$. It means we should not only deal with the return types of methods, but also look into the subtyping relation of interfaces. But for abstract methods, only return types are taken into consideration.
}
\end{comment}

%\text{\yanlin{shouldn't mostSpecific be: $\forall \method' \in \methods : \method \subtype
%  \method'$ ?}}

%(2) If $body_1.\textsf{returnType}=body_2.\textsf{returnType}$, \textsf{shadow} tends to return a default method. If both $body_1$ and $body_2$ are default methods, \textsf{shadow} throws an error.
%\begin{equation*}
%\begin{array}{ll}
%\textsf{shadow}(body_1, body_2)=\textsf{ERROR} & \textsf{if }body_1.\textsf{modifier}=body_2.\textsf{modifier}=\textbf{default}\\
%\textsf{shadow}(body_1, body_2)=body_1 \hspace{.1in} & \textsf{if }body_1.\textsf{modifier}=\textbf{default} \\
%\textsf{shadow}(body_1, body_2)=body_2 \hspace{.1in} & \textsf{if }body_2.\textsf{modifier}=\textbf{default} \\
%\textsf{shadow}(body_1, body_2)=body_1\textsf{ (or }body_2\textsf{)} \hspace{.1in} & \textsf{otherwise}
%\end{array}
%\end{equation*}
%
%(3) If $body_1.\textsf{returnType}<:body_2.\textsf{returnType}$, \textsf{shadow} tends to choose the one with the subtype (namely $body_1$), but only when both methods are abstract, otherwise it gives an error. The other direction $body_2.\textsf{returnType}<:body_1.\textsf{returnType}$ follows the same rule. It also gives an error if there is no subtyping relationship between two return types.
%\begin{equation*}
%\begin{array}{ll}
%\textsf{shadow}(body_1, body_2)=body_1 & \textsf{if }body_1.\textsf{modifier}=body_2.\textsf{modifier}=\emptyset\\
%& \textsf{and }body_1.\textsf{returnType}<:body_2.\textsf{returnType}\\
%\textsf{shadow}(body_1, body_2)=body_2 & \textsf{if }body_1.\textsf{modifier}=body_2.\textsf{modifier}=\emptyset\\
%& \textsf{and }body_2.\textsf{returnType}<:body_1.\textsf{returnType}\\
%\textsf{shadow}(body_1, body_2)=\textsf{ERROR} \hspace{.1in} & \textsf{otherwise}
%\end{array}
%\end{equation*}

%\subsubsection{Auxiliary function: \textsf{replace}}
%
%The \textsf{replace} function takes two same methods (with the same name and types of arguments), and gives the result of the first method overriding the second one.
%
%\begin{equation*}
%\begin{array}{ll}
%\textsf{replace}(body_1, body_2)=body_1 & \textsf{if }body_2=\emptyset\\
%\textsf{replace}(body_1, body_2)=body_2 & \textsf{if }body_1=\emptyset\\
%\textsf{replace}(body_1, body_2)=body_1 & \textsf{if }body_1.\textsf{returnType}<:body_2.\textsf{returnType}\\
%\textsf{replace}(body_1, body_2)=\textsf{ERROR} \hspace{.1in} & \textsf{otherwise}
%\end{array}
%\end{equation*}
