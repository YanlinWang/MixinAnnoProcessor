\section{Interaction of Interface Methods with Interface Composition}
Before formalizing Classless Java and object interfaces, it is helpful
to informally discuss the behavior of methods in Java 8
interfaces. In particular it is useful to understand how Java 8
interfaces differ from conventional trait models.

%We show several interesting cases
%and summarize compilation result of method composition into 3 categories:
%conflict error, accepted (both abstract) and conservative error. Meanwhile we
%also show different composition handling mechanism among traits, javac and ECJ.

\subsection{Methods in Java 8 Interfaces}
In Java 8 interfaces there are three types of methods: abstract, default, and static methods. Default and static methods were not allowed in interfaces in previous versions of Java.

\paragraph{Static methods} are handled in a very clean way: they are visible only in
  the interface in which they are explicitly defined. This means the following code
  is ill-typed.
\begin{lstlisting}
interface A0 { static int m(){return 1;} }
interface B0 extends A0 {}
... B0.m()//ill typed
\end{lstlisting}
This is different from the way static methods are handled in classes. Here
static methods have simply no interaction with interface
composition (\Q@extends@ or \Q@implements@).

\paragraph{Abstract method} composition is accepted when there exists a most specific one.
  For example, here method \texttt{Integer m()} from \texttt{B1} is visible in \Q@C1@.
\begin{lstlisting}
interface A1 { Object m(); }
interface B1 { Integer m();}
interface C1 extends A1,B1 {} //accepted
\end{lstlisting}

\paragraph{Default methods} conflict with any other default or abstract method. For
  example the following code is rejected due to method conflicts.
\begin{lstlisting}
interface A2 { default int m() {return 1;}}
interface B2 { int m(); }
interface C2 { default int m() {return 2;}}
interface D2 extends A2,B2 {} //rejected due to conflicting methods
interface E2 extends A2,C2 {} //rejected due to conflicting methods
\end{lstlisting}
Note how this is different from what happens in most trait models, where \Q@D2@
would be accepted, and the implementation in \Q@A2@ would be part of the
behavior of \Q@D2@.

\paragraph{Resolving conflicts:}
A method in the current interface wins over any method in its
super-interfaces, provided that the method
is the most specific one. This method also overrides conflict due to
inheritance. For example, the following code is accepted, but would be rejected
(see before) if the method \Q@m@ was not redefined in \texttt{D3} and
\texttt{E3}.
\begin{lstlisting}
interface D3 extends A2,B2 { int m(); } //accepted
interface E3 extends A2,C2 { default int m(){return 42;} } //accepted
\end{lstlisting}

\subsection{Classifying Outcomes of Interface Composition}
%When interfaces are composed and methods with the same name (and
%signature) exist, there are 3 possible outcomes.
%
We now try to classify possible outcomes for composition of methods with the same name (and
signature).
We will use the following (correct) declarations:
\begin{lstlisting}
interface A1 { T m(); }
interface A2 extends A1 { default T m(){ ... } }
interface A3 extends A2 { T m(); }

interface B1 { default T m(){ ... } }
interface B2 extends B1 { T m(); }
interface B3 extends B2 { default T m(){ ... } }
\end{lstlisting}

\noindent
What happens if a new interface \Q@M@ extends one \Q@A@${}_i$ and one
\Q@B@${}_j$?\\*
% Nine representative cases are shown next:\\*
\noindent
\begin{tabular}{|l|l|l|l|}
\hline
\textbf{M extends} & \textbf{A1}                  & \textbf{A2} & \textbf{A3} \\ \hline
\textbf{B1}        & conservative error                     & conflict error      & conservative error       \\ \hline
\textbf{B2}        & both abstract (accepted)     & conservative error       & both abstract (accepted)       \\ \hline
\textbf{B3}        & \textbf{conservative error} &conflict  error       & conservative error      \\ \hline
\end{tabular}
\\*
%We try to classify the results in the table:
\begin{itemize}
\item \textbf{conflict error} happens when the methods from both interfaces are implemented, which is also an error in most trait models.
\item \textbf{both abstract (accepted)} happens when the methods from both interfaces are abstract, which is also considered correct in all
  trait models.
\item \textbf{conservative error} happens when only one method is implemented
  (leaving another one abstract), which is different from what we would expect in
  a trait model, but is coherent with the conservative idea that a method
  defined in an interface should not silently satisfy a method in another one.
\end{itemize}

\paragraph{A bug:} During our experimentation, we found a bug in ECJ (Eclipse compiler for Java):
the case \textbf{M} extending \textbf{B3} and \textbf{A1} is accepted by
ECJ4.5.1 and rejected by javac.  By email communication with Brian Goetz
(leading Java 8 designer) we have confirmed that the expected behavior is
rejection, hence this is a bug in ECJ. This bug was also reported by
  others and is fixed in the ECJ developer branch, but not released as a stable
  version yet.
%Table~\ref{table:javabug} shows the method overriding bug in Java. In the
%example, there are 6 interfaces \texttt{A1,A2,A3,B1,B2,B3} with methods all
%named as \texttt{m} which are either abstract or default methods. If we define a
%new interface \texttt{M} that extends two of these interfaces, then the method
%overriding result is shown in the table. For example, row2 col2 means \texttt{M
%  extends A1,B1}. The result is \texttt{ERROR} because the abstract method
%\texttt{T m();} in \texttt{A1} conflicts with the method \texttt{default T m()
%  {...}} in \texttt{B1}. Readers may also figure out other extending cases by
%following this way of interpretation.

% The interesting case in row4 col2 where
%Indeed, to be coherent with the idea of \textbf{conservative error}, the case should not be accepted.
%We do not see how this should behave differently from \textbf{B1},\textbf{A1}, and \textbf{B3},\textbf{A3}.
%We fear that the only retro-compatible fix for this strange behaviour is to accept all the cases of \textbf{conservative error} in a future version of Java.
%the experimental result is \texttt{B3.m}. Because if we think consistently, this

%\lstinputlisting[linerange=4-9]{../UseMixinLombok/src/methodshadowing/test5/Test5.java} % APPLY:linerange=JAVABUG

%In our approach, we choose to not model this strange behaviour (a bug?).
%Our auxiliary function $\mBody(\m,\C)$ enforce the \textbf{conservative error} strategy.
%The rest of our formalization is parametric with the definition of $\mBody(\m,\C)$, thus if Java changes its resolution strategy to a more permissive one, only minor adaptations in $\mBody(\m,\C)$ would be needed.

\section{Formal Semantics}\label{sec:formal}

This section presents a formalization of Classless Java: a minimal
FeatherweightJava-like calculus which models the essence of Java
interfaces with default methods. This formalization is used in
Section~\ref{sec:translation} to define the semantics of object interfaces.


\begin{figure}[t]
\saveSpaceFig
\begin{grammar}
\production{
\e
}{
  \x\mid\MCall\e\m\es\mid\MCall{\C}\m\es\mid\MCall{\C\QM{.super}}\m\es\mid\x\QM=\e\QM;\e'\mid\obj
  }{expressions}\\
\production{
\obj
}{
\QM{new}\ \C\oR\cR\oC\fields\
\mh_1\oC\QM{return}\ \e_1\QM{;}\!\cC
\ldots
\mh_n\oC\QM{return}\ \e_n\QM{;}\!\cC
\cC
  }{object creation}\\
\production{\field}{\T\ \f \QM= \x\QM;}{field declaration}\\
\production{
\II
}{
 \ann\ \QM{interface}\ \C\ \QM{extends}\ \Cs\ \oC \methods\ \cC
  }{interface declaration}\\
\production{
\method
}{
 \QM{static}\ \mh\ \oC\QM{return}\ \e\QM{;}\!\cC
\mid
\QM{default}\ \mh\ \oC\QM{return}\ \e\QM{;}\cC
\mid
\mh\QM{;}
  }{method declaration}\\
\production{
\mh
}{
 \T_0\ \m\ \oR\T_1\ \x_1\ldots\T_n\ \x_n\cR
  }{method header}\\
\production{
\ann
}{
  \mixinAnn|\emptyset
  }{annotations}\\
\production{\Gamma}{
\x_1{:}\C_1\ldots\x_n{:}\C_n
}{environment}
\end{grammar}
\caption{Grammar of Classless Java}
\label{Grammar}
\end{figure}

\subsection{Syntax}

Figure~\ref{Grammar} shows the syntax of Classless Java.
The syntax formalizes a minimal
subset of Java 8, focusing on interfaces, default methods and object
creation literals.  There is no syntax for classes.
To help readability we use many metavariables to represent identifiers: $C,x,f$ and $m$; however they all map to a single set of identifiers as in Java.  Expressions
consist of conventional constructs such as variables ($\x$), method
calls ($\e\QM.\m\QM(\es)$) and static method calls
($\C\QM.\m\QM(\es)$). For simplicity the degenerate case of calling a
static method over the $\this$ receiver is not considered.  A more
interesting type of expressions is super calls
($\C\QM{.super.}\m\QM(\es)$), whose semantics is to call the (non-static) method $\m$ over the $\this$ receiver, but statically
dispatching to the version of the method as visible in the interface
$\C$. A simple form of field updates ($\x\QM=\e\QM;\e'$) is also
modeled. In the syntax of field updates $\x$ is expected to be a
field name. After updating the field $\x$ using the value of $\e$, the
expression $\e'$ is executed. To blend the statement based nature of
Java and the expression based nature of our language, we consider a
method body of the form \Q@return@ $\x\QM=\e\QM;\e'$ to represent
$\x\QM=\e\QM;\QM{return}\ \e'$ in Java.  Finally, there is an object
initialization expression from an interface $\C$, where (for
simplicity) all the fields are initialized with a variable present in
scope.
To  be fully compatible with Java, the concrete syntax for an interface
  declaration with empty supertype list  would also
  omit the \Q@extends@ keyword.
% The single
%non-Java 8 piece of syntax is the \mixin annotation, which is the only
%one interesting piece of syntax in this article.
%??? how is @Mixin not java8? in the sense that is java5?
  Following standard
practice, we consider a global Interface Table (\metaVar{IT}) mapping
from interface names $\C$ to interface declarations $\II$.

The environment $\Gamma$ is a mapping from variables to types.  As
usual, we allow a functional notation for $\Gamma$ to do variable
lookup.  Moreover, to help us define auxiliary functions, a functional
notation is also allowed for a set of methods $\methods$, using the
method name $\m$ as a key.  That is, we define $\methods(\m)=\method$
iff there is a unique $\method\in\methods$ whose name is $\m$.  For
convenience, we define $\methods(\m)=\none$ otherwise; moreover
$\m\in\dom(\methods)\ \miff\ \methods(\m)=\method$.
For simplicity, we do not model overloading, thus for an interface to be well formed its methods must be uniquely identified by their names.

\subsection{Typing}

Typing statement $\Gamma \vdash \e\in\C$ reads ``in the environment
$\Gamma$, expression $\e$ has type $\C$.''.
Before discussing the typing rules we discuss some of the used notation.
As a shortcut, we write
$\Gamma \vdash \e\in\C<:\C'$ instead of $\Gamma \vdash \e\in\C$ and
$\C<:\C'$.

We omit the definition of
the usual traditional subtyping relation between interfaces, that is the transitive and reflexive closure of the declared \Q@extends@ relation.
The auxiliary notation $\Gamma^\mh$ trivially
extracts the environment from a method header, by collecting the all types
and names of the method parameters.  The
notation $\m^\mh$ and $\C^\mh$ denotes respectively, extracting the
method name and the return type from a method header. $\mBody(\m,\C)$,
defined in Section~\ref{sec:auxiliary},
returns the full method declaration as seen by $\C$, that is the
method $\m$ can be declared in $\C$ or inherited from another
interface.
$\textsf{mtype}(\m,\C)$ and $\textsf{mtypeS}(\m,\C)$ return the type
signature from a method (using $\mBody(\m,\C)$ internally).
$\textsf{mtype}(\m,\C)$ is defined only for non static methods, while
$\textsf{mtypeS}(\m,\C)$ only for static ones. We use $\dom(\C)$ to
denote the set of methods that are defined for type $\C$, that is:
$\m\in\dom(\C)\ \miff \ \mBody(\m,\C)=\method$.

\begin{figure}[t]
\saveSpaceFig
$
\begin{array}{l}

%% T-Invk
\inferrule[(T-Invk)]{
 \Gamma \vdash \e \in \C_0 \\\\
\forall i\in 1..n\ \ \Gamma \vdash \e_i \in \_<:\C_i \\\\
  \textsf{mtype}(\m,\C_0) \!=\! \C_1\ldots\C_n \!\!\to\! \C
%\textsf{mmodifier}(\m,\C) \neq \textbf{static}
 }{
 \Gamma \vdash \e\QM.\m\QM(\e_1\ldots\e_n\QM) \in \C }
\quad\quad

%%T-StaticInvk
\inferrule[(T-StaticInvk)]{
\forall i\in 1..n\  \ \Gamma \vdash \e_i\in \_<:\C_i \\\\
\textsf{mtypeS}(\m,\C_0) \!=\! \C_1\ldots\C_n \!\to\! \C
%\textsf{mmodifier}(\m,\C) = \textbf{static}
}{
\Gamma \vdash \C_0\QM.\m\QM(\e_1\ldots\e_n\QM) \in \C}
\quad\quad

%%T-SuperInvk
\inferrule[(T-SuperInvk)]{
\Gamma(\this) <: \C_0 \\\\
\forall i\in 1..n\ \ \Gamma \vdash \e_i\in \_<:\C_i \\\\
  \textsf{mtype}(\m,\C_0) \!=\! \C_1\ldots\C_n \!\!\to\! \C
%\textsf{mmodifier}(\m,\C_0) \neq \textbf{static} \\\\
}{\Gamma \vdash \C_0\QM.\QM{super}\QM.\m\QM(\e_1\ldots\e_n\QM) \in \C}


%%T-Var
\\[5ex]
\inferrule[(T-Var)]{
\Gamma(\x)=\C
}{
\Gamma \vdash \x \in\C}
\quad\quad

%%T-Obj
\inferrule[(T-Obj)]{
\forall i\in 1..k\ \ \Gamma(\x_i)\subtype\T_i\\\\
\forall i\in 1..n\ \
%\Gamma_i
\Gamma,\f_1{:}\T_1,\ldots,\f_k{:}\T_k,\,\QM{this}{:}\C,\Gamma^{\mh_i}
\vdash\e_i\in \_\subtype\C^{\mh_i}\\\\
\sigvalid(\mh_1\ldots\mh_n,I)\quad\quad\quad\quad
%\forall i\in 1..n\ \mh_i\subtype\mBody(\m^{\mh_i},\C)\\\\
\alldefined(\mh_1\ldots\mh_n,I)
%\forall\m\mbox{ such that }
%\mBody(\m,\C)=\mh\QM; \exists i\in 1..n\ \m^{\mh_i}=\m
%\forall i\in 1\ldots n\ \Gamma_i=\Gamma,\f_1{:}\T_1,\ldots,\f_k{:}\T_k,\,\QM{this}{:}\C,\Gamma^{\mh_i}
}{
\Gamma \vdash\QM{new}\ \C\oR\cR\oC\T_1\ \f_1\QM=\x_1\QM;\ldots\T_k\ \f_k\QM=\x_k\QM;\
\mh_1\oC\QM{return}\ \e_1\QM{;}\!\cC
\ldots
\mh_n\oC\QM{return}\ \e_n\QM{;}\!\cC
\cC
\in\C
}
\\[5ex]



%%T-update
\quad
\inferrule[(T-update)]{
\Gamma \vdash \e\in\_<:\Gamma(\x)\\\\
\Gamma \vdash \e'\in\C
}{
\Gamma \vdash \x\QM=\e\QM;\e'\in\C }
\quad\quad\quad

%%T-Intf
 \inferrule[(T-Intf)]{
IT(\C) = \ann\ \QM{interface}\ \C\ \QM{extends}\ \C_1\ldots\C_n \
\oC\methods\ \cC\\\\
 \forall \QM{default}\ \mh\ \oC\QM{return}\ \e\QM;\cC \in \methods,
\ \ \Gamma^{\mh},\,\QM{this}{:}\C\vdash\e\in \_\subtype\C^{\mh} \\\\
 \forall \QM{static}\ \mh\ \oC\QM{return}\ \e\QM;\cC \in \methods,
\ \ \Gamma^{\mh}\vdash\e\in \_\subtype\C^{\mh} \\\\
\dom(\C)=\dom(\C_1)\cup\ldots\cup\dom(\C_n)\cup\dom(\methods)
 }{
\C \text{ OK}
}
\end{array}$
\caption{CJ Typing}
\label{ET}
\end{figure}

In Figure~\ref{ET} we show the typing rules.  We discuss the
most interesting rules, that is \rn{t-Obj} and \rn{t-Intf}. Rule
\rn{t-Obj} is the most complex typing rule. Firstly, we need to
ensure that all field initializations are type correct, by looking up the type of
each variable assigned to a field in the typing environment and verifying that such type is a
subtype of the field type. Finally, we check that all method bodies are
well-typed. To do this the environment used to check the method body
needs to be extended appropriately: we add all fields and their types;
add $\this:I$; and add the arguments (and types) of the respective
method.
Now we need to check if the object is a valid extension for that specific interface.
This can be logically divided into two steps.
First  we check that all method headers are valid
with respect to the corresponding method already present in $\C$:

\noindent$\begin{array}{l}
\InTextDef{10em}{\sigvalid(\mh_1\ldots\mh_n,I)
}{
\forall i\in 1..n\ \ \mh_i\QM;\subtype\mBody(\m^{\mh_i},\C)
}
\end{array}$

\noindent Here we require that for all newly declared methods, there is a method
with the same name defined in the interface $\C$, and that such method is a
supertype of the newly introduced one. We define subtyping between methods in a
general form that will also be useful later.

\noindent$\begin{array}{l}
\InTextDef{18em}{
\T\ \m\oR\T_1\x_1\ldots \T_n\x_n\cR\QM; \subtype \T' \m\oR\T_1\x_1'\ldots\T_n\x_n'\cR\QM;
}{
\T\subtype \T'
}\\
\InTextDef{18em}{\method \subtype
\QM{default}\ \mh\,\mbox{\Q@\{return \_;\}@}
}{
\method\subtype\mh\QM;
}\\
\InTextDef{18em}{\QM{default}\ \mh\,\mbox{\Q@\{return \_;\}@}\subtype\method
}{
\mh\QM;\subtype\method
}\\
\end{array}$

\noindent We allow return type specialization as introduced in Java 5. A method
header with return type $I$ is a subtype of another method header with return
type $I'$ if all parameter types are the same, and $I <: I'$. A default method
$\method_1$ is a subtype of another default method $\method_2$ iff
$\mh^{\method_1}$ is a subtype of $\mh^{\method_2}$. Secondly, we check that all
abstract methods (which need to be explicitly overridden) in the interface have
been implemented: %That is, we define a method with the same name.

\noindent$\begin{array}{l}
\InTextDef{11em}{\alldefined(\mh_1\ldots\mh_n,I)}{
\forall\m\mbox{ such that } \mBody(\m,\C)=\mh\QM; \exists i\in 1..n\ \m^{\mh_i}=\m}
\end{array}$

The rule \rn{t-intf} checks that an interface $\C$ is correctly
typed.  First we check that the body of all default and static
methods are well typed.  Then we check that $\dom(\C)$ is the same as
$\dom(\C_1)\cup\ldots\cup\dom(\C_n)\cup\dom(\methods)$.  This is not a
trivial check, since $\dom(\C)$ is defined using $\mBody$, which would be
undefined in many cases: notably if a method $\method\in\methods$ is
not compatible with some method in $\dom(\C_1)\ldots\dom(\C_n)$ or if
there are methods in any $\dom(\C_i)$ and $\dom(\C_j)$ ($i,j\in 1..n$) conflict.

\subsection{Auxiliary Definitions}\label{sec:auxiliary}

\begin{comment}
\subsubsection{Auxiliary function: \textsf{mtype}}
- \textsf{mtype(m, C)} : the signature of method m in C.

\[ \inferrule{
  IT(T) = \text{\emph{ann} interface } C \{ \overline{M} \} \\
  E \spc m(\overline{D} \spc \overline{x}) \{ \text{return } e; \} \in M}
{ \textsf{mtype(m,T)} = \overline{D} \to E } \]

\[ \inferrule{
  IT(T) = \text{\emph{ann} interface } C \{ \overline{M} \} \\
  m \notin M}
{ \textsf{mtype(m,T)} = \emptyset } \]

\[ \inferrule{
  IT(T) = \text{\emph{ann} interface } C \text{ extends } C_1,...,C_k \{ \overline{M} \} \\
  E \spc m(\overline{D} \spc \overline{x}) \{ \text{return } e; \} \in M}
{ \textsf{mtype(m,T)} = \overline{D} \to E } \]

\[ \inferrule{
  IT(T) = \text{\emph{ann} interface } C_0 \text{ extends } \overline{C} \{
  \overline{M} \} \\
  m \notin M}
{ \textsf{mtype(m,T)} = \bigcup \textsf{mtype}(m,\overline{D}) } \]
\end{comment}


Defining \mBody{} is not trivial, and requires quite a lot of attention to the
specific model of Java interfaces, and to how it differs w.r.t. Java Class model.
$\mBody(\m,\C)$ denotes the actual method $\m$ (body included) that
interface $\C$ owns. The method can either be defined originally in $\C$ or in its supertypes, and then passed to $\C$ via inheritance.
\noindent$\begin{array}{l}
\InTextDef{15ex}{
\mBody(\m,\C_0)
}{
\override(\methods(\m),
\tops(\m,\Cs))
}\\
\InTextWith{\metaVar{IT}(\C_0) =
\ \ann\ \QM{interface}\ \C_0\ \QM{extends}\ \C_1\ldots\C_n\ \oC\methods\ \cC \mbox{ and }
 \C\in\Cs \mbox{ if } \C_i\subtype\C, i\in 1..n

}
\end{array}$

\noindent The definition of $\mBody$ reconstructs the full set of supertypes $\Cs$ and then delegates the work to two other auxiliary functions:
 $\tops(m,\Cs)$ and $\override(\method,\methods)$.

\paragraph{\tops{}} recovers from the interface table only the ``needed'' methods, that is,
the non-static ones that are not reachable by another, less specific superinterface.
Since the second parameter of \tops{} is a set, we can choose an arbitrary element to be $\C_0$.
In the definition we denote by $\Aux{originalMethod}(\m,\C)=\method$ the non-static method called $\m$ defined directly in $\C$.
%\hl{Another two functions $\super(\C_0)$ and $\get(\m,\C_0)$ are used for the definition. $\super(\C_0)$ collects the
%super interfaces of $\C_0$ recursively, and $\get(\m,\C_0)$ returns the non-static method declared in $\C_0$ with name $\m$, or $\emptyset$ otherwise.
%Finally, $\tops(\m,\C_0)$ returns the topmost methods $m$ from the super interfaces.}
Formally:

\noindent$\begin{array}{l}
\InTextDef{20ex}{
\Aux{originalMethod}(\m,\C_0)}{
\method
}\\
\InTextWith{\metaVar{IT}(\C_0) =
\ \ann\ \QM{interface}\ \C_0\ \QM{extends}\ \Cs\ \oC\methods\ \cC, \method\in\methods\mbox{ not static},\m=\m^\method}
 \\

\InTextDef{44ex}{
\Aux{originalMethod}(\m,\C_0)\in\tops(\m,\C_0\dots\C_n)}{
 }{}\\
\tab\tab
\not\exists i\in1..n \mbox{ such that }
\Aux{originalMethod}(\m,\C_i) \mbox{ is defined and }
\C_i \subtype \C_0

%\\
%\InTextDef{13ex}{
%\tops(\m,\C_0)}{
%}{
%\{\ \get(\m, \C)\ |\ \C\in\super(\C_0), \get(\m, \C) \ne\emptyset,\mbox{ and }
%}\\
%\hspace{2.1in}\forall\C'\in\super(\C_0)\mbox{ and }\get(\m,\C')\ne\emptyset: \C'<:\C\Rightarrow\C'=\C\ \}\\
%\InTextDef{13ex}{
%\super(\C_0)}{
%}{
%\{\ \C'\ |\ \C'\in\Cs,\mbox{ or }\exists\C\in\Cs,\C'\in\super(\C)\ \}
%}\\
%\InTextWith{\metaVar{IT}(\C_0) =
%\ \ann\ \QM{interface}\ \C_0\ \QM{extends}\ \Cs\ \oC\methods\ \cC}\\
%\InTextDef{13ex}{
%\get(\m,\C_0)}{
%}{
%\method
%}\\
%\InTextWith{\metaVar{IT}(\C_0) =
%\ \ann\ \QM{interface}\ \C_0\ \QM{extends}\ \Cs\ \oC\methods\ \cC, \method\in\methods\mbox{ not static},\m=\m^\method}
\end{array}$


%\paragraph{\shadow{}} chooses the most specific version of a method,
%that is the unique version available, or a conflicted version from a
%set of possibilities. In this case all the conflicting defintions are returned.
%We do not model overloading, so it is an error if multiple versions are available with different parameter types. Formally:
%
%\noindent
%$\begin{array}{l}
%\InTextDef{15ex}{
%\shadow()}{\none}\\
%\InTextDef{15ex}{\shadow(\method)}{\method}\\
%\InTextDef{15ex}{\shadow(\overline{\mh\QM;})}{\Aux{mostSpecific}(\overline{\mh\QM;})}\\
%\InTextDef{15ex}{\shadow(\methods)}{\methods}\\
%\InTextWith{\mbox{ the former case not applicable} }
%%\methods\mbox{ not of the form }\overline{\mh\QM;}
%%\mbox{ and }
%%\Aux{mostSpecific}(\methods)\in\{\mh\QM;,\QM{default}\ \mh\mbox{\Q@\{return \_;\}@}\}}\\
%\end{array}$
%
\paragraph{override} models how a method in an interface can
override implementations in its superinterfaces, even in the case of
conflicts. Note how the special value $\none$ is used, and how (the 5th
case) overriding can solve a conflict.
% The notation in the last case below
%means $\method'$ is not static, and is supertype of $\method$.

\noindent$\begin{array}{l}
\InTextDef{20ex}{\override(\none,\emptyset)}{\none}\\
\InTextDef{20ex}{\override(\method,\emptyset)}{\method}\\
\InTextDef{20ex}{\override(\none,\method)}{\method}\\
\InTextDef{20ex}{\override(\none,\overline{\mh\QM;})}{\Aux{mostSpecific}(\overline{\mh\QM;})}\\
\InTextDef{20ex}{\override(\method,\methods)}{\method}\\
\InTextWith{
\forall \method' \in \methods :
%\method'\in\{\mh\QM;,\QM{default}\ \mh\,\mbox{\Q@\{return \_;\}@}, \conflicted\ \mh\QM; \},
\method\subtype\method'
}
\end{array}$

\noindent The definition $\Aux{mostSpecific}$ returns the most
specific method whose type is the subtype of all the
others.  Since method subtyping is a partial ordering,
$\Aux{mostSpecific}$ may not be defined, this in turn forces us to rely on the last clause of $\override$; otherwise
the whole \mBody{} would not be defined for that specific $\m$.
Rule \rn{t-intf} relies on this behavior.

\noindent$\begin{array}{l}
\InTextDef{18ex}{\Aux{mostSpecific}(\methods)}{\method}\\
\InTextWith{\method \in \methods\ \mand\ \forall \method' \in \methods :  \method \subtype \method'}
\end{array}$

To illustrate the mechanism of $\mBody$, we present an example. We compute $\mBody(\QM{m},\QM{D})$:
\begin{lstlisting}
interface A { Object m(); }
interface B extends A { default Object m() {return this.m();} }
interface C extends A {}
interface D extends B, C { String m(); }
\end{lstlisting}
\vspace{-2ex}
\begin{itemize}
\item First $\{\QM{A,B,C}\}$, the full set of supertypes of $\QM{D}$ is obtained.
\item Then we compute 
$\tops(\QM{m},\{\QM{A,B,C}\})= $
\Q@default Object m() {...}@,
 that is $\QM{B.m}$.
That is, we do not consider either
$\QM{C.m}$ (since $\QM{m}$ is not declared directly in $\QM{C}$, hence $\Aux{originalMethod}(\QM{m},\QM{C})$ is undefined) or $\QM{B.m}$ 
(that is a subtype of $\QM{A}$, thus $\QM{B.m}$ hides $\QM{A.m}$).
\item The final step computes $\override(\QM{D.m},\QM{B.m})=\QM{D.m}$,
by the last case of $\override$ we get that $\QM{D.m}$ hides $\QM{B.m}$ successfully ($\QM{String}$ is a subtype of $\QM{Object}$). Finally we get $\mBody(\QM{m},\QM{D}) = \QM{D.m}$.
\end{itemize}

%The \textsf{shadow} function takes two same methods (with the same name and types of arguments), and return the method which shadows the other during inheritance.
%exmples to motivate our design
%interface A{static String m(){return "A";}}
%interface C extends A{
%	default String dm(){
%		  this.m();//wrong in java
%		  A.m();
%		  C.m();//wrong in java
%		}
%}
%
%
%(1) Static methods are not inherited. Also, if one of $\{body_1,body_2\}$ is null, \textsf{shadow} simply returns the other one. Hence

\begin{comment}
%Now it is correct, but may be we do not need it?
We abbreviate typing statements on
sequences in a simple way, writing $\Gamma \vdash \overline{t}:\overline{C}$ as
shorthand for $\Gamma \vdash t_1:C_1,..., \Gamma \vdash t_n:C_n$.
\end{comment}



\begin{comment}
\subsubsection{Method Typing}


\[
\inferrule
{ }
{T_0 \spc m(\overline{T} \spc \overline{x}); \text{ OK IN I} }
\quad \textsc{(T-Meth)}
\]

\[
\inferrule
{\overline{x}:\overline{T} \vdash e:S \\ S <: T_0}
{T_0 \spc m(\overline{T} \spc \overline{x}) \text{ \{ return } e;\} \text{ OK IN
    I} \\\\ \Gamma \vdash \textbf{this}:I }
\quad \textsc{(T-MethBody)}
\]

\[
\inferrule
{IT(I)=\text{interface } I \text{ extends } \overline{J} \text{\{...\}} \\
\forall i,\text{if \textsf{mtype}}(m,J_i) = \overline(T) \to U_0, \text{then }
T_0 <: U_0 }
{T_0 \spc m(\overline{T} \spc \overline{x}); \text{ OK IN I} }
\quad \textsc{(T-MethExt)}
\]

\[
\inferrule
{\textbf{this}:I, \overline{x}:\overline{T} \vdash e:S \\ S <: T_0 \\\\
IT(I)=\text{interface } I \text{ extends } \overline{J} \text{\{...\}} \\\\
\forall i,\text{if \textsf{mtype}}(m,J_i) = \overline(T) \to U_0, \text{then }
T_0 <: U_0 }
{T_0 \spc m(\overline{T} \spc \overline{x}) \text{ \{ return } e;\} \text{ OK IN
    I} }
\quad \textsc{(T-MethBodyExt)}
\]

\[ \inferrule
{\textbf{interface } I \textbf{ extends } \overline{J} \{ \overline{M} \} \\\\
 \forall J_i \in \overline{J}, J_i \text{ OK} \\\\
 \forall m \in \overline{M}, \textsf{mbody}(m,I) \neq \error \\\\
 \forall J_i \in \overline{J}, \forall m \text{ inside } J_i,
 \textsf{mbody}(m,I) \neq \error }
{I \text{ OK}}
\quad \textsc{(T-Intf)}
 \]

Interface $I$ type checks well, if:
\begin{itemize}
\item All its super-interfaces $\overline{J}$ OK.
\item All methods inside interface $I$ are OK.
\item All methods that $I$ is inheriting from super-interfaces are OK.
\end{itemize}
\subsubsection{Subtyping}

\[ \inferrule{}{T <: T} \]

\[ \inferrule{S <: T \\ T <: U}{S <: U}\]

\[ \inferrule{\emph{ann} \spc \textbf{interface} \spc C_0 \spc \textbf{extends} \spc C_1,...,C_k \{...\}}
{C_0 <: C_1 \\ ... \\ C_0 <: C_k} \]
\end{comment}
%\subsubsection{Interface Table}

