\section{Interaction of Interface Methods with Interface Composition}
Before formalizing Classless Java and object interfaces, it is helpful
to informally discuss the behaviour of Java8
interfaces, in contrast with conventional trait models.

%We show several interesting cases
%and summarize compilation result of method composition into 3 categories:
%conflict error, accepted (both abstract) and conservative error. Meanwhile we
%also show different composition handling mechanism among traits, javac and ECJ.

\subsection{Methods in Java 8 Interfaces}
In Java 8 interfaces there are three types of methods: abstract, default, and static methods. Default and static methods were not allowed in interfaces in previous versions of Java.

\paragraph{Static methods} are handled in a very clean way: they are visible only in
  the interface in which they are explicitly defined. This means the following code
  is ill-typed.
\begin{lstlisting}
interface A0 { static int m(){return 1;} }
interface B0 extends A0 {}
... B0.m()//ill typed
\end{lstlisting}
This is different from the way static methods are handled in classes. Here
static methods have simply no interaction with interface
composition (\Q@extends@ or \Q@implements@).

\paragraph{Abstract method} composition is accepted when there exists a most specific one.
  For example, here method \texttt{Integer m()} from \texttt{B1} is visible in \Q@C1@.
\begin{lstlisting}
interface A1 { Object m(); }
interface B1 { Integer m();}
interface C1 extends A1,B1 {} //accepted
\end{lstlisting}

\paragraph{Default methods} conflict with any other default or abstract method. For
  example the following code is rejected due to method conflicts.
\begin{lstlisting}
interface A2 { default int m() {return 1;}}
interface B2 { int m(); }
interface C2 { default int m() {return 2;}}
interface D2 extends A2,B2 {} //rejected due to conflicting methods
interface E2 extends A2,C2 {} //rejected due to conflicting methods
\end{lstlisting}
Note how this is different from what happens in most trait models, where \Q@D2@
would be accepted, and the implementation in \Q@A2@ would be part of the
behavior of \Q@D2@.

\paragraph{Resolving conflicts:}
A method in the current interface wins over any method in its
super-interfaces, provided that the method
is the most specific one. This method also overrides conflict due to
inheritance. For example, the following code is accepted, but would be rejected
(see before) if the method \Q@m@ was not redefined in \texttt{D3} and
\texttt{E3}.
\begin{lstlisting}
interface D3 extends A2,B2 { int m(); } //accepted
interface E3 extends A2,C2 { default int m(){return 42;} } //accepted
\end{lstlisting}

\subsection{Classifying Outcomes of Interface Composition}
%When interfaces are composed and methods with the same name (and
%signature) exist, there are 3 possible outcomes.
%
We now try to classify possible outcomes for composition of methods with the same name (and
signature).
We will use the following (correct) declarations:
\begin{lstlisting}
interface A1 { T m(); }
interface A2 extends A1 { default T m(){ ... } }
interface A3 extends A2 { T m(); }

interface B1 { default T m(){ ... } }
interface B2 extends B1 { T m(); }
interface B3 extends B2 { default T m(){ ... } }
\end{lstlisting}

\noindent
What happens if a new interface \Q@M@ extends one \Q@A@${}_i$ and one
\Q@B@${}_j$?
% Nine representative cases are shown next:\\*
\begin{figure*}[htbp]
\centering
\begin{tabular}{|l|l|l|l|}
\hline
\textbf{M extends} & \textbf{A1}                  & \textbf{A2} & \textbf{A3} \\ \hline
\textbf{B1}        & conservative error                     & conflict error      & conservative error       \\ \hline
\textbf{B2}        & both abstract (accepted)     & conservative error       & both abstract (accepted)       \\ \hline
\textbf{B3}        & \textbf{conservative error} &conflict  error       & conservative error      \\ \hline
\end{tabular}
\end{figure*}
%We try to classify the results in the table:
\begin{itemize}
\item \textbf{conflict error} happens when the methods from both interfaces are implemented, which is also an error in most trait models.
\item \textbf{both abstract (accepted)} happens when the methods from both interfaces are abstract, which is also considered correct in all
  trait models.
\item \textbf{conservative error} happens when only one method is implemented
  (leaving another one abstract), which is different from what we would expect in
  a trait model, but is coherent with the conservative idea that a method
  defined in an interface should not silently satisfy a method in another one.
\end{itemize}

\paragraph{A bug:} During our experimentation, we found a bug in ECJ (Eclipse compiler for Java):
the case \textbf{M} extending \textbf{B3} and \textbf{A1} is accepted by
ECJ4.5.1 and rejected by javac.  By email communication with Brian Goetz
(leading Java 8 designer) we have confirmed that the expected behavior is
rejection, hence this is a bug in ECJ. This bug was also reported by
  others and is fixed in the ECJ developer branch, but not released as a stable
  version yet.
%Table~\ref{table:javabug} shows the method overriding bug in Java. In the
%example, there are 6 interfaces \texttt{A1,A2,A3,B1,B2,B3} with methods all
%named as \texttt{m} which are either abstract or default methods. If we define a
%new interface \texttt{M} that extends two of these interfaces, then the method
%overriding result is shown in the table. For example, row2 col2 means \texttt{M
%  extends A1,B1}. The result is \texttt{ERROR} because the abstract method
%\texttt{T m();} in \texttt{A1} conflicts with the method \texttt{default T m()
%  {...}} in \texttt{B1}. Readers may also figure out other extending cases by
%following this way of interpretation.

% The interesting case in row4 col2 where
%Indeed, to be coherent with the idea of \textbf{conservative error}, the case should not be accepted.
%We do not see how this should behave differently from \textbf{B1},\textbf{A1}, and \textbf{B3},\textbf{A3}.
%We fear that the only retro-compatible fix for this strange behaviour is to accept all the cases of \textbf{conservative error} in a future version of Java.
%the experimental result is \texttt{B3.m}. Because if we think consistently, this

%\lstinputlisting[linerange=4-9]{../UseMixinLombok/src/methodshadowing/test5/Test5.java} % APPLY:linerange=JAVABUG

%In our approach, we choose to not model this strange behaviour (a bug?).
%Our auxiliary function $\mBody(\m,\C)$ enforce the \textbf{conservative error} strategy.
%The rest of our formalization is parametric with the definition of $\mBody(\m,\C)$, thus if Java changes its resolution strategy to a more permissive one, only minor adaptations in $\mBody(\m,\C)$ would be needed.

\begin{comment}
\subsubsection{Auxiliary function: \textsf{mtype}}
- \textsf{mtype(m, C)} : the signature of method m in C.

\[ \inferrule{
  IT(T) = \text{\emph{ann} interface } C \{ \overline{M} \} \\
  E \spc m(\overline{D} \spc \overline{x}) \{ \text{return } e; \} \in M}
{ \textsf{mtype(m,T)} = \overline{D} \to E } \]

\[ \inferrule{
  IT(T) = \text{\emph{ann} interface } C \{ \overline{M} \} \\
  m \notin M}
{ \textsf{mtype(m,T)} = \emptyset } \]

\[ \inferrule{
  IT(T) = \text{\emph{ann} interface } C \text{ extends } C_1,...,C_k \{ \overline{M} \} \\
  E \spc m(\overline{D} \spc \overline{x}) \{ \text{return } e; \} \in M}
{ \textsf{mtype(m,T)} = \overline{D} \to E } \]

\[ \inferrule{
  IT(T) = \text{\emph{ann} interface } C_0 \text{ extends } \overline{C} \{
  \overline{M} \} \\
  m \notin M}
{ \textsf{mtype(m,T)} = \bigcup \textsf{mtype}(m,\overline{D}) } \]
\end{comment}

%The \textsf{shadow} function takes two same methods (with the same name and types of arguments), and return the method which shadows the other during inheritance.
%exmples to motivate our design
%interface A{static String m(){return "A";}}
%interface C extends A{
%	default String dm(){
%		  this.m();//wrong in java
%		  A.m();
%		  C.m();//wrong in java
%		}
%}
%
%
%(1) Static methods are not inherited. Also, if one of $\{body_1,body_2\}$ is null, \textsf{shadow} simply returns the other one. Hence

\begin{comment}
%Now it is correct, but may be we do not need it?
We abbreviate typing statements on
sequences in a simple way, writing $\Gamma \vdash \overline{t}:\overline{C}$ as
shorthand for $\Gamma \vdash t_1:C_1,..., \Gamma \vdash t_n:C_n$.
\end{comment}



\begin{comment}
\subsubsection{Method Typing}


\[
\inferrule
{ }
{T_0 \spc m(\overline{T} \spc \overline{x}); \text{ OK IN I} }
\quad \textsc{(T-Meth)}
\]

\[
\inferrule
{\overline{x}:\overline{T} \vdash e:S \\ S <: T_0}
{T_0 \spc m(\overline{T} \spc \overline{x}) \text{ \{ return } e;\} \text{ OK IN
    I} \\\\ \Gamma \vdash \textbf{this}:I }
\quad \textsc{(T-MethBody)}
\]

\[
\inferrule
{IT(I)=\text{interface } I \text{ extends } \overline{J} \text{\{...\}} \\
\forall i,\text{if \textsf{mtype}}(m,J_i) = \overline(T) \to U_0, \text{then }
T_0 <: U_0 }
{T_0 \spc m(\overline{T} \spc \overline{x}); \text{ OK IN I} }
\quad \textsc{(T-MethExt)}
\]

\[
\inferrule
{\textbf{this}:I, \overline{x}:\overline{T} \vdash e:S \\ S <: T_0 \\\\
IT(I)=\text{interface } I \text{ extends } \overline{J} \text{\{...\}} \\\\
\forall i,\text{if \textsf{mtype}}(m,J_i) = \overline(T) \to U_0, \text{then }
T_0 <: U_0 }
{T_0 \spc m(\overline{T} \spc \overline{x}) \text{ \{ return } e;\} \text{ OK IN
    I} }
\quad \textsc{(T-MethBodyExt)}
\]

\[ \inferrule
{\textbf{interface } I \textbf{ extends } \overline{J} \{ \overline{M} \} \\\\
 \forall J_i \in \overline{J}, J_i \text{ OK} \\\\
 \forall m \in \overline{M}, \textsf{mbody}(m,I) \neq \error \\\\
 \forall J_i \in \overline{J}, \forall m \text{ inside } J_i,
 \textsf{mbody}(m,I) \neq \error }
{I \text{ OK}}
\quad \textsc{(T-Intf)}
 \]

Interface $I$ type checks well, if:
\begin{itemize}
\item All its super-interfaces $\overline{J}$ OK.
\item All methods inside interface $I$ are OK.
\item All methods that $I$ is inheriting from super-interfaces are OK.
\end{itemize}
\subsubsection{Subtyping}

\[ \inferrule{}{T <: T} \]

\[ \inferrule{S <: T \\ T <: U}{S <: U}\]

\[ \inferrule{\emph{ann} \spc \textbf{interface} \spc C_0 \spc \textbf{extends} \spc C_1,...,C_k \{...\}}
{C_0 <: C_1 \\ ... \\ C_0 <: C_k} \]
\end{comment}
%\subsubsection{Interface Table}

