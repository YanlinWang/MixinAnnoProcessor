\section{Introduction}\label{sec:intro}

\bruno{Generaly speaking I like the flow of the new introduction!
I would say that we do need to defend ourselfves against Scala traits, 
but the differences are probably in three aspects:

\begin{itemize}

\item The abstract state operations and the novel technique to deal with 
covariant refinement of setters; Scala traits do support state (variables and 
immutable values), but there's no support for ``with'' (and clone-like) methods for example; and 
mutable state does not support type-refinement. 

\item Automatic type-refinements (although Scala does also support \emph{manual} type-refinement 
of immutable fields);

\item We support constructors; whereas Scala traits do not. This is one key reason why classes 
are still necessary in a lnaguage like Scala.

\end{itemize}

One final point about the introduction is that language tuning is no longer 
mentioned. I think this is fine, but we have to discuss the strategy.

}
Object oriented languages always strive to offer great code reuse,
they aim to couple flexibility and rigour, expressive power and modular reasoning.
Two main idea have emerged to this end: prototype based (PB) and class based (CB) languages.
In prototype based languages an objects inherits from other objects,
and thus objects own both behaviour and state (and objects is all you have).
In class based languages an object is instance of a specific class, and classes inherits from other classes.
Here objects own the state, while classes contains behaviour and the structure of the state.

We present here a third alternative: the concept of
 \textbf{Interface based object oriented languages} (IB), where 
  objects implements interfaces, and they own implementation for the behaviour,
that is structurally defined in their interface.
  
  State is just a special kind of behaviour, and is accessed by getter and setter methods.
  
  The literature provide good examples on how easy and modular can be combining
  multiple sources of pure behaviour (traits[]), and how
  hard it is to modularly combine multiple sources of behaviour \textbf{and} state (multiple inheritance[]).
  However, our object are the only responsible to define 
  the ultimate behaviour or a method, and if such method is just a getter/setter.

Our idea could be explained by defining a novel language, with new syntax and semantic.
However, this would have a
steep learning curve; for the sake of providing a more accessible explanation, we will encode
\footnote{
In the same sense, you can encode prototype based programming over class based with
the strategy pattern, and class based programming with prototypes by creating ``class'' objects.
} our ideas
over Java, relying on the new features of Java 8; namely
interface \emph{static methods}, and \emph{default methods}, allowing interfaces to have
method implementations. While the motivation for default methods was
to allow interfaces to be extended over time while preserving backwards
compatibility (\emph{interface evolution}),
 default methods can also emulate \emph{traits}~\cite{scharli03traits}.
\footnote{The
original notion of traits by Scharli et al. prescribes, among other
things, that: 1) a trait provides a set of methods that implement
behavior; and 2) a trait does not specify any state variables, so the
methods provided by traits do not access state variables
directly. Java 8 interfaces follow similar principles too. Indeed, a
detailed description of how to emulate trait-oriented programming in
Java 8 can be found in the work by Bono et al.~\cite{bono14}. The Java 8
team designing default methods, was also fully aware of that secondary
use of interfaces, but it was not their objective to model traits:
``The key goal of adding default methods to Java was "interface
evolution", not "poor man's traits"''~\cite{goetz13default}. As a result, 
they were happy to support the secondary use of interfaces with
default methods as long as it did not make the implementation and
language more complex.}



%Still, the design is quite conservative and appears to be quite limited
% in its current form to model advanced forms of multiple inheritance.
%Indeed, our own personal experience of combining default methods 
%and multiple interface inheritance in Java to achieve multiple implementation 
%inheritance is that many workarounds and boilerplate code are needed. 
%In particular, we encountered difficulties because:

%\begin{itemize}
%
%\item {\em Interfaces have no constructors.} As a result, classes are 
%still required to create objects, leading to substantial boilerplate 
%code for initialization.
%
%\item {\em Interfaces do not have state.} This creates a tension between 
% using multiple inheritance and having state. Using setter and
%  getter methods is a way out of this tension, but this workaround
%  requires tedious boilerplate classes that later implement those
%  methods.
%
%\item {\em Useful, general purpose methods require special care in
%  the presence of subtyping.} Methods such as
%  \emph{fluent} setters~\cite{fowler2005fluentinterface}, not only require access to the
%  internal state of an object, but they also require their return types to be
%  refined in subtypes.
%
%\end{itemize}

%\noindent Clearly, a way around those difficulties would be to change
%Java and just remove these limitations. Scala's own notion of
%traits~\cite{scala-overview}, for example, allows state in traits. Of
%course adding state (and other features) to interfaces would
%complicate the language and require changes to the compiler, and this
%would go beyond the goals of Java 8 development team.

%This paper takes a different approach. Rather than trying to get
%around the difficulties by changing the language in fundamental ways,
%we show that, with a simple language feature, default methods and
%interface inheritance are in fact very expressive.

We propose a programming style, where we never use classes\footnote{More precisely, we never use the \Q@class@ keyword.}.
We call this restricted version of Java \emph{Class-less Java}.
Our proposed style enables powerful object-oriented idioms, similar to 
multiple inheritance.

Since Java was not designed to be used in this way, our style can be verbose, especially about
object instantiation.
We will annotate with \Q|@Obj| certain interfaces and rely on
the Annotation processing to generate boring and repetitive code
allowing interface instantiation.
We call such annotated interface \emph{object interfaces};
they support \emph{abstract state operations}, providing a way
to use multiple inheritance with state in Java. The abstract state
operations include various common utility methods (such as getters and
setters, or clone-like methods). In the presence of subtyping, such
operations often require special care, as their types need to be
refined. Object interfaces provide support for type-refinement and can
automatically produce code that deals with type-refinement
adequately. %With object interfaces, many Java programs can be built
%without using a single class!


%
%Object interfaces do not require changes to the Java runtime or compiler, 
%and they also do not introduce any new syntax. All three features of object interfaces are
%achieved by reinterpreting existing Java syntax, and are translated
%into regular Java code without loss of type-safety. Since no new
%syntax is introduced, it would be incorrect to call object interfaces
%a language extension or syntactic sugar. So we use the term
%\emph{language tuning} instead. Language tuning sits in between a
%lightweight language extension and a glorified library. Language
%tuning can offer many features usually implemented by a real language
%extension, but because it does not modify the language syntax
%pre-existing tools can work transparently on the tuned language.  To
%exploit the full benefits of language tuning,

We formally define the behaviour of our \Q|Obj| annotation and we provide a
prototype
implementation using Java annotations to do AST
rewriting, allowing existing Java tools (such as IDEs) to work
out-of-the-box with our implementation. As a result, we could
experiment object interfaces with several interesting Java programs,
and conduct various case studies.

%To formalize object interfaces, we propose Classless Java (CJ): a
%FeatherweightJava-style~\cite{Igarashi01FJ} calculus, which captures the essence
%of interfaces with default methods. The semantics of object interfaces
%is given as a syntax-directed translation from CJ to itself. In the
%resulting CJ code, all object interfaces are translated into regular CJ
%(and Java) interfaces with default methods. The translation is proved
%to be type-safe, ensuring that the translation does not
%introduce type-errors in client code. 
%CJ's usefulness goes beyond serving as a
%calculus to formalize object interfaces. During the development
%process of CJ, we encountered a bug in the implementation of default
%methods for the Eclipse Compiler for Java (ECJ). For the program revealing the 
%bug, ECJ behaves differently from both our formalization and Oracle's 
%Java compiler.

To evaluate the usefulness of object interfaces, we illustrate 3
applications. The first application is a simple 
solution to the Expression Problem~\cite{wadler98expression}, supporting independent 
extensibility~\cite{zenger05independentlyextensible}, and without boilerplate code. The second
application shows how embedded DSLs using fluent interfaces~\cite{fowler2005fluentinterface} 
can be easily defined using object interfaces. The last
application is a larger case study for a simple Maze game implemented with 
multiple inheritance. For the last application we show that there is a
significant reduction in the numbers of lines of code when compared 
to an existing implementation~\cite{bono14} using plain Java 8. 
Noteworthy, all applications are implemented 
without defining a single class!

In summary, the contributions of this paper are:
\begin{itemize}

\item {\bf Object Interfaces:} A simple feature that allows various
  powerful multiple-inheritance programming idioms to be expressed
  conveniently in Java. We provide examples, informal and formal description.

\item{\bf Type preservation guarantees:}
We discuss our formalization of a subset of Java8 type system and how we use this
to characterize safety properties about our annotations.
%\item {\bf Classless Java (CJ):} A simple formal calculus that models 
%the essential features of Java 8 interfaces with default methods, and 
%can be used to formally define the translation of object interfaces. 
%We prove several properties of the translation\footnote{Proofs and prototype implementation are available in
%  the supplementary materials.}.

\item {\bf Implementation and Case Studies:} We have a prototype
  implementation of object interfaces, using Java
  annotations and AST rewriting. Moreover, the usefulness of object interfaces is
  illustrated through various examples and case studies.


%\item{\bf Language Tuning:} We identify the concept of language tuning 
%and describe object interfaces as an example. We also discuss 
%how other existing approaches, such as the annotations in project 
%Lombok~\cite{lombok},  can be viewed as language tuning.

\end{itemize}
