\section{Introduction}\label{sec:intro}

\Objectoriented languages strive to offer great code reuse.
They aim to couple flexibility and rigour, expressive power and
modular reasoning.  Two main ideas emerged to this end: 
\prototypebased (PB)~\cite{Ungar87self} and \classbased (CB) languages such as
Java, C\# or Scala.  In \prototypebased
languages objects inherit from other objects. Thus objects own
both behaviour and state (and objects are all you have).
In \classbased languages an object is an instance of a specific class,
and classes inherit from other classes.  Here objects own the state,
while classes contain behaviour and the structure of the state.

This paper presents a third alternative: the concept of
\textbf{\interfacebased} \objectoriented programming languages (in short IB), where objects 
implement interfaces directly. In IB interfaces own the implementation
for the behaviour, which is structurally defined in their
interface. Programmers do not define objects directly, but delegate
the task to \emph{object interfaces}, whose role is similar to non-abstract 
classes in CB. Instantiation of objects is
done using static factory methods in object interfaces.

A key challenge in IB lies in how to model state, which is
fundamental to have stateful objects. All abstract operations in an
object interface are interpreted as \emph{abstract state
  operations}. The abstract state operations include various common
utility methods (such as getters and setters, or clone-like
methods). Objects are the only responsible to define the ultimate
behaviour of a method and, for example, if such a method is just a
setter. Anything related to state is completely contained in the
instances and does not leak into the inheritance logic.  In CB, the structure of the state is fixed and can be only grown
by inheritance.  In contrast, in IB the state is never
fixed, and methods that look like abstract setters and getters
can always receive an explicit implementation down in the inheritance
chain, improving \textbf{modularity and flexibility}.  That is, the
concept of abstract state is more fluid.

In the presence of subtyping, abstract state operations often require
special care, as their types need to be refined. Object interfaces
provide support for type-refinement and can automatically produce code
that deals with type-refinement adequately. In contrast,
 verbose explicit type-refinement
 is typically required in CB.
  We believe that such verbosity hindered and slowed
down the discovery of useful programming patterns involving
type-refinement. In particular, a recently discovered
solution~\cite{eptrivially} to the Expression
Problem~\cite{wadler98expression} in Java-like languages, shows how easy it is to solve
the problem using only type-refinement. However it took nearly 20
years since the formulation of the problem for that
solution to be presented in the literature. In IB, due to
its emphasis on type-refinement, that solution should have been much
more obvious.

One important advantage of the use of abstract state operations and
type-refinement is that it allows a new approach to
\emph{type-safe covariant mutable state}. That is, in IB,
it is possible to type-refine \emph{mutable} ``fields'' in subtypes. This is
typically forbidden in CB: it is widely known that \emph{naive} type-refinement of
mutable fields is not type-safe. While covariant refinement of mutable
fields is supported by some type systems~\cite{bruce98astatically,bruce1994paradigmatic,ernst06virtual,Saito2013933}, this requires
significant complexity and restrictions to ensure that all uses of
covariant state are indeed type-safe. 

\begin{comment}
\marcoT{%

In this paper we show how to support type-safe
\textbf{field removal},
\textbf{field type refinement} allowing a kind of covariant setters refinement,
and \textbf{multiple inheritance}.}
\end{comment}

Finally, another advantage of IB is its support for
multiple inheritance. The literature provides good examples on how
easy and modular it is to combine multiple sources of pure behaviour,
using mechanisms such as traits~\cite{scharli03traits}. In Java
multiple \emph{interface} inheritance has been supported since
inception, and in Java 8 default methods~\cite{goetz12fdefenders} bring some of the
advantages of traits into Java. In contrast, the literature~\cite{Sak89dis,bracha90mixin,malayeri2009cz}
is also rich on how hard it is to modularly combine multiple sources
of behaviour \textbf{and} state with multiple \emph{implementation}
inheritance of classes. In IB there is only multiple
interface inheritance, yet programmers can still use state via the
abstract state operations. IB enables powerful 
idioms using multiple inheritance and state.

%Traits or Java8 interfaces still assume a CB model: 
%But traits still require classes, which
%are responsible for object construction and adding state.

%To retain the easiness and modularity of combining multiple sources of
%pure behaviour, in IB state is just a special kind of behaviour
%implementation.
%\marco{NO NO!
%the advantage is not just multiple composition 
%(by the way, that is the term you are searching for, not multiple inheritance)
%By not having state reified in the ideal/platonic level, it is not fixed
%what is state and what is not, thus you can (model an encoding of) REMOVE fields and you can refine
%fields.}
% Objects are the only responsibles to define the
%ultimate behaviour of a method and, for example, if such method is
%just a setter. Anything related to state is completely contained in
%the instances and do not leak in the interfaces.

IB could be explained by defining a novel language, with new syntax
and semantics. However, this would have a steep learning curve.  We
take a different approach instead. For the sake of providing a more
accessible explanation, we will embed our ideas directly into Java. 
Our IB embedding relies on the
new features of Java 8 for interfaces: interface \emph{static methods}; and
\emph{default methods}, which allow interfaces to have method
implementations. In the context of Java, what we propose is a programming
style, where we never use classes\footnote{More precisely, we never
  use the \Q@class@ keyword.}.  We call this restricted version of
Java \emph{Classless Java}. 

%Still, the design is quite conservative and appears to be quite limited
% in its current form to model advanced forms of multiple inheritance.
%Indeed, our own personal experience of combining default methods 
%and multiple interface inheritance in Java to achieve multiple implementation 
%inheritance is that many workarounds and boilerplate code are needed. 
%In particular, we encountered difficulties because:

%\begin{itemize}
%
%\item {\em Interfaces have no constructors.} As a result, classes are 
%still required to create objects, leading to substantial boilerplate 
%code for initialization.
%
%\item {\em Interfaces do not have state.} This creates a tension between 
% using multiple inheritance and having state. Using setter and
%  getter methods is a way out of this tension, but this workaround
%  requires tedious boilerplate classes that later implement those
%  methods.
%
%\item {\em Useful, general purpose methods require special care in
%  the presence of subtyping.} Methods such as
%  \emph{fluent} setters~\cite{fowler2005fluentinterface}, not only require access to the
%  internal state of an object, but they also require their return types to be
%  refined in subtypes.
%
%\end{itemize}

%\noindent Clearly, a way around those difficulties would be to change
%Java and just remove these limitations. Scala's own notion of
%traits~\cite{scala-overview}, for example, allows state in traits. Of
%course adding state (and other features) to interfaces would
%complicate the language and require changes to the compiler, and this
%would go beyond the goals of Java 8 development team.

%This paper takes a different approach. Rather than trying to get
%around the difficulties by changing the language in fundamental ways,
%we show that, with a simple language feature, default methods and
%interface inheritance are in fact very expressive.





  %With object interfaces, many Java programs can be built
%without using a single class!

Using Java annotation processors, we produce an implementation of
Classless Java, which allow us to stick to pure Java 8. By annotating
with \Q|@Obj| the interfaces that represent object interfaces, we can
automatically generate code for interface instantiation and
type refinement. Such code should not be needed in the first place in
a real IB language, and the annotation processor allows us to
transparently hide it from Java programmers.  The implementation works
by performing AST rewriting, allowing most existing Java tools (such as
IDEs) to work out-of-the-box with our implementation. Moreover, the
implementation blends Java's conventional CB style and IB smoothly.
As a result, we experiment object interfaces with several interesting
Java programs, and conduct various case studies.  Finally, we also
formally define the behaviour of our \mixin annotation and discuss
its properties.

%Since Java was not designed to be used in this way, our style can be verbose, especially about object instantiation.


%
%Object interfaces do not require changes to the Java runtime or compiler, 
%and they also do not introduce any new syntax. All three features of object interfaces are
%achieved by reinterpreting existing Java syntax, and are translated
%into regular Java code without loss of type-safety. Since no new
%syntax is introduced, it would be incorrect to call object interfaces
%a language extension or syntactic sugar. So we use the term
%\emph{language tuning} instead. Language tuning sits in between a
%lightweight language extension and a glorified library. Language
%tuning can offer many features usually implemented by a real language
%extension, but because it does not modify the language syntax
%pre-existing tools can work transparently on the tuned language.  To
%exploit the full benefits of language tuning,



%To formalize object interfaces, we propose Classless Java (CJ): a
%FeatherweightJava-style~\cite{Igarashi01FJ} calculus, which captures the essence
%of interfaces with default methods. The semantics of object interfaces
%is given as a syntax-directed translation from CJ to itself. In the
%resulting CJ code, all object interfaces are translated into regular CJ
%(and Java) interfaces with default methods. The translation is proved
%to be type-safe, ensuring that the translation does not
%introduce type-errors in client code. 
%CJ's usefulness goes beyond serving as a
%calculus to formalize object interfaces. During the development
%process of CJ, we encountered a bug in the implementation of default
%methods for the Eclipse Compiler for Java (ECJ). For the program revealing the 
%bug, ECJ behaves differently from both our formalization and Oracle's 
%Java compiler.

\begin{comment}
To evaluate the usefulness of object interfaces, we illustrate
\numOfCaseStudies \bruno{needs updates}\yanlin{updated} 
applications. The first application is a simple 
solution to the Expression Problem~\cite{wadler98expression}, supporting independent 
extensibility~\cite{zenger05independentlyextensible}, and without boilerplate code. The second
application shows how embedded DSLs using fluent interfaces~\cite{fowler2005fluentinterface} 
can be easily defined using object interfaces. The third
application is a case study for a simple Maze game implemented with 
multiple inheritance. For this application we show that there is a
significant reduction in the number of lines of code when compared 
to an existing implementation~\cite{bono14} using plain Java 8. The last
application is a relatively larger case study of refactoring of a simple interpreter, showing that our
approach can benefit modularity and scale to real code base. Noteworthy, the first three applications are implemented 
without defining a single class! The last one contains several classes that cannot be 
converted to interfaces due to the limitation of our implementation of \mixin.
\end{comment}

While the Java embedding has obvious advantages from the practical
point-of-view, it also imposes some limitations that a new IB language
would not have. In particular, supporting proper encapsulation is
difficult in Java due to limitations of Java interfaces. We discuss
these limitations, possible workarounds, and native language support in Section~\ref{sec:futurework}.

In summary, the contributions of this paper are:
\begin{itemize}

\item {\bf IB and Object Interfaces:} a novel take on object orientation, allowing
  powerful programming idioms using multiple-inheritance, 
  type-refinement and abstract state operations.

\item {\bf Classless Java:} a practical realization of IB in
  Java. Classless Java is implemented using annotation processing, 
  allowing most tools to work transparently with our approach. 
  We provide examples, informal and formal descriptions of Classless
  Java.
  Existing Java projects can use our approach and still be
  backward compatible with their client, in a way that is formally specified
  by our safety properties.

\item {\bf Type-safe covariant mutable state:} we show how the 
 combination of abstract state operations and type-refinement enables 
 a form of mutable state that can be covariantly refined in a type-safe way.

%\item{\bf Type preservation guarantees:}\bruno{needs rephrasing}
%We discuss our formalization of a subset of Java8 type system and how we use this
%to characterize safety properties about our annotations.
%\item {\bf Classless Java (CJ):} A simple formal calculus that models 
%the essential features of Java 8 interfaces with default methods, and 
%can be used to formally define the translation of object interfaces. 
%We prove several properties of the translation\footnote{Proofs and prototype implementation are available in
%  the supplementary materials.}.

\item {\bf Applications and case studies:} we illustrate the usefulness of IB
  through various examples and case studies.

%\item{\bf Language Tuning:} We identify the concept of language tuning 
%and describe object interfaces as an example. We also discuss 
%how other existing approaches, such as the annotations in project 
%Lombok~\cite{lombok},  can be viewed as language tuning.

\end{itemize}
