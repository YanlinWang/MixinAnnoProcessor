\section{Conclusion}\label{sec:conclusion}


Before Java 8, concrete methods and static methods were not allowed
to appear in interfaces.  Java 8 allows static interface methods and
introduces \emph{default methods}, which enables implementations
inside interfaces. This had an important positive consequence that
was probably overlooked: the concept of class
(in Java) is now (almost) redundant and unneeded.
We proposed a programming style, called ClassLess Java, where
truly \objectoriented programs and (reusable) libraries
can be defined and used without ever defining a single class.

However, using this programming style directly in Java is very verbose.
To avoid syntactic boilerplate
caused by Java not being originally designed to work without classes,
we introduce a new annotation, \mixin, which provides default implementations
for various methods (e.g. getters, setters, with-methods) and a
mechanism to instantiate objects. 
We leverage on annotation processing and the Lombok library, in this way
\mixin is just a normal Java library; thus our proposed style can be integrated
in any Java project.

The \mixin annotation helps programmers
to write less cumbersome code while coding in Classless Java. Indeed, 
we think the obtained gain is so high that Classless Java with the \mixin
annotation can be less cumbersome than full Java.


Class less Java is just a programming style, but is 
showing the way for a new flavour of object orientation:
We propose \interfacebased \objectoriented languages (IB),
as opposed to \classbased or \prototypebased.
In IB state is not modelled at the platonic/ideal level
but is handled excursively by instances.
This unlock useful code reuse patterns, as shown in~\ref{sec:ep}.



Interestingly, without classes there is also no subclassing. This scratches an old
  itching point in the long struggle of subtyping versus subclassing:
  according to some authors, from a software engineering perspective,
  interfaces are just a kind of classes. Others consider more
  opportune to consider interfaces as pure types. We do not know how to conciliate
  those two viewpoints and Classless Java design.
  Classless Java does not have classes purely in the Java sense.  
% Classless Java encourages coding in a more flexible way by either
% keeping a higher abstraction level (interfaces are a more abstract
% concept than classes), or relying on concrete object initialization
% (the \Q@new I(){...}@ construct).


% More generally, we identify the concept of \emph{language tuning}.
% We identify libraries that are already performing language tuning (Lombok and Cofoja), and 
% we forecast many different kinds of language tuning will emerge on suitable platforms like Java or the C\# CLR.
% We identify various kinds of safety guarantees that can be offered by language tuning, but the door is open for more flavors of safety guarantees to emerge.
