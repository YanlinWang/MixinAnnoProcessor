\section{Future work}\label{sec:futurework}
In this section we discuss potential future work.
\begin{comment}
\subsection{Encapsulation}
The biggest limitation of our approach is the inability to model visibility restrictions. For example, the absence
of support for private/protected methods in Java 8 interfaces forces all
members of interfaces to be public, including static methods. Since we use
abstract methods to encode state, our state is always all public. Still, because
the state can only be accessed by methods, it
is impossible for the user to know if a certain method maps directly to a field
or if it has a default implementation.  If the user wants a constructor that
does not directly maps to the fields, (as for secondary constructors in Scala)
he can simply define its own \Q@of@ method and delegate on the generated one:
\begin{lstlisting}
@Obj interface Point {  int x(); int y();
    static Point of(int val){return Point.of(val,val);} }
\end{lstlisting}
However, the generated \Q@of@ method would also be present and public.  If a
future version of Java was to support \emph{static private methods in
  interfaces} we could extend our code generation to handle also encapsulation.

However, interfaces as a whole can have public or
package private (java default) visibility.

We can add a second annotation \Q!@Exposed!  that leverages on this edge: An
interface without exposed works as usual, but if any method of a public \mixin
interface is annotated with \Q!@Exposed!, we can apply a translation where a new
(package private) interface type is introduced, and the original interface become
just a facade.  For example:
\begin{lstlisting}
@Obj public interface Person{
  void name(String val);
  @Exposed default void rename(String newName){ if(/*valid name*/){ this.name(val);}}
  @Exposed String name();
  @Exposed static Person from(String val){ if(/*valid name*/){return Person.of(val);}
    throw /*invalid name*/}  }
\end{lstlisting}
becomes
\begin{lstlisting}
public interface Person{
  void rename(String newName)
  String name();
  static Person from(String val){ return Person$.from(val);} }

@Obj interface Person$ extends Person{//will be further expanded by @Obj
  void name(String val);
  default void rename(String newName){ if(/*valid name*/){ this.name(val);}}
  String name();
  static Person from(String val){ if(/*valid name*/){return Person$.of(val);}
    throw /*invalid name*/}  }
\end{lstlisting}

This is not a perfect solution, since
\Q@Person$@ can still be seen inside the \Q@Person@ package and heirs of
\Q@Person$@,
however it is surprising we achieve such of a good result without any language
support for privacy in interfaces.
\end{comment}

%\subsection{Qualifiers in Methods} %{Private state
\paragraph{Encapsulation} %{Private state
Since we are using Java8 to encode IB, we are unable to directly support qualifies
like private or protected. In Java 9 private methods will be allowed in
interfaces. Abstract state operations are, however, abstract methods,
so they can not be private.
Still, because the state can only be accessed by methods, it
is impossible for the user to know if a certain method maps directly to a field
or if it has a default implementation.

As shown by Ferruccio in [...] interfaces are enough to support privateness by a
 variation of the facade pattern:
\begin{lstlisting}
@Obj interface InternalData implements ExposedData {
    /*all the methods*/} //package visibility
public interface ExposedData {
  /*just the exposed interface*/
  static ExposedData of(...) {
    return InternalData.of(...);}  }
\end{lstlisting}

This is a safe and explicit way to list all the public interfaces and to hide all the implementation
details from outside of the package.\footnote{
Note how \Q@ExposedData.of@ do not need to map directly to the fields, and can also perform
invariant checking.}

Using packages as a boundary we can make all object interfaces ``with privateness''
invisible types for the rest of the program.
This preserve encapsulation, but limits inheritance:
package protected interfaces can not be seen,
thus can not be implemented from outside the package.

Using Java9 private interface methods%
\footnote{Or some ugly pattern available today like public static classes with private methods}
one can define (and inherit) private behaviour; but private state inheritance in IB poses a logical challenge:
How can the heir produce initialization code for a field whose very existence is secret?

CB get around this problem in two ways: the constructor have to call a super-constructor, and
some fields can be pre-initialized.
The first idea would not provide any encapsulation benefit in IB,
since generated \Q@of@ methods expose all the field information anyway.

Pre-initialized fields could be supported by IB if we support a cache expression,
that is \Q@cache(e)@ would evaluate the expression a single time for each object, and cache the result
somewhere.\footnote{
For example, in a the compiled bytecode for objects
could have an extra field for each cached expression in its methods.
Such cache expression would be quite useful for other reasons,
including replacing involved initialization patterns.}

In this way
\marco{I fixed the code down, was wrong, there was no mention of hello in the original source}
\begin{lstlisting}
@Obj interface Text{
  private String msg();//assuming Java9
  private Text msg(String val);{f("hello");}
  }
\end{lstlisting}
Would be expanded into
\begin{lstlisting}
@Obj interface Text{
  private Box<String> _msg(){
    return cache(Box<String>.of("hello"));}
  private String msg(){return _msg().val();}
  private Text msg(String val){
    return _msg().val(val);}
  }
\end{lstlisting}

Where we could have a predefined \Q@Box@ type as\\*
\Q|@Obj interface Box<T>{ T val(); Box<T> val(T val);}|




\begin{comment}
\subsection{Class Invariants in ClassLess Java}
Since objects are created by automatically generated methods, another limitation
of our current approach is that there is no place where the user can dynamically
check for class invariants. In Java often we see code like
\begin{lstlisting}
class Point{ int x; int y;
  Point(int x; int y){this.x=x;this.y=y; assert this.checkInvariant();}
  private boolean checkInvariant(){... x>0,y>0...}
}
\end{lstlisting}

We are considering an extension of our annotation where
default methods with the special name \Q@checkInvariant()@ will be called inside the \Q@of@ methods.
If multiple interfaces are implemented, and more then one offers
\Q@checkInvariant()@,  a composed implementation could be automatically generated, composing by \Q@&&@ the various competing implementations.
\end{comment}
%\bruno{removed the invariants stuff; we need space, I think.}

%\subsection{Clone, toString, equals and hashCode}
\paragraph{Clone, toString, equals and hashCode}
Methods originally defined in Java class \Q@Object@, as \Q@clone@ and
\Q@toString@, can be supported by our approach with special care. If an
interface annotated with \mixin asks an implementation for \Q@clone@,
\Q@toString@, \Q@equals@ or \Q@hashCode@ we can easily generate one from the
fields.\footnote{In particular, for clone we can do automatic return type
  refinement as we do for \Q@with-@ and fluent setters. Note how this would
  solve most of the Java ugliness related to \Q@clone@ methods.}  However, if
the user wishes to provide his own implementation, since the method is also
implemented in \Q@Object@, a conflict arises. The generated code can resolve the
conflict inside \Q@of@, by implementing the method and delegating it to the user
implementation, thus

\begin{lstlisting}
@Obj interface Point { int x(); int y();
    default Point clone() { return Point.of(0,0); } } //user defined clone
\end{lstlisting}
would expand into

\begin{lstlisting}
interface Point { ...//as before
    public static Point of(int _x, int _y) {
        return new Point() {...
            public Point clone() {
                return Point.super.clone();}};} }
\end{lstlisting}

%\paragraph{A real IB language}
%In this article we presented the concept of IB as a programming pattern over Java.
%We do not wish to define here a complete IB language, just to show the reader IB as
%a promising way to design a new oo language.
%The first task is to define the set of abstract state operations.
%Of course we need getters; \Q@withX@ methods seams to be pretty convenient, and they enable a
%clean function programming style that would be otherwise too cumbersome.
%
%Currently we are offering both setters and Fluent setters.
%We believe be fluent setters to be sufficient; we could drop the concept of (non fluent)setters.
%
%Manually defining abstract state methods can be verbose, we could offer a convenient syntax to
%declare get/set/with.
%For example
%\begin{lstlisting}
%obj interface Foo{
%  String bar;//== String bar(); Foo withBar(String val);
%  var String beer;//also fluent setter
%  }
%\end{lstlisting}
%
%We may also want to offer a compact syntax for property updaters and generalized with methods;
%\begin{lstlisting}
%obj interface Foo{
%  updater Bar, Beer//== Foo set(Bar val); Foo set(Beer val);
%  with Foo, Beer//== Foo with(Bar val); Foo with(Beer val);
%  }
%\end{lstlisting}
%Finally, certain methods may have a specific meaning, for example
%if a private method called \Q@init@ is present, it will be called after construction,
%and it may be handled specially during multiple implementations, so that all relevant \Q@init@ methods are
%called in some order.
