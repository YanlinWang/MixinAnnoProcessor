\documentclass[a4paper,UKenglish]{lipics-v2016}
%This is a template for producing LIPIcs articles. 
%See lipics-manual.pdf for further information.
%for A4 paper format use option "a4paper", for US-letter use option "letterpaper"
%for british hyphenation rules use option "UKenglish", for american hyphenation rules use option "USenglish"
% for section-numbered lemmas etc., use "numberwithinsect"
 
\usepackage{microtype}%if unwanted, comment out or use option "draft"

%\graphicspath{{./graphics/}}%helpful if your graphic files are in another directory

\bibliographystyle{plainurl}% the recommended bibstyle

\usepackage{url}
\usepackage{makeidx}  % allows for indexgeneration
\usepackage{listings}
\usepackage{color}
\usepackage{xspace}
\usepackage{times}
\usepackage{comment}
\usepackage{rotating} 
\addtolength{\textwidth}{0.5ex}
\usepackage{enumitem}
\setlist{nosep}
\setlist{leftmargin=1ex}
\setlist[itemize]{noitemsep,nolistsep}
\setlist[enumerate]{noitemsep,nolistsep}
%\let\oldparagraph\paragraph
%\renewcommand{\paragraph}[1]{\vspace{-5pt}\oldparagraph{#1}}
\renewcommand{\paragraph}[1]{\noindent\emph{#1}}

\let\oldsubsection\subsection
\renewcommand{\subsection}[1]{\vspace{-5pt}\oldsubsection{#1}\vspace{-5pt}}

\newenvironment{listing}{\vspace{-3pt}\begin{lstlisting}}{\end{lstlisting}\vspace{-3pt}}
%
\lstset{language=Java,
  basicstyle=\ttfamily\footnotesize,%\small,%\scriptsize,
  keywordstyle=\bfseries,%\color{darkRed},
  showstringspaces=false,
  mathescape=true,
  xleftmargin=0pt,
  xrightmargin=0pt,
  breaklines=false,
  breakatwhitespace=false,
  breakautoindent=false,
  linewidth=4\textwidth,% should be enough
%  identifierstyle=\idstyle
 morekeywords={method,Use,This}
}
%
\newcommand\saveSpace{}
%\newcommand\saveSpace{\vspace{-3pt}}

\newcommand\Rotated[1]{\begin{turn}{90}\begin{minipage}{12em}#1\end{minipage}\end{turn}}
%
%marco
\newcommand{\Q}{\lstinline}
\newcommand\Opt[1]{#1 ?}
%bnf
\newenvironment{bnf}{$\begin{array}{lcll}}{\end{array}$}
\newcommand{\prodFull}[3]{#1&::=&#2&\mbox{#3}}
%\newcommand{\prodFull}[3]{#1&::=&\mbox{#2}&\mbox{#3}}
\newcommand{\prodInline}[2]{#1::=#2}
\newcommand{\prodNextLine}[2]{&&#1&\mbox{#2}}
\newcommand{\terminal}[1]{%
\ensuremath{$\texttt{#1}$}%
}
\newcommand{\terminalCode}[1]{\mbox{\lstinline{#1}}}
\newcommand{\metavariable}[1]{%
\ensuremath{\mathit{#1}}%
}
%----------------------
\newcommand\mL{\metavariable{L}}
\newcommand\mC{\metavariable{C}}
\newcommand\mT{\metavariable{T}}
\newcommand\mV{\metavariable{V}} %code value t or L
\newcommand\mMD{\metavariable{MD}}
\newcommand\mTD{\metavariable{TD}}
\newcommand\mCD{\metavariable{CD}}
\newcommand\mD{\metavariable{D}}
\newcommand\me{\metavariable{e}}
\newcommand\mx{\metavariable{x}}
\newcommand\mm{\metavariable{m}}
\newcommand\mt{\metavariable{t}}
\newcommand\use{\terminalCode{Use}}
\newcommand\oC{\mbox{\lstinline@\{@}}
\newcommand\cC{\mbox{\lstinline@\}@}}
\newcommand\oR{\mbox{\lstinline@(@}}
\newcommand\cR{\mbox{\lstinline@)@}}
%--------------------------
\newcommand{\mynotes}[3]{{\color{#2} {\sc #1}: #3}}
\newcommand\bruno[1]{\mynotes{Bruno}{red}{#1}}
\newcommand\marco[1]{\mynotes{Marco}{blue}{#1}}

\newcommand{\syndef}{$::=$}

\newcommand\name{{\bf $42_{\mu}$}\xspace}


% Author macros::begin %%%%%%%%%%%%%%%%%%%%%%%%%%%%%%%%%%%%%%%%%%%%%%%%
\title{Separating Use and Reuse to Improve Both}
\titlerunning{Separating Use and Reuse to Improve Both}

\author{Authors omitted for double-bind review.}
%% Please provide for each author the \author and \affil macro, even when authors have the same affiliation, i.e. for each author there needs to be the  \author and \affil macros
%\author[1]{John Q. Open}
%\author[2]{Joan R. Access}
%\affil[1]{Dummy University Computing Laboratory, Address/City, Country\\
%  \texttt{open@dummyuniversity.org}}
%\affil[2]{Department of Informatics, Dummy College, Address/City, Country\\
%  \texttt{access@dummycollege.org}}
%\authorrunning{J.\,Q. Open and J.\,R. Access} %mandatory. First: Use abbreviated first/middle names. Second (only in severe cases): Use first author plus 'et. al.'
\begin{comment}
\author{Marco Servetto\inst{1} \and Bruno C. d. S. Oliveira\inst{2}}
%
\authorrunning{M. Servetto and B. Oliveira} % abbreviated author list (for running head)
%
%%%% list of authors for the TOC (use if author list has to be modified)
\tocauthor{Marco Servetto and Bruno C. d. S. Oliveira}
%
\institute{Victoria University of Wellington, New Zealand,\\
\and
The University of Hong Kong, Hong Kong\\}
\end{comment}

\Copyright{John Q. Open and Joan R. Access}%mandatory, please use full first names. LIPIcs license is "CC-BY";  http://creativecommons.org/licenses/by/3.0/

\subjclass{Dummy classification -- please refer to \url{http://www.acm.org/about/class/ccs98-html}}% mandatory: Please choose ACM 1998 classifications from http://www.acm.org/about/class/ccs98-html . E.g., cite as "F.1.1 Models of Computation". 
\keywords{Dummy keyword -- please provide 1--5 keywords}% mandatory: Please provide 1-5 keywords
% Author macros::end %%%%%%%%%%%%%%%%%%%%%%%%%%%%%%%%%%%%%%%%%%%%%%%%%

%Editor-only macros:: begin (do not touch as author)%%%%%%%%%%%%%%%%%%%%%%%%%%%%%%%%%%
\EventEditors{John Q. Open and Joan R. Acces}
\EventNoEds{2}
\EventLongTitle{42nd Conference on Very Important Topics (CVIT 2016)}
\EventShortTitle{CVIT 2016}
\EventAcronym{CVIT}
\EventYear{2016}
\EventDate{December 24--27, 2016}
\EventLocation{Little Whinging, United Kingdom}
\EventLogo{}
\SeriesVolume{42}
\ArticleNo{23}
% Editor-only macros::end %%%%%%%%%%%%%%%%%%%%%%%%%%%%%%%%%%%%%%%%%%%%%%%

\begin{document}

\maketitle

\begin{abstract}
\saveSpace\saveSpace\saveSpace
In most OO languages subclassing/inheritance implies
subtyping. This is considered by many a conceptual design error, but it
seems required for technical reasons, due to what we call the
\emph{this-leaking problem}. This problem shows that separating
inheritance from subtyping is non-trivial and requires a significant
departure from the OO models in existing mainstream OO languages.
We are aware of at least 3 independently designed research languages 
addressing this limitation: \emph{TraitRecordJ}, \emph{Package Templates}, \emph{DeepFJig}.
The goal of this paper is to synthesize and improve on
the main ideas of those very different designs into a nominally typed
minimalist language, called \name, which is quite
natural to OO programmers.
By making our type system distinguish between code-use and code-reuse
we can separate inheritance and subtyping, while avoiding 
redundant abstract declarations required in TraitRecordJ and
DeepFJig. At the same time \emph{self construction},
\emph{binary methods} and \emph{recursive types} are also supported.
Moreover, we provide a novel and elegant solution to uniformly
handle behaviour and state within trait composition.
These ideas have been implemented in the 42 language, 
which supports all the examples we show in the paper.
\saveSpace\saveSpace\saveSpace

%\keywords{Code Reuse, Object-Oriented Programming}
\saveSpace\saveSpace\saveSpace
\end{abstract}

\section{Introduction}\label{sec:intro}

In mainstream OO languages like Java, C++ or C\# subclassing 
implies subtyping. For example, in Java a subclass definition, such as 
\Q@class A extends B {}@
\noindent does two things at the same time:
1) {\bf inherits} \lstinline{B} code; and {\bf create
a subtype} of \emph{B}. Therefore in a language like Java 
a subclass is \emph{always} a subtype of the extended class.

Historically, there has been a lot of focus on the importance of
separing subtyping from subclassing~\cite{cook}.  This is claimed to be
good for code-reuse, design and reasoning. There are at
least two distinct situations where the separation of subtyping and 
subclassing is helpful.

\begin{itemize}

\item {\bf Allowing inheritance/reuse even when subtyping is impossible:} 
In some situations a subclass contains methods whose signatures 
are incompatible with the superclass, yet inheritance is still
possible. A typical example, which was illustrated by Cook et al.~\cite{cook}, are 
classes with \emph{binary methods}~\cite{bruce96binary}.

\item {\bf Preventing unintended subtyping:} For certain classes we
  would like to inherit code without creating a subtype even if, from
  the typing point of view, subtyping is still possible. A typical
  example~\cite{LaLonde:1991:SSS:110673.110679} of this are methods for collection classes such as Sets and
  Bags. Bag implementations can often inherit 
  from Set implementations, and the interfaces of the two collection types are
  similar and type compatible. 
  However, from the logical point-of-view a Bag is \emph{not a
    subtype} of a Set. 

\end{itemize}

Type systems based on structural typing can deal with the first
situation well, but not the second. Since structural subtyping
accounts for the types of the methods only, a Bag would be a subtype
of a Set if the two interfaces are type compatible. For dealing with
the second situation nominal subtyping is preferable. With nominal
subtyping an explicit subtyping relation must be signalled by the
programmer. Thus if subtyping is not desired, the idea is that 
programmer can simply {\bf not} declare a subtyping relationship.

While there is no problem in subtyping without subclassing, in the design
of most nominal OO languages subtyping implies subtyping in a
fundamental way. This is because of what we call the
\emph{this-leaking problem}, illustrated by the following
(Java) code:

\begin{lstlisting}[language=Java]
  class A{ int ma(){return Utils.m(this);} }
  class Utils{static int m(A a){..}}
\end{lstlisting}

Method \lstinline{A.ma} passes \lstinline{this} as \lstinline{A} to \Q@Util.m@.
This code seems correct, and there is no subtyping/subclassing.
Now, lets add a class \Q@B@

\begin{lstlisting}[language=Java]
  class B extends A{ int mb(){return this.ma();} }  
\end{lstlisting}

%%Class \lstinline{B} does two things at the same time: 
%%1) it {\bf inherits} the method \lstinline{ma} from
%%\lstinline{A}; and 2) it creates a {\bf subtype} of \lstinline{A}.
We can see an invocation of \lstinline{A.ma} inside
\lstinline{B.mb}, where the self-reference \lstinline{this} is of type \lstinline{B}. 
The execution will eventually call \lstinline{Util.m} with an
instance of \lstinline{B}. However, this can be correct only if \lstinline{B} is a subtype of
\lstinline{A}. 

%If Java was to support a mechanism to allow reuse/inheritance 
%without introducing subtyping, such as:
%
%\begin{lstlisting}[language=Java]
%  class B inherits A{ int mb(){return this.ma();} }
%\end{lstlisting}
%
%\noindent Then an invocation of 
%\lstinline{mb} would be type-unsafe (i.e. it would 
%result in a run-time type error). 
%Note that here the intention of using the imaginary keyword {\bf
%  inherits} is to allow the code from \lstinline{A} to be inherited 
%without \lstinline{B} becoming a subtype of \lstinline{A}. 
%However this breaks type-safety. The problem is that the
%self-reference \lstinline{this} in class \lstinline{B} has 
%type \lstinline{B}. Thus, when \lstinline{this} is passed as an argument to 
%the method \lstinline{Utils.m} as a result of the invocation of
%\lstinline{mb}, it will have a type that is incompatible with the
%expected argument of type \lstinline{A}.  


As a though experiment, imagine that Java code-reuse (extends keyword) was not introducing subtyping: then an invocation of 
\lstinline{B.mb} would result in a run-time type error.
The problem is that the
self-reference \lstinline{this} in class \lstinline{B} has 
type \lstinline{B}. Thus, when \lstinline{this} is passed as an argument to 
the method \lstinline{Utils.m} as a result of the invocation of
\lstinline{mb}, it will have a type that is incompatible with the
expected argument of type \lstinline{A}.  

Every OO language with the minimal features exposed in the example (using \lstinline{this},
extends and method calls) is forced to accept that subclassing implies
subtyping.\footnote{C++ allows to "extends privately", but it is a limitation over
  subtyping visibility, not over subtyping itself.  The
  former example would be accepted even if \lstinline{B} was to
  "privately extends" \lstinline{A}}.
  
In essence we believe that a problem with the design of classes in
languages like Java is that classes
% do too many things at once. In particular they 
act both as units of \emph{use} and \emph{reuse}.
That is, in languages like Java, class names can be \emph{use}d as
type and instantiated.  Classes can also be subclassed to provide
\emph{reuse} of code.

What the this-leaking problem shows is that 
adopting a more flexible nominally typed OO model where subclassing
does not imply subtyping is not trivial, and a more substantial change
in the language design is necessary. 
We are aware of at least 3 independently designed research languages 
that address this limitation
\begin{itemize}
\item Delta-Trait (DT)~\cite{Bettini:2010:ISP:1774088.1774530,BETTINI2013521,Bettini2015282}:
Each  constructs  has one single responsibilty: classes instantiate objects,
interfaces induce types, records express state and traits are unit of reuse.
\item Package Templates (PT)~\cite{KrogdahlMS09,DBLP:journals/taosd/AxelsenSKM12,DBLP:conf/gpce/AxelsenK12}:
An extension of (full) Java where new packages can be ``syntetized'' by mixing
and integrating code templates. As an extension of java, PT allows but not requires
separation of inheritance and subtyping.
\item DeepFjig~\cite{deep,servetto2014meta,DBLP:journals/iandc/LagorioSZ12}:
A module composition language supporting composing nested classes, where the main idea is that
nested classes with the same name are recursivelly composed.
\end{itemize}

This paper aims at showing a simple language design, called \name, to
completely decouple subtyping and subclassing in a nominally typed OO
language. The key idea is to divide between code designed for
\textbf{USE} and code designed for \textbf{REUSE}. 
In \name there are two separate concepts: classes
and traits~\cite{Traits:ECOOP2003}. Classes are meant for code use, and cannot be used
for reuse. In some sense classes in \name are like final classes in
Java. Traits are meant for code reuse and multiple traits can be
composed to form a class which can then be instantiated. Traits 
cannot be instantiated (or used) directly. Such design allows the
subtyping and code reuse to be treated separately, which in turn
brings several benefits in terms of flexibility and code reuse.
%\marco{We need to talk of unanticipated extensions?}
%\bruno{Talk more about typing aspects here. Summarize the important 
%aspects of the design of \name.}

\bruno{Need to say something about state?}

The language design of \name is adopted by the 42 programming 
language, All the examples shown in this paper can be run in 42.
The compiler for 42 is available at: 

\url{http://l42.is}


\begin{comment}
First we show a minimal language that illustrates the basic ideas. 
Then we show how mutually recursive types
are supported, how state/constructors/fields are supported,
how we can extend the language with nested classes.
Finally we show 42, a full blown language build around our ideas of
reuse.

Our design leverage on traits~\cite{ducasse2006traits}: a well know mechanisms for pure
code reuse.
\end{comment}

\subsection{Contributions:}
\begin{itemize}
\item We identify the {\bf this-leaking} problem and argue why it
  makes the separation of subclassing and subtyping difficult.
\item Inspired by previous language designs that solve the
  this-leaking problem, we synthesize the key ideas in {\bf a novel and
  simple language design}. This language is the logic core of the language 42, and 
  all the examples in the paper can be directly encoded as valid 42 programs. 

\item We illustrate how the new design {\bf improves both code use and code
  reuse}.
\item We propose {\bf a clean and elegant approach to handle of state} in a
  languages that support traits.
%\item The new language design is {\bf adopted by the 42 language}. All the
 % examples in the paper are valid 42 programs. 

\begin{comment}
\item simple approach to fully separate inheritance and subtyping
\item easy to grasp model of compilation were typing is late but not
  too late (lazy?)\bruno{this idea is important: it's a key design
    decision. We need to talk about it in the introduction and offer
    some motivation/justification. }
\item clean and elegant handling of state in trat/module composition language
\item extension with nested classes is natural and
  powefull\bruno{wondering if this is necessary. I think there are
    several interesting/novel ideas already. Covering too much will
    result on not saying enough about each individual ideas.}
\item more operators can be accomodated without changing the general
  model
\end{comment}
\end{itemize}






\section{The Design of \name: Separating Use and Reuse}

This section presents the overview of \name, and it illustrates the
key ideas of its design. In particular we illustrate how to separate code use and 
code reuse.
% in \name. %, and how this allows improving both. 

%\subsection{The Design of \name: Decoupling Use from Reuse}

\subsection{Classes in \name: A mechanism for code Use}
%\name has a substantially different design from Java-like languages.
The concept of a class in \name provides a mechanism for code-use
only. This means that there is actually no subclassing:
classes are roughly equivalent to final classes in Java.  Thus,
compared to Java-like languages, the most noticeable difference is the
absence of the \Q@extends@ keyword in \name. 

To illustrate classes in \name consider the example in Section~\ref{}:

\begin{lstlisting}
 A: { method int ma() Utils.m(this) }//note, no {return _}
 Utils:{ class method int m(A a)/*method body here*/ }
\end{lstlisting} 

\noindent Classes in \name use a slightly different declaration style compared
to Java: there is no \lstinline{class} keyword, and a colon separe the class name (which must always start with
an uppercase letter) and the class implementation, which is used to specify the
definitions of the class. In our example, in the class declaration
for \lstinline{A}, the name of the class is \lstinline{A} and the code 
literal associatted with the class (\Q@{ method int ma()   Utils.m(this) }@) contains the definitions associated to the
class.

%We will see next some 
%important differences to Java-like languages in the way 
%classes and code-literals are type-checked, as we shall see next. 
%Nevertheless, for this example, things still work in a similar way to Java. 

The \name code above is fine, but there is no way to add a class 
\Q@B@ reusing the code of \Q@A@, since
\Q@A@ is designed for code \emph{use} and not \emph{reuse}. So, unlike
the Java code, introducing a subclass
\lstinline{B} is not possible. At first, this may seem like a severe restriction, but
\name has a different mechanism for code \emph{reuse} that 
is more appropriate when code reuse is intended. 

\subsection{Traits in \name: A mechanim of code Reuse}

Unlike classes, traits in \name cannot be instantiated and do not introduce new
types. However they provide code reuse.
%So, lets try again encoding the code for the leaking problem, but this
%time aiming at code reuse. 
Trait declarations 
look very much like class declarations, but trait names 
start with a lowercase letter. An obvious first attempt 
to model the example in Section~\ref{} (for a Java programmer) 
with traits and code reuse is:

\begin{lstlisting}
 ta:{ method int ma() Utils.m(this) }//type error
 A:Use ta
 Utils:{ class method int m(A a)/*method body here*/ }
\end{lstlisting}

\noindent Here \lstinline{ta} is a trait intended to replace the
original class \lstinline{A} so that the code of the method
\lstinline{ma} can be reused. Then the class \lstinline{A} 
is created by inheriting the code from the trait using the keyword 
{\bf Use}. Note that \use\ cannot contain class names: only trait
names are allowed.
That is, using a trait is the only way to induce code reuse.
Unfortunatelly, this code does not work, 
because \lstinline{Utils.m} requires an \lstinline{A} and the type of \lstinline{this} in
\Q@ta@ has no relationship to the type \lstinline{A}.

A Java programmer may then try to write
\begin{lstlisting}
 ta:{ method int ma() Utils.m(this) }//type error
 A:Use ta
 Utils:{ class method int m(ta a)//syntax error
   /*method body here*/ }
\end{lstlisting}
But this does not work either, indeed \Q@ta@ is not a type in the first place, since it is a (lowercase) trait name.
Indeed
since the trait name is not a type, no code external to that trait can
refer to it. This is one of the key design decisions in \name. 

% this text may go after, while explaining the This type?
%Code inside traits/classes can refer to (uppercase) class names, but not to (lowercase) trait names.
%%The point is that the trait can refer to the program via \lstinline{this},
%??????????????
% This means that the
%program is agnostic to what the trait type is going to be, so it can
%be later assigned to any (or many) classes. This improves the
%flexibility of reuse as illustrated in Section~\ref{}.
%However, 
%to solve the more immediate typing issue above we need one more round of refactoring, as we shall see next. 


%: in \name all declarations 
%are sequentially type-checked and if a definition occurs afterwards
%another, then it is not visible on all previous definitions.

%\paragraph{Type of the self-reference} 
%The code above does not work.
%because the way type-checking works in \name is that type-checking of 
%code literals is independent of the class/trait names associated to it.
%A Java programmer may expect that the type of \lstinline{this} 
%in the previous definition of \lstinline{ta} is \lstinline{ta}
%itself. However, this intuition brought from Java is wrong in \name
%for two reasons:

%\begin{itemize}

%\item {\bf Traits are not types}. Traits in
%  \name are simply units of reuse and cannot be used as types. Types
%  are only introduced by class/interface declarations.

%\item The second reason is that {\bf the type of self-references is the self type of the
%    code literal.} Unlike Java-like languages, where the body of a
%  class declaration is intrinsically coupled with the class itself, in
%  \name code literals are first-class\bruno{Is it appropriate to say
%    that they are first class?} and are type-checked independently.
%  \name has a notion of self-type, which is closely related to
%  approaches such as ThisType~\cite{} and ...\cite{others}. Therefore
%  in \name the self-reference does not have the type of the class
%  being defined, but rather it has the self-type.
%\end{itemize}

With this in mind, we can try to model the example in Section~\ref{} again:

\begin{lstlisting}
 IA:{interface method int ma()}//interface with abstract method
 Utils:{ class method int m(IA a)/*method body here*/}
 ta:{implements IA
   method int ma() Utils.m(this) }
 A:Use ta
\end{lstlisting}

This code works: \Q@Utils@ rely over interface \Q@IA@ and the trait \Q@ta@
implements \Q@IA@.
\Q@ta@ is well typed: independently of what class name(s) will be associated to its code, we knew such class(es) will implement \Q@IA@, thus while typechecking \Q@Utils.m(this)@ we can assume
\Q@this<:IA@.
 It is also possible to add a \Q@B@ as follows
\begin{lstlisting}
  B:Use ta, { method int mb(){return this.ma();} }
\end{lstlisting}
This also works.  \Q@B@ reuses the code of \Q@ta@, but has no knowledge of \Q@A@.
Since \Q@B@ reuses \Q@ta@, and \Q@ta@ implements \Q@IA@, also \Q@B@ implements \Q@IA@. 

Later, in section \ref{} we will provide a more formal detail over type system. For now notice that in the former example the code is correct even if no method called \Q@ma@ is explicitly declared.
The idea is that such method is imported from trait \Q@ta@, exactly as in the java equivalent
\begin{lstlisting}[language=Java]
  class B extends A{ int mb(){return this.ma();} }  
\end{lstlisting}
method \Q@ma@ is imported from \Q@A@.
This concept is natural for a Java programmer, but was not supported in previous work \cite{deep, ferruccio}

\paragraph*{Semantic of Use}
Albeit alternative semantic models for traits~\cite{} have been proposed,
here we use the flattening model.
This means that 
\begin{lstlisting}
A:Use ta
B:Use ta, { method int mb(){return this.ma();} }
\end{lstlisting}
 
\noindent reduces/is equivalent to/is flatted into
  
 \begin{lstlisting}
A:{implements IA method int ma() Utils.m(this) }
B:{implements IA
  method int ma() Utils.m(this)
  method int mb() this.ma() } 

 \end{lstlisting}
 
 This code seems correct, and there is no mention of the trait
 \Q@ta@. In some sense, all the information about code-reuse/inheritance
  is just a private implementation detail of \Q@A@
 and \Q@B@; while subtyping is part of the class interface.


%To finish this section, Figure \ref{fig:compare} provides a summary of
%the differences between classes and traits. The comparison focus on
%the roles of traits and classes with respect to instantiation,
%reusability and whether the declarations also introduce new types or
%not.





\section{Improving Use}

To illustrates how 
\name \emph{improves the use} of classes we model a simplified version of
Set and Bag collections first in Java, and then in \name.
The main benefit of the \name solution is that we can get reuse 
without introducing subtyping between Bags and Sets. This improves the 
use of Bags and eliminates logical errors arizing from incorrect
subtyping relations that are allowed in the Java solution. 

\subsection{Sets and Bags in Java}
An iconic example on why connecting inheritance/code reuse and
subtpying is problematic is provided by
LaLonde~\cite{LaLonde:1991:SSS:110673.110679}.  A reasonable
implementation for a \Q@Set@ is easy to extend into a \Q@Bag@ by
keeping track of how many times an element occurs.  We just need to
add some state and override a few methods.
For example in Java one could have:

\begin{lstlisting}
class Set {...//usual hashmap implementation
  private Elem[] hashMap;
  void put(Elem e) /*body*/
  boolean isIn(Elem e)/*body*/}
class Bag extends Set{...//for each element in the hash map,
  private int[] countMap;// keep track of how many times is there
  @Override void put(Elem e)/*body*/
  int howManyTimes(Elem e)/*body*/}
\end{lstlisting}

\noindent Coding \Q@Bag@ in this way avoids a lot of code
duplication, but \emph{we induced unintented subtyping}! 
Since subclassing implies subtyping, our code break Liskov substitution principle (LSP)~\cite{martin2000design}: not all bags are sets!
Indeed, the following is allowed:

\begin{lstlisting}
Set mySet=new Bag(); //OK for the type system but not for LSP
\end{lstlisting}


\paragraph{A (broken) attempt to fix the Problem in Java.}
One could \emph{retroactivelly} fix this problem by introducing \Q@AbstractSetOrBag@
and making both \Q@Bag@ and \Q@Set@ inherit from it:
\begin{lstlisting}
abstract class AbstractSetOrBag{/*old set code goes here*/}
class Set extends AbstractSetOrBag{} //empty body
class Bag extends AbstractSetOrBag{/*old bag code goes here*/}
...
//AbstractSetOrBag type not designed to be used.
AbstractSetOrBag unexpected=new Bag(); 
\end{lstlisting}

This looks unnatural, since \Q@Set@ would extend \Q@AbstractSetOrBag@ without adding anything,
and we would be surprised to find a use of the type \Q@AbstractSetOrBag@.
Worst, if we are to constantly apply this mentalty, we would introduce a very high number
of abstract classes that are not supposed to be used as types. Those classes would clutter the 
public interface of our classes and our code project as a whole.
In our example the information \Q@Set<:AbstractSetOrBag@ would be present in the public interface
of the class \Q@Set@, but it is not needed to use the class properly!

Moreover, the original problem is not really solved, but only moved 
further away. For example, one day  we may need bags that can only store up to 5 copies of the same element.
We are now at the starting point again:
\begin{itemize}
\item either we insert \Q@class Bag5 extends Bag@ and we break LSP; 
\item or we duplicate the code of the \Q@Bag@ implementation with minimal
  adjustments in \Q@class Bag5 extends AbstractSetOrBag@;
\item or we introduce a
\Q@abstract class BagN extends AbstractSetOrBag@ and \Q@class Bag5 extends BagN@
and we modify \Q@Bag@ so that  \Q@class Bag extends BagN@.
Note that this last solution is changing the public interface of the formerly released \Q@Bag@ class, and
this may even break retro-compatibility (if a client program was using
reflection, for example).
\end{itemize}

\subsection{Sets and Bags in \name}
Instead, in \name, if we was to originally declare
\begin{lstlisting}
Set:{/*set implementation*/} 
\end{lstlisting}
Then our code would be impossible to reuse in the first place for any user of our library.
We consider this an advantage, since unintended code reuse runs into underdocumented behaviour nearly all the time\footnote{See
``Design and document for inheritance or else prohibit
it''\cite{Bloch08}: the
self use of public methods is rarelly documented, thus is hard to understand the effects of overriding a library metod.
}!

If the designer of the set class wishes to make it reusable, it can do it explicitly by providing a set trait:
\begin{lstlisting}
set:{/*set implementation*/} 
Set:Use set
\end{lstlisting}
Notice how since \Q@set@ can never be used as a type, there is no reason to give it a fancy-future-aware name like
\Q@AbstractSetOrBag@.
When bag will be added, the code will look either like
\begin{lstlisting}
set:{/*set implementation*/} 
Set:Use set
Bag: Use set, {/*bag implementation*/}
\end{lstlisting}
or 
\begin{lstlisting}
set:{/*set implementation*/} 
Set:Use set
bag: Use set, {/*bag implementation*/}
Bag: Use bag
\end{lstlisting}
Notice how, thanks to flattening, the resulting code for \Q@Bag@ is identical in both versions, 
and as shown in Section 2, there is no trace of trait \Q@bag@. Thus if we are the developers of bags, we can temporarly go for the first version, and move when needed to the second without adding new undesired complexity for our old clients. 


%As a pleasaruble accident, avoid such code gift us simple support for
%This type and (in the extensions with nested classes seen later)
%family polimporphism.


\saveSpace\saveSpace\section{Improving Reuse}\saveSpace
\name allows reuse even when subtyping is impossible.
\name traits do not induce a new (externally visible) type.
However, locally in a trait, programmers can use the special self-type \Q@This@~\cite{bruce_1994,Saito:2009,ryu16ThisType} in order to denote the 
type of \Q@this@.
That is, a program is agnostic to what the \Q@This@ type is, so that it can
be later assigned to any (or many) classes. 
The idea is that during flattening, \Q@This@ will be replaced with the actual class name.
In this way, \name allows reuse even when subtyping is
impossible. For example for \emph{binary
  methods}~\cite{bruce96binary} where the method parameter has type \Q@This@. 
This type of situations is the primary motivator
for previous work aiming at separating inheritance from subtyping~\cite{cook}.
Leveraging on the \Q@This@ type, we can also provide self-instantiation (trait methods can create instances of the class using them) and smoothly integrate state and traits: a challenging problem that has limited the flexibility of traits and
reuse in the past.

\subsection{Managing State}

To illustrate how \name improves reuse, we will show a novel approach
to deal with \emph{state} in traits.  The idea of summing pieces of
code is very elegant, and has proven very successful in module
composition languages~\cite{ancona2002calculus} and several trait
models~\cite{Traits:ECOOP2003,Bergel2007,BETTINI2013521,fjig}.  However the research
community is struggling to make it work with object state (constructors
and fields) while achieving the following goals:

\begin{itemize}
%complicated discussions on this point \item keep sum associative and commutative,
\item managing fields in a way that borrows the elegance of summing methods;
\item actually initialize objects, leaving no null fields;
\item making it easy to add new fields;
\item allowing a class to create instances of itself.
\end{itemize}

In the related work we will show some alternative ways to handle
state.  However the purest solution requires methods only. The idea is
that the trait code just uses getter/setters/factories, while leaving
to classes the role to finally define the fields/constructors. That
is, classes have syntax richer than traits, allowing
declaration for fields and constructors.  This approach is very
powerful as illustrated by Wang et al.~\cite{wang2016classless}.

\paragraph{Modelling Points} Consider, for example, two simple 
traits that deal with \emph{point} objects. That is, points
in the cartesian plane (with coordinates \lstinline{x} and
\lstinline{y}). The first trait provides a \emph{binary method} that 
sums the point object with another point to return a new point. 
The second trait provides a similar operation that does multiplication 
instead.
\saveSpace 
\begin{lstlisting}
  pointSum: { method int x()  method int y()//getters
    class method This of(int x,int y)//factory method
    method This sum(This that)
      This.of(this.x()+that.x(),this.y()+that.y())//self instantiation
    }
  pointMul: { method int x() method int y()//repeating getters
    class method This of(int x,int y)//repeating factory
    method This mul(This that)
      This.of(this.x()*that.x(),this.y()*that.y())
    }
\end{lstlisting}
\saveSpace
\noindent As we can see, all the state operations (the getters for the 
\lstinline{x} and \lstinline{y} coordinates) are represented as {\bf abstract} methods.
Notice the abstract \Q@class method This of(..)@ which acts as a constructor
for points:
a class method is similar to a \Q@static@ method in Java but can be abstract. 
As for instance methods, they are late bound:  flattening can provide an implementation for them.
Abstract class methods are very similar to the original concept of member functions in the module composition setting~\cite{ancona2002calculus}.

\subsection{A First Attempt at Composition with a Traditional Model of
Traits}
According to the general ideas about trait composition presented in
Section~\ref{sec:separate}, a first attempt at composing the two traits providing
two different operations on points is:
\saveSpace\begin{lstlisting}
  Point:Use pointSum,pointMul
\end{lstlisting}  \saveSpace

\noindent However \emph{in a traditional trait model}~\cite{Traits:ECOOP2003} this would fail since methods \Q@x@,\Q@y@ and \Q@of@ are still abstract!
Instead a user could write something similar to:

\begin{lstlisting}
  CPoint:Use pointSum,pointMul, {//not our suggested solution
    int x   int y
    method int x() x       
    method int y() y
    class method This of(int x, int y) new Point(x,y)
    constructor Point(int x, int y){ this.x=x   this.y=y }
    }
\end{lstlisting}
%\bruno{We talk about withers later on. So I think we should consider
 % having withers in this code, so that readers can understand what 
%withers are!}
%\marco{with withers it will look more complicated}

\noindent This approach works, and it as some advantages, but also
some disadvantages: 

\begin{itemize}

\item {\bf Advantages:} This approach is associative and commutative, even self construction
  can be allowed if the trait requires a static/class method
  returning \Q@This@. The class will then implement the methods returning \Q@This@
  by forwarding a call to the constructor.
  
\item {\bf Disadvantages:} Writing such obvious definitions to close
  the state/fixpoint in the class 
   with the constructors and fields and getter/setters and factories is tedious.
   Moreover, there is no way for a trait to specify a default value for a field,
   the class need to handle all the state, even state that is conceptually
   "private" of such trait. 
   Previous work shows that such code can be automatically
   generated~\cite{wang2016classless}.
   Also a model with fields and constructors is necessarily
   more complex than a model with methods only.

\end{itemize}

\subsection{The \name Approach to State: Coherent Classes}

In our model we go one step further: there is no need to generate
code, or to explicitly write down constructors and fields. In fact in
\name there is not even syntax for those constructs!  The idea is that
any class that ``can" be completed in the obvious way  is \emph{a
  complete ``coherent" class}.  In most other languages, a class is
abstract if it has abstract methods.  Instead, we call a class
abstract only when the set of abstract methods is not coherent. That
is, the unimplemented methods cannot not be automatically recognised
as factory, getters and setters. Methods recognised as factory, getters and setters are called
\emph{abstract state operations}.
  
\paragraph{Coherent classes} A more detailed definition of coherent
classes is given next:
\begin{itemize}
\item a class with no abstract methods is coherent (just like Java
  \Q@Math@, for example). Such classes are useful for calling class/static methods.
\item a class with a single abstract \Q@class@ method returning \Q@This@
is coherent if all the other abstract methods can be seen as \emph{abstract state
operations} over one of its argument.
For example,
if there is a \Q@class method This of(int x, int y)@ as before,
then
\begin{itemize}
\item a method \Q@int x()@ is interpreted as an abstract state method: a \emph{getter} for \Q@x@.
\item a method \Q@Void x(int that)@ is a \emph{setter} for x.
\end{itemize}
\end{itemize}
\noindent
While getters and setters are fundamental operations, it is possible to
support more operations. For example:
\begin{itemize}
\item \Q@method This withX(int that)@
may be a ``wither", doing a functional field update: it creates a new instance that is like \Q@this@ but where field \Q@x@ has now \Q@that@ value.
\item \Q@method Void update(int x,int y)@
may do two field updates at a time.
\item\Q@method This clone()@ may do a shallow clone of the object.
\end{itemize}

We are not sure what is the best set of abstract state operations yet,
but we think this could become a very interesting area of research.
The work by Wang et al.~\cite{wang2016classless} explores a particular
set of such abstract state operations.

\paragraph{Points in \name:}
In \name and with our approach to handle the state, 
\lstinline{pointSum} and \lstinline{pointMul} can indeed be directly composed:
\saveSpace
\begin{lstlisting}
  //Same code as before! Works because resulting class is coherent.
  PointAlgebra:Use pointSum,pointMul 
\end{lstlisting}  
\saveSpace
\noindent
  Note how we can declare the methods independently and compose the result
  as we wish. 

  \paragraph{Improved solution} So far the current solution still
  repeats the abstract methods \Q@x@, \Q@y@ and \Q@of@.
  Moreover, in addition to \Q@sum@ and \Q@mul@ we may want many
  operations over points. It is possible to improve reuse
  and not repeat such abstract definitions by abstracting the common
  abstract definitions into a trait \Q@p@: 
\saveSpace
\begin{lstlisting}
  p: { method int x() method int y()
    class method This of(int x,int y)
    }
  pointSum:Use p, { method This sum(This that)
      This.of(this.x()+that.x(),this.y()+that.y())
    }
  pointMul:Use p, { method This mul(This that)
      This.of(this.x()*that.x(),this.y()*that.y())
    }
  pointDiv: ...
  PointAlgebra:Use pointSum,pointMul,pointDiv,...
\end{lstlisting}
\saveSpace      
Now the code is fully modularized, and each trait handles exactly one method.

\subsection{State Extensibility}
Programmers may want to extend points with more state. For example 
they may want to add colors to the points. A first attempt at doing
this would be:
\saveSpace
\begin{lstlisting}
  colored:{ method Color color() }
  Point:Use pointSum,colored //Fails: class not coherent
\end{lstlisting}
\saveSpace
This first attempt does not work: the abstract color method
is not a getter for any of the parameters of 
\Q@ class method This of(int x,int y)@. 
A solution is to provide a richer factory:
\saveSpace
\begin{lstlisting}
  CPoint:Use pointSum,colored,{
    class method This of(int x,int y) This.of(x,y,Color.of(/*red*/))
    class method This of(int x, int y,Color color)
    }
\end{lstlisting}
\saveSpace
\noindent 
where we assume support for overloading on different number of parameters.
This is a reasonable solution, however the method \Q@CPoint.sum@ resets
the color to red: we call the \Q@of(int,int)@ method, that now
delegates to \Q@of(int,int,Color)@ by passing red as the default field
value.  What should be the behaviour in this case?  If our abstract
state supports withers, we can use
\Q@this.withX(newX).withY(newY)@, instead of writing \Q@This.of(...)@, in order to preserve the color from
\Q@this@.  This solution is still not satisfactory: this design ignores
the color from \Q@that@.

\paragraph{A better design}
If the point designer is designing for reuse and extensibility, then 
a better design would be the following:  
\saveSpace\begin{lstlisting}
  p: { method int x() method int y() //getters
    method This withX(int that) method This withY(int that)//withers
    class method This of(int x,int y)
    method This merge(This that) //new method merge!
    }
  pointSum:Use p, { method This sum(This that)
      this.merge(that).withX(this.x()+that.x()).withY(this.y()+that.y())
    }
  colored:{method Color color()
    method This withColor(Color that)
    method This merge(This that) //how to merge color handled here
      this.withColor(this.color().mix(that.color())
    }
  CPoint:/*as before*/
\end{lstlisting}  \saveSpace
  \noindent This design allows merging colours, or any other kind of state we may want to add
  following this pattern.%\bruno{worried that withers are not explained enough.}

\paragraph{Independent Extensibility}
  Of course, quite frequently there can be multiple independent
  extensions~\cite{Zenger-Odersky2005} that need to be composed. Lets suppose that 
  we could have a notion of flavoured points as well.   
  In order to compose, let say \Q@colored@ with \Q@flavored@ we would
  need to compose the merge operation inside of both of them.
  The simple model we are presenting could accommodate this with an
  extension allowing code literals inside of a \use\ expression to use some form of super call to compose conflicting implementations. This is similar to the \emph{override} operation present in the original trait model~\cite{ducasse2006traits}.

The following code shows how to mix colours and flavours. The syntax \Q@_2merge@ and   \Q@_3merge@
call the version of merge as defined in the second/third element of \use.
\saveSpace\begin{lstlisting}
  p: {/*as before*/ }
  pointSum:Use p, {/*as before*/ }
  colored:{/*as before*/}
  flavored:{method Flavor flavor() //very similar to colored
    method This withFlavor(Flavor that)
    method This merge(This that) //merging flavors handled here
      this.withFlavor(that.flavor())}//inherits ``that'' flavor
  FCPoint:Use pointSum,colored,flavored{
    class method This of(int x,int y)
      This.of(x,y,Color.of(/*red*/),Flavor.none())
    class method This of(int x, int y,Color color,Flavor flavor)
    //resolves the conflict about two implementations for merge
    //by proving our own implementation here
    method This merge(This that) this._2merge(that)._3merge(that)
    }
\end{lstlisting}  \saveSpace\saveSpace

\saveSpace\saveSpace\section{Intuitions on formalization}\label{sec:formal}
\saveSpace\saveSpace

This section sketches some of the key ideas in \name. 

%In this article we dedicate more space to examples and informal presentation and motivations;
%so we do not have space to provide a full formalizations.
%We will provide here some hints on how the formalization works.

\subsection{Syntax}

In the following, we present a very simplified grammar:
%\begin{bnf}
%\prodFull{aa}{bb}{Declaration}\\
%\end{bnf}

%\begin{comment}
\begin{bnf}
\prodFull\mTD{\mt\terminalCode{:}\mL \mid \mt\terminalCode{:} \use\ \overline\mV}{Trait Decl}\\
\prodFull\mCD{\mC\terminalCode{:}\mL \mid \mC\terminalCode{:} \use\ \overline\mV}{Class Decl}\\
\prodFull\mV{\mt \mid \mL}{Code Value}\\
\prodFull\mL{
\oC
\Opt{\terminalCode{interface}}\ \terminalCode{implements} \overline\mT\ \overline\mMD
\cC
}{Code Literal}\\
\prodFull\mT{\mC}{types are class names}\\
\prodFull\mMD{\Opt{\terminalCode{class}}\ \terminalCode{method}\ \mT\ \mm\oR\overline{\mT\,\mx}\cR \Opt\me}{Method Decl}\\

\prodFull\me{\mx\mid\mT\mid\me\terminalCode{.}\mm\oR\overline\me\cR}{expressions}\\
\prodFull\mD{\mCD\mid\mTD}{Declaration}\\
\end{bnf}
%\end{comment}

To declare a trait \mTD\ or a class \mCD, we can use either a code literal \mL\ or a trait
expression.  Traits come with various operators (restrict, hide,
alias) but in this work we focus on the single operator 
$\use$, taking a set
of code values: that is trait names \mt\ or literals \mL\ and composing them.  This operation, sometimes called \emph{sum}, is the simplest and most elegant
trait composition operator.  $\use\ \overline\mV$ composes the content of $\overline\mV$
by taking the union of the methods and the union of the implementations.

\use\ can not be applied if multiple versions of the same method are
present in different traits.  An exception is done for abstract methods:
methods where the implementation \me\ is missing. In this case (if the
headers are compatible) the implemented version is selected.  In a sum
of two abstract methods with compatible headers, the one with the more
specific type is selected.

Code literals \mL\ can be marked as interfaces. 
That is, the interface keyword is inside the curly brakets, so an upper case name associated with an interface literal is a class-interface, while a lowercase one is a trait-interface.
In our simple model, we consider an error trying to merge an interface with a non-interface.
 Then we have a set of implemented interfaces and a set of member
  declarations. In this simple language, the only members are methods.
If there are no implemented interfaces, in the concrete syntax we will omit the \Q@implements@ keyword.

Methods \mMD~can be instance methods or \Q@class@ methods. A class method is similar to a \Q@static@ method in Java but can be abstract. This is very usefull in the context of code composition.
To denote a method as abstract, instead of an optional keyword we just omit the implementation \me.

A version of this language where there are no traits can be seen 
as a restriction/variation of FJ~\cite{igarashi2001featherweight}.

\subsection{Well-formedness}
Basic well formedness rules apply:
\begin{itemize}
\item all method parameters have unique names and the special parameter name \Q@this@ is not declared
 in the parameter list,
\item all methods in a code literal have unique names,
\item all used variables are in scope,
\item all methods in an interface are abstract, and there are no interface class methods.
\end{itemize}
Those rules can be applied on any given \mL~individually and in full isolation.

We expect the type system to enforce: 
\begin{itemize}
\item all the traits and classes have unique names in a program $\overline\mD$, and the special class name
\Q@This@ is reserved,
\item all used types correspond to class declarations in the program, or are \Q@This@, 
\item subtyping between interfaces and classes,
\item method call typechecking,
\item no circular implementation of interfaces,
\item type signature of methods from interfaces can be refined following the well known variant-contravariant rules,
\item only interfaces can be implemented.
\item \marco{I'm sure I'm missing something}
\end{itemize}
While classes are typed assuming \lstinline{this} is of the nominal type of the
class, trait declarations, do not introduce any nominal type.  \lstinline{this}
in a trait is typed with a special type \lstinline{This} that is visible only
inside such trait. Syntactically, \Q@This@ is just a special, reserved, class name $\mC$.
A Literal can use the \lstinline{This} type,
and when flattening completes creating a class definition, \Q@This@ will be replaced with such class name.

For the sake of simplicity, method bodies are just simple expressions
\me: they can be just variables, types and method calls. We need types as part of expressions in order to use them as receivers for class methods.

\subsection{Remarks on Typing}
 Our typing discipline is 
what distinguishes our approach from a simple minded code composition macro~\cite{bawden1999quasiquotation}
or a rigid module composition~\cite{ancona2002calculus}.

The are two core ideas:
\paragraph{1: traits are \emph{well-typed} before being reused.}${}_{}$\\*
 For example in

\saveSpace\begin{lstlisting}
t:{method int m() 2 
   method int n() this.m()+1}
\end{lstlisting}\saveSpace

\noindent \Q@t@ is well typed since \Q@m()@ is declared inside of \Q@t@, while

\saveSpace\begin{lstlisting}
t1:{method int n() this.m()+1} 
\end{lstlisting}\saveSpace
\noindent would be ill typed.

\paragraph{2: code literals are not required to be \emph{well-typed} before flattening.}${}_{}$\\*
In class expressions  $\use\ \overline\mV$
an \mL\ in $\overline\mV$ is not typechecked before flattening, and only the result is expected to be well-typed.
While this seems a very dangerous approach at first, consider that also Java have the same behaviour:
for example in
\saveSpace\begin{lstlisting}[language=Java]
  class A{ int m() {return 2;}  int n(){return this.m()+1;} }
  class B extends A{ int mb(){return this.ma();} }
\end{lstlisting}\saveSpace
\noindent in \Q@B@ we can call \lstinline{this.ma()} even if in the curly braces there is no declaration for \Q@ma()@.
In our example, using the trait \Q@t@ of before

\saveSpace\begin{lstlisting}
C: Use t {method int k() this.n()+this.m()}
\end{lstlisting}\saveSpace
\noindent would be correct: even if n,m are not defined inside
\Q@{method int k() this.n()+this.m()}@,
the result of the flattening is well typed.

This is not the case in many similar works in literature~\cite{deep,Bettini2015282,Bergel2007} where the
literals have to be self complete. In this case we would have been forced to
declare abstract methods \Q@n@ and \Q@m@.

Our typing strategy has two important properties:
\begin{itemize}
\item If a class is declared by using $\mC : \use\ \overline\mt$, that is, without literals,
and the flattening is successfull, \mC\ is well typed, no need of further checking.
\item On the other side, if a class is declared by $\mC : \use\ \overline\mV$, with
$\mL_1\ldots\mL_n \in \overline\mV$, and after successfull flattening $\mC : \mL$ can not be typechecked,
then the issue was originally present in one of $\mL_1\ldots\mL_n$.
It may be that the result is intrinsically ill-typed, if one of the methods in $\mL_1\ldots\mL_n$ is not well typed,
but it may also happen that a type referred from one of those methods is declared \emph{after} the current class. As we will see later, this is how our relaxation allows to support (indirectly) recursive types.

This also means that as an optimization strategy
 we may remember what method bodies come from traits and what method bodies come from code literals, in order to typecheck only the latter.
 \end{itemize}

 \subsection{Recursive types}

OO languages leverage on recursive types most of the times.
For example in a pure OO language, \Q@String@ may offers a \Q@Int size()@
method, and \Q@Int@ may offer a \Q@String toString()@ method.

This means that is not possible to type in (full) isolation classes
\Q@String@ and \Q@Int@.

The most expressive compilation process may divide the program in groups of mutually 
dependent classes.
Each group may also depend from a number of other groups.
This would form a Direct Aciclyc Graph of groups.
To type a group, we first need to type all depended groups, then
we can extract the structure/signature/structural type of all
the classes of the group.
Now, with the information of the depended groups and the one extracted
from the current group, it is possible to typecheck the implementation
 of each class in the group.
In this model, it is reasonable to assume that flattening happens group by group, before
extracting the class signatures.

Here we go for a much simpler simple top down execution/interpretation for flattening, where flattening
happen one at the time, and classes are typechecked where their type is first needed.
That is, In our approach typing and flattening interleaves. We assume our compilation process to stop as soon as 
an error arise. There are two main kinds of errors: Type errors (like method not found) or Composition errors (like summing two conflicting implementation for the same method).
For example
\saveSpace\begin{lstlisting}
A:{method int ma(B b) b.mb()+1}
tb:{method int mb() 2}
tc:{method int mc(A a,B b) a.ma(b)}
B: Use tb
C: Use tc, {method int hello() 1}
\end{lstlisting}\saveSpace
In this scenario, since we go top down, we first need to generate \Q@B@.
To generate \Q@B@, we need to use \Q@tb@;
In order to modularly ensure well typedness,
we require \Q@tb@ to be well typed at this stage. If \Q@tb@ was not well typed
a type error could be generated at this stage.
In this moment, \Q@A@ can not be compiled/checked alone,
we need informations about \Q@B@, but \Q@A@ is not used in \Q@tb@,
thus we do not need to type \Q@A@ and we can type \Q@tb@ with
 the available informations and proceed to generate \Q@B@.
Now, we need to generate \Q@C@, and we need to ensure well typedness of \Q@tc@.
Now \Q@B@ is alreay well typed (since generated by \use\ \Q@tb@, with no \mL),
and \Q@A@ can be typed;  finally \Q@tc@ can be typed and used.
If \use\ could not be performed (for example it \Q@tc@ had a method \Q@hello@ too)
a composition error could be generated at this stage.
On the opposite side, if \Q@B@ and \Q@C@ was swapped, as in
\saveSpace\begin{lstlisting}
C: Use tc, {method int hello() 1}  
B: Use tb
\end{lstlisting}\saveSpace
\noindent
now the first task would be to generate \Q@C@, but 
to type \Q@tc@ we need to know the type of \Q@A@ and \Q@B@.
But they are both unavailable: \Q@B@ is still not computed and 
\Q@A@ can not be compiled/checked without information about \Q@B@.
A type error would be generated, on the line of ``flattening of \Q@C@
requires \Q@tc@, \Q@tc@ requires \Q@A@,\Q@B@, but \Q@B@ is still in need of flattening".

In this example, a more expressive compilation/precompilation process 
could compute a dependency graph and, if possible, reorganize the list,
but for simplicity lets consider to always provide the declarations
in the right order, if one exists.

\paragraph{Criticism: existence of an order is restrictive.}${}_{}$\\*
Some may find the requirement of the existence of an order restrictive;
An example of a ``morally correct" program where no right order exists is the following:
\saveSpace\begin{lstlisting}
t:{ int mt(A a) a.ma()}
A:Use t {int ma() 1}
\end{lstlisting}\saveSpace

In a system without inference for method types,
if the result of composition operators depends only on the
structural shape of their input (as for \use)
is indeed possible to optimistically compute the resulting structural shape of the classes
and use it to type involved examples like the former.
We stick to our simple approach, since we believe such typing discipline would be fragile,
and could make human understanding the code-reuse process much harder/involved.
Indeed we just wrote an involved program where the correctness of trait \Q@t@ depends of 
\Q@A@, that is in turn generated using trait \Q@t@.

\paragraph{Criticism: it would be better to typecheck before flattening.}${}_{}$\\*
In the world of strongly typed languages we are tempted to
first check that all can go well, and then perform the flattening.
This would however be overcompicated for no observable difference:
Indeed, in the \Q@A,B,C@ example above there is no difference
between
\begin{itemize}
\item  (1)First check \Q@B@ and produce \Q@B@ code (that also contains \Q@B@ structural shape),
  (2) then use \Q@B@ shape to check \Q@C@ and produce \Q@C@ code;\ 
or a more involved
\item  (1)First check \Q@B@ and discover just \Q@B@ structural shape as result of the checking,
  (2)then use \Q@B@ shape to check \Q@C@.
  (3) Finally produce both \Q@B@ and \Q@C@ code.
\end{itemize}

Note that we can reuse code only by naming traits; but our only point of relaxation is {\bf only} the code literal: there is no way an error can ``move around'' and be duplicated during the compilation process.
In particular, our approach allows for safe libraries of traits and classes to be fully typechecked, deployed and reused by multiple clients: no type error will emerge from library code.
On the other side, we do not enforce the programmer to write always self-contained code where all the abstract method definition are explicitly declared.

\section{Extensions to our model}
  One of the main feature of our simple reuse/use model is that it can be
  easly extended. One simple but amazingly expressive extension is nested classes

\subsection{Nested classes}

A nested class will be another kind of member in the Literal, so  
the grammar can be updated as following:

\begin{bnf}
\prodFull\mMD{
\Opt{\terminalCode{class}}\ \terminalCode{method}\ \mT\ \mm\oR\overline{\mT\,\mx}\cR \Opt\me
\mid \mCD
}{Member Decl}\\
\prodFull\mT{\mC\mid\mC\terminalCode{.}\mT}{types are now paths}
\end{bnf}\\

The general idea is that by composing code with \use, nested classes with the same name are recursivelly composed.
Note that while we have nested classes, we do not have nested traits: all traits are still
at top level.
Untypable/unresolved Traits are also the only``dependency"
the type system keeps track of, this means that when a nested class at an arbirary
nested level is flattend, as in
\Q@C:{ D:{ E:Use t1,t2,L}}@
t1 and t2 must be defined before C at top level; and they may require classes (and their
nested) defined before C. This means that the type system can still consider
the class table as a simple map from Types T to their definition.

This extension lets us challenge the expression problem[]:
in the expression problem we have data variants and operations and....

Let see how to easily encode and solve the expression problem:

\begin{lstlisting}
exp:{Exp:{interface}}//Exp declared once, reused everywhere
lit:Use exp,{ Lit:{implements Exp //Exp not explicitly declared
    class method Lit of(int inner) //Lit abstract state
    method int inner()}
  }
sum:Use  exp,{ Sum:{implements Exp 
    class method This of(Exp left, Exp right)
    method Exp left() //Sum abstract state
    method Exp right()}
  }  
uminus:Use exp,{ UMinus:{implements Exp 
    class method This of(exp inner)//and so on for
    method Exp inner()}//all the needed datavariants
  }   
  
expToS:{Exp:{interface method String toString()}}
//concept of toString declared once

sumToS:Use sum,expToS,{ Sum:{implements Exp//with Exp.toString
    method String toString()//just the implementation of the
      left.toString()+"+"+right.toString()//specific method
  }
uminusToS:...//implement toString for all the datavariants

expEval:{Exp:{interface method int eval()}}
//declare the next operation and implement it for all the datavariant
\end{lstlisting}

Now that you have nicely modularized the code, just compose all the traits you need.
\begin{lstlisting}
MySolution:Use sumToS,litToS
//sum,lit and exp traits are alread included
\end{lstlisting}

The expression problem presented up to now is the traditional challenge proposed by [];
this has been criticized to not really address the fundamental issues since it does not handle ....
Now we show how we can go behond the traditional expression problem by encoding transformer methods:
For example, lets add 1 to all literals
\begin{lstlisting}
expAdd1:{Exp:{interface method Exp add1()}}
sumAdd1:Use sum,expAdd1,{Sum:{implements Exp
    method Exp add1()
      Sum.of(left.add1,right.add1())
  }
litAdd1:Use lit,expAdd1,{Lit:{implements Exp
    method Exp add1()
      Lit.of(inner()+1);
    }

MySolutionAdd1:Use sumToS,litToS,sumAdd1,litAdd1
\end{lstlisting}

This nicely solve our problem. 
However, notice how if we wished to add many similar operations we would 
have to repeat the propagation code (as in \Q@sumAdd1@) many times
just changing the name of the operation.
In the next section we will show how to improve on this point.


\subsection{More composition operators}
\use\ is amazing, elegant and simple, but our system can be easily enriched with more 
operators: while most approaches in literature presents a fixed set of operators, we do not need such restriction.
We just need to be sure that every newly added operator respects the following criteria:

\begin{itemize}
\item As for \use, the operato does not need to be total, but if it fails it needs to provide an error that will be reported to the programmer.
\item When the operator takes in input only traits (they are going to be well typed), if a result is produced,
 such result is well typed.
\item When the operator takes in input also code literal, if a non well typed result is produced,
the type error must be tracked back to code in one of those non typed yet code literals.
 \end{itemize}


Here we see some example of other composition operators:
\paragraph*{Restrict}${}_{}$\\*
Restrict makes a method abstract
\begin{lstlisting}
t:{method bool geq(This x) x.leq(this)   method bool leq(This x) x.qeq(this) }
C:Use t[restrict geq],{method bool geq(This x)  /*actual geq impl*/}
\end{lstlisting}

A variant of this operator allows to move the implementation to another name. This is very usefull to implement \Q@super@. This also is enough to suppor the example of before, when we try to compose \Q@colored@
with \Q@flavoured@.

\begin{lstlisting}
  colored:{/*as before*/}
  flavoured:{
  method Flavour flavour()
    method This withFlavour(Flavour that)
    method This merge(This other)
      this.withFlavour(this.flavour().mix(that.flavour())
    }
  CPoint:Use colored[restrict merge as _1merge],flavoured[restrict merge as _2merge],
  pointSum,{
    class method This of(int x,int y) This.of(x,y,Color.of(/*red*/))
    class method This of(int x, int y,Color color)
    method This merge(This other)
      this._1merge(other)._2merge(other)
    }
\end{lstlisting}

Note how we are leveraging on the fact that the code literal does not need to be complete, thus we can just call \Q@_1merge()@ and \Q@_2merge()@


\paragraph*{Rename}${}_{}$\\*
Rename allows to make some form of ``compile time'' refactoring
There are a lots of different forms of rename in literature,
sometime allowing only to rename specific methods, sometime allowing to rename
nested classes into other nested classes either at the same or at a different nesting level.
Renaming in the context of nested classes also means that when renaming a method of an interface, all the 
nested classes implementing such interface inside of that code literal need to be adjusted.
Renaming need to rename not only the method headers, but all the method calls inside of method bodies.
At first glimps, this seams to be not always possible since we are considering to be able to apply those
operators also to non well typed code.
However, if the expression language is simple enough, it is possible to pre process the code to
annotate the expected receiver type on all method calls by doing a purelly sintactic analysis
on a single code literal in isolation. 
All the expression whose type is guessed to be out of the border of the literal can stay unannotated; they are not going to be renamed anyway.

\begin{lstlisting}
t:{ I:{interface method int mI() }
     A:{implements I  method int mI() 42}
     B:{ method int mB(I i, A a, C c) i.mI()+a.mI()+c.mI()}
     //mB would be annotated i[I].mI()+a[A].mI()+c.mI()
     }
D:t[rename A.mI kI]
\end{lstlisting}
 Notice how we are sure that c does not implements I since it is invisible from the outside: traits does not introduce nominal types!
 
 We expect the flattened version for \Q@D@ to be
\begin{lstlisting}
D:{ I:{interface method int kI() }
     A:{implements I  method int kI() 42}
     B:{ method int mB(I i, A a, C c) i.kI()+a.kI()+c.mI()}
     }
\end{lstlisting}

Hide can be seen as a variation of rename, where the method/class is renamed to a fresh unguessable name.

\paragraph*{Redirect}${}_{}$\\*
Redirect allows to emulate generics; the main idea is that a (fully abstract) class can be redirect to another one external to the trait/code literal.
For example a linked list can be implement as
\begin{lstlisting}
list:{ Elem:{}
     Cell:{class method Cell of(Elem e,Cell c) 
       method Elem e()  method Cell c()
       }
   method Elem get(int x) ...
   ...more methods..
   }
ListString:list[redirect Elem to String]
\end{lstlisting}

%An expressive form of Redirect can be multiple, that is, can redirect may interdependent classes at the same time.
%We show a graph example, where also we can show how to propagate generics:
%For example
%\begin{lstlisting}
%t:{ method boolean reachable(Node start, Node end)/*implements reachability*/
%     Node:{method ListEdge nodes()}
%     Edge:{method Node in()  method Node out()}
%     ListEdge:list[redirect Elem to Edge]
%     }
%\end{lstlisting}



%Redirect can be multiple.

\paragraph*{Application on the expression problem}${}_{}$\\*
With redirect, rename and restrict we can have the general operator propagator
\begin{lstlisting}
operation:{//for sum and lit, easy to extends as before
  T:{}
  Exp:{interface method Exp op(T x)}}
  Sum:Use sum,{ extends Exp sum,expAdd1,{
    method Exp op(T x)
      Sum.of(left.op(x),right.op(x))  }
  Lit:Use lit,{
    method Exp op(T x)  this
  }
\end{lstlisting}

Now, to have my addN I can

\begin{lstlisting}
opAddn: Use
  operation[redirect T to Int]
    [rename Exp.op(x) to addN(x)][restrict Lit.op(x)], {
  Lit:{method Exp addN(Int x) Lit.of(inner())+x}
  }
\end{lstlisting}  



\paragraph*{Full power of redirect}${}_{}$\\*
An expressive form of Redirect can be multiple, that is, can redirect may interdependent classes at the same time.
We show an example where a specific kind of \Q@Service@ can produce a \Q@Report@, and 
\Q@Report@s can be combined together.
The goal is to execute a list of such services and produce a collated report.
This example also show how to propagate generics:

\begin{lstlisting}
Service:{interface method Void performService()}
serviceCombinator:{
  S:{implements Service method R report()  }
  
  R:{method R combine(R that)   class method R empty() }
  
  ListS:list[redirect Elem to S]
  
  class method R doAll(ListS ss){//here we use extended java like syntax
    R r=R.empty()
    for(S s in ss){
      s.performService();
      r=r.combine(s.report())
      }
    return r;
  }
}
PaintingService:serviceCombinator[redirect S to PaintingService]
PaintingService:{... method PaintingReport report()..}
PaintingReport:{..}
\end{lstlisting}

The flattened version of \Q@PaintingService@ would look like:
\begin{lstlisting}
PaintingService:{
  ListS:/*the expansion of list[redirect Elem to PaintingService]*/
  
  class method PaintingReport doAll(ListS ss){
    PaintingReport r=PaintingReport.empty()
    for(PaintingService s in ss){
      s.performService();
      r=r.combine(s.report())
      }
    return r;
  }
}
\end{lstlisting}
Where you can note how redirect figured out \Q@R=PaintingReport@ by comparing the structural shape of
classes \Q@PaintingService@ and \Q@S@.

To encode the former generic code in java you need to write
the following headeche inducing interfaces for RService and Report.
and require that the services you want to serve implement those.
\begin{lstlisting}
interface Service{ void performService();}
interface Report<R extends Report<R>>{R combine(R that);}
interface RService<R extends Report<R>> extends Service{ R report();}
\end{lstlisting}
Note how we still can not encode the method \Q@empty@.

\section{Related Work}\label{sec:related}
In this section we discuss related work and how it compares to Classless Java.

%\yanlin{need to discuss: whether the comments for FTJ is fair.}

%No, was not fair, I know that work, it does modular type checking, and
%more work after that they extend with more safe/modular typing things.
%also, Java8 IS a language extension on Java
%However, language extensions (including FTJ)
%have a natural drawback: the programmer has to learn new syntax. In contrast,
%our approach is completely compatible with the current Java language, so that
%programmers don't need to pay any learning cost to adapt to this new classless
%programming pattern. Another drawback which is particular for FTJ is that FTJ
%doesn't have type for traits, hence the correctness check of trait is done when
%type-checking classes. This choice makes the design of FTJ simpler but lost
%typechecking efficiency (one trait will be potentially checked multiple times if
%it is used in multiple classes).

\paragraph{Multiple Inheritance in Object Oriented Languages}
Many authors have argued in favor or against multiple inheritance.  Multiple
inheritance is very expressive, but difficult to model and implement, and can
cause difficulty (including the famous diamond (fork-join)
problem~\cite{bracha90mixin,Sak89dis}, conflicting methods, etc.) in reasoning about
programs. To conciliate the need for expressive power and simplicity, many
models have been proposed, including C++ virtual inheritance,
mixins~\cite{bracha90mixin}, traits~\cite{scharli03traits}, and hybrid models
such as CZ~\cite{malayeri2009cz}.  They provide novel programming architecture
models in the OO paradigm. In terms of restrictions set on these models, C++
virtual inheritance aims at a relative general model; the mixin model adds some
restrictions; and the trait model is the most restricted one (excluding
state, instantiation, etc).

%\paragraph{C++ Approach.}
C++ tries to provide a general solution to multiple inheritance by
virtual inheritance, dealing with the diamond problem by keeping only
one copy of the base class~\cite{ellis1990annotated}. However C++
suffers from the object initialization problem~\cite{malayeri2009cz}.
It bypasses all constructor calls to virtual superclasses, which can
potentially cause serious semantic errors. In our approach, the \mixin
annotation has full control over object initialization,
and the mechanism is transparent to users. Moreover, customized factory
methods are also allowed: if users are not satisfied with the default
generated \texttt{of} method, they can implement their own.

%\paragraph{Mixins.}
Mixins are a more restricted model than the C++ approach. Mixins allow to name
components that can be applied to various classes as reusable functionality
units. However, % they suffer from linearization: the order of mixin application
% is relevant in often subtle and undesired ways. This hinders their usability
% and the ability of resolving conflicts:
the linearization (total ordering) of
mixin inheritance cannot provide a satisfactory resolution in some cases and
restricts the flexibility of mixin composition. To fight this limitation, an
algebra of mixin operators is introduced~\cite{ancona2002calculus}, but this
raises the complexity, especially when constructors and fields
are considered~\cite{marco09FJigsaw}. Scala traits are in fact more like linearized mixins.
Scala avoids the object initialization
problem by disallowing constructor parameters, causing no ambiguity in cases
such as diamond problem. However this approach has limited expressiveness, and
suffers from all the problems of linearized mixin composition. % Other languages, such as
% Python, also use linearized mixins.
\begin{comment}
Python also offers multiple inheritance via linearized mixins. Indeed, in python any class is implicitly a mixin, and mixin composition informally expressed as\\*
\Q@ class A use B,C {...new methods...}@\\*
can be expressed in python as \\*
\Q@ class Aux: ...new methods...@\\*
\Q@ class A(B,C,Aux): pass@
\end{comment}
\noindent Java interfaces and default methods do not use
linearization: the semantics of Java \textbf{extends} clause in
interfaces is unordered and symmetric.

% However, in pure Java, there is no mechanism for creating objects in
% interfaces. Also, our approach supports proper constructor mechanism.

%\paragraph{CZ.}
Malayeri and Aldrich proposed a model CZ~\cite{malayeri2009cz} which
aims to do multiple inheritance without the diamond problem. They
divide inheritance into two separate concepts: inheritance dependency
and implementation inheritance. Using a combination of
\texttt{requires} and \texttt{extends}, a program with diamond
inheritance can be transformed to one without diamonds. Moreover,
fields and multiple inheritance can coexist. However untangling
inheritance also untangles the class structure. In CZ, not only the
number of classes, but also the class hierarchy complexity
increases. \mixin does not complicate the class structure, and state
can also coexist with multiple inheritance.

%\paragraph{Traits and Java's default methods}
Simplifying the mixins approach, traits~\cite{scharli03traits} draw a strong
line between units of reuse and object factories. Traits, as units of reusable
code, contain only methods as reusable functionality, ignoring state and state
initialization. Classes, as object factories, require functionality from
(multiple) traits. Java 8 interfaces are closely related to
traits: concrete method implementations are allowed (via the \textbf{default}
keyword) inside interfaces. The introduction of default methods opens the gate
for various flavors of multiple inheritance in Java. Traits offer an algebra
of composition operations like sum, alias and exclusion, providing explicit conflict
resolution. Former work~\cite{bono14} provides details on mimicking the trait
algebra through Java 8 interfaces.  We briefly recall the main points of their
encoding; however we propose a different representation of \textbf{exclusion}.
The first author of~\cite{bono14} agreed (via personal communication) that
our revised version for exclusion is cleaner, typesafe and more direct.
\newcommand\shortItem{\vspace{-1ex}}
\begin{itemize}
\item \textbf{Symmetric sum} can be obtained by simple multiple inheritance between interfaces.
    \begin{lstlisting}
    interface A { int x(); }    interface B { int y(); }    interface C extends A, B {}
    \end{lstlisting}
\shortItem
\item \textbf{Overriding} a conflict is obtained by specifying which super interface take precedence.
    \begin{lstlisting}
    interface A { default int m() {return 1;} }
    interface B { default int m() {return 2;} }
    interface C extends A, B { default int m() {return B.super.m();} }
    \end{lstlisting}
\shortItem
\item \textbf{Alias} is creating  a new method delegating to the existing super interface.
    \begin{lstlisting}
    interface A { default int m() {return 1;} }
    interface B extends A { default int k() {return A.super.m();} }
    \end{lstlisting}
\shortItem

\item \textbf{Exclusion}: exclusion is also supported in Java, where method declarations can hide the default methods correspondingly in the super interfaces.
    \begin{lstlisting}
    interface A { default int m() {return 1;} }
    interface B extends A { int m(); }
    \end{lstlisting}
\shortItem
\end{itemize}

There are also proposals for extending Java with traits. For
example, FeatherTrait Java (FTJ)~\cite{Liquori08ftj} extends
FJ~\cite{Igarashi01FJ} with statically-typed traits, adding trait-based
inheritance in Java.  Except for few, mostly syntactic details, their work can
be emulated with Java 8 interfaces. There are also extensions to the original
trait model, with operations (e.g. renaming~\cite{reppy2006foundation}, which breaks
structural subtyping) that default methods and interfaces cannot
model.

\emph{Traits vs Object Interfaces.}
We consider object interfaces to be an alternative to traits or mixins.  In the
trait model two concepts (traits and classes) coexist and cooperate. Some
authors~\cite{BettiniDSS13} see this as good language design fostering good
software development by helping programmers to think about the structure of
their programs.  However, other authors see the need of two concepts and the
absence of state as drawbacks of this model~\cite{malayeri2009cz}.  Object
interfaces are units of reuse, and at the same time they provide factory methods
for instantiation and support state.  Our approach promotes the use of
interfaces instead of classes, in order to rely on the modular composition
offered by interfaces. Since Java was designed for classes, a direct classless
programming style is verbose and unnatural. However, annotation-driven code
generation is enough to overcome this difficulty, and the resulting programming
style encourages modularity, composability and reusability, by keeping a strong
OO feel. In that sense, we promote object interfaces as being both units of
reusable code and object factories. Our practical experience is that, in Java,
separating the two notions leads to a lot of boilerplate code, and is quite
limiting when multiple inheritance with state is required.  Abstract state
operations avoid the key difficulties associated with multiple inheritance and
state, while still being quite expressive.  Moreover the ability to support
constructors adds expressivity, which is not available in approaches
such as Scala's traits/mixins.

%Nevertheless, we
%believe that in other languages the separation can be more practical
%than in Java, and we do not necessarily advocate that merging the two
%notions is a better approach in general.

%There are other limitations of our current approach, but they may be addressed
%in future work (see Section~\ref{sec:futurework}).


\begin{comment}
Besides, we support more features than the original trait model:
\begin{itemize}
\item We provide \texttt{of} methods for the annotated interfaces. During annotation processing time, the ``fields'' inside an interface are observed and a static method \texttt{of} is automatically injected to the interface with its arguments correspondingly. Such a method is a replacement to the constructors in original traits, making instantiation more convenient to use.
That is, in our approach there are only interfaces, our model requires a single concept, while the trait model requires traits \emph{and} classes.

\item We provide \texttt{with-} methods as auxiliary constructors. A \texttt{with-} method is generated for each field, just like record update, returning the new object with that field updated
%. A \texttt{clone} method is generated for the interface, returning a copy of the current object.
Furthermore, we do automatic return type refinement for this kind of methods. This feature is comparatively useful in big examples, making operations and behaviors more flexible.%, which we will demonstrate later.
\item We provide two options for generating setters. There are two kinds of setters which are commonly used, namely \textit{void setters} and \textit{fluent setters}. The only difference is that a fluent setter returns the object itself after setting, thus supporting a pipeline of such operations. The generation depends on which type of setter is declared in the interface by users.
\end{itemize}
\end{comment}



% \begin{enumerate}
% \item Mixins allows to name components that can be applied to various classes as
%   reusable functionality units. However, they suffer from linearization: the
%   order of mixin application is relevant in often subtitle and undesired
%   ways. constraints. This hinders their usability and their ability of resolving
%   conflicts: the linearization (total ordering) of mixin inheritance cannot
%   provide a satisfactory resolution in all cases and restricts the flexibility
%   of mixin composition.  To fight those limitations, an algebra of mixin
%   operators is introduced~\cite{ancona2002calculus}, but this raised the
%   complexity of the approach, especially when constructors and fields are
%   considered~\cite{marco09FJigsaw}.  Our approach does not have the
%   linearization problem, in that the semantics of Java \textbf{extends} clause
%   is unordered and symmetric.
% \item Simplifying the mixin algebras approach, traits draw a strong line between
%   units of reuse (traits) and object factories (classes) In this model,
%   traits~\cite{scharli03traits} units of reusable code, containing only methods
%   as reusable functionalities. Thus, no state/state initialization is
%   considered.

%   Classes act as object factories, requiring functionalities from multiple
%   traits.  Traits offers a trait algebra with operations like sum, alias and
%   exclusion, provided for explicit conflict resolution.

%   Concluding, (pure) traits do not allow state and they do not offers any reuse
%   instrument to ensure that state is coherently initialized when finally defined
%   in classes.  Traits can't be instantiated. This model requires two concepts
%   (traits and classes) to coexist and cooperate.

%   Some authors see this as good language design fostering good software
%   development by helping programmers to think about the structure of their
%   programs.  However, other authors see the need of two concepts and the absence
%   of state as drawbacks of this model. Our approach takes interfaces as units of
%   reuse, and meanwhile generates factory method for instantiation.

% \item C++ and Scala also try to provide solutions to multiple inheritance, but
%   both suffer from object initialization problems. Virtual inheritance in C++
%   provides another solution to multiple inheritance (especially the diamond
%   problem by keeping only one copy of the base class)~\cite{ellis1990annotated},
%   however suffers from object initialization problem as pointed out by Malayeri
%   et al.~\cite{malayeri2009cz}. It bypasses all constructor calls to virtual
%   superclasses, which would potentially cause serious semantic error. Scala
%   solution (very similar to linearised mixins, but misleadingly called traits in
%   the language) avoids this problem by disallowing constructor parameters,
%   causing no ambiguity in cases such as diamond problem.  This approach has
%   limited expressiveness, and suffers from all the problems of linearised mixin
%   composition.
%   Python also offers multiple inheritance via linearised mixins. Indeed in python any class is implicitly a mixin, and mixin composition informally expressed as\\*
%   \Q@ class A use B,C {...new methods...}@\\*
%   can be expressed in python as \\*
%   \Q@ class Aux: ...new methods...@\\*
%   \Q@ class A(B,C,Aux): pass@
%   Our approach not only does not involve linearised mixin problem, but also
%   support proper constructor mechanism.
% \end{enumerate}

% %\section{Comparing to traits and mixins}\label{sec:comparison}
\subsection{Multiple Inheritance in ClassLess Java}
\begin{comment}
Haoyuan

   - vs both: we do automatic return type refinement, which has useful applications
   (example: Expression Problem)

   - vs traits: we support of methods to create new objects (a replacement to constructors);
   Moreover we have the with and clone methods (we miss more applications for those). Show
   how to model the operations on traits; discuss operations that we cannot model
   (example: renaming).

   - vs mixins: we use the trait model of explicitly resolving conflicts. This is arguably
   better for reasoning.
\end{comment}

Our approach is based on forbidding the use of certain language constructs (classes), in order to rely on the modular composition offered by the rest of the language (interfaces).
Since Java was designed for classes, a direct class-less programming style is verbose and feel unnatural. However, annotation driven code generation is enough to overcome this difficulty, and
the resulting programming style encourages modularity, composability and reusability, by keeping a strong object oriented feel.
We consider our approach as an alternative to traits or mixins.

A comparison on how to to emulate the original operations on traits using Java8 can be find in\cite{bono14}. We briefly recall the main points of their encoding; however we propose a different representation of
 \textbf{exclusion}.
The author of ~\cite{bono14} agree that our revised version is cleaner, typesafe and more direct.\bruno{is this something to 
say in the paper?}

%%\newcommand\shortItem{\vspace{-1ex}}
\begin{itemize}
\begin{comment}%I shoreten up a lot after. I think this version is better but longer, and this content is just repeating bono at all.
\item \textbf{Symmetric sum}: the symmetric composition of two disjoint traits is achieved by simply implementing two interfaces in Java correspondingly, without overriding any method. The composition relies on multiple inheritance on interfaces, which is supported by Java. Below is a simple example.
    \begin{lstlisting}
    interface A { int x(); }
    interface B { int y(); }
    interface C extends A, B {}
    \end{lstlisting}
\item \textbf{Override}: the overriding operation (also known as asymmetric/preferential sum) is modelled by implementing many interfaces, while overriding some methods inside. The code below gives an example of explicitly specifying which super interface take precedence, regarding conflict on a specific method.
    \begin{lstlisting}
    interface A { default int m() {return 1;} }
    interface B { default int m() {return 2;} }
    interface C extends A, B { default int m() {return B.super.m();} }
    \end{lstlisting}
    Here the method \texttt{m()} in interface \texttt{C} simply inherits from \texttt{B.m()}.
\item \textbf{Alias}: an alias operation adds a new name to an old method when creating the new trait. In Java, we just create a new method with reference to the existing method in its super interface. See the example below, where the new method \texttt{k()} is an alias of the existing method \texttt{m()}.
    \begin{lstlisting}
    interface A { default int m() {return 1;} }
    interface B extends A { default int k() {return A.super.m();} }
    \end{lstlisting}
\end{comment}

\item \textbf{Symmetric sum} can be obtained by simple multiple inheritance between interfaces.
\shortItem
    \begin{lstlisting}
    interface A { int x(); }    interface B { int y(); }    interface C extends A, B {}
    \end{lstlisting}
\shortItem
\item \textbf{Overriding} a conflict is obtained by specifying which super interface take precedence.
\shortItem
    \begin{lstlisting}
    interface A { default int m() {return 1;} } 
    interface B { default int m() {return 2;} }
    interface C extends A, B { default int m() {return B.super.m();} }
    \end{lstlisting}
\shortItem
\item \textbf{Alias} is creating  a new method delegating to the existing super interface.
\shortItem
    \begin{lstlisting}
    interface A { default int m() {return 1;} }
    interface B extends A { default int k() {return A.super.m();} }
    \end{lstlisting}
\shortItem

\item \textbf{Exclusion}: exclusion is also supported in Java, where method declarations can hide the default methods correspondingly in the super interfaces.
\shortItem
    \begin{lstlisting}
    interface A { default int m() {return 1;} }
    interface B extends A { int m(); }
    \end{lstlisting}
\shortItem
\end{itemize}

Besides, we support more features than the original trait model:
\begin{itemize}
\item We provide \texttt{of} methods for the annotated interfaces. During annotation processing time, the ``fields'' inside an interface are observed and a static method \texttt{of} is automatically injected to the interface with its arguments correspondingly. Such a method is a replacement to the constructors in original traits, making instantiation more convenient to use.
That is, in our approach there are only interfaces, our model requires a single concept, while the trait model requires traits \emph{and} classes.

\item We provide \texttt{with-} methods as auxiliary constructors. A \texttt{with-} method is generated for each field, just like record update, returning the new object with that field updated
%. A \texttt{clone} method is generated for the interface, returning a copy of the current object.
Furthermore, we do automatic return type refinement for these kind of methods. This feature is comparatively useful in big examples, making operations and behaviours more flexible.%, which we will demonstrate later.
\item We provide two options for generating setters. There are two kind of setters which are commonly used, namely \textit{void setters} and \textit{fluent setters}. The only difference is that a fluent setter returns the object itself after setting, thus supporting a pipeline of such operations. The generation depends on which type of setter is declared in the interface by users.
\end{itemize}

These are the additional features supported by our model, conversely, there are certain operations we cannot model, such as method renaming (as in [Reppy2006]), which breaks structural subtyping.

There are other limitations of our current approach, but they may be addressed
in future work (see Section~\ref{sec:futurework}).



\paragraph{ThisType and MyType}
%\marco{ this section is unfair. There are unsolved problems with ThisType in negative positions, namely it breaks subtyping. The most common solution is to allow calling some methods only when the exact type is known. This demolishes most advantages of interfaces.}
In certain situations, object interfaces allow automatic refinement for \emph{return
types}. This is part of a bigger topic in class-based languages: expressing and
preserving type recursion and (nominal/structural) subtyping at the same time.

One famous attempt in this direction is provided by
\emph{MyType}~\cite{bruce1994paradigmatic}, representing the type of
\textbf{this}, changing its meaning along with inheritance. However when
invoking a method with MyType in a parameter position, the exact type of the
receiver must be known.  This is a big limitation in class based object oriented
programming, and is exasperated by the interface-based programming we propose: no
type is ever going to be exact since classes are not explicitly used. A recent
article~\cite{Saito2013933} lights up this topic, proposing two
new features: exact statements and nonheritable methods. Both are
related to our work: 1) any method generated inside of the \Q@of@ method is indeed
non-inheritable since there is no class name to extend from; 2) exact
statements (a form of wild-card capture on the exact run-time type) could
capture the ``exact type'' of an object even in a class-less
environment.
Admittedly, MyType greatly enhances the expressiveness and extensibility of
object-oriented programming languages. Object interfaces use covariant-return types to
simulate some uses of MyType. But our approach only works for refining
return types, whereas MyType is more general, as it also works for
parameter types. Nevertheless, as illustrated with our examples and
case studies, object interfaces are still very useful in many
practical applications.
% The addition of MyType to a language will allow easy definition of
%binary methods, methods with recursive types (i.e., the same type of the
%receiver appears in the argument or return positions of methods), etc.


\begin{comment}
\subsection{Type-Directed Translations/Syntactic Sugar}
\marco{I'm tring to merge this and the next one}
Language extensions are often implemented as syntactic sugar of the base
language. For example, Scala compiler supports XML syntax in normal Scala code
directly (after Scala ?, users need to import \texttt{scala-xml} library
manually). However, this approach is hard in terms of implementation, because it
requires extending the compiler. Also, this approach does not support combining
multiple extensions into one.

SugarJ~\cite{erdweg11sugarj} is a Java-based extensible programming language
that allows programmers to extend the base language with custom language
features by definitions in meta-DSLs (SDF, Stratego, etc).
\yanlin{Is new syntax really a ``drawback''? I think for some system, like
  SugarJ, one of purpose IS to introduce these new syntax.}  Drawbacks: new
syntax. To create the extension, programmers have to work with multiple
languages (SDF, stratego, etc) while our approach works totally in Java
environment.

We can model certain types of language extensions with annotations
only, but those extensions do not introduce new syntax: they
merely do automatic code generation.
\end{comment}

\paragraph{Meta-Programming Competes with Language Extensions}
The most obvious solution to adding features to a language is via syntactic
extensions. Syntactic extensions are often implemented as
desugarings to the base language. For example,
the Scala compiler was extended to directly support XML syntax.
However, when syntactic extensions are independently created it is hard
to combine multiple extensions into one.
SugarJ~\cite{erdweg11sugarj} is a Java-based extensible language that
aims at making the creation and composition of syntactic sugar
extensions easy, by allowing programmers to extend Java with custom
features (typically for DSLs). However SugarJ's goals are different
from language tuning: SugarJ aims at creating and composing new syntax;
whereas language tuning merely reinterprets existing syntax. It is
clear that reinterpreting existing syntax only can be limiting
for some applications. However, when language tuning is possible
it has the advantage that existing tools for the language work
out-of-the-box (since the syntax is still the same); and composition
of independently developed tunings is straightforward.

Scala-Virtualized~\cite{Rompf2012scalavirtualized} is an extension to Scala,
which allows blending shallow and deep embedding of DSLs. It redefines
some of Scala's language constructs to method calls,
which can be overridden by DSL implementer. Thus Scala-Virtualized
can also reinterpret syntax, and be seen as a form of language tuning.
However, although many Scala's language constructs are supported,
not all language constructs can be virtualized.

%These extensions are admittedly useful, however they are essentially
%creating a fork of the language, and rarely the new fork gains enough
%traction to become the main language release. % On this
% topic we mention


When the base language has a flexible enough syntax and a fast and
powerful enough reflection mechanism, we may just need to play with
operator overloading and other language tricks to discover that the
language feature we need can be expressed as a simple library in our
language. An example of this is SQLAlchemy~\cite{bayer2012sqlalchemy} in
Python, which uses operator overloading to dynamically turn normal python expressions into database queries
without requiring any syntactic extensions to Python.
Java-like languages tend to sit in the middle of two extremes:
libraries can not influence the type system, so many solutions valid
in Python or other dynamic languages are not applicable, or have the
cost of losing type-safety.

In Java-like languages compile time code generation comes at the rescue: if, for a certain feature
(\mixin in our case), it is possible to use the original language syntax to
\emph{express/describe} any specific instantiation of such feature (annotating a
class and providing getters), then we can insert in the compilation process a tool
that examines and enriches the code before compilation. No need to modify the
original source (for example we can work on temporary files).
Java is a particular
good candidate for this kind of manipulation since it already provide ways to
define and integrate such tools in its own compilation process via annotation
processing. In this way
there is no need of temporary files, and there is a well-defined way of putting
multiple extensions together.

Other languages offer even stronger support for safe code manipulation:
Template Haskell~\cite{sheard2002template}, F\# (type providers)~\footnote{http://research.microsoft.com/fsharp/} and MetaFjig (Active Libraries)~\cite{servetto2010metafjig}
all allow to execute code at compile time. They generate code that is
transparently integrated in the program that is being
generated/processed/compiled. In particular, MetaFjig offers a property called
\emph{meta-level-soundness}, ensuring by construction that library code (even if wrong or unreasonable) never generates ill-typed code. This is roughly equivalent to
\textit{Self coherence}, that we have to manually prove.
Since MetaFjig is not working on annotated classes, there is not a ``semantic with/without annotations''. Our \mixin tuning theorem does not make sense in such context.

\paragraph{Formalization of Java 8}
We provide a simple formalization for a subset of Java including
default/static interface methods and object initialization literals
(often called anonymous local inner classes).  A similar formalization
was drafted by Goetz and Field~\cite{goetz12fdefenders} to formalize
defender (default) methods in Java. In their formalization, classes
and interfaces can have only one method \Q@m()@ without arguments, so
as to simplify method overloading and renaming. Classless Java is more
general, as it supports multiple methods with arguments, it supports
static methods, and features such as multiple inheritance of
interfaces and reabstraction of default methods are also modeled.

\bruno{next comes some text that may be handy for related work}

While the motivation for default methods was to allow
interfaces to be extended over time while preserving backwards
compatibility (\emph{interface evolution}), default methods can also
emulate \emph{traits}~\cite{scharli03traits}.  \footnote{The original
  notion of traits by Scharli et al. prescribes, among other things,
  that: 1) a trait provides a set of methods that implement behavior;
  and 2) a trait does not specify any state variables, so the methods
  provided by traits do not access state variables directly. Java 8
  interfaces follow similar principles too. Indeed, a detailed
  description of how to emulate trait-oriented programming in Java 8
  can be found in the work by Bono et al.~\cite{bono14}. The Java 8
  team designing default methods, was also fully aware of that
  secondary use of interfaces, but it was not their objective to model
  traits: ``The key goal of adding default methods to Java was
  "interface evolution", not "poor man's
  traits"''~\cite{goetz13default}. As a result, they were happy to
  support the secondary use of interfaces with default methods as long
  as it did not make the implementation and language more complex.}
\marcoT{We could use Scala Traits to encode IB, but the simpler Java8
  model is a better starting point.}  \footnote{ \marco{may go in
    related work} Scala traits can declare \Q@def, val, var@ and
  \Q@lazy val@ However, while \Q@def@s can be overridden by
  \Q@val@/\Q@var@s, \Q@val@/\Q@var@s can not be overridden.\marco{we
    need to check this} A\Q@val@/\Q@var@ member is predestined to be
  instance state, and this is in sharp contrast with our model where
  the state is only a private detail of instances, and do not
  contaminate interface/trait/classes.}  \footnote{Scala traits have
  no constructor, but \Q@var@s and (\Q@lazy@) \Q@val@s can be
  ``pre-initialized''.}



\section{Interaction of Interface Methods with Interface Composition}
\bruno{I don't think this section makes alot of sense in the context 
of the new paper! I think perhaps some of this could be mentioned 
in related work.

In my opinion, instead of this section, we could have one section 
that demonstrates our implementation and its advantages.

\begin{itemize}

\item Show a diagram of how the approach works? 

\item Syntax and type system reuse; (good for error messages, for example)
\item On-the-fly AST rewriting (means no new files are created, and functionality
is immediatelly available for IDE's for example); this is better than conventional 
annotation processors.
\item Eclipse integration; code completion.
\item No integration issues in typical pre-processors; (the process is transparent; 
no need for an external tool that generates code in a first step).
\item Separate compilation (which relates to the properties?)

\end{itemize}

}
Before formalizing Classless Java and object interfaces, it is helpful
to informally discuss the behaviour of Java8
interfaces, in contrast with conventional trait models.

%We show several interesting cases
%and summarize compilation result of method composition into 3 categories:
%conflict error, accepted (both abstract) and conservative error. Meanwhile we
%also show different composition handling mechanism among traits, javac and ECJ.

\subsection{Methods in Java 8 Interfaces}
In Java 8 interfaces there are three types of methods: abstract, default, and static methods. Default and static methods were not allowed in interfaces in previous versions of Java.

\paragraph{Static methods} are handled in a very clean way: they are visible only in
  the interface in which they are explicitly defined. This means the following code
  is ill-typed.
\begin{lstlisting}
interface A0 { static int m(){return 1;} }
interface B0 extends A0 {}
... B0.m()//ill typed
\end{lstlisting}
This is different from the way static methods are handled in classes. Here
static methods have simply no interaction with interface
composition (\Q@extends@ or \Q@implements@).

\paragraph{Abstract method} composition is accepted when there exists a most specific one.
  For example, here method \texttt{Integer m()} from \texttt{B1} is visible in \Q@C1@.
\begin{lstlisting}
interface A1 { Object m(); }
interface B1 { Integer m();}
interface C1 extends A1,B1 {} //accepted
\end{lstlisting}

\paragraph{Default methods} conflict with any other default or abstract method. For
  example the following code is rejected due to method conflicts.
\begin{lstlisting}
interface A2 { default int m() {return 1;}}
interface B2 { int m(); }
interface C2 { default int m() {return 2;}}
interface D2 extends A2,B2 {} //rejected due to conflicting methods
interface E2 extends A2,C2 {} //rejected due to conflicting methods
\end{lstlisting}
Note how this is different from what happens in most trait models, where \Q@D2@
would be accepted, and the implementation in \Q@A2@ would be part of the
behavior of \Q@D2@.

\paragraph{Resolving conflicts:}
A method in the current interface wins over any method in its
super-interfaces, provided that the method
is the most specific one. This method also overrides conflict due to
inheritance. For example, the following code is accepted, but would be rejected
(see before) if the method \Q@m@ was not redefined in \texttt{D3} and
\texttt{E3}.
\begin{lstlisting}
interface D3 extends A2,B2 { int m(); } //accepted
interface E3 extends A2,C2 { default int m(){return 42;} } //accepted
\end{lstlisting}

\subsection{Classifying Outcomes of Interface Composition}
%When interfaces are composed and methods with the same name (and
%signature) exist, there are 3 possible outcomes.
%
We now try to classify possible outcomes for composition of methods with the same name (and
signature).
We will use the following (correct) declarations:
\begin{lstlisting}
interface A1 { T m(); }
interface A2 extends A1 { default T m(){ ... } }
interface A3 extends A2 { T m(); }

interface B1 { default T m(){ ... } }
interface B2 extends B1 { T m(); }
interface B3 extends B2 { default T m(){ ... } }
\end{lstlisting}

\noindent
What happens if a new interface \Q@M@ extends one \Q@A@${}_i$ and one
\Q@B@${}_j$?
% Nine representative cases are shown next:\\*
\begin{figure*}[htbp]
\centering
\begin{tabular}{|l|l|l|l|}
\hline
\textbf{M extends} & \textbf{A1}                  & \textbf{A2} & \textbf{A3} \\ \hline
\textbf{B1}        & conservative error                     & conflict error      & conservative error       \\ \hline
\textbf{B2}        & both abstract (accepted)     & conservative error       & both abstract (accepted)       \\ \hline
\textbf{B3}        & \textbf{conservative error} &conflict  error       & conservative error      \\ \hline
\end{tabular}
\end{figure*}
%We try to classify the results in the table:
\begin{itemize}
\item \textbf{conflict error} happens when the methods from both interfaces are implemented, which is also an error in most trait models.
\item \textbf{both abstract (accepted)} happens when the methods from both interfaces are abstract, which is also considered correct in all
  trait models.
\item \textbf{conservative error} happens when only one method is implemented
  (leaving another one abstract), which is different from what we would expect in
  a trait model, but is coherent with the conservative idea that a method
  defined in an interface should not silently satisfy a method in another one.
\end{itemize}

\paragraph{A bug:} During our experimentation, we found a bug in ECJ (Eclipse compiler for Java):
the case \textbf{M} extending \textbf{B3} and \textbf{A1} is accepted by
ECJ4.5.1 and rejected by javac.  By email communication with Brian Goetz
(leading Java 8 designer) we have confirmed that the expected behavior is
rejection, hence this is a bug in ECJ. This bug was also reported by
  others and is fixed in the ECJ developer branch, but not released as a stable
  version yet.

\section{Conclusion}\label{sec:conclusion}


Before Java 8, concrete methods and static methods were not allowed
to appear in interfaces.  Java 8 allows static interface methods and
introduces \emph{default methods}, which enables implementations
inside interfaces. This had an important positive consequence that
was probably overlooked: the concept of class
(in Java) is now (almost) redundant and unneeded.
We proposed a programming style, called ClassLess Java, where
truly \objectoriented programs and (reusable) libraries
can be defined and used without ever defining a single class.

However, using this programming style directly in Java is very verbose.
To avoid syntactic boilerplate
caused by Java not being originally designed to work without classes,
we introduce a new annotation, \mixin, which provides default implementations
for various methods (e.g. getters, setters, with-methods) and a
mechanism to instantiate objects. 
We leverage on annotation processing and the Lombok library, in this way
\mixin is just a normal Java library; thus our proposed style can be integrated
in any Java project.

The \mixin annotation helps programmers
to write less cumbersome code while coding in Classless Java. Indeed, 
we think the obtained gain is so high that Classless Java with the \mixin
annotation can be less cumbersome than full Java.


Class less Java is just a programming style, but is 
showing the way for a new flavour of object orientation:
We propose \interfacebased \objectoriented languages (IB),
as opposed to \classbased or \prototypebased.
In IB state is not modelled at the platonic/ideal level
but is handled excursively by instances.
This unlock useful code reuse patterns, as shown in~\ref{sec:ep}.



Interestingly, without classes there is also no subclassing. This scratches an old
  itching point in the long struggle of subtyping versus subclassing:
  according to some authors, from a software engineering perspective,
  interfaces are just a kind of classes. Others consider more
  opportune to consider interfaces as pure types. We do not know how to conciliate
  those two viewpoints and Classless Java design.
  Classless Java does not have classes purely in the Java sense.  
% Classless Java encourages coding in a more flexible way by either
% keeping a higher abstraction level (interfaces are a more abstract
% concept than classes), or relying on concrete object initialization
% (the \Q@new I(){...}@ construct).


% More generally, we identify the concept of \emph{language tuning}.
% We identify libraries that are already performing language tuning (Lombok and Cofoja), and 
% we forecast many different kinds of language tuning will emerge on suitable platforms like Java or the C\# CLR.
% We identify various kinds of safety guarantees that can be offered by language tuning, but the door is open for more flavors of safety guarantees to emerge.


%\subparagraph*{Acknowledgements.}

%I want to thank \dots

%\appendix
%\section{Morbi eros magna}



%%
%% Bibliography
%%

%% Please use bibtex, 

\bibliography{main}


\end{document}