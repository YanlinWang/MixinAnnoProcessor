%
\documentclass{llncs}
%
\usepackage{makeidx}  % allows for indexgeneration
\usepackage{listings}
\usepackage{color}
\usepackage{xspace}

\newcommand{\mynotes}[3]{{\color{#2} {\sc #1}: #3}}
\newcommand\bruno[1]{\mynotes{Bruno}{red}{#1}}
\newcommand\marco[1]{\mynotes{Marco}{blue}{#1}}

\newcommand{\syndef}{$::=$}

\newcommand\name{{\bf DTrait}\xspace}

%
\begin{document}
%
\frontmatter          % for the preliminaries
%
\pagestyle{headings}  % switches on printing of running heads
%\addtocmark{Hamiltonian Mechanics} % additional mark in the TOC
%

\title{Separating Use and Reuse to Improve Both}
%
\titlerunning{Hamiltonian Mechanics}  % abbreviated title (for running head)

\author{Marco Servetto\inst{1} \and Bruno C. d. S. Oliveira\inst{2}}
%
\authorrunning{Ivar Ekeland et al.} % abbreviated author list (for running head)
%
%%%% list of authors for the TOC (use if author list has to be modified)
\tocauthor{Marco Servetto and Bruno C. d. S. Oliveira}
%
\institute{Victoria University of Wellington, New Zealand,\\
\email{marco.servetto@ecs.vuw.ac.nz},\\ WWW home page:
\texttt{https://ecs.victoria.ac.nz/Main/MarcoServetto}
\and
The University of Hong Kong, Hong Kong\\}

\maketitle              % typeset the title of the contribution

\begin{abstract}
Abstract here!
\dots
\keywords{Code Reuse, Object-Oriented Programming}
\end{abstract}

\section{Introduction}

Historically, we have seen a lot of focus on the importance of
separing subtype from subclassing~\cite{cook}.  This is claimed to be
good for code reuse, design and reasoning.  While there is no problem
in subtyping without subclassing, in most OO languages code reuse
(inheritance/extends) imply subtyping.  Consider the following (Java)
code:

\begin{lstlisting}[language=Java]
  class A{ int ma(){return Utils.m(this);} }
  class Utils{static int m(A a){..}}
\end{lstlisting}

This code seems correct, and there is no subtyping/subclassing.
Now, lets add a class B

\begin{lstlisting}[language=Java]
  class B extends A{ int mb(){return this.ma();} }
\end{lstlisting}

Class \lstinline{B} now have a method \lstinline{ma} inherited from \lstinline{A}.
This method passes \lstinline{this} as \lstinline{A}; we can see an invocation of \lstinline{ma}
inside \lstinline{mb}, where \lstinline{this} is of type \lstinline{B}.
The execution will eventually call \lstinline{Util.m} with a \lstinline{B} instance.
This can be correct only if \lstinline{B} is a subtype of \lstinline{A}.
Thus, every OO language with the minimal features 
exposed in this example (using this, extends, method call)
is forced to accept that subclassing implies subtyping.
Note how this is true also in C++, where is possible to
"extends privately". Such is a limitation of the visibility of
subtyping but not over subtyping itself, and the former example
would be accepted by C++ even if \lstinline{B} was to "privately extends" \lstinline{A}.
\marco{Cite some work of bruceTHisType and show how he also fails.}
\bruno{I was expecting some text/argument here on why subclassing
  implies subtyping is bad. Perhaps another example where subclassing 
must not imply subtyping to type-check.}


We will show here a simple design to completley uncouple subtyping
and subclassing in a nominally typed OO language,
by dividing code designed for USE
from code designed for REUSE.
First we show a minimal language, then we show how mutually recursive types
are supported, how state/constructors/fields are supported,
how we can extend the language with nested classes.
Finally we show 42, a full blown language build around our ideas of reuse.

Our design leverage on traits~\cite{}: a well know mechanisms for pure
code reuse

\section{Overview of \name}

Present basic examples and syntax?

\section{A Decoupled Trait Calculus}\bruno{Probably we need a name 
for the language/calculus}

Lets consider a simple language where trait names start lowercase (t)
and class names start uppercase (C).  
The syntax of the language is:\\

\begin{tabular}{lrcl}
      Code Literal & $L$ & \syndef & ${{\bf interface}?~{\bf implements}~\overline{C}~\overline{M}}$ \\
%      Types & $\tau$ & \syndef & $C$ \\
      Method Declaration & $M$ & \syndef & ${\bf class}?~{\bf method}~C~m(C~\overline{x})~e?$ \\

      Traits & $T$ & \syndef & $t \mid L$ \\
      Trait Declarations & $TD$ & \syndef & $t:L \mid t:Sum~\overline{T}$ \\
      Class Declarations & $CD$ & \syndef & $C:L \mid C:Sum~\overline{T}$ \\
      Expressions & $e$ & \syndef & $x \mid C \mid e.m(\overline{e})$\\
      Declarations & $D$ & \syndef & $CD \mid TD$\\
\end{tabular}

\bruno{Original is below for double-checking.}
\begin{verbatim}
L::={interface? implements Ts Ms} //code literal
T::=C //types are class names
M::= class? method T m(Txs) e? //method declaration
tL::=t | L  //trait name or literal
TD::= t:L | t:Sum tLs  //trait declaration
CD::= C:L | C:Sum tLs
e::= x | T | e.m(es)
D::=CD | TD
\end{verbatim}

To declare a trait or a class, we can use a code literal (or a trait
expression).  Traits come with various operators (restrict, hide,
alias) but for now we focus on the single operator Sum, taking a set
of trait names or literals and composing them in an \emph{associative}
and \emph{commutative} way.  Sum is the simplest and most elegant
composition operator.  $Sum \overline{T}$ composes the content of $\overline{T}$
by taking the union of the methods and the union of the implementations.
The Sum can not be applied if multiple versions of the same method are
present in different traits.  An exception is done for abstract methods:
methods where the implementation e is missing; In this case, if the
headers are compatible the implemented version is selected.  In a sum
of two abstract methods with compatible headers, the one with the more
specific type is selected. We will discuss later the formal details of
Sum.

A version of this language where there are no traits can be seen 
as a restriction/variation of FJ~\cite{}.

As you can see code literals can be marked as interfaces.  We will call class
interface, or simply interface a class declaration of for C:{interface
  ...}.

While classes are typed assuming \lstinline{this} is of the nominal type of the
class, trait declarations, do not introduce any nominal type.  \lstinline{this}
in a trait is typed with a special type \lstinline{This} that is visible only
inside such trait~\bruno{{\bf This} does not show up anywhere in
  the syntax!}.  Syntactically, a Literal can use the \lstinline{This} type,
and when the trait is reuse to create a class, This will become just
an alias for the class name.

In a program $\overline{D}$, a code literal is not well typed if it
implements some \lstinline{C} that is not an interface; that is,
implementing interface is the only way to induce subtyping in this
language literals can implement classes.

\subsection{Code Use and Code Reuse in \name}

As you can see there is no extends keyword, and \lstinline{Sum} can not contain class names.
In a sense, all classes are final (as in Java).
That is, using a trait is the only way to induce code reuse.

Lets now try to encode the example in Section~\ref{} e in our language:

\begin{lstlisting}
 A:{ method int ma(){return Utils.m(this);} }
 Utils:{ class method int m(A a){..}}
\end{lstlisting} 

This code is fine, but there is no way to add a B reusing code of A, since
A is designed for code \emph{use} and not \emph{reuse}. Lets try
again, but this time aiming at code reuse:

\begin{lstlisting}
 ta:{ method int ma(){return Utils.m(this);} }
 A:SUM ta
 Utils:{ class method int m(A a){..}}
\end{lstlisting}

This does not work, because \lstinline{Utils.m} requires an \lstinline{A} and \lstinline{this} in
ta have no knowlegde of \lstinline{A}. Lets try again

\begin{lstlisting}
 IA:{interface method int ma()}
 ta:{implements IA
   method int ma(){return Utils.m(this);} }
 A:SUM ta
 Utils:{ class method int m(IA a){..}}
\end{lstlisting}

This code works: Utils is using an interface IA and the trait ta
is implementing it. It is also possible to add a B as follow
Abeit other semantic models for traits~\cite{} have been proposed,
here we use the flattening model.
This means that 
 A:SUM ta
  reduces/is equivalent to/is flatted into
 A:{implements IA method int ma(){return Utils.m(this);} }
This code seems correct, and there is no subtyping/subclassing.
Now, lets add a class B

\begin{lstlisting}
  B:Sum ta, { int mb(){return this.ma();} }
\end{lstlisting}

This also works.  B reuses the code of ta, but has no knowledge of A.
Since B reuses ta, and ta implements IA, also B implements IA.  By
flattening, B is reduced to

\begin{lstlisting}
B:{implements IA
  method int ma(){return Utils.m(this);}
  int mb(){return this.ma();} } 
\end{lstlisting}

\subsection{Remarks on Typing}

Now some important notes on typing:

-traits are \emph{well-typed} before being reused. For example in

\begin{lstlisting}
t:{int m() 2 
   int n() this.m()+1}
\end{lstlisting}

\noindent t is well typed since m() is declared inside of t, while

\begin{lstlisting}
t1:{int n() this.m()+1} would be ill typed.
\end{lstlisting}

\noindent Note how in the former example ta is well typed just by knowing itself and IA.

-in class expressions C:Sum tLs
L in tLs are not well typed (on their own) before flattening, only
the result is supposed to be well typed.
Also Java have the same behaviour:
for example in

\begin{lstlisting}[language=Java]
  class B extends A{ int mb(){return this.ma();} }
\end{lstlisting}

\noindent we can call \lstinline{this.ma()} even if in the curly braces there is no declaration for ma.
In our example,with the trait t of before

\begin{lstlisting}
C: Use t {int k() this.n()+this.m()}
\end{lstlisting}
\bruno{Do we mean Sum instead of Use?}

\noindent would be well typed, even if n,m are not defined inside
\begin{lstlisting}
{int k() this.n()+this.m()}
\end{lstlisting}
This is not the case in may similar works in literature~\cite{} where the
literals have to be self complete. In this case we would have been forced to
declare abstract methods n and m.

This idea that code literals wrote
in the declaration of a class
do not have to be well typed on their own also lets us
reason about mutually recursive types, as shown in Section~\ref{}. 

-if a class is declared by using C:Sum ts, that is, without literals,
If the Sum is successfull, C is well typed, no need of further checking.
-on the other side, if a class is declared by C:Sum tLs, with
L1..Ln in tLs, and after successfull flattening C:L is not well typed,
the error was originally present in one of L1..Ln.

Recursive types:
OO language leverage on recursive types most of the times.
For example in a pure OO language, String may offer a Int size()
method, and Int may offer a String toString() method.

This means that is not possible to type in (full) isolation classes
String and Int.

The most expressive compilation process may divide the classes in groups of mutually 
dependent classes.
Each group may also depend from a number of other groups.
This would form a Direct Aciclyc Graph of groups.
To type a group, we first need to type all depended groups, then
we can extract the structure/signature/structural type of all
the classes of the group.
Now, with the information of the depended groups and the one extracted
from the current group, it is possible to typecheck the implementation
 of each class in the group.

In this model, it is reasonable to assume that flattening happens group by group, before
extracting the class signatures.

Here we go for a much simpler simple top down execution/interpretation for flattening, where flattening
happen one at the time, and classes are typechecked where their type is first needed.

For example

A:{int ma(B b) b.mb()+1}
tb:{int mb() 2}
tc:{int mc(A a,B b) a.ma(b)}
B: Sum tb
C: Sum tc, {method int hello() 1}

In this scenario, since we go top down, we first need to generate B.
To generate B, we need to use tb;
In order modularly ensure well typedness,
we require tb to be well typed at this stage. If tb was not well typed
a compilation error could be generated at this stage.
In this moment, A can not be compiled/checked alone,
we need informations about B, but A is not used in tb,
thus we do not need to type A and we can type tb with
 the available informations and proceed to generate B.
Now, we need to generate C, and we need to ensure well typedness of tc.
Now B is alreay well typed (since generated by Sum tb, with no Ls),
and A can be typed;  finally tc can be typed and used.
If sum could not be performed (for example it tc had a method hello too)
a compilation error could be generated at this stage.

On the opposite side, if B and C was swapped, as in
C: Sum tc, {method int hello() 1}  
B: Sum tb
now the first task would be to generate C, but 
to type tc we need to know the type of A and B.
But they are both unavailable: B is still not computed and 
A can not be compiled/checked alone, without information about B.
A compilation error would be generated, on the line of "flattening of C
requires tc, tc requires A,B, B is still in need of flattening".

In this example, a more expressive compilation process 
could compute a dependency graph and, if possible, reorganize the list,
but for simplicity lets consider to always provide the declarations
in the right order, if one exists.

Some may find the requirement of the existence of an order restrictive;
An example of a "morally correct" program where no right order exists is the following:

t:{ int mt(A a) a.ma()}
A:Sum t {int ma() 1}

We expect two (related) criticism to our compilation model:
1
In a system without inference for method types,
if the result of composition operators depends only on the
structural shape of their input (as for Sum)
is indeed possible to optimistically compute the resulting structural shape of the classes
and use this information to type involved examples like the former.
We believe such typing discipline could be fragile,
and could make human understanding the code reuse process much harder/involved.
2
In the world of strongly typed languages we could be tempted to
first check that all can go well, and then perform the flattening.
This would however be overcompicated for no observable difference:
Indeed, in the former example there is no difference
between
  First checking B and produce B code (that also contains B structural shape),
  then use B shape to check C and produce C code
or a more involved
  First check B and discover B structural shape as result of the checking,
  then use B shape to check C.
  Finally produce both B and C code.

\section{Managing State}

The Sum operator is very elegant, but our research community is struggling to
make it work with constructor/state/fields.
The goals fo this struggle are as follow:

\begin{itemize}

\item keep sum associative and commutative.
\item allowing a class to create instances of itself.
\item actually initialize objects, not leaving no null fields
\item managing fields in a way that borrows the elengance of summing methods
\item make easy to add new fields

\end{itemize}

In the related work we will show some alternative ways to handle state.
However the purest solution just requires methods:
  The idea is that
  the trait code just uses getter/setters/factories, while leaving
  to classes the role to finally define the fields/constructors.
  That is, the  the class has syntax richer that the trait one,
  allowing declaration for fields and constructors.
  This approach is very powefull[see class less java]
 
  Pro: This approach is associative and commutative, even self construction
  can be allowed if the trait requires a static/class method
  returning This; the class will then implement this method by forwarding
  a call to the constructor.
  Negative: writing the class code with the constructors and
   fields and getter/setters and factories can be quite tedious.
   There is no way for a trait to specify a default value for a field,
   the class need to handle all the state, even state that is conceptually
   "private" of such trait.

  for example 

\begin{lstlisting}
  pointSum: { method int x(); method int y();
    static method This of(int x,int y);
    method This sum(This other)
      This.of(this.x+other.x,this.y+other.y);
    }
  pointMul: { method int x(); method int y();
    static method This of(int x,int y);
    method This mul(This other)
      This.of(this.x*other.x,this.y*other.y);
    }
\end{lstlisting}

\subsection{A First Attempt at Composition}

\begin{lstlisting}
  Point:Use pointSum,pointMul
\end{lstlisting}  

\noindent would fail since methods x,y and of are still abstract.
In this model the user is required to write something similar to

\begin{lstlisting}
  CPoint:Use point,colored, {
    int x; int y;
    method int x()x; method int y()y;
    class method This of(int x, int y)
      new Point(x,y);
    constructor Point(int x, int y){
      this.x=x;this.y=y;this.color=color;
      }
    }
\end{lstlisting}

\noindent after a while programming in this style, those "fixpoint" close the state in
the obvious way classes becomes quite repetitive, and one wonder
if it could be possible to automatically generate such code[]
In our model we go one step further:
In our model there is no need to generate code, or to explicitly
write down constructors and fields; there is not even syntax for those
constructor.
The idea is that any class that "could" be completed in the obvious way
is a complete "coherent" class.
Others call abstract a class that have abstract methods.
We call abstract a class that have a set of abstract methods that are not
coherent, that is, you can not interpret those as factory,getters and setters.
  
In the detail: a class with no abstract method is coherent, and like Java Math
will just be usefull for calling class/static methods.
a class with a single abstract class method returning This
is coherent if all the other abstract methods can be seen as "abstract state
operations" over one of its argument.
For example,
if there is a class method This of(int x, int y) as before,
then
a method int x() is an abstract state method, getter for x.
a method Void x(int that) is abstract state method, setter for x.
while getters and setters are fundamental operations, we can immagine
more operations to be supported:

\begin{lstlisting}
method This withX(int that) 
\end{lstlisting}
may be a "wither", doing a functional field update
  method Void update(int x,int) may do two field update at a time
  method This clone() may do a shallow clone of the object.
  We are not sure what is the best set of abstract state operations yet.
  
  
  lets play with the points of before, to see what good can we do with the current
  instruments:

\begin{lstlisting}
  pointSum: { method int x(); method int y();
    static method This of(int x,int y);
    method This sum(This other)
      This.of(this.x+other.x,this.y+other.y);
    }
  pointMul: { method int x(); method int y();
    static method This of(int x,int y);
    method This mul(This other)
      This.of(this.x*other.x,this.y*other.y);
    }
  PointAlgebra:Sum pointSum,pointMul
\end{lstlisting}  

  As you can see, we can declare the methods independently and compose the result
  as we wish. However we have to repeat the abstract methods x,y and of.
  In addition of Sum,Mul we may want many operations over points; can we improve our reuse
  and not repeat such abstract definitions? of course!

\begin{lstlisting}
  p: { method int x(); method int y();
    static method This of(int x,int y);
    }
  pointSum:Sum p, { method This sum(This other)
      This.of(this.x+other.x,this.y+other.y);
    }
  pointMul:Sum p, { method This mul(This other)
      This.of(this.x*other.x,this.y*other.y);
    }
  pointDiv: ...
  PointAlgebra:Sum pointSum,pointMul,pointDiv,...
\end{lstlisting}
      
now our code is fully modularized, and each trait handle exactly one method.

What happens if we want to add fields instead of just operations?

\begin{lstlisting}
  colored:{ method Color color() }
  Point:Sum pointSum,colored
\end{lstlisting}

This first attempt does not work: the abstract color method
is not a getter for any of the parameters of 

\begin{lstlisting}
  class method This of(int x,int y) 
\end{lstlisting}

A solution is to provide a richer factory:

\begin{lstlisting}
  CPoint:Sum pointSum,colored,{
    class method This of(int x,int y) This.of(x,y,Color.of(/*red*/)
    class method This of(int x, int y,Color color)
    }
\end{lstlisting}

where we assume to support overloading on different parameter number.
This is a good solution, we think is better that any alternatives in literature,
however the method CPoint.sum reset the color to red.
What should be the behaviour in this case?
If instead of writing This.of() we used 
this.withX(newX).withY(newY) we could preserve the color from this.
Not sure if that would be better.
If the point designer could predict this kind of extension, then we could
use the following design:  
\begin{lstlisting}
  p: { method int x(); method int y();
    method This withX(int that);
    method This withY(int that);
    static method This of(int x,int y);
    method This merge(This other);
    }
  pointSum:Sum p, { method This sum(This other)
      this.merge(other).withX(this.x+other.x).withY(this.y+other.y);
    }
  colored:{method Color color();
    method This withColor(Color that);
    method This merge(This other)
      this.withColor(this.color().mix(that.color());
    }
  CPoint:/*as before*/
\end{lstlisting}  
  Now we can merge colors, or any other kind of state we may want to add
  following this pattern.

\end{document}
