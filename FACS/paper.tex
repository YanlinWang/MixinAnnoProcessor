%
\documentclass{llncs}
%
\usepackage{makeidx}  % allows for indexgeneration
\usepackage{listings}
\usepackage{color}
\usepackage{xspace}
\usepackage{times}
\usepackage{comment}
\lstset{language=Java,
  basicstyle=\ttfamily\small,%\scriptsize,
  keywordstyle=\bfseries,%\color{darkRed},
  showstringspaces=false,
  mathescape=true,
  xleftmargin=0pt,
  xrightmargin=0pt,
  breaklines=false,
  breakatwhitespace=false,
  breakautoindent=false,
  linewidth=4\textwidth,% should be enough
%  identifierstyle=\idstyle
 morekeywords={method,Use,This}
}

%marco
\newcommand{\Q}{\lstinline}
\newcommand\Opt[1]{#1 ?}
%bnf
\newenvironment{bnf}{$\begin{array}{lcll}}{\end{array}$}
\newcommand{\prodFull}[3]{#1&::=&#2&\mbox{#3}}
%\newcommand{\prodFull}[3]{#1&::=&\mbox{#2}&\mbox{#3}}
\newcommand{\prodInline}[2]{#1::=#2}
\newcommand{\prodNextLine}[2]{&&#1&\mbox{#2}}
\newcommand{\terminal}[1]{%
\ensuremath{$\texttt{#1}$}%
}
\newcommand{\terminalCode}[1]{\mbox{\lstinline{#1}}}
\newcommand{\metavariable}[1]{%
\ensuremath{\mathit{#1}}%
}
%----------------------
\newcommand\mL{\metavariable{L}}
\newcommand\mC{\metavariable{C}}
\newcommand\mT{\metavariable{T}}
\newcommand\mV{\metavariable{V}} %code value t or L
\newcommand\mMD{\metavariable{MD}}
\newcommand\mTD{\metavariable{TD}}
\newcommand\mCD{\metavariable{CD}}
\newcommand\mD{\metavariable{D}}
\newcommand\me{\metavariable{e}}
\newcommand\mx{\metavariable{x}}
\newcommand\mm{\metavariable{m}}
\newcommand\mt{\metavariable{t}}
\newcommand\use{\terminalCode{Use}}
\newcommand\oC{\mbox{\lstinline@{@}}
\newcommand\cC{\mbox{\lstinline@}@}}
\newcommand\oR{\mbox{\lstinline@(@}}
\newcommand\cR{\mbox{\lstinline@)@}}
%--------------------------
\newcommand{\mynotes}[3]{{\color{#2} {\sc #1}: #3}}
\newcommand\bruno[1]{\mynotes{Bruno}{red}{#1}}
\newcommand\marco[1]{\mynotes{Marco}{blue}{#1}}

\newcommand{\syndef}{$::=$}

\newcommand\name{{\bf DTrait}\xspace}

%
\begin{document}
%
\frontmatter          % for the preliminaries
%
\pagestyle{headings}  % switches on printing of running heads
%\addtocmark{Hamiltonian Mechanics} % additional mark in the TOC
%

\title{Separating Use and Reuse to Improve Both}
%
\titlerunning{Separating Use and Reuse to Improve Both}  % abbreviated title (for running head)

\author{Marco Servetto\inst{1} \and Bruno C. d. S. Oliveira\inst{2}}
%
\authorrunning{M. Servetto and B. Oliveira} % abbreviated author list (for running head)
%
%%%% list of authors for the TOC (use if author list has to be modified)
\tocauthor{Marco Servetto and Bruno C. d. S. Oliveira}
%
\institute{Victoria University of Wellington, New Zealand,\\
\email{marco.servetto@ecs.vuw.ac.nz},\\ WWW home page:
\texttt{https://ecs.victoria.ac.nz/Main/MarcoServetto}
\and
The University of Hong Kong, Hong Kong\\}

\maketitle              % typeset the title of the contribution

\begin{abstract}
In most object oriented languages subclassing/inheritance implies
subtyping. This is considered by many a conceptual design error, but it
seems required for technical reasons, due to what we call the
\emph{this-leaking problem}, showing that separating
code-reuse from subtyping is non-trivial and requires a significant
departure from the OO models in existing mainstream OO languages.

We are aware of at least 3 independently designed research languages 
that address this limitation:Delta-Trait, Package Template, DeepFjig.

Leveraging on \emph{traits}, in this work we aim to synthesizes
the main ideas of those very different language designs.
Our result is a nominally typed minimalist language, natural to OO programmers.
By making our type system distinguish between code-use and code-reuse
we can separate inheritance and subtyping, while avoiding 
redundant abstract declarations (required in Delta-Trait and DeepFjig), while supporting self construction,
binary methods and recursive types.
Moreover, we provide a novel and elegant solution to uniformly
handle behavior and state within trait composition.


\begin{comment}
Two distinct groups of researchers independently [.marco..] and
[..ferruccio.] used traits to provide
a sound language design where code-reuse and subtyping are decoupled.
In this work we distill their work in a simpler model that is far more
natural for a OO programmer.
For example, in [ marco ] class A {A m(){return this;}} is not well typed code,
and in [.ferruccio..] all the logic about object construction have to
be provided after all trait composition.

We provide an novel and elegant solution to uniformly handle behavior
and state within trait composition,
Making our type system distinguish between used code and reused code
we  can avoid many redundant
abstract declarations that was required by [ marco ] and [ ferruccio ],
while supporting self construction, binary methods and recursive types;

We then show that our model can easily support nested classes and a
rich composition algebra as in [ marco ]
\end{comment}

\keywords{Code Reuse, Object-Oriented Programming}
\end{abstract}

\section{Introduction}

Historically, we have seen a lot of focus on the importance of
separing subtype from subclassing~\cite{cook}.  This is claimed to be
good for code reuse, design and reasoning.  While there is no problem
in subtyping without subclassing, in most OO languages code reuse
(inheritance/extends) imply subtyping.  Consider the following (Java)
code:

\begin{lstlisting}[language=Java]
  class A{ int ma(){return Utils.m(this);} }
  class Utils{static int m(A a){..}}
\end{lstlisting}

This code seems correct, and there is no subtyping/subclassing.
Now, lets add a class \Q@B@

\begin{lstlisting}[language=Java]
  class B extends A{ int mb(){return this.ma();} }
\end{lstlisting}

Class \lstinline{B} now have a method \lstinline{ma} inherited from \lstinline{A}.
This method passes \lstinline{this} as \lstinline{A}; we can see an invocation of \lstinline{ma}
inside \lstinline{mb}, where \lstinline{this} is of type \lstinline{B}.
The execution will eventually call \lstinline{Util.m} with a \lstinline{B} instance.
This can be correct only if \lstinline{B} is a subtype of \lstinline{A}.
Thus, every OO language with the minimal features 
exposed in this example (using this, extends, method call)
is forced to accept that subclassing implies subtyping.
Note how this is true also in C++, where is possible to
"extends privately". Such is a limitation of the visibility of
subtyping but not over subtyping itself, and the former example
would be accepted by C++ even if \lstinline{B} was to "privately extends" \lstinline{A}.
\marco{Cite some work of bruceTHisType and show how he also fails.}
\bruno{I was expecting some text/argument here on why subclassing
  implies subtyping is bad. Perhaps another example where subclassing 
must not imply subtyping to type-check.}


We will show here a simple design to completley uncouple subtyping
and subclassing in a nominally typed OO language,
by dividing code designed for \textbf{USE}
from code designed for \textbf{REUSE}.\footnote{We need to talk of unanticipated extensions?}
First we show a minimal language, then we show how mutually recursive types
are supported, how state/constructors/fields are supported,
how we can extend the language with nested classes.
Finally we show 42, a full blown language build around our ideas of reuse.

Our design leverage on traits~\cite{}: a well know mechanisms for pure
code reuse


\section{Intuitions on formalization}
In this article we dedicate more space to examples and informal presentation and motivations;
so we do not have space to provide a full formalizations.
We will provide here some hints on how the formalization works.

In the following, we present a very simplified grammar:
%\begin{bnf}
%\prodFull{aa}{bb}{Declaration}\\
%\end{bnf}

%\begin{comment}
\begin{bnf}
\prodFull\mTD{\mt\terminalCode{:}\mL \mid \mt\terminalCode{:} \use\ \overline\mV}{Trait Decl}\\
\prodFull\mCD{\mC\terminalCode{:}\mL \mid \mC\terminalCode{:} \use\ \overline\mV}{Class Decl}\\
\prodFull\mV{\mt \mid \mL}{Code Value}\\
\prodFull\mL{\Opt{\terminalCode{interface}}\ \terminalCode{implements} \overline\mT\ \overline\mMD}{Code Literal}\\
\prodFull\mT{\mC}{types are class names}\\
\prodFull\mMD{\Opt{\terminalCode{class}}\ \terminalCode{method}\ \mT\ \mm\oR\overline{\mT\,\mx}\cR \Opt\me}{Method Decl}\\

\prodFull\me{\mx\mid\mT\mid\me\terminalCode{.}\mm\oR\overline\me\cR}{expressions}\\
\prodFull\mD{\mCD\mid\mTD}{Declaration}\\
\end{bnf}
%\end{comment}

To declare a trait \mTD\ or a class \mCD, we can use either a code literal \mL\ or a trait
expression.  Traits come with various operators (restrict, hide,
alias) but in this work we focus on the single operator 
$\use$, taking a set
of code values: that is trait names \mt\ or literals \mL\ and composing them.  This operation, sometimes called \emph{sum}, is the simplest and most elegant
trait composition operator.  $\use\ \overline\mV$ composes the content of $\overline\mV$
by taking the union of the methods and the union of the implementations.

\use can not be applied if multiple versions of the same method are
present in different traits.  An exception is done for abstract methods:
methods where the implementation \me\ is missing. In this case (if the
headers are compatible) the implemented version is selected.  In a sum
of two abstract methods with compatible headers, the one with the more
specific type is selected.

Code literals \mL\ can be marked as interfaces. 
That is, the interface keyword is inside the curly brakets, so an upper case name associated with an interface literal is a class-interface, while a lowercase one is a trait-interface.
In our simple model, we consider an error trying to merge an interface with a non-interface.
 Then we have a set of implemented interfaces and a set of member
  declarations. In this simple language, the only members are methods.
If there are no implemented interfaces, in the concrete syntax we will omit the \Q@implements@ keyword.

Methods \mMD~can be instance methods or \Q@class@ methods. A class method is similar to a \Q@static@ method in Java but can be abstract. This is very usefull in the context of code composition.
To denote a method as abstract, instead of an optional keyword we just omit the implementation \me.

A version of this language where there are no traits can be seen 
as a restriction/variation of FJ~\cite{igarashi2001featherweight}.

\paragraph{Well-formedness}
Basic well formedness rules apply:
\begin{itemize}
\item all the traits and classes have unique names in a program $\overline\mD$,
\item all method parameters have unique names and the special parameter name \Q@this@ is not declared
 in the parameter list,
\item all methods in a code literal have unique names,
\item all used variables are in scope,
\item all methods in an interface are abstract, and there are no interface class methods.
\end{itemize}
Those rules can be applied on any given \mL~individually and in full isolation.

We expect the type system to enforce: 
\begin{itemize}
\item subtyping between interfaces and classes,
\item method call typechecking,
\item no circular implementation of interfaces,
\item type signature of methods from interfaces can be refined following the well known variant-contravariant rules,
\item only interfaces can be implemented.
\item \marco{I'm sure I'm missing something}
\end{itemize}
While classes are typed assuming \lstinline{this} is of the nominal type of the
class, trait declarations, do not introduce any nominal type.  \lstinline{this}
in a trait is typed with a special type \lstinline{This} that is visible only
inside such trait. Syntactically, \Q@This@ is just a special, reserved, class name $\mC$.
A Literal can use the \lstinline{This} type,
and when flattening completes creating a class definition, \Q@This@ will be replaced with such class name.

For the sake of simplicity, method bodies are just simple expressions
\me: they can be just variables and method calls.

\begin{comment}
\subsection{Code Use and Code Reuse in \name}

The great absent in our language is the  \Q@extends@ keyword.
In a sense, all classes are final (as in Java).
Moreover  \use\ cannot contain class names.
That is, using a trait is the only way to induce code reuse.

Lets now try to encode the example in Section~\ref{} e in our language:

\begin{lstlisting}
 A:{ method int ma() Utils.m(this) }//note, no {return _}
 Utils:{ class method int m(A a)/*method body here*/ }
\end{lstlisting} 

This code is fine, but there is no way to add a class \Q@B@ reusing
the code of \Q@A@, since
\Q@A@ is designed for code \emph{use} and not \emph{reuse}. Lets try
again, but this time aiming at code reuse:

\begin{lstlisting}
 ta:{ method int ma() Utils.m(this) }//type error
 A:Use ta
 Utils:{ class method int m(A a)/*method body here*/ }
\end{lstlisting}

This does not work, because \lstinline{Utils.m} requires an \lstinline{A} and \lstinline{this} in
\Q@ta@ have no knowlegde of \lstinline{A}. Lets try again

\begin{lstlisting}
 IA:{interface method int ma()}//interface with abstract method
 ta:{implements IA
   method int ma() Utils.m(this) }
 A:Use ta
 Utils:{ class method int m(IA a)/*method body here*/}
\end{lstlisting}

This code works: \Q@Utils@ is using an interface \Q@IA@ and the trait \Q@ta@
is implementing it. It is also possible to add a \Q@B@ as follows
\begin{lstlisting}
  B:Use ta, { method int mb(){return this.ma();} }
\end{lstlisting}
This also works.  \Q@B@ reuses the code of \Q@ta@, but has no knowledge of \Q@A@.
Since \Q@B@ reuses \Q@ta@, and \Q@ta@ implements \Q@IA@, also \Q@B@ implements \Q@IA@. 

\paragraph*{Semantic of Use}
Abeit alternative semantic models for traits~\cite{} have been proposed,
here we use the flattening model.
This means that 
\begin{lstlisting}
A:Use ta
B:Use ta, { method int mb(){return this.ma();} }
\end{lstlisting}
 
  reduces/is equivalent to/is flatted into
  
 \begin{lstlisting}
A:{implements IA method int ma() Utils.m(this) }
B:{implements IA
  method int ma() Utils.m(this)
  method int mb() this.ma() } 

 \end{lstlisting}
 
This code seems correct, and there is no mention of the trait \Q@ta@. In some sense, all the information about code reuse/subclassing is just a private implementation detail of \Q@A@ and \Q@B@; while subtyping is part of the class interface.
\end{comment}


\subsection{Remarks on Typing}
 Our typing discipline is 
what distinguishes our approach from a simple minded code composition macro~\cite{bawden1999quasiquotation}
or a rigid module composition~\cite{ancona2002calculus}.

The are two core ideas:
\paragraph*{1: traits are \emph{well-typed} before being reused.}${}_{}$\\*
 For example in

\begin{lstlisting}
t:{method int m() 2 
   method int n() this.m()+1}
\end{lstlisting}

\noindent \Q@t@ is well typed since \Q@m()@ is declared inside of \Q@t@, while

\begin{lstlisting}
t1:{method int n() this.m()+1} 
\end{lstlisting}
\noindent would be ill typed.

\paragraph*{2: code literals are not required to be \emph{well-typed} before flattening.}${}_{}$\\*
In class expressions  $\use\ \overline\mV$
an \mL\ in $\overline\mV$ is not typechecked before flattening, and only the result is expected to be well-typed.
While this seems a very dangerous approach at first, consider that also Java have the same behaviour:
for example in
\begin{lstlisting}[language=Java]
  class A{ int m() {return 2;}  int n(){return this.m()+1;} }
  class B extends A{ int mb(){return this.ma();} }
\end{lstlisting}

\noindent in \Q@B@ we can call \lstinline{this.ma()} even if in the curly braces there is no declaration for \Q@ma()@.
In our example, using the trait \Q@t@ of before

\begin{lstlisting}
C: Use t {method int k() this.n()+this.m()}
\end{lstlisting}
\noindent would be correct: even if n,m are not defined inside
\Q@{method int k() this.n()+this.m()}@,
the result of the flattening is well typed.

This is not the case in many similar works in literature~\cite{} where the
literals have to be self complete. In this case we would have been forced to
declare abstract methods \Q@n@ and \Q@m@.

Our typing strategy has two important properties:
\begin{itemize}
\item if a class is declared by using $\mC : \use\ \overline\mt$, that is, without literals,
and the flattening is successfull, \mC\ is well typed, no need of further checking.
\item on the other side, if a class is declared by $\mC : \use\ \overline\mV$, with
$\mL_1\ldots\mL_n \in \overline\mV$, and after successfull flattening $\mC : \mL$ can not be typechecked,
then the issue was originally present in one of $\mL_1\ldots\mL_n$.
It may be that the result is intrinsically ill-typed, if one of the methods in $\mL_1\ldots\mL_n$ is not well typed,
but it may also happen that a type referred from one of those methods is declared \emph{after} the current class. As we will see later, this is how our relaxation allows to support (indirectly) recursive types.

This also means that as an optimization strategy
 we may remember what method bodies come from traits and what method bodies come from code literals, in order to typecheck only the latter.
 \end{itemize}

 \subsection{Recursive types}

OO language leverage on recursive types most of the times.
For example in a pure OO language, \Q@String@ may offers a \Q@Int size()@
method, and \Q@Int@ may offer a \Q@String toString()@ method.

This means that is not possible to type in (full) isolation classes
\Q@String@ and \Q@Int@.

The most expressive compilation process may divide the classes in groups of mutually 
dependent classes.
Each group may also depend from a number of other groups.
This would form a Direct Aciclyc Graph of groups.
To type a group, we first need to type all depended groups, then
we can extract the structure/signature/structural type of all
the classes of the group.
Now, with the information of the depended groups and the one extracted
from the current group, it is possible to typecheck the implementation
 of each class in the group.

In this model, it is reasonable to assume that flattening happens group by group, before
extracting the class signatures.

Here we go for a much simpler simple top down execution/interpretation for flattening, where flattening
happen one at the time, and classes are typechecked where their type is first needed.
That is, In our approach typing and flattening interleaves. We assume our compilation process to stop as soon as 
an error arise. There are two main kinds of errors: Type errors (like method not found) or Composition errors (like summing two conflicting implementation for the same method).

For example
\begin{lstlisting}
A:{method int ma(B b) b.mb()+1}
tb:{method int mb() 2}
tc:{method int mc(A a,B b) a.ma(b)}
B: Use tb
C: Use tc, {method int hello() 1}
\end{lstlisting}
In this scenario, since we go top down, we first need to generate \Q@B@.
To generate \Q@B@, we need to use \Q@tb@;
In order to modularly ensure well typedness,
we require \Q@tb@ to be well typed at this stage. If \Q@tb@ was not well typed
a type error could be generated at this stage.
In this moment, \Q@A@ can not be compiled/checked alone,
we need informations about \Q@B@, but \Q@A@ is not used in \Q@tb@,
thus we do not need to type \Q@A@ and we can type \Q@tb@ with
 the available informations and proceed to generate \Q@B@.
Now, we need to generate \Q@C@, and we need to ensure well typedness of \Q@tc@.
Now \Q@B@ is alreay well typed (since generated by \use \Q@tb@, with no \mL),
and \Q@A@ can be typed;  finally \Q@tc@ can be typed and used.
If \use\ could not be performed (for example it \Q@tc@ had a method \Q@hello@ too)
a composition error could be generated at this stage.

On the opposite side, if \Q@B@ and \Q@C@ was swapped, as in
\begin{lstlisting}
C: Use tc, {method int hello() 1}  
B: Use tb
\end{lstlisting}

now the first task would be to generate \Q@C@, but 
to type \Q@tc@ we need to know the type of \Q@A@ and \Q@B@.
But they are both unavailable: \Q@B@ is still not computed and 
\Q@A@ can not be compiled/checked alone, without information about \Q@B@.
A type error would be generated, on the line of ``flattening of \Q@C@
requires \Q@tc@, \Q@tc@ requires \Q@A@,\Q@B@, but \Q@B@ is still in need of flattening".

In this example, a more expressive compilation/precompilation process 
could compute a dependency graph and, if possible, reorganize the list,
but for simplicity lets consider to always provide the declarations
in the right order, if one exists.

\paragraph*{Criticism: existence of an order is restrictive.}${}_{}$\\*
Some may find the requirement of the existence of an order restrictive;
An example of a ``morally correct" program where no right order exists is the following:
\begin{lstlisting}
t:{ int mt(A a) a.ma()}
A:Use t {int ma() 1}
\end{lstlisting}

In a system without inference for method types,
if the result of composition operators depends only on the
structural shape of their input (as for \use)
is indeed possible to optimistically compute the resulting structural shape of the classes
and use this information to type involved examples like the former.
We stick to our simple approach, since we believe such typing discipline would be fragile,
and could make human understanding the code-reuse process much harder/involved.
Indeed we just wrote an involved program where the correctness of trait \Q@t@ depends of 
\Q@A@, that is in turn generated using trait \Q@t@.

\paragraph*{Criticism: it would be better to typecheck before flattening.}${}_{}$\\*
In the world of strongly typed languages we could be tempted to
first check that all can go well, and then perform the flattening.
This would however be overcompicated for no observable difference:
Indeed, in the \Q@A,B,C@ example above there is no difference
between
\begin{itemize}
\item  (1)First check \Q@B@ and produce \Q@B@ code (that also contains \Q@B@ structural shape),
  (2) then use \Q@B@ shape to check \Q@C@ and produce \Q@C@ code;\ 
or a more involved
\item  (1)First check \Q@B@ and discover just \Q@B@ structural shape as result of the checking,
  (2)then use \Q@B@ shape to check \Q@C@.
  (3) Finally produce both \Q@B@ and \Q@C@ code.
\end{itemize}

Note that we can reuse code only by naming traits; but our only point of relaxation is the class literal: there is no way an error can ``move around'' and be duplicated during the compilation process.
In particular, our approach allows for safe libraries of traits and classes to be fully typechecked, deployed and reused by multiple clients: no type error will emerge from library code.
On the other side, we do not enforce the programmer to write always self-contained code where all the abstract method definition are explicitly declared.

\begin{comment}
\section{Managing State}
This idea of summing pieces of code is very elegant,
and has proven very successfull in module composition languages~\cite{ancona2002calculus} but our research community is struggling to
make it work with object state (constructor and fields) while acheving the following goals:

\begin{itemize}
\item keep sum associative and commutative,
\item allowing a class to create instances of itself,
\item actually initialize objects, leaving no null fields,
\item managing fields in a way that borrows the elengance of summing methods,
\item make easy to add new fields.
\end{itemize}

In the related work we will show some alternative ways to handle state.
However the purest solution just requires methods:
  The idea is that
  the trait code just uses getter/setters/factories, while leaving
  to classes the role to finally define the fields/constructors.
  That is, the  the class has syntax richer that the trait one,
  allowing declaration for fields and constructors.
  This approach is very powefull~\cite{wang2016classless}
 
\paragraph*{Advantages:} 
this approach is associative and commutative, even self construction
  can be allowed if the trait requires a static/class method
  returning This; the class will then implement this method by forwarding
  a call to the constructor.
  
\paragraph*{Negatives:} writing the class code with the constructors and
   fields and getter/setters and factories can be quite tedious.
   Moreover, there is no way for a trait to specify a default value for a field,
   the class need to handle all the state, even state that is conceptually
   "private" of such trait.

Here in the following an exaple of such approach:

\begin{lstlisting}
  pointSum: { method int x()  method int y()
    class method This of(int x,int y)
    method This sum(This other)
      This.of(this.x+other.x,this.y+other.y)
    }
  pointMul: { method int x() method int y()
    class method This of(int x,int y)
    method This mul(This other)
      This.of(this.x*other.x,this.y*other.y)
    }
\end{lstlisting}

As you see, all the state operations are represented as abstract methods.

\subsection{A First Attempt at Composition}
According to the general ideas expressed before,
\begin{lstlisting}
  Point:Use pointSum,pointMul
\end{lstlisting}  

\noindent would fail since methods \Q@x@,\Q@y@ and \Q@of@ are still abstract.
In this mindset, the user would be required to write something similar to

\begin{lstlisting}
  CPoint:Use pointSum,pointMul, {//not our suggested solution
    int x   int y
    method int x()x   method int y()y
    class method This of(int x, int y)
      new Point(x,y)
    constructor Point(int x, int y){ this.x=x   this.y=y }
    }
\end{lstlisting}

\noindent after a while programming in this style, 
writing those obvious ``close the state'' classes become a
repetivie boring job, and one wonder
if it could be possible to automatically generate such code~\cite{wang2016classless}.
Indeed those classes are just a form of ``fixpoint''.

In our model we go one step further: there is no need to generate code, or to explicitly
write down constructors and fields; there is not even syntax for those
constructor.
The idea is that any class that ``could" be completed in the obvious way
\emph{is a complete ``coherent" class}.
In most other languages, a class is abstract if have abstract methods.
Instead, we call abstract a class whose set of abstract methods is not
coherent, that is, can not be automatically recognized as factory,getters and setters.
  
\paragraph*{Detaild definition of coherent:}
\begin{itemize}
\item a class with no abstract method is coherent, and like Java \Q@Math@.
Will just be usefull for calling class/static methods.
\item a class with a single abstract \Q@class@ method returning \Q@This@
is coherent if all the other abstract methods can be seen as \emph{abstract state
operations} over one of its argument.
For example,
if there is a \Q@class method This of(int x, int y)@ as before,
then
\item a method \Q@int x()@ is intepreted as an abstract state method: a getter for \Q@x@.
\item a method \Q@Void x(int that)@ is a setter for x.
\end{itemize}

While getters and setters are fundamental operations, we can immagine
more operations to be supported; for example
\begin{itemize}
\item \Q@method This withX(int that)@
may be a ``wither", doing a functional field update.
\item \Q@method Void update(int x,int y)@
may do two field update at a time.
\item\Q@method This clone()@ may do a shallow clone of the object.
\end{itemize}

We are not sure what is the best set of abstract state operations yet, but we think this could become a very interesting area of research.
  
  
  lets play with the points of before, to see what good can we do with the current
  instruments:

\begin{lstlisting}
 //same code as before for pointSum and pointMul
  pointSum: { method int x() method int y()
    class method This of(int x,int y)
    method This sum(This other)
      This.of(this.x+other.x,this.y+other.y)
    }
  pointMul: { method int x() method int y()//look we are repeating
    class method This of(int x,int y)//the abstract method declarations.
    method This mul(This other)
      This.of(this.x*other.x,this.y*other.y)
    }
  PointAlgebra:Sum pointSum,pointMul
\end{lstlisting}  

  As you can see, we can declare the methods independently and compose the result
  as we wish. However we have to repeat the abstract methods \Q@x@,\Q@y@ and \Q@of@.
  In addition of \Q@Sum@ and \Q@Mul@ we may want many operations over points; can we improve our reuse
  and not repeat such abstract definitions? of course!

\begin{lstlisting}
  p: { method int x() method int y()
    class method This of(int x,int y)
    }
  pointSum:Use p, { method This sum(This other)
      This.of(this.x+other.x,this.y+other.y)
    }
  pointMul:Use p, { method This mul(This other)
      This.of(this.x*other.x,this.y*other.y)
    }
  pointDiv: ...
  PointAlgebra:Use pointSum,pointMul,pointDiv,...
\end{lstlisting}
      
Now our code is fully modularized, and each trait handle exactly one method.

What happens if we want to add fields instead of just operations?

\begin{lstlisting}
  colored:{ method Color color() }
  Point:Sum pointSum,colored //fails
\end{lstlisting}

This first attempt does not work: the abstract color method
is not a getter for any of the parameters of 
\Q@ class method This of(int x,int y)@
A solution is to provide a richer factory:

\begin{lstlisting}
  CPoint:Use pointSum,colored,{
    class method This of(int x,int y) This.of(x,y,Color.of(/*red*/))
    class method This of(int x, int y,Color color)
    }
\end{lstlisting}

where we assume to support overloading on different parameter number.
This is a good solution, we think is better that any alternatives in literature,
however the method \Q@CPoint.sum@ resets the color to red.
What should be the behaviour in this case?
If we support withers, instead of writing \Q@This.of()@ we can use
\Q@this.withX(newX).withY(newY)@ in order to preserve the color from \Q@this@.
Sadly, if we use this design inside of \Q@sum(This other)@ we would loose the color from \Q@other@.

If the point designer could predict this kind of extension, then we could use the following design:  
\begin{lstlisting}
  p: { method int x() method int y()
    method This withX(int that)
    method This withY(int that)
    static method This of(int x,int y)
    method This merge(This other)
    }
  pointSum:Use p, { method This sum(This other)
      this.merge(other).withX(this.x+other.x).withY(this.y+other.y)
    }
  colored:{method Color color()
    method This withColor(Color that)
    method This merge(This other)
      this.withColor(this.color().mix(that.color())
    }
  CPoint:/*as before*/
\end{lstlisting}  
  Now we can merge colors, or any other kind of state we may want to add
  following this pattern.
  In order to compose, let say \Q@colored@ with \Q@flavored@ we would
  need to compose the merge operation inside of both of them.
  The simple model we are presenting could accomodate this with an
  extension allowing code literals inside of a \use\ expression to use some form of super call to compose conflicting implementations. This is similar to the \emph{override} operation present in the original trait model~\cite{ducasse2006traits}.
\end{comment}
  
\section{Extensions to our model}
  One of the main feature of our simple reuse/use model is that it can be
  easly extended. One simple but amazingly expressive extension is nested classes

\subsection{Nested classes}

A nested class will be another kind of member in the Literal, so  
the grammar can be updated as following:

\begin{bnf}
\prodFull\mMD{
\Opt{\terminalCode{class}}\ \terminalCode{method}\ \mT\ \mm\oR\overline{\mT\,\mx}\cR \Opt\me
\mid \mCD
}{Member Decl}\\
\prodFull\mT{\mC\mid\mC\terminalCode{.}\mT}{types are now paths}
\end{bnf}\\

The general idea is that by composing code with \use, nested classes with the same name are recursivelly composed.
Note that while we have nested classes, we do not have nested traits: all traits are still
at top level.
Untypable/unresolved Traits are also the only``dependency"
the type system keeps track of, this means that when a nested class at an arbirary
nested level is flattend, as in
\Q@C:{ D:{ E:Use t1,t2,L}}@
t1 and t2 must be defined before C at top level; and they may require classes (and their
nested) defined before C. This means that the type system can still consider
the class table as a simple map from Types T to their definition.

This extension lets us challenge the expression problem[]:
in the expression problem we have data variants and operations and....

Let see how to easily encode and solve the expression problem:

\begin{lstlisting}
exp:{Exp:{interface}}//Exp declared once, reused everywhere
lit:Use exp,{ Lit:{implements Exp //Exp not explicitly declared
    class method Lit of(int inner) //Lit abstract state
    method int inner()}
  }
sum:Use  exp,{ Sum:{implements Exp 
    class method This of(Exp left, Exp right)
    method Exp left() //Sum abstract state
    method Exp right()}
  }  
uminus:Use exp,{ UMinus:{implements Exp 
    class method This of(exp inner)//and so on for
    method Exp inner()}//all the needed datavariants
  }   
  
expToS:{Exp:{interface method String toString()}}
//concept of toString declared once

sumToS:Use sum,expToS,{ Sum:{implements Exp//with Exp.toString
    method String toString()//just the implementation of the
      left.toString()+"+"+right.toString()//specific method
  }
uminusToS:...//implement toString for all the datavariants

expEval:{Exp:{interface method int eval()}}
//declare the next operation and implement it for all the datavariant
\end{lstlisting}

Now that you have nicely modularized the code, just compose all the traits you need.
\Q@MySolution:Use sumToS,litToS@ //sum,lit and exp traits are alread included

The expression problem presented up to now is the traditional challenge proposed by [];
this has been criticized to not really address the fundamental issues since it does not handle ....
Now we show how we can go behond the traditional expression problem by encoding transformer methods:
For example, lets add 1 to all literals
\begin{lstlisting}
expAdd1:{Exp:{interface method Exp add1()}}
sumAdd1:Use sum,expAdd1,{Sum:{implements Exp
    method Exp add1()
      Sum.of(left.add1,right.add1())
  }
litAdd1:Use lit,expAdd1,{Lit:{implements Exp
    method Exp add1()
      Lit.of(inner()+1);
    }

MySolutionAdd1:Use sumToS,litToS,sumAdd1,litAdd1
\end{lstlisting}

This nicely solve our problem. 
However, notice how if we wished to add many similar operations we would 
have to repeat the propagation code (as in \Q@sumAdd1@) many times
just changing the name of the operation.
In the next section we will show how to improve on this point.


\section{More composition operators}
\use is amazing, elegant and simple, but the system can be easily enriched with more 
operators: while most approaches in literator present a fixed set of operators, in our
system we do not need such restriction.
We just need to be sure that every newly added operator respects the following criteria:
if starts form well typed code, it produces well typed code.\marco{not enough? rephrease to respect the two quality at the start?}


---------
Restrict make abstract method

--------
Rename, like refactoring
Rename can invoke sum inside
---------
Redirect, like generics
Redirect can be multiple.

With redirect+rename we can have the general operator propagator
 
operation:{//for sum and lit, easy to extends as before
  T:{}
  Exp:{interface method Exp op(T x)}}
  Sum:Sum sum,{ extends Exp sum,expAdd1,{
    method Exp op(T x)
      Sum.of(left.op(x),right.op(x));
    }
  Lit:Sum lit,{
    method Exp op(T x)
      this;
  }
Now, to have my addN I can
opAddn: Sum
  operation[redirect T to Int]
    [rename Exp.op(x) to addN(x)][restrict Lit.op(x)], {
  Lit:{method Exp addN(Int x) Lit.of(inner())+x}
  }
  
Full power of redirect

serviceCombinator:{ S:{implements Service
  method R report()
  }
  
  R:{method R combine(R that)
    class method R empty()}
  
  ListS:list[redirect Elem to S]
  
  class R method(ListS ss){//here we use extended java like syntax
    R r=R.empty()
    for(S s in ss){
      s.performService();
      r=r.combine(s.report())
      }
    return r;
  }
}
PaintingService:serviceCombinator[redirect S to Painting.Service]

To encode the former generic code in java you need to write
the following headeche inducing interfaces for RService and Report (and still we miss the empty method)
and require that the services you want to serve implement those.
  interface Service{ void performService();}
  interface Report<R extends Report<R>>{R combine(R that);}
  interface RService<R extends Report<R>> extends Service{ R report();}


\section{Related Work}\label{sec:related}
In this section we discuss related work and comparison to Classless Java.




%\yanlin{need to discuss: whether the comments for FTJ is fair.}

%No, was not fair, I know that work, it does modular type checking, and 
%more work after that they extend with more safe/modular typing things.
%also, Java8 IS a language extension on Java
%However, language extensions (including FTJ)
%have a natural drawback: the programmer has to learn new syntax. In contrast,
%our approach is completely compatible with the current Java language, so that
%programmers don't need to pay any learning cost to adapt to this new classless
%programming pattern. Another drawback which is particular for FTJ is that FTJ
%doesn't have type for traits, hence the correctness check of trait is done when
%type-checking classes. This choice makes the design of FTJ simpler but lost
%typechecking efficiency (one trait will be potentially checked multiple times if
%it is used in multiple classes).

\subsection{Multiple Inheritance in Object Oriented Languages}
Many authors have argued in favour or against multiple inheritance.  It provides
expressive power, but it is difficult to model and implement, and can create
programs that are hard to reason about.  These difficulties include the famous
diamond (fork-join) problem~\cite{bracha90mixin,Sak89dis}, conflicting methods, etc.
%and the yo-yo problem~\cite{taenzer1989problems}.  
To conciliate the need for
expressive power and the need for simplicity, lots of models have been proposed
in the past few years, including mixins~\cite{bracha90mixin} and
traits~\cite{scharli03traits}.  They provide novel programming architecture
models in the OO paradigm.

\begin{enumerate}
\item Mixins allows to name components that can be applied to various classes as
  reusable functionality units. However, they suffer from linearisation: the
  order of mixin application is relevant in often subtitle and undesired
  ways. constraints. This hinders their usability and their ability of resolving
  conflicts: the linearisation (total ordering) of mixin inheritance cannot
  provide a satisfactory resolution in all cases and restricts the flexibility
  of mixin composition.  To fight those limitations, an algebra of mixin
  operators is introduced~\cite{ancona2002calculus}, but this raised the
  complexity of the approach, especially when constructors and fields are
  considered~\cite{marco09FJigsaw}.  Our approach does not have the
  linearisation problem, in that the semantics of Java \textbf{extends} clause
  is unordered and symmetric.
\item Simplifying the mixin algebras approach, traits draw a strong line between
  units of reuse (traits) and object factories (classes) In this model,
  traits~\cite{scharli03traits} units of reusable code, containing only methods
  as reusable functionalities. Thus, no state/state initialization is
  considered.

  Classes act as object factories, requiring functionalities from multiple
  traits.  Traits offers a trait algebra with operations like sum, alias and
  exclusion, provided for explicit conflict resolution.

  Concluding, (pure) traits do not allow state and they do not offers any reuse
  instrument to ensure that state is coherently initialized when finally defined
  in classes.  Traits can't be instantiated. This model requires two concepts
  (traits and classes) to coexist and cooperate.

  Some authors see this as good language design fostering good software
  development by helping programmers to think about the structure of their
  programs.  However, other authors see the need of two concepts and the absence
  of state as drawbacks of this model. Our approach takes interfaces as units of
  reuse, and meanwhile generates factory method for instantiation.

\item C++ and Scala also try to provide solutions to multiple inheritance, but
  both suffer from object initialization problems. Virtual inheritance in C++
  provides another solution to multiple inheritance (especially the diamond
  problem by keeping only one copy of the base class)~\cite{ellis1990annotated},
  however suffers from object initialization problem as pointed out by Malayeri
  et al.~\cite{malayeri2009cz}. It bypasses all constructor calls to virtual
  superclasses, which would potentially cause serious semantic error. Scala
  solution (very similar to linearised mixins, but misleadingly called traits in
  the language) avoids this problem by disallowing constructor parameters,
  causing no ambiguity in cases such as diamond problem.  This approach has
  limited expressiveness, and suffers from all the problems of linearised mixin
  composition.
  Python also offers multiple inheritance via linearised mixins. Indeed in python any class is implicitly a mixin, and mixin composition informally expressed as\\*
  \Q@ class A use B,C {...new methods...}@\\*
  can be expressed in python as \\*
  \Q@ class Aux: ...new methods...@\\*
  \Q@ class A(B,C,Aux): pass@
  Our approach not only does not involve linearised mixin problem, but also
  support proper constructor mechanism.
\end{enumerate}

\subsection{Multiple Inheritance in Java}
Since Java 8 default methods are introduced, concrete method implementation are
allowed to be defined (via the \textbf{default} keyword) inside
interfaces. 
% Since Java supports implementation of multiple interfaces (instead
% of extension of a single class), 
The introduction of default methods opens the
gate for various flavours of multiple inheritance in Java, using interfaces.
Former work by Bono.et. al.~\cite{bono14}. provides details on mimicking traits
through interfaces.

There are proposals for extending Java (before Java8) with traits. For example,
FeatherTrait Java (FTJ)~\cite{Liquori08ftj} by Liquori et al. extends the
calculus of Featherweight Java (FJ)~\cite{Igarashi01FJ} with statically-typed
traits, adding trait-based inheritance in Java.  Except for few, mostly
syntactic details, their work can be emulated/rephrased with Java8 interface.


\section{Comparing to traits and mixins}

\begin{comment}
Haoyuan

   - vs both: we do automatic return type refinement, which has useful applications
   (example: Expression Problem)

   - vs traits: we support of methods to create new objects (a replacement to constructors);
   Moreover we have the with and clone methods (we miss more applications for those). Show
   how to model the operations on traits; discuss operations that we cannot model
   (example: renaming).

   - vs mixins: we use the trait model of explicitly resolving conflicts. This is arguably
   better for reasoning.
\end{comment}

Our approach is based on code generation by Java annotations. The model we generate encourages composability and reusability in object-oriented programming, and is considered to be an alternative to traits or mixins,  meanwhile achieving better performance in some situations. Hence it is necessary for us to make a comparison between this approach and traits (or mixins) we commonly used before.

Our approach is quite different from mixins, in the sense that we use the trait model of explicitly resolving conflicts. Just as [Scharli2003] demonstrated in the paper, mixin inheritance is a good approach of achieving code reuse, nevertheless, the mixin model is not so expressive to resolve conflicts from many mixins. In the trait model, aliases and exclusions are provided for explicit conflict resolution. Such operations can actually be modelled in the mechanism of our approach.

Here we present how the original operations on traits are supported by our model.
\begin{itemize}
\item \textbf{Symmetric sum}: the symmetric composition of two disjoint traits is achieved by simply implementing two interfaces in Java correspondingly, without overriding any method. The composition relies on multiple inheritance on interfaces, which is supported by Java.
\item \textbf{Override}: the overriding operation (also known as asymmetric sum) is modelled by implementing many interfaces, while overriding some methods inside. The code below gives an example of explicitly specifying which super interface to refer to, regarding two methods with the same name.
    \begin{lstlisting}
    interface A { default int m() {return 1;} }
    interface B { default int m() {return 2;} }
    interface C extends A, B { default int m() {return B.super.m();} } 
    \end{lstlisting}
    Here the method \texttt{m()} in interface \texttt{C} simply inherits from \texttt{B.m()}.
\item \textbf{Alias}: an alias operation adds a new name to an old method when creating the new trait. In Java, we just create a new method with reference to the existing method in its super interface. See the example below, where the new method \texttt{k()} is an alias of the existing method \texttt{m()}.
    \begin{lstlisting}
    interface A { default int m() {return 1;} }
    interface B extends A { default int k() {return A.super.m();} }
    \end{lstlisting}
\item \textbf{Exclusion}: exclusion is also supported in Java, where method declarations can hide the default methods correspondingly in the super interfaces. See the example below.
    \begin{lstlisting}
    interface A { default int m() {return 1;} }
    interface B extends A { int m(); }
    \end{lstlisting}
\end{itemize}

Besides, we support \texttt{of} methods in our model, as a replacement to the constructors in original traits. Furthermore, we also support \texttt{with} and \texttt{clone} methods as auxiliary constructors, making the creation of instances more flexible and convenient. Conversely, there are certain operations we cannot model, such as method renaming (as in [Reppy2006]), which breaks structural subtyping.

A further feature leads to return type refinement in our model. Generally speaking, when we use inheritance to create a new trait, with the return type of an existing method being refined, the new \texttt{of} method keeps this consistency. This feature is very useful in many applications; we will see how it makes a difference in our Expression Problem example, in Section [Case Study]. 

\subsection{ThisType/MyType/Extensibility}
%\marco{ this section is unfair. There are unsolved problems with ThisType in negative positions, namelly it breaks subtyping. The most common solution is to allow calling some methods only when the exact type is known. This demolish most advantages of interfaces.}

In certain situations, our approach allows automatic refinement for return types. This is part of a bigger topic in class based languages: expressing and preserving type recursion and (nominal/structural) subtyping at the same time. 

One famous attempt in this direction is provided by
\emph{MyType}~\cite{bruce1994paradigmatic}, representing the type of
\textbf{this}, changing its meaning along with inheritance.  However when
invoking a method with MyType in a parameter position, the exact type of the
receiver must be known.  This is huge limitation in class based object oriented
programming, and is exasperated by interface-only programming as we propose: no
type is ever exact since classes are not part of the language.  A recent
article~\cite{Saito2013933} lights up this topic, since they propose two
features: exact statements \yanlin{do we support exact statements?}\marco{no we do not. When is that it may seams we do?} and
non-inheritable methods; both related to our work: any method generated inside
of the \Q@of@ method is indeed non-inheritable, since there is no class name to
extends from, and exact statements (a form of wild-card capture on the exact
run-time type) could capture the ``exact type'' of an object even in a
class-less environment.



% The addition of MyType to a language will allow easy definition of
%binary methods, methods with recursive types (i.e., the same type of the
%receiver appears in the argument or return positions of methods), etc. MyType
%greatly enhances the expressiveness and extensibility of object-oriented
%programming languages. 

%In our approach, we are using covariant-return types to simulate some uses of
%MyType. But our approach only works on method positive positions, whereas MyType
%is more general, as it works on any positions. Nevertheless our approach is
%still useful for modeling fluent interfaces and solving expression
%problems,etc. 

\begin{comment}
\subsection{Type-Directed Translations/Syntactic Sugar}
\marco{I'm tring to merge this and the next one}
Language extensions are often implemented as syntactic sugar of the base
language. For example, Scala compiler supports XML syntax in normal Scala code
directly (after Scala ?, users need to import \texttt{scala-xml} library
manually). However, this approach is hard in terms of implementation, because it
requires extending the compiler. Also, this approach does not support combining
multiple extensions into one.

SugarJ~\cite{erdweg11sugarj} is a Java-based extensible programming language
that allows programmers to extend the base language with custom language
features by definitions in meta-DSLs (SDF, Stratego, etc). 
\yanlin{Is new syntax really a ``drawback''? I think for some system, like
  SugarJ, one of purpose IS to introduce these new syntax.}  Drawbacks: new
syntax. To create the extension, programmers have to work with multiple
languages (SDF, stratego, etc) while our approach works totally in Java
environment.

We can model certain types of language extensions with annotations 
only, but those extensions do not introduce new syntax: they 
merely do automatic code generation. 
\end{comment}
\subsection{Meta-programming competes with Language extensions}
The most obvious solution to add features to a language is language extension.
They are often implemented as syntactic extensions that can be desugared to the base
language. For example, the Scala compiler was extended to directly supports XML syntax. However, this approach does not support combining multiple extensions into one. We are de facto creating a fork in the language, and rarely the new fork gain enough traction to become the main language release.
On this topic we mention SugarJ~\cite{erdweg11sugarj}; a Java-based extensible language allowing programmers to extend it with custom features by definitions in meta-DSLs (SDF, Stratego, etc). 

On the other side, when the starting language have a flexible enough syntax and a fast and powerful enough reflection, we may just need to play with operator overloading and other language tricks to discover that the language feature we need can be expressed as a simple library in our language. For example, consider SQL alchemy in python.

Java-like language tends to sits in the middle of this two extremes:
libraries can not influence the type system, so many solutions valid in python or other languages could not be applicable, or may be applicable at the cost of loosing safety.

Here (compile/load time) code generation come at the rescue: 
if for a certain feature (\mixin in our case) it is possible to use the original language syntax to
\emph{express-describe} any specific instantiation of such feature
(annotating a class and provide getters), then we can insert in the compilation process a tool that exam and enrich the code before compilation. No need to modify the original source; for example we can work on temporary files.
Java is a particular good candidate for this kind of manipulation since it already provide ways to define and integrate such tools in its own compilation process: in this way there is no need of temporary files, and there is a well defined way of putting multiple extensions together.

Other languages offers even stronger support to safe code manipulation:
Template Haskell~\cite{}, F\# (type providers)~\cite{} and MetaFjig (Active Libraries)~\cite{}
all allows to execute code at compile time and to generate code/classes that are transparently integrated in the program that is being generated/processed/compiled.
In particular, MetaFjig offers a property called \emph{meta-level-soundness}. In short this property ensures by construction that library code (even if wrong or non nonsensical) would never generate ill-typed code. This is roughly equivalent to what we state and manually proof in Lemma 2 for our particular transformation.
Since MetaFjig is not working on annotated classes, there is no direct equivalence on the overall theorem of safety we shown.

\subsection{Formalization of Java8}
We provide a simple and well designed formalization for a subset of Java including default/static interface methods and object initialization literals (often called anonymous local inner classes).
A similar formalization was drafted by\marco{ I use the term draft because I seams to remember it was just a technical report, I'm right?}
Goetz and Field~\cite{goetz12fdefenders} to formalize defender (default) methods
in Java. However this formalization is limited to model exactly one
method inside classes/interfaces.

As a evidence of the attention and care present in our formalization work, while double-checking the behaviour of Java in side cases we have discovered a likely bug in the current \texttt{javac}.\marco{refer to before when we explain the issue.}

\section{Future work}

\subsection{Private state}
The biggest limitation of our approach is the absence in Java8 of support for private/protected methods in interfaces.That is, in Java8 all members of interfaces are public, including static methods.
Since we use abstract methods to encode the state, our state is always all public; however is impossible for the user to know if a certain method maps directly to a field or if it have a default implementation.
If the use wants a constructor that does not directly maps to the fields, (as for secondary constructors in Scala) he can simply define its own \Q@of@ method and delegate on the generated one, as in
\begin{lstlisting}
@Mixin interface Point{
  int x(); int y();
  static Point of(int val){return Point.of(val,val);}  
  }
\end{lstlisting}
However, the generated \Q@of@ method would also be present and public.
If a future version of Java was to support \emph{static private methods in interfaces} we could extend our code generation to handle also encapsulation.
Currently, is possible to use a public nested class with private static methods inside, but this is ugly and cumbersome. We are considering if our annotation processor can take code with \Q!@Private! annotation and turn it into static private methods of a nested class. In this extension,  also the of method could be made private following the same pattern. 

\subsection{State initialization}

As discuss before, the user can trivially define its own \Q@of@ method, and initialize a portion of the state with default values.
However, the initialization code would not be reused/reusable, and subinterfaces would have to repeat such initialization code.
If a field has no setters, a simple alternative is to just define the ``field'' as a default method as in 
\begin{lstlisting}
@Mixin interface Box{ default int val(){return 0;} }
\end{lstlisting}
if setters are required, a possible extension of our code expansion could recognize a field if the getter is provided and the setter is required, and could generate the following code:
\begin{lstlisting}
interface Box{ 
  default int val(){return 0;} //provided
  void val(int _val);//provided
  static Box of(){return new Box(){//generated
    int val=Box.super.val();
    int val(){return val;}
    void val(int _val){val=_val;}
    };}}
\end{lstlisting}
We are unsure of the value of this solution: is very tricky, the user define a method that (contrary to our usual expectation) is actually overridden in a way that the behaviour change, but change only after the first setter is called, plus this code would cache the result instead of re-computing it every time. This can be very relevant and tricky in a non functional setting.

\subsection{Class invariants in ClassLess Java}
Since the objects are created by automatically generated methods,
another limitation of our current approach is that there is no place where the user can dynamically check for class invariants.
In Java often we see code like
\begin{lstlisting}
class Point{ int x; int y;
  Point(int x; int y){this.x=x;this.y=y; assert this.checkInvariant();}
  private boolean checkInvariant(){... x>0,y>0...}
\end{lstlisting} 

We are considering an extension of our annotation where 
default methods with the special name \Q@checkInvariant()@ will be called inside of the \Q@of@ methods.
if multiple interfaces are implemented, and more then one offers
\Q@checkInvariant()@,  a composed implementation could be automatically generated, composing by \Q@&&@ the various competing implementations.

\subsection{Clone, toString, equals and hashCode}
Methods originally defined in \Q@Object@, as \Q@clone@ and \Q@toString@ can be supported by our approach, but they need special care. If an interface annotated with \mixin ask an implementation for \Q@clone@, \Q@toString@, \Q@equals@ or \Q@hashCode@ we can easily generate one from the fields.\footnote{In particular, for clone we can do automatic return type refinement as we do for \Q@with-@ and fluent setters. Note how this would solve most of the Java ugliness related to \Q@clone@ methods.}

However, if the user wish to provide its own implementation, since the method is also implemented in \Q@Object@ we would have a conflict, that we have to explicitly resolve inside of \Q@of@, by implementing the method and delegating it to the user implementation, thus

\begin{lstlisting}
@Mixin interface Point{ int x(); int y();
  default Point clone(){ return Point.of(0,0);}//user defined clone
  }
\end{lstlisting} 
Would expand into 

\begin{lstlisting}
interface Point{ int x(); int y();
  default Point clone(){ return Point.of(0,0);}//user defined clone
  public static Point of(int _x,int_y){
    return new Point(){...
      public Point clone(){ return Point.super.clone();}
      };  }  }
\end{lstlisting} 

\section{Conclusion}\label{sec:conclusion}

Before Java 8, concrete methods and static methods where not allowed
to appear in interfaces.  Java 8 allows static interface methods and
introduces \emph{default methods}, which allow for implementation
insides interfaces. This had an important positive consequence that
was probably overlooked by the Java design team: the concept of class
(in java) is now redundant and unneeded.  We define a subset of Java,
called ClassLess Java, where programs and (reusable) libraries can be
easily defined and used.  To avoid for some syntactic boilerplate
caused by Java not being originally designed to work without classes,
we introduce a new annotation:\mixin provide default implementations
for various methods (e.g. getters, setters, with-methods) and a
mechanism to instantiate objects. \mixin annotation helps programmers
to write less cumbersome code while coding in ClassLess Java; indeed
we think the obtained gain is so high that ClassLess Java with \mixin
annotation can be less cumbersome than full Java.\bruno{May need rewriting}


\section*{Bibliography}
\bibliographystyle{elsart-num-sort}
\bibliography{main}

\end{document}
