\section{Improving Use by Preventing Unintended Subtyping}
%%Preventing Unintended Subtyping}
As stated in Section~\ref{} a possible benefit of a nominally typed OO language
that separates inheritance from subtyping is that it is possible to 
have reuse, while preventing unintended subtyping. The design of \name 
allows this, thus it \emph{improves the use} of classes, while
retaining the benefits of reuse when compared to Java-like languages. 
Here we illustrate this benefit by modelling a simplified version of
Set and Bag collections. 

\bruno{Plan: Show simplified Java code. Show how the Java code forces
  us to make Bags a subtype of Set, if we want reuse. 
Then show the 42/language in this paper/ code and solution.  
}

An iconic example on why connecting inheritance/code reuse and
subtpying is problematic is provided by the
historic\cite{LaLonde:1991:SSS:110673.110679}: A reasonable
implementation for a \Q@Set@ may be easy to extend into a \Q@Bag@
keeping tracks of how many times an element occurs.  We would just add
some state and override a couple of methods.\bruno{Are we going to
  present this example in the paper solved in 42, for example? 
I think I would expect to see it.}

However, our subclassing would break Liskov substitution principle (LSP)~\cite{martin2000design}: not all bags are sets!
Of course, one could retroactivelly fix this problem by introducing \Q@AbstractSetOrBag@
and making both \Q@Bag@ and \Q@Set@ inherit from it.
This looks unnatural, since \Q@Set@ would extend it without adding anything,
and we would be surprised to find a use of the type \Q@AbstractSetOrBag@.
Worst, if we was to constantly apply this mentalty, we would introduce a very high number
of abstract classes that are not supposed to be used as types, and that will clutter the 
public interface of our classes and our code project as a whole.

\begin{lstlisting}
class Set {.../* Elem put() boolean isIn(Elem)*/ ...}
class Bag extends Set{.../*override something to keep track of duplications*/..}
\end{lstlisting}

but now, 

\begin{lstlisting}
Set mySet=new Bag(); //OK for the type system but not for LSP
\end{lstlisting}

as a pleasaruble accident, avoid such code gift us simple support for
This type and (in the extensions with nested classes seen later)
family polimporphism.

