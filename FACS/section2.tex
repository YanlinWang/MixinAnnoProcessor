\saveSpace\saveSpace\saveSpace
\section{The Design of \name: Separating Use and Reuse}\label{sec:separate}
\saveSpace\saveSpace\saveSpace
This section presents the overview of \name, and it illustrates the
key ideas of its design. In particular we illustrate how to separate code use and 
code reuse, and how \name solves the this-leaking problem.
% in \name. %, and how this allows improving both. 

%\subsection{The Design of \name: Decoupling Use from Reuse}
\saveSpace\saveSpace
\subsection{Classes in \name: A mechanism for code Use}
\saveSpace
%\name has a substantially different design from Java-like languages.
The concept of a class in \name provides a mechanism for code-use
only. This means that there is actually no subclassing:
classes are roughly equivalent to final classes in Java.  Thus,
compared to Java-like languages, the most noticeable difference is the
absence of the \Q@extends@ keyword in \name. 
Consider the example in Section~\ref{sec:intro}:
\saveSpace
\begin{lstlisting}
 A: { method int ma() Utils.m(this) }//note, no need for {return _}
 Utils:{ class method int m(A a)/*some method body here*/ }
\end{lstlisting} 
\saveSpace
\noindent Classes in \name use a slightly different declaration style compared
to Java: there is no \lstinline{class} keyword, and a colon separates the class name (which must always start with
an uppercase letter) and the class implementation, which is used to specify the
definitions of the class. In our example, in the class declaration
for \lstinline{A}, the name of the class is \lstinline{A} and the code 
literal associated with the class (\Q@{ method int ma()   Utils.m(this) }@) contains the definitions associated to the
class.

%We will see next some 
%important differences to Java-like languages in the way 
%classes and code-literals are type-checked, as we shall see next. 
%Nevertheless, for this example, things still work in a similar way to Java. 

The \name code above is fine, but there is no way to add a class 
\Q@B@ reusing the code of \Q@A@, since
\Q@A@ is designed for code \emph{use} and not \emph{reuse}. So, unlike
the Java code, introducing a subclass
\lstinline{B} is not possible. This may seem like a severe restriction, but
\name has a different mechanism for \emph{code-reuse} that 
is more appropriate when \emph{code-reuse} is intended. 

\subsection{Traits in \name: A mechanism of code Reuse}

Unlike classes, traits in \name cannot be instantiated and do not introduce new
types. However they provide code reuse.
%So, lets try again encoding the code for the leaking problem, but this
%time aiming at code reuse. 
Trait declarations 
look very much like class declarations, but trait names 
start with a lowercase letter. An obvious first attempt 
to model the example in Section~\ref{sec:intro} 
with traits and code reuse is:
\saveSpace
\begin{lstlisting}
 ta:{ method int ma() Utils.m(this) }//type error
 A:Use ta
 Utils:{ class method int m(A a)/*method body here*/ }
\end{lstlisting}
\saveSpace
\noindent Here \lstinline{ta} is a trait intended to replace the
original class \lstinline{A} so that the code of the method
\lstinline{ma} can be reused. Then the class \lstinline{A} 
is created by inheriting the code from the trait using the keyword 
{\bf Use}. Note that \use\ cannot contain class names: only trait
names are allowed.
That is, using a trait is the only way to induce code reuse.
Unfortunately, this code does not work, 
because \lstinline{Utils.m} requires an \lstinline{A} and the type of \lstinline{this} in
\Q@ta@ has no relationship to the type \lstinline{A}.
A Java programmer may then try to write:
\saveSpace
\begin{lstlisting}
 ta:{ method int ma() Utils.m(this) }//type error
 A:Use ta
 Utils:{ class method int m(ta a)//syntax error
   /*method body here*/ }
\end{lstlisting}
\saveSpace
But this does not work either: \Q@ta@ is not a type in the first place, since it is a (lowercase) trait name.
Indeed
since the trait name is not a type, no code external to that trait can
refer to it. This is one of the key design decisions in \name. 

% this text may go after, while explaining the This type?
Code inside traits/classes can refer to (uppercase) class names, but not to (lowercase) trait names.
The point is that the trait can refer to the program via \lstinline{this},
%??????????????
 This means that the
program is agnostic to what the trait type is going to be, so it can
be later assigned to any (or many) classes. This improves the
flexibility of reuse as illustrated in Section~\ref{}.
However, 
to solve the more immediate typing issue above we need one more round of refactoring, as we shall see next. 


in \name all declarations 
are sequentially type-checked and if a definition occurs afterwards
another, then it is not visible on all previous definitions.

\paragraph{Type of the self-reference} 
The code above does not work.
because the way type-checking works in \name is that type-checking of 
code literals is independent of the class/trait names associated to it.
A Java programmer may expect that the type of \lstinline{this} 
in the previous definition of \lstinline{ta} is \lstinline{ta}
itself. However, this intuition brought from Java is wrong in \name
for two reasons:

\begin{itemize}

\item {\bf Traits are not types}. Traits in
  \name are simply units of reuse and cannot be used as types. Types
  are only introduced by class/interface declarations.

\item The second reason is that {\bf the type of self-references is the self type of the
    code literal.} Unlike Java-like languages, where the body of a
  class declaration is intrinsically coupled with the class itself, in
  \name code literals are first-class\bruno{Is it appropriate to say
    that they are first class?} and are type-checked independently.
  \name has a notion of self-type, which is closely related to
  approaches such as ThisType~\cite{} and ...\cite{others}. Therefore
  in \name the self-reference does not have the type of the class
  being defined, but rather it has the self-type.
\end{itemize}

With this in mind, we can try to model the example again:
\saveSpace
\begin{lstlisting}
 IA:{interface method int ma()}//interface with abstract method
 Utils:{class method int m(IA a)/*method body here*/}
 ta:{implements IA //This line is the core of the solution
     method int ma() Utils.m(this) }
 A:Use ta
\end{lstlisting}
\saveSpace
This code works: \Q@Utils@ relies on interface \Q@IA@ and the trait \Q@ta@
implements \Q@IA@.
\Q@ta@ is well typed: independently of what class name(s) will be
associated to its code, we know that such class(es) will implement
\Q@IA@. 
Therefore, while typechecking \Q@Utils.m(this)@ we can assume
\Q@this<:IA@.
 It is also possible to add a \Q@B@ as follows:
\saveSpace
\begin{lstlisting}
  B:Use ta, { method int mb(){return this.ma();} }
\end{lstlisting}
\saveSpace
This also works.  \Q@B@ reuses the code of \Q@ta@, but has no knowledge of \Q@A@.
Since \Q@B@ reuses \Q@ta@, and \Q@ta@ implements \Q@IA@, also \Q@B@ implements \Q@IA@. 

Later, in Section \ref{sec:formal} we will provide more details on the type
system. 
For now notice that in the former example the code is correct even if
no method called \Q@ma@ is explicitly declared.
DeepFJig and TR would require instead to explicitly declare an abstract \Q@ma@:
\saveSpace
\begin{lstlisting}
  B:Use ta, { method int ma()//not required by us
      method int mb(){return this.ma();} }
\end{lstlisting}\saveSpace
\noindent
The idea in \name is that such method is imported from trait \Q@ta@, exactly as in the Java equivalent
\saveSpace\begin{lstlisting}[language=Java]
  class B extends A{ int mb(){return this.ma();} }  
\end{lstlisting}
\saveSpace
where method \Q@ma@ is imported from \Q@A@.
This concept is natural for a Java programmer, but was not supported
in previous work \cite{BETTINI2013521,deep}. Those works require all
dependencies in code literals to be explicitly declared, so that the
code literal is completely self-contained. However, this results in
many redundant abstract method declarations.

\paragraph{Semantic of Use}
Albeit alternative semantic models for traits~\cite{Traits:ECOOP2003} have been proposed,
here we use the flattening model. This means that 
\saveSpace\begin{lstlisting}
A:Use ta
B:Use ta, { method int mb(){return this.ma();} }
\end{lstlisting}\saveSpace
 
\noindent is equivalent to:
  
\saveSpace \begin{lstlisting}
A:{implements IA method int ma() Utils.m(this) }
B:{implements IA
  method int ma() Utils.m(this)
  method int mb() this.ma() } 
 \end{lstlisting}
\saveSpace 
 This code is correct, and {\bf in the resulting code there is no mention of the trait
 \Q@ta@}. In some sense, all the information about code-reuse/inheritance
  is just a private implementation detail of \Q@A@
 and \Q@B@; while subtyping is part of the class interface.


To finish this section, Figure \ref{fig:compare} provides a summary of
the differences between classes and traits. The comparison focus on
the roles of traits and classes with respect to instantiation,
reusability and whether the declarations also introduce new types or
not.




