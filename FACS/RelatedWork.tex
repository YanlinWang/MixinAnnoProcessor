\section{Related Work}
The literature on code reuse is too vast to let us do justice of it in a fiew pages,
Our work is clearly inspired on Traits~\cite{ducasse2006traits}; they are in turn insipired on module composition languages~\cite{ancona2002calculus}
and multiple inheritance~\cite{}.

Our idea of fully separate inheritance and subtyping has some precedence
In all the line of work of Ferruccio [cite many] neither traits nor classes introduce nominal types,
that are provided only by interfaces.
In DeepFjig[] classes introduce nominal types but in an innatural way, the type of \Q@this@ is only \Q@This@ (sometimes called \Q@<>@) and not the nominal type of its class.

Other work cite~\cite{odersky2008programming,nystrom2006j}
do no completly break the relation between inheritance and subtyping, but only prevent subtyping where 
it would be unsound.

Our compilation model is a clear step forward with respect to 
\cite[metaFjig, deepFjig,myThesis], where all code was required to declare all the abstract dependencies all of the time.

%\item clean and elegant handling of state in trat/module composition language
The idea of abstract state operation has emerged from CJ~\cite{wang2016classless}, and offer a clean solution to handling state
in a trait composition setting.
Note how abstract state operation are different from just hiding fields under getter and setters: 
in our model the programmer simply never declare what is the state of the class, not even what information is stored in fields.
The state is computed by the system as an overall result of the whole code composition process.

In literature there has been many attempts to add state in traits/module composition languages.
\begin{itemize}  
\item No constructor: all the fields start at null/a default specified value.
  In this model fields are like another kind of member, and two fields
  with identical type can be merged by sum; \Q@new C()@ can be used for all classes, and \Q@init@ method may be called later, as in
  \Q@Point p=new Point(); p.init(10,30)@.
  
  To its credit, this simple approach is commutative and associative and do not disrupt elegance of summing methods.
  However, objects are created "broken" and the user is trusted with fixing them.
  While is easy to add fields, the load of inizializing them is on the user; moreover
    all the objects are intrinsically mutable, so this model is unfriendly
    to a functional programming style.
\item Constructor compose fields:
In this approach the fields are declared but not initialized, and
a canonical constrctor constructor taking a value for each field and just initializing such field
is automatically generated in the resulting class.

In this approach it is easy to add fields, however this approach is associative but not commutative, since the order of composition will change the order of the fields; moreover there is no support for self construction
since the signature of the constuctor change during composition.

\item Constructor can be composed if they offers the same exact parameters:
In this approach traits declare field and constructors.
The constructor initialize the fields but can do any other computation.
Traits whose constructors have the same signature can be composed.
The composed constructor will execute both constructor bodies in order.

This approach is designed to allows Self construction.
It is also associative and mostly commutative: the order of composition is relevant only
in the order of execution of side effects during construction.
However the requirement that the constructors need the same parameters 
hamper reuse, and if a field is added, its initial value need to be
somehow syntetized from the old constructor parameters.

\end{itemize}

%\item extension with nested class is natural and powefull
We claim that our solution to the expression problem is the most natural in literature to date:
While a similar sintax can be achieved with the scandinavian style~\cite{ernst2004expression}, their dependent type system makes reasoning quite complex, and indeed more recent solutions have accepted a more involved syntax in order to have an understandable type system~\cite{igarashi2005lightweight}.

Our close contendant is DeepFJig~\cite{corradi2011deepfjig}: all our gain over their model is based on our relaxation over abstract signatures.
Note how we solve the expression problem in the most radical way possible \marco{bruno, here you can insert some of the criticism over expression problem trivially and explain how is not applicable to us here.}



