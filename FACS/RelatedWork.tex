
\saveSpace\saveSpace
\section{Related Work}
\saveSpace\saveSpace
Literature on code reuse is too vast to let us do justice of it in a few pages.
Our work is inspired by traits~\cite{ducasse2006traits}, which in turn
are inspired by module composition languages~\cite{ancona2002calculus}.
\saveSpace
\subsection{Separating Inheritance and Subtyping}
\saveSpace
We are aware of at least 3 independently designed research languages 
that address the this-leaking problem: Delta-Trait~(DT)\cite{Bettini:2010:ISP:1774088.1774530,BETTINI2013521,Bettini2015282}, Package Templates~(PT)\cite{KrogdahlMS09,DBLP:journals/taosd/AxelsenSKM12,DBLP:conf/gpce/AxelsenK12}, DeepFjig\cite{deep,servetto2014meta,fjig}.
Leveraging on \emph{traits}, in this work we aim to synthesize
the best ideas of those very different designs, while at the same time 
coming up with a simpler and improved design for separating
subclassing from subtyping, which also addresses various limitations of those
3 particular language designs.
The following compares 
various aspects of the language designs:

\begin{itemize}
\item {\bf A simple uniform syntax for code literals}
DeepFjig is best in this sense, since DT has separate syntax for class literals, trait literals and record literals.
PT on the other hand is built on top of full Java, thus has a very
involved syntax.
\name leverages on DeepFjig's approach but,
\emph{thanks to our novel representation of state}, \name also offers a much simpler and uniform syntax than
all other approaches: everyhing is just a method.
\item 
{\bf Reusable code cannot be ``used'', that is instantiated or used as a type.}
This happens in DT and in PT, but not in DeepFjig. To allow reusable code to be directly 
usable, in DeepFjig
classes introduce nominal types in an unnatural way: the type of
\Q@this@ is only \Q@This@ (sometimes called \Q@<>@) and not the
nominal type of its class. 
That is in DeepFjig 
\Q@A:{ method A m()this}@ is not well typed. This is because
\Q@B: Use A@ flattens to \Q@B:{ method A m()this}@, which is clearly not well typed.
Looking to this example is clear why we need reusable code to be agnostic of its name.
Then, either reusable code has no name (as in DT, PT and \name)
or all code is reusable and usable, and all code needs to be awkwardly agnostic of its name.

\item 
{\bf Requiring abstract signatures is a left over of module composition mindset.}
DT and Deepfjig comes from a tradition of functional module composition, where 
modules are typed in isolation under an environment, and then the composition is performed.
As we show in this work, this ends up requiring verbose repetition of abstract signatures,
which (for highly modularized code) may end up constituting most of the program.
Simple Java (and thus PT, since it is a Java extension) shows us a better way:
the meaning of names can be understood from the reuse context.
The typing strategy of PT offers the same advantages of our typing model, 
but is more involved and indirect. This may be caused by the
heavy task of integrating with full Java.
Recent versions of Delta-Traits are trying to address this issue too.
\item {\bf Composition algebra.}
The idea of using composition operators over atomic values as in an arithmetic expression is very powerful,
and makes it easy to extend languages with more operators. Deepfjig and DT embrace this idea, while PT takes the traditional Java/C++ approach of using enhanced class/package declaration syntax.
The typing strategy of PT also seems to be connected with this
decision, so it would be hard to move their approach in a composition
algebra setting.
\item {\bf Complete ontological separation between use and reuse}
While all 3 works allow separating inheritance and subtyping only DT properly enforces 
separation between use (classes and interfaces) and reuse (traits).
This is because in DeepFjig all classes are both units of use and reuse (however, subtyping is not induced).
PT imports all the complexity of Java, so although is possible to separate use and reuse, the model have powerful but non-obvious implications where (conventional Java) \Q@extends@ and PT are used together.
\item {\bf Naming your type, even if there is none yet.}
Both Deepfjig and PT allow a class to refer to its name, albeit this is
less obvious in PT since both a package and a class have to be introduced to express it.
This allows encoding binary methods, express patterns like withers or fluent setters and to instantiate instances of the (future) class(es)  using the reused code.

\end{itemize}

\subsection{State and traits}

The idea of abstract state operations emerged from Classless
Java~\cite{wang2016classless}. This approach offers a clean solution to handle state
in a trait composition setting.
Note how abstract state operations are different from just hiding fields under getter and setters: 
in our model the programmer simply never has to declare what is the state of the class, not even what information is stored in fields.
The state is computed by the system as an overall result of the whole code composition process.

In the literature there have been many attempts to add state in traits/module composition languages:
\begin{itemize}  
\item An early approach is to have {\bf no constructors}: all the fields start with {\bf null} or a default specified value.
  Fields are just like another kind of (abstract) member, and two fields
  with identical types can be merged by sum/use; \Q@new C()@ can be used for all classes, and \Q@init@ methods may be called later, as in
  \Q@Point p=new Point(); p.init(10,30)@.
  
  To its credit, this simple approach is commutative and associative and does not disrupt elegance of summing methods.
  However, objects are created "broken" and the user is trusted with fixing them.
  While is easy to add fields, the load of inizializing them is on the user; moreover
    all the objects are intrinsically mutable, so this model is unfriendly
    to a functional programming style.
\item {\bf Constructors compose fields}:
In this approach (used by \cite{fjig}) the fields are declared but not initialized, and
a canonical constructor (as in FJ) taking a value for each field and just initializing such field
is automatically generated in the resulting class.
It is easy to add fields, however this model is associative but not commutative: composition order influences field order, and thus the construtor signature.
Self construction is not possible 
since the signature of the constuctors changes during composition.

\item {\bf Constructors can be composed if they offer the same exact parameters}:
In this approach (used by DeepFJig) traits declare fields and constructors.
The constructor initializes the fields but can do any other computation.
Traits whose constructors have the same signature can be composed.
The composed constructor will execute both constructor bodies in order.
This approach is designed to allows Self construction.
It is also associative and mostly commutative: composition order only influences execution order of side effects during construction.
However constructor composition requires identical constructor signatures: this
hampers reuse, and if a field is added, its initial value needs to be
somehow syntetized from the constructor parameters.

\end{itemize}

%\item extension with nested class is natural and powefull
%We claim that our solution to the expression problem is the most natural in literature to date:
%While a similar sintax can be achieved with the scandinavian style~\cite{ernst2004expression}, their dependent type system makes reasoning quite complex, and indeed more recent solutions have accepted a more involved syntax in order to have an understandable type system~\cite{igarashi2005lightweight}.

%Our close contendant is DeepFJig~\cite{corradi2011deepfjig}: all our gain over their model is based on our relaxation over abstract signatures.
%Note how we solve the expression problem in the most radical way possible \marco{bruno, here you can insert some of the criticism over expression problem trivially and explain how is not applicable to us here.}

%Add that: original trait model has no self construction and types introduction not relevant since dynamicly typed language

\subsection{ThisType with Subclassing implying Subtyping}
With the exception of those mentioned 3 lines of work, to the best of our understading
other famous work in literature, like~\cite{odersky2008programming,nystrom2006j}
does not completely break the relation between inheritance and subtyping, but only prevents subtyping where 
it would be unsound.
Recently work on ThisType \cite{ToplasAndIgarashi}
also continue on this line. In those works, ``subtyping by subclassing'' is considered a valuable feature.

\begin{comment}
\subsection{Tablular comparision of many approaches}
\begin{minipage}[t]{0.4\textwidth}
In this table we show if some approach or part-there-of support certain critical features:
Direct instantation (as in \Q@new C()@),
Self instantiation (as in \Q@new This()@),
Unit of use, Unit of reuse,
Introduce type and if the Induced type is the type of \Q@this@,
support for binary methods,
does inheritance induce subtype?,
is code required to be well-typed before being inherited /imported in a new context?
is code required to be well-typed before composed with other code?

We use Y and X to mean yes and no, and we use ``-'' were the question is not really applicable to the current approach. For example the original trait model was untyped, so typing questions makes little sense.
\end{minipage}
%second column
\begin{minipage}[t]{0.6\textwidth}
${}_{}$
\begin{center}
\begin{tabular}{c|c|c|c|c|c|c|c|c|c|c}
&\Rotated{direct instantation}
&\Rotated{self instantiation}
&\Rotated{unit of use}
&\Rotated{unit of reuse}
&\Rotated{introduce type}
&\Rotated{induced type is this type}
&\Rotated{binary methods}
&\Rotated{inheritance induce subtype}
&\Rotated{well-typed before imported}
&\Rotated{well-typed before composed} 
\\
\hline
java/scala class&Y&X&Y&Y&Y&Y&X&Y&Y&X\\
java8 interface &X&X&X&Y&Y&Y&X&Y&Y&X\\
scala trait        &X&X&X&Y&Y&Y&X&Y&X&X\\
original trait     &X&X&X&Y&-&-&X&X&-&-\\
Ferruccio trait  &X&X&X&Y&X&-&X&X&Y&Y\\
\name trait            &X&Y&X&Y&X&-&Y&X&Y&X\\
\name class           &Y&Y&Y&X&Y&Y&Y&-&Y&X\\
module composition
                      &-&-&Y&Y&-&-&-&-&Y&Y\\
deepFJig class &Y&Y&Y&Y&Y&X&Y&X&Y&Y\\
package template
                      &X&Y&X&Y&X&-&X&X&Y&X\\

\end{tabular}
\end{center}
\end{minipage}

\bruno{
We can also mention what happens in structural types:
"System with structural types would *be required to* guarantee that
the structural type of A is a supertype of the structural type of B”.
}
\end{comment}

