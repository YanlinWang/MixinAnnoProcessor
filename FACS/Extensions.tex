\section{Extensions to our model}
  One of the main feature of our simple reuse/use model is that it can be
  easly extended. One simple but amazingly expressive extension is nested classes

\subsection{Nested classes}

A nested class will be another kind of member in the Literal, so  
the grammar can be updated as following:

\begin{bnf}
\prodFull\mMD{
\Opt{\terminalCode{class}}\ \terminalCode{method}\ \mT\ \mm\oR\overline{\mT\,\mx}\cR \Opt\me
\mid \mCD
}{Member Decl}\\
\prodFull\mT{\mC\mid\mC\terminalCode{.}\mT}{types are now paths}
\end{bnf}\\

The general idea is that by composing code with \use, nested classes with the same name are recursivelly composed.
Note that while we have nested classes, we do not have nested traits: all traits are still
at top level.
Untypable/unresolved Traits are also the only``dependency"
the type system keeps track of, this means that when a nested class at an arbirary
nested level is flattend, as in
\Q@C:{ D:{ E:Use t1,t2,L}}@
t1 and t2 must be defined before C at top level; and they may require classes (and their
nested) defined before C. This means that the type system can still consider
the class table as a simple map from Types T to their definition.

This extension lets us challenge the expression problem~\cite{wadler1998expression}:
in the expression problem we have data variants and operations and....

Let see how to easily encode and solve the expression problem:

\begin{lstlisting}
exp:{Exp:{interface}}//Exp declared once, reused everywhere
lit:Use exp,{ Lit:{implements Exp //Exp not explicitly declared
    class method Lit of(int inner) //Lit abstract state
    method int inner()}
  }
sum:Use  exp,{ Sum:{implements Exp 
    class method This of(Exp left, Exp right)
    method Exp left() //Sum abstract state
    method Exp right()}
  }  
uminus:Use exp,{ UMinus:{implements Exp 
    class method This of(exp inner)//and so on for
    method Exp inner()}//all the needed datavariants
  }   
  
expToS:{Exp:{interface method String toString()}}
//concept of toString declared once

sumToS:Use sum,expToS,{ Sum:{implements Exp//with Exp.toString
    method String toString()//just the implementation of the
      left.toString()+"+"+right.toString()//specific method
  }
uminusToS:...//implement toString for all the datavariants

expEval:{Exp:{interface method int eval()}}
//declare the next operation and implement it for all the datavariant
\end{lstlisting}

Now that you have nicely modularized the code, just compose all the traits you need.
\begin{lstlisting}
MySolution:Use sumToS,litToS
//sum,lit and exp traits are alread included
\end{lstlisting}

The expression problem presented up to now is the traditional challenge proposed by~\cite{wadler1998expression};
this has been criticized to not really address the fundamental issues since it does not handle ....
Now we show how we can go behond the traditional expression problem by encoding transformer methods:
For example, lets add 1 to all literals
\begin{lstlisting}
expAdd1:{Exp:{interface method Exp add1()}}
sumAdd1:Use sum,expAdd1,{Sum:{implements Exp
    method Exp add1()
      Sum.of(left.add1,right.add1())
  }
litAdd1:Use lit,expAdd1,{Lit:{implements Exp
    method Exp add1()
      Lit.of(inner()+1);
    }

MySolutionAdd1:Use sumToS,litToS,sumAdd1,litAdd1
\end{lstlisting}

This nicely solve our problem. 
However, notice how if we wished to add many similar operations we would 
have to repeat the propagation code (as in \Q@sumAdd1@) many times
just changing the name of the operation.
In the next section we will show how to improve on this point.


\subsection{More composition operators}
\use\ is amazing, elegant and simple, but our system can be easily enriched with more 
operators: while most approaches in literature presents a fixed set of operators, we do not need such restriction.
We just need to be sure that every newly added operator respects the following criteria:

\begin{itemize}
\item As for \use, the operato does not need to be total, but if it fails it needs to provide an error that will be reported to the programmer.
\item When the operator takes in input only traits (they are going to be well typed), if a result is produced,
 such result is well typed.
\item When the operator takes in input also code literal, if a non well typed result is produced,
the type error must be tracked back to code in one of those non typed yet code literals.
 \end{itemize}


Here we see some example of other composition operators, inspired from~\cite{servetto2014meta}:
\paragraph*{Restrict}${}_{}$\\*
Restrict makes a method abstract
\begin{lstlisting}
t:{method bool geq(This x) x.leq(this)   method bool leq(This x) x.qeq(this) }
C:Use t[restrict geq],{method bool geq(This x)  /*actual geq impl*/}
\end{lstlisting}

A variant of this operator allows to move the implementation to another name. This is very usefull to implement \Q@super@. This also is enough to suppor the example of before, when we try to compose \Q@colored@
with \Q@flavoured@.

\begin{lstlisting}
  colored:{/*as before*/}
  flavoured:{
  method Flavour flavour()
    method This withFlavour(Flavour that)
    method This merge(This other)
      this.withFlavour(this.flavour().mix(that.flavour())
    }
  CPoint:Use colored[restrict merge as _1merge],flavoured[restrict merge as _2merge],
  pointSum,{
    class method This of(int x,int y) This.of(x,y,Color.of(/*red*/))
    class method This of(int x, int y,Color color)
    method This merge(This other)
      this._1merge(other)._2merge(other)
    }
\end{lstlisting}

Note how we are leveraging on the fact that the code literal does not need to be complete, thus we can just call \Q@_1merge()@ and \Q@_2merge()@


\paragraph*{Rename}${}_{}$\\*
Rename allows to make some form of ``compile time'' refactoring
There are a lots of different forms of rename in literature,
sometime allowing only to rename specific methods, sometime allowing to rename
nested classes into other nested classes either at the same or at a different nesting level.
Renaming in the context of nested classes also means that when renaming a method of an interface, all the 
nested classes implementing such interface inside of that code literal need to be adjusted.
Renaming need to rename not only the method headers, but all the method calls inside of method bodies.
At first glimps, this seams to be not always possible since we are considering to be able to apply those
operators also to non well typed code.
However, if the expression language is simple enough, it is possible to pre process the code to
annotate the expected receiver type on all method calls by doing a purelly sintactic analysis
on a single code literal in isolation. 
All the expression whose type is guessed to be out of the border of the literal can stay unannotated; they are not going to be renamed anyway.

\begin{lstlisting}
t:{ I:{interface method int mI() }
     A:{implements I  method int mI() 42}
     B:{ method int mB(I i, A a, C c) i.mI()+a.mI()+c.mI()}
     //mB would be annotated i[I].mI()+a[A].mI()+c.mI()
     }
D:t[rename A.mI kI]
\end{lstlisting}
 Notice how we are sure that c does not implements I since it is invisible from the outside: traits does not introduce nominal types!
 
 We expect the flattened version for \Q@D@ to be
\begin{lstlisting}
D:{ I:{interface method int kI() }
     A:{implements I  method int kI() 42}
     B:{ method int mB(I i, A a, C c) i.kI()+a.kI()+c.mI()}
     }
\end{lstlisting}

Hide can be seen as a variation of rename, where the method/class is renamed to a fresh unguessable name.

\paragraph*{Redirect}${}_{}$\\*
Redirect allows to emulate generics; the main idea is that a (fully abstract) class can be redirect to another one external to the trait/code literal.
For example a linked list can be implement as
\begin{lstlisting}
list:{ Elem:{}
     Cell:{class method Cell of(Elem e,Cell c) 
       method Elem e()  method Cell c()
       }
   method Elem get(int x) ...
   ...more methods..
   }
ListString:list[redirect Elem to String]
\end{lstlisting}

%An expressive form of Redirect can be multiple, that is, can redirect may interdependent classes at the same time.
%We show a graph example, where also we can show how to propagate generics:
%For example
%\begin{lstlisting}
%t:{ method boolean reachable(Node start, Node end)/*implements reachability*/
%     Node:{method ListEdge nodes()}
%     Edge:{method Node in()  method Node out()}
%     ListEdge:list[redirect Elem to Edge]
%     }
%\end{lstlisting}



%Redirect can be multiple.

\paragraph*{Application on the expression problem}${}_{}$\\*
With redirect, rename and restrict we can have the general operator propagator
\begin{lstlisting}
operation:{//for sum and lit, easy to extends as before
  T:{}
  Exp:{interface method Exp op(T x)}}
  Sum:Use sum,{ extends Exp sum,expAdd1,{
    method Exp op(T x)
      Sum.of(left.op(x),right.op(x))  }
  Lit:Use lit,{
    method Exp op(T x)  this
  }
\end{lstlisting}

Now, to have my addN I can

\begin{lstlisting}
opAddn: Use
  operation[redirect T to Int]
    [rename Exp.op(x) to addN(x)][restrict Lit.op(x)], {
  Lit:{method Exp addN(Int x) Lit.of(inner())+x}
  }
\end{lstlisting}  



\paragraph*{Full power of redirect}${}_{}$\\*
An expressive form of Redirect can be multiple, that is, can redirect may interdependent classes at the same time.
We show an example where a specific kind of \Q@Service@ can produce a \Q@Report@, and 
\Q@Report@s can be combined together.
The goal is to execute a list of such services and produce a collated report.
This example also show how to propagate generics:

\begin{lstlisting}
Service:{interface method Void performService()}
serviceCombinator:{
  S:{implements Service method R report()  }
  
  R:{method R combine(R that)   class method R empty() }
  
  ListS:list[redirect Elem to S]
  
  class method R doAll(ListS ss){//here we use extended java like syntax
    R r=R.empty()
    for(S s in ss){
      s.performService();
      r=r.combine(s.report())
      }
    return r;
  }
}
PaintingService:serviceCombinator[redirect S to PaintingService]
PaintingService:{... method PaintingReport report()..}
PaintingReport:{..}
\end{lstlisting}

The flattened version of \Q@PaintingService@ would look like:
\begin{lstlisting}
PaintingService:{
  ListS:/*the expansion of list[redirect Elem to PaintingService]*/
  
  class method PaintingReport doAll(ListS ss){
    PaintingReport r=PaintingReport.empty()
    for(PaintingService s in ss){
      s.performService();
      r=r.combine(s.report())
      }
    return r;
  }
}
\end{lstlisting}
Where you can note how redirect figured out \Q@R=PaintingReport@ by comparing the structural shape of
classes \Q@PaintingService@ and \Q@S@.

To encode the former generic code in java you need to write
the following headeche inducing interfaces for RService and Report.
and require that the services you want to serve implement those.
\begin{lstlisting}
interface Service{ void performService();}
interface Report<R extends Report<R>>{R combine(R that);}
interface RService<R extends Report<R>> extends Service{ R report();}
\end{lstlisting}
Note how we still can not encode the method \Q@empty@.
