\section{Introduction}

Historically, we have seen a lot of focus on the importance of
separing subtype from subclassing~\cite{cook}.  This is claimed to be
good for code reuse, design and reasoning.  While there is no problem
in subtyping without subclassing, in most OO languages code reuse
(inheritance/extends) imply subtyping.  Consider the following (Java)
code:

\begin{lstlisting}[language=Java]
  class A{ int ma(){return Utils.m(this);} }
  class Utils{static int m(A a){..}}
\end{lstlisting}

This code seems correct, and there is no subtyping/subclassing.
Now, lets add a class \Q@B@

\begin{lstlisting}[language=Java]
  class B extends A{ int mb(){return this.ma();} }
\end{lstlisting}

Class \lstinline{B} now have a method \lstinline{ma} inherited from \lstinline{A}.
This method passes \lstinline{this} as \lstinline{A}; we can see an invocation of \lstinline{ma}
inside \lstinline{mb}, where \lstinline{this} is of type \lstinline{B}.
The execution will eventually call \lstinline{Util.m} with a \lstinline{B} instance.
This can be correct only if \lstinline{B} is a subtype of \lstinline{A}.
Thus, every OO language with the minimal features 
exposed in this example (using this, extends, method call)
is forced to accept that subclassing implies subtyping.
Note how this is true also in C++, where is possible to
"extends privately". Such is a limitation of the visibility of
subtyping but not over subtyping itself, and the former example
would be accepted by C++ even if \lstinline{B} was to "privately extends" \lstinline{A}.
\marco{Cite some work of bruceTHisType and show how he also fails.}
\bruno{I was expecting some text/argument here on why subclassing
  implies subtyping is bad. Perhaps another example where subclassing 
must not imply subtyping to type-check.}


We will show here a simple design to completley uncouple subtyping
and subclassing in a nominally typed OO language,
by dividing code designed for \textbf{USE}
from code designed for \textbf{REUSE}.\footnote{We need to talk of unanticipated extensions?}
First we show a minimal language, then we show how mutually recursive types
are supported, how state/constructors/fields are supported,
how we can extend the language with nested classes.
Finally we show 42, a full blown language build around our ideas of reuse.

Our design leverage on traits~\cite{}: a well know mechanisms for pure
code reuse

