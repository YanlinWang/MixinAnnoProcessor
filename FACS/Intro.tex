\section{Introduction}\label{sec:intro}
\saveSpace\saveSpace
In mainstream OO languages like Java, C++ or C\# subclassing 
implies subtyping. For example, in Java a subclass definition, such as 
\Q@class A extends B {}@
\noindent does two things at the same time:
it {\bf inherits} code from \lstinline{B}; and it {\bf creates
a subtype} of \emph{B}. Therefore in a language like Java 
a subclass is \emph{always} a subtype of the extended class.

Historically, there has been a lot of focus on
separating subtyping from subclassing~\cite{cook}.  This is claimed to be
good for code-reuse, design and reasoning. There are at
least two distinct situations where the separation of subtyping and 
subclassing is helpful.

\begin{itemize}

\item {\bf Allowing inheritance/reuse even when subtyping is impossible:} 
In some situations a subclass contains methods whose signatures 
are incompatible with the superclass, yet inheritance is still
possible. A typical example, which was illustrated by Cook et al.~\cite{cook}, are 
classes with \emph{binary methods}~\cite{bruce96binary}.

\item {\bf Preventing unintended subtyping:} For certain classes we
  would like to inherit code without creating a subtype even if, from
  the typing point of view, subtyping is still possible. A typical
  example~\cite{LaLonde:1991:SSS:110673.110679} of this are methods for collection classes such as Sets and
  Bags. Bag implementations can often inherit 
  from Set implementations, and the interfaces of the two collection types are
  similar and type compatible. 
  However, from the logical point-of-view a Bag is \emph{not a
    subtype} of a Set. 

\end{itemize}

Type systems based on structural typing can deal with the first
situation well, but not the second. Since structural subtyping
accounts for the types of the methods only, a Bag would be a subtype
of a Set if the two interfaces are type compatible. For dealing with
the second situation nominal subtyping is preferable. With nominal
subtyping an explicit subtyping relation must be signalled by the
programmer. Thus if subtyping is not desired, the idea is that 
programmer can simply {\bf not} declare a subtyping relationship.

While there is no problem in subtyping without subclassing, in the design
of most nominal OO languages subtyping implies subtyping in a
fundamental way. This is because of what we call the
\emph{this-leaking problem}, illustrated by the following
(Java) code:

\begin{lstlisting}[language=Java]
  class A{ int ma(){return Utils.m(this);} }
  class Utils{static int m(A a){..}}
\end{lstlisting}

Method \lstinline{A.ma} passes \lstinline{this} as \lstinline{A} to \Q@Util.m@.
This code seems correct, and there is no subtyping/subclassing.  Now, lets add a class \Q@B@

\begin{lstlisting}[language=Java]
  class B extends A{ int mb(){return this.ma();} }  
\end{lstlisting}

%%Class \lstinline{B} does two things at the same time: 
%%1) it {\bf inherits} the method \lstinline{ma} from
%%\lstinline{A}; and 2) it creates a {\bf subtype} of \lstinline{A}.
We can see an invocation of \lstinline{A.ma} inside
\lstinline{B.mb}, where the self-reference \lstinline{this} is of type \lstinline{B}. 
The execution will eventually call \lstinline{Util.m} with an
instance of \lstinline{B}. However, this can be correct only if \lstinline{B} is a subtype of
\lstinline{A}. 

%If Java was to support a mechanism to allow reuse/inheritance 
%without introducing subtyping, such as:
%
%\begin{lstlisting}[language=Java]
%  class B inherits A{ int mb(){return this.ma();} }
%\end{lstlisting}
%
%\noindent Then an invocation of 
%\lstinline{mb} would be type-unsafe (i.e. it would 
%result in a run-time type error). 
%Note that here the intention of using the imaginary keyword {\bf
%  inherits} is to allow the code from \lstinline{A} to be inherited 
%without \lstinline{B} becoming a subtype of \lstinline{A}. 
%However this breaks type-safety. The problem is that the
%self-reference \lstinline{this} in class \lstinline{B} has 
%type \lstinline{B}. Thus, when \lstinline{this} is passed as an argument to 
%the method \lstinline{Utils.m} as a result of the invocation of
%\lstinline{mb}, it will have a type that is incompatible with the
%expected argument of type \lstinline{A}.  


As a though experiment, imagine that Java code-reuse (the {\bf extends} keyword) was not introducing subtyping: then an invocation of 
\lstinline{B.mb} would result in a run-time type error.
The problem is that the
self-reference \lstinline{this} in class \lstinline{B} has 
type \lstinline{B}. Thus, when \lstinline{this} is passed as an argument to 
the method \lstinline{Utils.m} as a result of the invocation of
\lstinline{mb}, it will have a type that is incompatible with the
expected argument of type \lstinline{A}.  
Therefore, every OO language with the minimal features exposed in the example (using \lstinline{this},
extends and method calls) is forced to accept that subclassing implies
subtyping\footnote{C++ allows to "extends privately", but it is a limitation over
  subtyping visibility, not over subtyping itself.  The
  former example would be accepted even if \lstinline{B} was to
  "privately extends" \lstinline{A}}.
  
In essence we believe that in languages like Java classes do too many
things at once. In particular they act both as units of \emph{use} and
\emph{reuse}: classes can be \emph{use}d as types and can be instantiated;
classes can also be subclassed to provide \emph{reuse} of code.

What the this-leaking problem shows is that adopting a more flexible
nominally typed OO model where subclassing does not imply subtyping is
not trivial, and a more substantial change in the language design is
necessary.  We are aware of at least 3 independently designed research
languages that address this limitation:
\begin{itemize}
\item In {\bf Delta-Trait (DT)}~\cite{Bettini:2010:ISP:1774088.1774530,BETTINI2013521,Bettini2015282}
each  construct has a single responsability: classes instantiate objects,
interfaces induce types, records express state and traits are reuse units.
\item {\bf Package Templates (PT)}~\cite{KrogdahlMS09,DBLP:journals/taosd/AxelsenSKM12,DBLP:conf/gpce/AxelsenK12}:
are an extension of (full) Java where new packages can be ``syntetized'' by mixing
and integrating code templates. As an extension of java, PT allows but not requires
separation of inheritance and subtyping.
\item {\bf
    DeepFjig}~\cite{deep,servetto2014meta,fjig} is
a module composition language where the main idea is that
nested classes with the same name are recursivelly composed.
\end{itemize}
\bruno{We need here a better motivation. From the reviewer perspactive
he will think: ``So, there are already 3 solutions, why should I care
about a new one?''}
This paper aims at showing a simple language design, called \name, to
completely decouple subtyping and subclassing in a nominally typed OO
language. The key idea is to divide between code designed for
\textbf{USE} and code designed for \textbf{REUSE}. 
In \name there are two separate concepts: classes
and traits~\cite{Traits:ECOOP2003}. Classes are meant for code use, and cannot be used
for reuse. In some sense classes in \name are like final classes in
Java. Traits are meant for code reuse and multiple traits can be
composed to form a class which can then be instantiated. Traits 
cannot be instantiated (or used) directly. Such design allows the
subtyping and code reuse to be treated separately, which in turn
brings several benefits in terms of flexibility and code reuse.
In summary, our contributions are:

%\marco{We need to talk of unanticipated extensions?}
%\bruno{Talk more about typing aspects here. Summarize the important 
%aspects of the design of \name.}

%\bruno{Need to say something about state?}

%The language design of \name is adopted by the 42 programming 
%language, All the examples shown in this paper can be run in 42.
%The compiler for 42 is available at: 

%\url{http://l42.is}



\begin{itemize}
\item We identify the {\bf this-leaking} problem and argue why it
  makes the separation of subclassing and subtyping difficult.
\item We synthesize the key ideas of previous designs that solve the
  this-leaking problem into {\bf a novel and
  simple language design}. This language is the logic core of the language 42, and 
  all the examples in the paper can be directly encoded as valid 42 programs. 

\item We illustrate how the new design {\bf improves both code use and code
  reuse}.
\item We propose {\bf a clean and elegant approach to handle of state} in a trait based language.

%\item The new language design is {\bf adopted by the 42 language}. All the
 % examples in the paper are valid 42 programs. 
\end{itemize}


\saveSpace
\saveSpace

