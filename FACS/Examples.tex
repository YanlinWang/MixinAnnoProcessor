\saveSpace\saveSpace\section{Intuitions on formalization}\label{sec:formal}
\saveSpace\saveSpace
In this article we dedicate more space to examples and informal presentation and motivations;
so we do not have space to provide a full formalizations.
We will provide here some hints on how the formalization works.

In the following, we present a very simplified grammar:
%\begin{bnf}
%\prodFull{aa}{bb}{Declaration}\\
%\end{bnf}

%\begin{comment}
\begin{bnf}
\prodFull\mTD{\mt\terminalCode{:}\mL \mid \mt\terminalCode{:} \use\ \overline\mV}{Trait Decl}\\
\prodFull\mCD{\mC\terminalCode{:}\mL \mid \mC\terminalCode{:} \use\ \overline\mV}{Class Decl}\\
\prodFull\mV{\mt \mid \mL}{Code Value}\\
\prodFull\mL{
\oC
\Opt{\terminalCode{interface}}\ \terminalCode{implements} \overline\mT\ \overline\mMD
\cC
}{Code Literal}\\
\prodFull\mT{\mC}{types are class names}\\
\prodFull\mMD{\Opt{\terminalCode{class}}\ \terminalCode{method}\ \mT\ \mm\oR\overline{\mT\,\mx}\cR \Opt\me}{Method Decl}\\

\prodFull\me{\mx\mid\mT\mid\me\terminalCode{.}\mm\oR\overline\me\cR}{expressions}\\
\prodFull\mD{\mCD\mid\mTD}{Declaration}\\
\end{bnf}
%\end{comment}

To declare a trait \mTD\ or a class \mCD, we can use either a code literal \mL\ or a trait
expression.  Traits come with various operators (restrict, hide,
alias) but in this work we focus on the single operator 
$\use$, taking a set
of code values: that is trait names \mt\ or literals \mL\ and composing them.  This operation, sometimes called \emph{sum}, is the simplest and most elegant
trait composition operator.  $\use\ \overline\mV$ composes the content of $\overline\mV$
by taking the union of the methods and the union of the implementations.

\use\ can not be applied if multiple versions of the same method are
present in different traits.  An exception is done for abstract methods:
methods where the implementation \me\ is missing. In this case (if the
headers are compatible) the implemented version is selected.  In a sum
of two abstract methods with compatible headers, the one with the more
specific type is selected.

Code literals \mL\ can be marked as interfaces. 
That is, the interface keyword is inside the curly brakets, so an upper case name associated with an interface literal is a class-interface, while a lowercase one is a trait-interface.
In our simple model, we consider an error trying to merge an interface with a non-interface.
 Then we have a set of implemented interfaces and a set of member
  declarations. In this simple language, the only members are methods.
If there are no implemented interfaces, in the concrete syntax we will omit the \Q@implements@ keyword.

Methods \mMD~can be instance methods or \Q@class@ methods. A class method is similar to a \Q@static@ method in Java but can be abstract. This is very usefull in the context of code composition.
To denote a method as abstract, instead of an optional keyword we just omit the implementation \me.

A version of this language where there are no traits can be seen 
as a restriction/variation of FJ~\cite{igarashi2001featherweight}.

\paragraph{Well-formedness}
Basic well formedness rules apply:
\begin{itemize}
\item all method parameters have unique names and the special parameter name \Q@this@ is not declared
 in the parameter list,
\item all methods in a code literal have unique names,
\item all used variables are in scope,
\item all methods in an interface are abstract, and there are no interface class methods.
\end{itemize}
Those rules can be applied on any given \mL~individually and in full isolation.

We expect the type system to enforce: 
\begin{itemize}
\item all the traits and classes have unique names in a program $\overline\mD$, and the special class name
\Q@This@ is reserved,
\item all used types correspond to class declarations in the program, or are \Q@This@, 
\item subtyping between interfaces and classes,
\item method call typechecking,
\item no circular implementation of interfaces,
\item type signature of methods from interfaces can be refined following the well known variant-contravariant rules,
\item only interfaces can be implemented.
\item \marco{I'm sure I'm missing something}
\end{itemize}
While classes are typed assuming \lstinline{this} is of the nominal type of the
class, trait declarations, do not introduce any nominal type.  \lstinline{this}
in a trait is typed with a special type \lstinline{This} that is visible only
inside such trait. Syntactically, \Q@This@ is just a special, reserved, class name $\mC$.
A Literal can use the \lstinline{This} type,
and when flattening completes creating a class definition, \Q@This@ will be replaced with such class name.

For the sake of simplicity, method bodies are just simple expressions
\me: they can be just variables, types and method calls. We need types as part of expressions in order to use them as receivers for class methods.

\subsection{Remarks on Typing}
 Our typing discipline is 
what distinguishes our approach from a simple minded code composition macro~\cite{bawden1999quasiquotation}
or a rigid module composition~\cite{ancona2002calculus}.

The are two core ideas:
\paragraph{1: traits are \emph{well-typed} before being reused.}${}_{}$\\*
 For example in

\saveSpace\begin{lstlisting}
t:{method int m() 2 
   method int n() this.m()+1}
\end{lstlisting}\saveSpace

\noindent \Q@t@ is well typed since \Q@m()@ is declared inside of \Q@t@, while

\saveSpace\begin{lstlisting}
t1:{method int n() this.m()+1} 
\end{lstlisting}\saveSpace
\noindent would be ill typed.

\paragraph{2: code literals are not required to be \emph{well-typed} before flattening.}${}_{}$\\*
In class expressions  $\use\ \overline\mV$
an \mL\ in $\overline\mV$ is not typechecked before flattening, and only the result is expected to be well-typed.
While this seems a very dangerous approach at first, consider that also Java have the same behaviour:
for example in
\saveSpace\begin{lstlisting}[language=Java]
  class A{ int m() {return 2;}  int n(){return this.m()+1;} }
  class B extends A{ int mb(){return this.ma();} }
\end{lstlisting}\saveSpace
\noindent in \Q@B@ we can call \lstinline{this.ma()} even if in the curly braces there is no declaration for \Q@ma()@.
In our example, using the trait \Q@t@ of before

\saveSpace\begin{lstlisting}
C: Use t {method int k() this.n()+this.m()}
\end{lstlisting}\saveSpace
\noindent would be correct: even if n,m are not defined inside
\Q@{method int k() this.n()+this.m()}@,
the result of the flattening is well typed.

This is not the case in many similar works in literature~\cite{deep,Bettini2015282,Bergel2007} where the
literals have to be self complete. In this case we would have been forced to
declare abstract methods \Q@n@ and \Q@m@.

Our typing strategy has two important properties:
\begin{itemize}
\item If a class is declared by using $\mC : \use\ \overline\mt$, that is, without literals,
and the flattening is successfull, \mC\ is well typed, no need of further checking.
\item On the other side, if a class is declared by $\mC : \use\ \overline\mV$, with
$\mL_1\ldots\mL_n \in \overline\mV$, and after successfull flattening $\mC : \mL$ can not be typechecked,
then the issue was originally present in one of $\mL_1\ldots\mL_n$.
It may be that the result is intrinsically ill-typed, if one of the methods in $\mL_1\ldots\mL_n$ is not well typed,
but it may also happen that a type referred from one of those methods is declared \emph{after} the current class. As we will see later, this is how our relaxation allows to support (indirectly) recursive types.

This also means that as an optimization strategy
 we may remember what method bodies come from traits and what method bodies come from code literals, in order to typecheck only the latter.
 \end{itemize}

 \subsection{Recursive types}

OO languages leverage on recursive types most of the times.
For example in a pure OO language, \Q@String@ may offers a \Q@Int size()@
method, and \Q@Int@ may offer a \Q@String toString()@ method.

This means that is not possible to type in (full) isolation classes
\Q@String@ and \Q@Int@.

The most expressive compilation process may divide the program in groups of mutually 
dependent classes.
Each group may also depend from a number of other groups.
This would form a Direct Aciclyc Graph of groups.
To type a group, we first need to type all depended groups, then
we can extract the structure/signature/structural type of all
the classes of the group.
Now, with the information of the depended groups and the one extracted
from the current group, it is possible to typecheck the implementation
 of each class in the group.
In this model, it is reasonable to assume that flattening happens group by group, before
extracting the class signatures.

Here we go for a much simpler simple top down execution/interpretation for flattening, where flattening
happen one at the time, and classes are typechecked where their type is first needed.
That is, In our approach typing and flattening interleaves. We assume our compilation process to stop as soon as 
an error arise. There are two main kinds of errors: Type errors (like method not found) or Composition errors (like summing two conflicting implementation for the same method).
For example
\saveSpace\begin{lstlisting}
A:{method int ma(B b) b.mb()+1}
tb:{method int mb() 2}
tc:{method int mc(A a,B b) a.ma(b)}
B: Use tb
C: Use tc, {method int hello() 1}
\end{lstlisting}\saveSpace
In this scenario, since we go top down, we first need to generate \Q@B@.
To generate \Q@B@, we need to use \Q@tb@;
In order to modularly ensure well typedness,
we require \Q@tb@ to be well typed at this stage. If \Q@tb@ was not well typed
a type error could be generated at this stage.
In this moment, \Q@A@ can not be compiled/checked alone,
we need informations about \Q@B@, but \Q@A@ is not used in \Q@tb@,
thus we do not need to type \Q@A@ and we can type \Q@tb@ with
 the available informations and proceed to generate \Q@B@.
Now, we need to generate \Q@C@, and we need to ensure well typedness of \Q@tc@.
Now \Q@B@ is alreay well typed (since generated by \use\ \Q@tb@, with no \mL),
and \Q@A@ can be typed;  finally \Q@tc@ can be typed and used.
If \use\ could not be performed (for example it \Q@tc@ had a method \Q@hello@ too)
a composition error could be generated at this stage.
On the opposite side, if \Q@B@ and \Q@C@ was swapped, as in
\saveSpace\begin{lstlisting}
C: Use tc, {method int hello() 1}  
B: Use tb
\end{lstlisting}\saveSpace
\noindent
now the first task would be to generate \Q@C@, but 
to type \Q@tc@ we need to know the type of \Q@A@ and \Q@B@.
But they are both unavailable: \Q@B@ is still not computed and 
\Q@A@ can not be compiled/checked without information about \Q@B@.
A type error would be generated, on the line of ``flattening of \Q@C@
requires \Q@tc@, \Q@tc@ requires \Q@A@,\Q@B@, but \Q@B@ is still in need of flattening".

In this example, a more expressive compilation/precompilation process 
could compute a dependency graph and, if possible, reorganize the list,
but for simplicity lets consider to always provide the declarations
in the right order, if one exists.

\paragraph{Criticism: existence of an order is restrictive.}${}_{}$\\*
Some may find the requirement of the existence of an order restrictive;
An example of a ``morally correct" program where no right order exists is the following:
\saveSpace\begin{lstlisting}
t:{ int mt(A a) a.ma()}
A:Use t {int ma() 1}
\end{lstlisting}\saveSpace

In a system without inference for method types,
if the result of composition operators depends only on the
structural shape of their input (as for \use)
is indeed possible to optimistically compute the resulting structural shape of the classes
and use it to type involved examples like the former.
We stick to our simple approach, since we believe such typing discipline would be fragile,
and could make human understanding the code-reuse process much harder/involved.
Indeed we just wrote an involved program where the correctness of trait \Q@t@ depends of 
\Q@A@, that is in turn generated using trait \Q@t@.

\paragraph{Criticism: it would be better to typecheck before flattening.}${}_{}$\\*
In the world of strongly typed languages we are tempted to
first check that all can go well, and then perform the flattening.
This would however be overcompicated for no observable difference:
Indeed, in the \Q@A,B,C@ example above there is no difference
between
\begin{itemize}
\item  (1)First check \Q@B@ and produce \Q@B@ code (that also contains \Q@B@ structural shape),
  (2) then use \Q@B@ shape to check \Q@C@ and produce \Q@C@ code;\ 
or a more involved
\item  (1)First check \Q@B@ and discover just \Q@B@ structural shape as result of the checking,
  (2)then use \Q@B@ shape to check \Q@C@.
  (3) Finally produce both \Q@B@ and \Q@C@ code.
\end{itemize}

Note that we can reuse code only by naming traits; but our only point of relaxation is {\bf only} the code literal: there is no way an error can ``move around'' and be duplicated during the compilation process.
In particular, our approach allows for safe libraries of traits and classes to be fully typechecked, deployed and reused by multiple clients: no type error will emerge from library code.
On the other side, we do not enforce the programmer to write always self-contained code where all the abstract method definition are explicitly declared.
